\newcommand\N{\ensuremath{\mathbb{N}}} 
\newcommand\R{\ensuremath{\mathbb{R}}} 
\newcommand\A{\ensuremath{\mathbb{A}}} %affine space
\newcommand\Z{\ensuremath{\mathbb{Z}}} 
\renewcommand\O{\ensuremath{\emptyset}} 
\newcommand\Q{\ensuremath{\mathbb{Q}}} 
\newcommand\C{\ensuremath{\mathbb{C}}}
\newcommand\F{\ensuremath{\mathbb{F}}} %field
\newcommand\E{\ensuremath{\mathbb{E}}} %field extension
\renewcommand\P{\ensuremath{\mathbb{P}}} %projective space
\renewcommand\H{\ensuremath{\mathbb{H}}} %hyperbolic space
\newcommand\im{\ensuremath{\operatorname{im}}} %image
\newcommand\rank{\ensuremath{\operatorname{rank}}} %rank
\newcommand\id{\ensuremath{\operatorname{id}}} %identity map
\newcommand\grad{\ensuremath{\operatorname{grad}}} %gradient
\newcommand\curl{\ensuremath{\operatorname{curl}}} %gurl
\renewcommand\div{\ensuremath{\operatorname{div}}} %divergence
\newcommand\Gr{\ensuremath{\operatorname{Gr}}} %grassmannian
\newcommand\Hom{\ensuremath{\operatorname{Hom}}} %linear mappings

\documentclass[xcolor=dvipsnames]{beamer} 
\usetheme{Copenhagen} 

\definecolor{burntorange}{RGB}{191,87,0}
\definecolor{utgrey}{RGB}{51,63,72} 

\setbeamercolor{palette primary}{bg=burntorange,fg=white}
\setbeamercolor{palette secondary}{bg=burntorange,fg=white}
\setbeamercolor{palette tertiary}{bg=burntorange,fg=white}
\setbeamercolor{palette quaternary}{bg=burntorange,fg=white}
\setbeamercolor{structure}{fg=burntorange} % itemize, enumerate, etc
\setbeamercolor{section in toc}{fg=burntorange} % TOC sections
\setbeamercolor{subsection in head/foot}{bg=burntorange,fg=white}
\setbeamercolor{block title example}{fg=white,bg=burntorange}
%\setbeamercolor{block body example}{fg=white,bg=burntorange}

\usepackage[utf8]{inputenc} 
\title{Motivating de Rham cohomology} 
\subtitle{why?} 
\author{Simon Xiang} 
\institute{University of Texas at Austin} 
\begin{document} 
    \begin{frame}
        \titlepage
    \end{frame}

    \begin{frame}
        \frametitle{Prerequisites} 
        Here are some things I'll assume you know about:\pause
        \begin{itemize}
            \item Linear algebra, including quotient spaces, exact sequences\pause
            \item Multivariable calculus up to Green's theorem and friends\pause
            \item Analysis, including the notions of open and closed sets\pause
        \end{itemize}
        It would be helpful to know (algebra, exact sequences)
        \begin{itemize}
            \item What groups are,\pause
            \item And basic topology.
        \end{itemize}
    \end{frame}

    \begin{frame}
        \frametitle{Motivation} 
        \begin{exampleblock}{Question} 
            Does there exist a function that is the gradient of some other function? More precisely, when does $F \colon U \to \R^2$ for $U \subseteq \R^2$ satisfy \[
                \frac{\partial F}{\partial x}=f_1,\quad \frac{\partial F}{\partial y}=f_2 \quad \text{for some} \ f=(f_1,f_2)?
            \] (You could also think of this question as asking when vector fields have potential.)
        \end{exampleblock}\pause
        \begin{exampleblock}{Answer} 
           It depends on the topology of $U$! 
        \end{exampleblock}
    \end{frame}
    
    \begin{frame}
    \frametitle{Some vector calculus} 
    Note that $\frac{\partial F}{\partial x}=f_1,\ \frac{\partial F}{\partial y}=f_2$ implies $\frac{\partial f_1}{\partial y}=\frac{\partial f_2}{\partial x}$. Is this condition sufficient to show $F$ is the gradient of some other function?
    \begin{example}
        easy example of something ssatisfying on $\R^2$.
    \end{example}\pause
    \begin{example}
        However, consider $f(x,y)=\left( \frac{-y}{x^2+y^2},\frac{x}{x^2+y^2} \right) $.
    \end{example}\pause
    By a theorem, the above condition turns out to be sufficient if $U$ looks like a ball (convex).
    \end{frame}

    \begin{frame}
        \frametitle{more vector cal} 
        define div grad and curl
    \end{frame}

    \begin{frame}
        \frametitle{sneak peek of de rham cohomology} 
        Define cohomology groups, do some theorem, show reason why our ealire rfucntion is false is because $H(\R^2\setminus \{0\} )\neq 0$
    \end{frame}

    \begin{frame}
       Define the re Rham complexa nd bieng generated by the $dx^i $ 
    \end{frame}

    \begin{frame}
        calculate div grad curl using abstract, link between aglebra and calc: $d^2=0$
    \end{frame}
    \begin{frame}
        \frametitle{testing testing} 
        \begin{definition}
            what\alert{no} 
        \end{definition}
        \begin{itemize}
            \item<1->test 
            \item<2-> ok
            \item<3->te
            \end{itemize} 
    \end{frame}

    \begin{frame}
        \frametitle{what's the point?} 
        \begin{columns}
            \column{0.5\textwidth} 
            text
            \[
            R_{\mu \nu}-\frac{1}{2}R g_{\mu\nu}+\Lambda g_{\mu\nu}=\frac{8 \pi G}{c^4}T_{\mu\nu}
            \] 
            \begin{enumerate}
                \item what's the
                \item point?
            \end{enumerate}
            \column{0.5\textwidth} 
            text
            \[
            \nabla _{\beta }T^{\alpha \beta }=T^{\alpha \beta }_{\beta }=0
            \] 
            \begin{enumerate}
                \item of
            \end{enumerate}
        \end{columns}
    \end{frame}
\end{document}
