\documentclass[xcolor=dvipsnames]{beamer} 
\usetheme{Copenhagen} 

\definecolor{burntorange}{RGB}{191,87,0}
\definecolor{utgrey}{RGB}{51,63,72} 

\setbeamercolor{palette primary}{bg=burntorange,fg=white}
\setbeamercolor{palette secondary}{bg=burntorange,fg=white}
\setbeamercolor{palette tertiary}{bg=burntorange,fg=white}
\setbeamercolor{palette quaternary}{bg=burntorange,fg=white}
\setbeamercolor{structure}{fg=burntorange} % itemize, enumerate, etc
\setbeamercolor{section in toc}{fg=burntorange} % TOC sections
\setbeamercolor{subsection in head/foot}{bg=burntorange,fg=white}

\usepackage[utf8]{inputenc} 
\title{Motivating de Rham cohomology} 
\subtitle{why?} 
\author{Simon Xiang} 
\institute{University of Texas at Austin} 
\begin{document} 
    \begin{frame}
        \titlepage
    \end{frame}

    \begin{frame}
        \frametitle{Prerequisites} 
        Here are some things I'll assume you know about:\pause
        \begin{itemize}
            \item Linear algebra, including quotient spaces, exact sequences\pause
            \item Multivariable calculus up to Green's theorem and friends\pause
            \item Analysis, including the notions of open and closed sets\pause
        \end{itemize}
        It would be helpful to know
        \begin{itemize}
            \item What groups are,\pause
            \item And basic topology.
        \end{itemize}
    \end{frame}
    \begin{frame}
        \frametitle{Some vector calculus} 
        \begin{example}
            consider
        \end{example}\pause
        \small{why is this green??} \pause
        \begin{theorem}
            ok
        \end{theorem}\pause
        \begin{proof}
            pls
        \end{proof}
    \end{frame}
    

    \begin{frame}
        \frametitle{testing testing} 
        \begin{definition}
            what\alert{no} 
        \end{definition}
        \begin{itemize}
            \item<1-> hora
            \item<2-> mitete
            \item<3->te ne
            \end{itemize} 
    \end{frame}

    \begin{frame}
        \frametitle{what's the point?} 
        \begin{columns}
            \column{0.5\textwidth} 
            text
            \[
            R_{\mu \nu}-\frac{1}{2}R g_{\mu\nu}+\Lambda g_{\mu\nu}=\frac{8 \pi G}{c^4}T_{\mu\nu}
            \] 
            \begin{enumerate}
                \item what's the
                \item point?
            \end{enumerate}
            \column{0.5\textwidth} 
            text
            \[
            \nabla _{\beta }T^{\alpha \beta }=T^{\alpha \beta }_{\beta }=0
            \] 
            \begin{enumerate}
                \item of
                \item living?
            \end{enumerate}
        \end{columns}
    \end{frame}
\end{document}
