\newcommand\sgn{\ensuremath{\operatorname{sgn}}} %sign

\section{Motivation}
\subsection{Calculus}
\begin{namedthing}{Question}
    Let $f \colon U \to \R^2$ be a smooth function, where $U \subseteq \R^2$ is open. Is there a smooth function $F \colon U \to \R$ such that $\partial _{x_1}F=f_1, \ \partial _{x_2}F=f_2,$ where $f=(f_1,f_2)$? Note that this implies $\partial _{x_2}f_1=\partial _{x_1}f_2$. Is this a sufficient condition to show the existence of $F$?
\end{namedthing}
\begin{example}\label{failure}
    Consider $f \colon \R^2 \to \R^2$, where \[
        f(x_1,x_2)= \left( \frac{-x_2}{x_1^2+x_2^2} , \frac{x_1}{x_1^2+x_2^2}\right) 
    \] Now  
    \begin{gather*}
        \partial _{x_2}f_1= \frac{-(x_1^2+x_2^2)+2x_2^2}{(x_1^2+x_2^2)^2}=\frac{x_2^2-x_1^2}{(x_1^2+x_2^2)^2},\\ 
        \partial _{x_1}f_2= \frac{(x_1^2+x_2^2)-2x_1^2}{(x_1^2+x_2^2)^2}=\frac{x_2^2-x_1^2}{(x_1^2+x_2^2)^2}.
    \end{gather*} So $f$ satisfies $\partial _{x_2}f_1=\partial _{x_1}f_2$. However, we have no $F \colon \R^2 \setminus \{0\} \to \R$: assume there was such an $F$, then \[
    \int_{0}^{2\pi } \frac{d}{d\theta}F(\cos \theta, \sin \theta) \, d\theta=F(1,0)-F(1,0)=0.
    \] But \[
    \frac{d}{d\theta}F( \cos \theta ,\sin  \theta)= \frac{dF}{dx}(-\sin \theta)+\frac{\partial F}{\partial y}\cos \theta=-f_1(\cos \theta, \sin \theta) \sin \theta+f_2 (\cos \theta, \sin \theta) \cos \theta=1
    \] by the chain rule, a contradiction. So we have procured a counterexample.
\end{example}
\begin{definition}[Star-shaped]
    A subset $X\subseteq \R^n $ is \textbf{star-shaped} with respect to $x_0\in X$ if the line segment $\{tx_0+(1-t)x\mid t \in [0,1]\} $ is contained in $X$ for all $x \in X$.
\end{definition}
\begin{theorem}\label{star}
    Let $U \subseteq \R^2$ be open and star-shaped. Then for any smooth function $(f_1,f_2) \colon U \to \R^2$ satisfying $\partial _{x_2}f_1=\partial _{x_1}f_2$, there exists a smooth function $F \colon U \to \R$ such that $\partial _{x_1}F=f_2, \ \partial _{x_2}F=f_1$.
\end{theorem}
\begin{proof}
    Messy.
\end{proof}

\subsection{Sneak peek of cohomology}
Say $U \subseteq R^2$ is open, then let $C^{\infty}(U, \R^k)$ be the vector space of smooth functions $\phi \colon U \to \R^k$. Define the \textbf{gradient} and \textbf{curl} functions\footnote{The book uses \emph{rotation} instead of curl, but I think this is the standard notation.} $\operatorname{grad}\colon C^{\infty} (U,\R)\to C^{\infty}(U, \R^2), \ \operatorname{curl}\colon C^{\infty}(U,\R^2) \to C^{\infty}(U,\R)$ by \[
    \operatorname{grad}(\phi) = \left( \frac{\partial \phi}{\partial x_1}, \frac{\partial \phi}{\partial x_2} \right) , \qquad \operatorname{curl}(\phi_1,\phi_2)= \frac{\partial \phi_1}{\partial x_2}-\frac{\partial \phi_2}{\partial x_1}.
\] Note that the curl of the gradient is zero, or $\operatorname{curl}\circ \operatorname{grad}=0$. So the kernel of the curl contains the image of the gradient, since mapping $\operatorname{im}(\operatorname{grad})$ by $\operatorname{curl}$ gives zero. Since $\operatorname{curl}$ and $\operatorname{grad}$ are linear, both $\ker (\operatorname{curl})$ and $\operatorname{im}(\operatorname{grad})$ are (infinite-dimensional) vector spaces, furthermore, $\operatorname{im}(\operatorname{grad})$ is a subspace of $\ker (\operatorname{curl})$. So we can consider the quotient space (since vector spaces are abelian groups) $H_1(U)=\ker (\operatorname{curl}) / \operatorname{im}(\operatorname{grad})$. This is a sneak peek of the \emph{cohomology} groups (in this case, vector spaces) assigned to a space. Somehow the cohomology groups tend to be finite-dimensional.
\begin{figure}[H]
\centering
\begin{tikzcd}
\overset{\ker(\grad)}{H^0(U)} \arrow[r]                       & \overset{\ker(\operatorname{curl})/\operatorname{im}(\operatorname{grad})}{H^1(U)=0} \arrow[r] & H^2(U)                       \\
{C^{\infty}(U,\R)} \arrow[r, "\operatorname{grad}"] \arrow[u] & {\underset{\text{exact}}{C^{\infty}(U,\R^2)}} \arrow[r, "\operatorname{curl}"] \arrow[u]       & {C^{\infty}(U,\R)} \arrow[u]
\end{tikzcd}
\caption{The commutative diagram of gradient and curl for $U$ star-shaped.}
\end{figure}
Now \cref{star} becomes the statement ``$H^1(U)=0$ whenever $U \subseteq \R^2$ is star-shaped''. To see this, note that $\ker (\operatorname{curl})$ consists of precisely the functions $\phi \colon U \to \R^2$ such that $\partial _{x_2}\phi_1=\partial _{x_1}\phi_2$, and if the image of $\operatorname{grad}$ are such functions $\phi$ (since $\ker(\curl)=\im (\grad)$), then there must exist an $F \in C^{\infty}(U, \R) $ mapping onto $\phi=(f_1,f_2)$, where $\partial _{x_1}F=f_2, \ \partial _{x_2}=f_1$.

We know that $H^1(\R^2 \setminus \{0\} )\neq 0$, since \cref{failure} details a function in $\ker(\curl)$ that doesn't get mapped onto by $\im(\grad)$. We will see later that $H^1(\R^2 \setminus \{0\} )$ is $1$-dimensional as a vector space, and that $H^1(\R^2 \setminus \bigcup_{i=1} ^k \{x_i \} )\cong \R^k $. So the dimension of the cohomology groups gives us data about how many ``holes'' a space has. We will introduce cochain complexes and coboundaries later, but for now let us define $H^0(U)=\ker(\grad)$ analagously. This is well-defined for open sets $U\subseteq \R^k$ for $k\geq 1$, for \[
    \grad(f)= \left( \frac{\partial f}{\partial x_1},\cdots ,\frac{\partial f}{\partial x_n } \right).
\] 
\begin{theorem}
    An open set $U\subseteq \R^k$ is connected iff $H^0(U)=\R$.
\end{theorem}
\begin{proof}
    If $f \in \ker(\grad)$ (so $\operatorname{grad}(f)=0$), then $f$ is locally constant, that is, every $x_0 \in U$ has a neighborhood $V(x_0)$ such that $f(x)=f(x_0)$ for $x \in V(x_0)$. This makes sense because having zero derivative geometrically means ``zero rate of change'', so the function will be constant if we ``zoom in close enough''. To see this, apply the mean value theorem to the closure of a neighborhood around $x_0$, say $[a,b]\subseteq U$. Then $f'(c)=\frac{f(b)-f(a)}{b-a}$, and since $f'(c)=0$, $f(b)-f(a)$. Since the derivative is zero everywhere, this implies the image of the neighborhood (and then $x_0$) is constant. This generalizes to multiple variables by parametrizing by one variable.

Now suppose $U$ is connected. Then locally constant functions are actually constant, since for $x_0 \in U$, the set \[
    \{x\in U \mid f(x)=f(x_0)\} =f^{-1}(f(x_0))
\] is closed since it's the preimage of a closed set by the continuity of $f$, and open since $f$ is locally constant (every neightborhood has apoint). So since this set is nonempty, by connectedness this is all of $U$, and $H^0(U)=\R$.

Conversely, if $U$ is not connected, then we have a smooth, surjective function $f \colon U \to \{0,1\} $ defined by taking all but one of the connected components to $0$, and the other to $1$. Since $f$ is locally constant, $\grad (f)=0$, so $\dim H^0 (U)>1$. We can easily extend this to show $\dim H^0(U)>1$ by replacing $\{0,1\} $ with $\{1,\cdots ,n\} $, where $n$ is the number of connected components of $U$.
\end{proof}

Now let's move on to functions of three variables. Let $U \subseteq \R^3$ be open. Define 
\begin{align*}
    \grad  \colon C ^{\infty}(U,\R) \to C^{\infty}(U, \R^3),\quad &f \mapsto \left( \frac{\partial f}{\partial x_1},\frac{\partial f}{\partial x_2},\frac{\partial f}{\partial x_3} \right), \\
    \curl \colon  C^{\infty}(U,\R^3)\to C^{\infty}(U,\R^3),\quad &(f_1,f_2,f_3) \mapsto \left( \frac{\partial f_3}{\partial x_2}-\frac{\partial f_2}{\partial f_3},\frac{\partial f_1}{\partial x_3}-\frac{\partial f_3}{\partial x_1},\frac{\partial f_2}{\partial x_1}-\frac{\partial f_1}{\partial x_2}\right), \\
\operatorname{div} \colon C^{\infty}(U,\R^3) \to C^{\infty}(U,\R),\quad &(f_1,f_2,f_3) \mapsto \frac{\partial f_1}{\partial x_1}+\frac{\partial f_2}{\partial x_2}+\frac{\partial f_3}{\partial x_3}.
\end{align*}
Note that $\curl \circ \grad=0$, and $\operatorname{div}\circ \curl=0$. Most textbooks leave this as an exercise but let's work this out in detail. 
\begin{gather*}
    \curl\left( \frac{\partial f}{\partial x_1},\frac{\partial f}{\partial x_2},\frac{\partial f}{\partial x_3} \right) =\left( \frac{\partial ^2f}{\partial x_2\partial x_3}-\frac{\partial ^2f}{\partial x_3\partial x_2},\frac{\partial ^2f}{\partial x_3\partial x_1}-\frac{\partial ^2f}{\partial x_1\partial x_3},\frac{\partial ^2f}{\partial x_1\partial x_2} -\frac{\partial ^2f}{\partial x_2\partial x_1}\right) =0,\\
    \operatorname{div}\left( \frac{\partial f_3}{\partial x_2}-\frac{\partial f_2}{\partial x_3},\frac{\partial f_1}{\partial x_3}-\frac{\partial f_3}{\partial x_1},\frac{\partial f_2}{\partial x_1}-\frac{\partial f_1}{\partial x_2}\right)=0.
\end{gather*}The first equality is true because mixed partial derivatives commute, and the second because the first component in the expression for curl has no part containing $x_1$. So $\frac{\partial }{\partial x_1}\left( \frac{\partial f_3}{\partial x_2}-\frac{\partial f_2}{\partial x_3} \right) =\frac{\partial ^2 f_3}{\partial x_1\partial x_2}-\frac{\partial ^2f_2}{\partial x_1\partial x_3}=0,$ and so on.

Define $H^0(U),\ H^1(U)$ as earlier and set $H^2(U)=\ker (\operatorname{div}) / \im (\curl)$. 
\begin{theorem}
    For an open star-shaped set in $\R^3$ we have $H^0(U)=\R$, $H^1(U)=0$, and $H^2(U)=0$.
\end{theorem}
\begin{figure}[H]
\centering
\begin{tikzcd}
\overset{\ker(\grad)}{H^0(U)} \arrow[r]                       & \overset{\ker(\operatorname{curl})/\operatorname{im}(\operatorname{grad})}{H^1(U)=0} \arrow[r] & \overset{\ker(\div)/\im(\curl)}{H^2(U)=0} \arrow[r]                       & H^3(U)                       \\
{C^{\infty}(U,\R)} \arrow[r, "\operatorname{grad}"] \arrow[u] & {\underset{\text{exact}}{C^{\infty}(U,\R^3)}} \arrow[r, "\operatorname{curl}"] \arrow[u]       & {\underset{\text{exact}}{C^{\infty}(U,\R^3)}} \arrow[u] \arrow[r, "\div"] & {C^{\infty}(U,\R)} \arrow[u]
\end{tikzcd}
\caption{The updated commutative diagram for $U$ star-shaped, now with divergence.} 
\end{figure}
\begin{proof}
    Since $U$ is star-shaped by assumption (and therefore connected), we immediately have $H^0(U)=\R$ and $H^1(U)=0$ by our previous theorems. We want to show that $H^2(U)=0$, or $\ker(\div)=\im(\curl)$. Since $\div(\im(\curl))=0$, we have $\im (\curl) \subseteq \ker(\div)$. So the goal has been reduce to showing that $\ker(\div) \subseteq \im(\curl)$. To accomplish this, it suffices to exhibit a smooth function $U\subseteq \R^3\to \R^3$ such that the curl of this function is equal to some chosen element of $\ker(\div) .$

    Assume $U$ is star-shaped with respect to 0, and let $F \colon U \to\R^3$ such that $\div F=0$. Consider $G \colon U \to \R^3$ defined by \[
        G(\mathbf x)= \int_{0}^{1} (F(t\mathbf x)\times t\mathbf x) \, dt.
    \] Then if $\mathbf x=(x_1,x_2,x_3)$, $F=(f_1,f_2,f_3)$, we have
    \begin{align*}
        \curl(F(t\mathbf x)\times t\mathbf x)&=\curl((f_1(tx_1),f_2(tx_2),f_3(tx_3))\times (tx_1,tx_2,tx_3))\\
                                             &=\curl (f_2(tx_2)tx_3-f_3(tx_3)tx_2,f_3(tx_3)tx_1-f_1(tx_1)tx_3,f_1(tx_1)tx_2-f_2(tx_2)tx_1)\\
                                             &=\left( \left( tf_1(tx_1) -tx_1\frac{\partial f_2}{\partial x_2}(tx_2)\right) -\left( tx_1 \frac{\partial f_3}{\partial x_3}(tx_3)-tf_1(tx_1) \right) ,\cdots \right) \\
                                             &=\left( 2tf_1(tx_1)-tx_1\left( \frac{\partial f_2}{\partial x_2}(tx_1)+\frac{\partial f_3}{\partial x_3}(tx_3) \right)  ,\cdots \right) \\
                                             &=2t F(t\mathbf x)+{\color{red}??} \\
                                             &=\frac{d}{dt}(t^2 F(t\mathbf x)).
    \end{align*}
Therefore \[
    \curl G(\mathbf x)=\int_{0}^{1} \frac{d}{dt}(t^2F(t\mathbf x)) \, dt=F(\mathbf x).\qedhere
\] 
\end{proof}
\begin{example}
    If $U$ is not star-shaped then we can have nontrivial first and second cohomology groups. Consider $f \colon (\R^3\setminus S^1 ) \to \R^3$ by \[
        f(x_1,x_2,x_3) =\left( \frac{-2x_1x_3}{x_3^2+(x_1^2+x_2^2-1)^2},\frac{-2x_2x_3}{x_3^2+(x_1^2+x_2^2-1)^2},\frac{x_1^2+x_2^2-1}{x_3^2+(x_1^2+x_2^2-1)^2} \right) .
    \] By some calculation we have $\curl(f)=0$. So $f$ defines some element $[f] \in H^1(U)$. To show $[f]$ is nontrivial, we integrate along a curve $\gamma \subseteq U$ ``linked'' to the missing $S^1 $. Define $\gamma (t)=\left( \sqrt{1+\cos t} ,0,\sin t \right) $ for $t \in [-\pi,\pi]$. Assume $\grad(F)=f$ for some function $F$. One one hand, we have \[
    \int_{-\pi +\varepsilon }^{\pi-\varepsilon } \frac{d}{dt}F(\gamma (t)) \, dt=F(\gamma (\pi-\varepsilon ))-F(\gamma (-\pi+\varepsilon ))\to 0 \quad \text{for} \ \varepsilon \to 0,
    \] and on the other hand we have by the chain rule \[
    \frac{d}{dt}F(\gamma (t))=f_1(\gamma (t))\cdot \gamma_1 '(t)+\cdots =\sin ^2 t+0+ \cos ^2t=1.
    \] So the integral also converges to $2\pi$, a contradiction.
\end{example}
\begin{example}
    Let $U \subseteq \R^k$ be open and $X \colon U \to \R^k$ be smooth ($X$ is a smooth vector field). The \textbf{energy} $A_{\gamma }(X)$ of $X$ along a smooth curve $\gamma  \colon [a,b] \to U$ is defined by \[
        A_{\gamma} (X)=\int_{a}^{b} \langle X \circ \gamma (t),\gamma '(t) \rangle  \, dt.
    \] If $X=\grad (\Phi)$ and $\Phi_{\gamma (a)}=\Phi_{\gamma (b)}$, then the energy of $X$ is zero, since \[
    \langle X \circ \gamma (t),\gamma '(t) \rangle =\frac{d}{dt}\Phi(\gamma (t)).
    \] 
\end{example}

