\section{Some algebra} 
\subsection{The alternating algebra}
Let $V$ be a real vector space. A map $
f \colon \overset{k \, \text{times} }{\overbrace{V\times V\times \cdots \times V} }  \to \R$ is $\mathbf k$\textbf{-linear} (or multilinear) if $f$ is linear in each factor.
\begin{definition}[]
    A $k$-linear map $\omega \colon V^k \to \R$ is \textbf{alternating} if $\omega(\xi_1,\cdots ,\xi_k)=0$ whenever $\xi_i =\xi_j $ for some pair $i\neq j$. Denote the vector space of alternating $k$-linear maps as $A_k(V)$.
\end{definition}
Note that $A_k(V)=0$ if $k>\dim V$, since two vectors in the domain have to be linearly dependent. Recall that $\operatorname{sgn} \colon S_k \to \{\pm 1\} $ is a homomorphism, since $\sgn(\sigma \circ \tau)=\sgn(\sigma) \circ \sgn(\tau)$.
\begin{lemma}\label{alt} 
    If $\omega \in A_k(V)$ and $\sigma \in S_k$, then \[
        \omega(\xi _{\sigma(1)},\cdots ,\xi_{\sigma(k)})=\sgn(\sigma)\omega(\xi_1,\cdots ,\xi_k).
    \] 
\end{lemma}
\begin{proof}
    It is sufficient to show this is true for $\sigma=(i,j)$. Let $\omega_{i,j}(\xi,\xi')=\omega(\xi_1,\cdots ,\xi,\cdots ,\xi',\cdots ,\xi_k)$, where $\xi$ and $\xi'$ occur at positions $i,j$ respectively. The remaining $\xi_v \in V$ are arbitrary fixed vectors. Now $\omega _{i,j}\in A_2(V)$ since $\omega \in A_k(V)$, so $\omega _{i,j}(\xi_i +\xi_j ,\xi_i +\xi_j )=0$. By bilinearity, we have $\omega_{i,j}(\xi_i ,\xi_j )+\omega_{i,j}(\xi_j ,\xi_i )=0,$ and so $\omega _{i,j}(\xi_i ,\xi_j )=-\omega _{i,j}(\xi_j ,\xi_i )=\sgn(\sigma)\omega _{i,j}(\xi_j ,\xi_i )$.
\end{proof}
\begin{example}
    If $V=\R^k$ and $\xi_i =(\xi_{i1},\cdots ,\xi_{ik})$, the determinant function $(\xi_1,\cdots ,\xi_k) \mapsto \det(\xi_{ij})$ is alternating.
\end{example}
\begin{definition}[]
    A $\mathbf { (p,q)}$\textbf{-shuffle} $\sigma$ is a permutation in $S_{p+q}$ such that $\sigma(1)< \cdots < \sigma(p)$ and $\sigma(p+1)< \cdots  < \sigma(p+q)$. Denote the set of all $(p,q)$-shuffles by $S_{(p,q)}$. Since a $(p,q)$-shuffle is uniquely determined by the set $\{\sigma(1),\cdots \sigma(p)\} $, to form $S_{(p,q)}$ we choose subsets of order $p$ from  $S_{p+q}$. So $|S_{(p,q)}|={p+q\choose p} $.
\end{definition}
\begin{definition}[]
    For $\omega_1\in A_p(V)$ and $\omega_2 \in A_q(V)$, define \[
        (\omega_1\wedge\omega_2)(\xi_1,\cdots ,\xi_{p+q})=\sum_{\sigma \in S_{(p,q)}}^{} \sgn(\sigma)\omega_1 \left( \xi_{\sigma(1)}, \cdots  , \xi_{\sigma(p)} \right) \omega_2\left( \xi_{\sigma(p+1)}, \cdots ,\xi_{\sigma(p+q)} \right) .
    \] 
    Note that $\omega_1\wedge \omega_2$ is $(p+q)$-linear. This product is called the \textbf{exterior product} or \textbf{wedge product}.
\end{definition}
\begin{remark}
    Often you also see the exterior product defined as \[
        \omega_1\wedge \omega_2(\xi_1,\cdots ,\xi_{p+q}):=\frac{1}{p!q!}\sum _{\sigma \in S_{p+q}}\operatorname{sgn}(\sigma)\omega_1(\xi_{\sigma(1)},\cdots ,\xi_{\sigma(p)})\omega_2(\xi_{\sigma(p+1)},\cdots ,\xi_{\sigma(p+q)}).
    \] This definition compensates for the $|S_p|=p!$ and $|S_q|=q!$ repetitions by dividing by them in the coefficient.
\end{remark}
\begin{lemma}
    If $\omega_1 \in A_p(V)$ and $\omega_2 \in A_q(V)$, then $\omega_1\wedge \omega_2 \in A_{p+q}(V)$.
\end{lemma}
\begin{proof}
    We show that $(\omega_1\wedge\omega_2)(\xi_1,\xi_2, \cdots ,\xi_{p+q})=0$ when $\xi_1=\xi_2$. Let $S_{12}=\{\sigma \in S_{(p,q)}\mid \sigma(1)=1,\sigma(p+1)=2\} $, $S_{21}=\{\sigma\in S_{(p,q)}\mid \sigma(1)=2,\sigma(p+1)=1\} $, and $S_0=S_{(p,q)}$ {\color{red}todo:algebra proof} 
\end{proof}
\begin{lemma}\label{transp} 
    A $k$-linear map $\omega$ is alternating if $\omega(\xi_1,\cdots ,\xi_k)=0$ for all $k$-tuples with $\xi_i =\xi_{i+1}$ for some $1\leq i \leq k-1$.
\end{lemma}
\begin{proof}
    Recall that $S_k$ is generated by the transpositions $(i,i+1)$, and so by \cref{alt}, we have e \[
        \omega(\xi_1,\cdots ,\xi_i ,\xi_{i+1},\cdots ,\xi_k)=-\omega(\xi_1,\cdots ,\xi_{i+1},\xi_i ,\cdots ,\xi_k).
    \] Then \cref{alt} holds for all $\sigma \in S_k$, so $\omega$ is alternating.\footnote{Isn't this lemma true by definition?}
\end{proof}
\begin{lemma}
    The exterior product is anticommutative. That is, for $\omega_1 \in A_P(V)$ and $\omega_2 \in A_q(V)$, we have  $\omega_1\wedge \omega_2=(-1)^{pq}\omega_2\wedge \omega_1$.
\end{lemma}
\begin{proof}
        Define $\tau \in S_{p+q}$ to be the permutation \[
    \tau=
    \begin{pmatrix}
        1 & \cdots  & q & q+1 & \cdots  & q +p\\
        p+1 & \cdots  & p+q & 1 & \cdots  & p
    \end{pmatrix}.
\] For any $\xi_1,\cdots ,\xi_{p+q}\in V$,
\begin{align*}
    \omega_1\wedge \omega_2(\xi_1,\cdots ,\xi_{p+q})&=\sum _{\sigma \in S_{p+q}}(\operatorname{sgn}\sigma)f(\xi_{\sigma(1)},\cdots ,\xi_{\sigma(p)})g(\xi_{\sigma(p+1)},\cdots ,\xi_{p\sigma(p+q)})\\
                                         &= \sum _{\sigma \in S_{p+q}}(\operatorname{sgn}\sigma)f (\xi_{\sigma\tau(q+1)},\cdots , \xi _{\sigma\tau(q+p)})g(\xi_{\sigma\tau(1)},\cdots ,\xi_{\sigma\tau(q)})\\
                                         &=(\operatorname{sgn}\tau) \sum _{\sigma\in S_{p+q}}(\operatorname{sgn}\sigma\tau) g (\xi_{\sigma\tau(1)},\cdots , \xi_{\sigma\tau(q)})f(\xi_{\sigma\tau(q+1)},\cdots ,\xi_{\sigma\tau(q+p)})\\
                                         &=(\operatorname{sgn}\tau)A (g\otimes f)(\xi_1,\cdots ,\xi_{p+q}).
\end{align*}{\color{red}todo:adapt this proof to the $(p,q)$-shuffle definition} 
\end{proof}
\begin{lemma}
    The exterior product is associative. That is, for $\omega_1\in A_p(V),\,\omega_2\in A_q(V),$ $\omega_3 \in A_r(V)$, we have \[
        \omega_1\wedge (\omega_2\wedge \omega_3)=(\omega_1\wedge\omega_2)\wedge\omega_3.
    \] 
\end{lemma}
\begin{proof}
    {\color{red}todo:spending less time on the algebra to get to the good stuff} 
\end{proof}
On top of making sure $A_k(V)$ is closed under multiplication and being associative, the exterior product is also associative and satisfies homogeneity, making it $A_k(V)$ into an algebra. What's an algebra? An $\R$\textbf{-algebra}  $A$ is a real vector space with an associative bilinear map $\mu \colon A\times A \to A$. The algebra is \textbf{unitary} if there exists a unit element (say 1) such that $\mu(1,a)=\mu(a,1)=a$ for all $a \in A$.
\begin{definition}[]
    \,
    \begin{enumerate}[label=(\roman*)]
        \item A \textbf{graded} $\R$\textbf{-algebra} $A_*$ is a sequence of vector spaces $A_k,\, k=0,1,\cdots $ and bilinear maps $\mu \colon A_k \times A_{\ell} \to A_{k+\ell}$ which are associative.
        \item The graded algebra $A_*$ is \textbf{connected} if there exists a unit element $1 \in A_0$, and the map $\varepsilon \colon \R \to A_0,\ r\mapsto r\cdot 1$ is an isomorphism.
        \item The graded algebra $A_*$ is \textbf{commutative} (resp \textbf{anti-commutative}) if $\mu(a,b)=(-1)^{k\ell}\mu(b,a)$ for $a \in A_k$ and $b \in A_{\ell}$.
    \end{enumerate}
    Elements in $A_k$ are said to have degree $k$.
\end{definition}
Note that $A_k(V)$ is a real vector space since 
\begin{gather*}
    (\omega_1+\omega_2)(\xi_1,\cdots ,\xi_k)=\omega_1(\xi_1,\cdots ,\xi_k)+\omega_2(\xi_1,\cdots ,\xi_k),\\
    (\lambda\omega)=(\xi_1,\cdots ,\xi_k)=\lambda\omega(\xi_1,\cdots \xi_k),\quad \lambda \in \R.
\end{gather*}
\begin{theorem}
    $A_*(V)$ with the exterior product is an anti-commutative and connected graded algebra.
\end{theorem}
\begin{proof}
    Set $A_0(V)=\R$, since maps that take no vectors and output a scalar are just scalars themselves. Expand the product to $A_0(V)\times A_p(V)$ using the vector space structure. We have seen above that the exterior product is closed, associative, distributive, and anticommutative.
\end{proof}
$A_*(V)$ is the \textbf{exterior algebra} or \textbf{alternating algebra} associated with $V$. Elements of $A_1(V)$ are called $\mathbf 1$\textbf{-forms}.
\begin{lemma}
    For $1-$ forms $\omega_1,\cdots ,\omega_p \in A_1(V)$, we have \[
        (\omega_1\wedge\cdots \wedge\omega_p)(\xi_1,\cdots ,\xi_p)=\det 
        \begin{pmatrix}
            \omega_1(\xi_1) & \omega_1(\xi_2) & \cdots  & \omega_1(\xi_p)\\
            \omega_2(\xi_1) & \omega_2(\xi_2) & \cdots  & \omega_2(\xi_p)\\
            \vdots & \vdots & & \vdots \\
            \omega_p(\xi_1) & \omega_p(\xi_2) & \cdots  & \omega_p(\xi_p)\\
        \end{pmatrix}.
    \] 
\end{lemma}
\begin{proof}
    We use induction on $p$. If $p=2$, then the two elements $(12),(21)$ of $S_2$ are $(1,1)$-shuffles. So $(\omega_1\wedge\omega_2)(\xi_1,\xi_2)=\omega_1(\xi_1)\omega_2(\xi_2)-\omega_1(\xi_2)\omega_2(\xi_1)=\det \left( 
    \begin{smallmatrix}
        \omega_1(\xi_1) & \omega_1(\xi_2)\\
        \omega_2(\xi_1) & \omega_2(\xi_2)
    \end{smallmatrix}\right) $. Now \[
    \omega_1\wedge (\omega_2\wedge \cdots \wedge \omega_p)(\xi_1,\cdots ,\xi_p)=\sum_{j=1}^{p} (-1)\omega_1(\xi_j )(\omega_2\wedge \cdots \wedge \omega_p)\big(\xi_1,\cdots ,\hat{\xi_j },\cdots ,\xi_p\big).
    \] Expanding the determinant along the first row gives our result.
\end{proof}
This lemma shows that if the $1$-forms $\omega_1,\cdots ,\omega_p $ are linearly independent, then $\omega_1\wedge\cdots \wedge\omega_p\neq 0$. This is an equivalence: we can choose elements $\xi_i \in V$ with $\omega_i (\xi_j )=0$ for $i\neq j$ and $\omega_j (\xi_j )=1$, which implies that $\det (\omega_i (\xi_j ))=1$. Conversely, if the $\omega_i $ were linearly dependent, we could write $\omega_p=\sum _{i=1}^{p-1}r_i \omega_i $. So the determinant in the previous lemma would have two equal rows and be zero. To summarize:
\begin{lemma}
    For $1$-forms $\omega_1,\cdots ,\omega_p$ on $V$, we have $\omega_1\wedge \cdots \wedge \omega_p \neq 0$ iff they are linearly independent.
\end{lemma}
\begin{theorem}\label{basis} 
    For $\{e_i \} $ a basis of $V$ and $\{\phi_i \} $ the dual basis of $A_1(V)$ (as $i$ varies over $n$), we have \[
        \{\phi_{\sigma_1 }\wedge \phi_{\sigma(2)}\wedge \cdots \wedge \phi_{\sigma(p)}\} _{\sigma \in S_{(p,n-p)}}
    \] a basis of $A_p(V)$. In particular, $\dim A_p(V)={\dim V\choose p}  $. 
\end{theorem}
\begin{proof}
    {\color{red}todo:less time on algebra} 
\end{proof}
This tells us that $A_n (V)\cong \R$ if $n=\dim V$ (since they're both one dimensional real vector spaces, ${n\choose n} =1$) and $A_p(V)=0$ for $p>n$ (since two factors will be the same). A linear map $f \colon V \to W$ induces the linear map  \[
    A_p(f) \colon A_p(W) \to A_p(V)
\] by setting $A_p(f)(\omega(\xi_1,\cdots ,\xi_p))=\omega(f(\xi_1),\cdots ,f(\xi_p)).$ We have $A_p(g \circ f)=A_p(f) \circ A_p(g),$ and $A_p(\id)=\id$. This is equivalent to saying that $A_p(-)$ is a \textbf{contravariant functor}. For $\dim V=n$, $f \colon V \to V$ linear, the induced map $A_n (f) \colon A_n (V) \to A_n (V)$ is a linear endormorphism of a $1$-dimensional vector space, and is therefore just scalar multiplication. It follows from the theorem below that this scalar is $\det f$.

\begin{theorem}
    The characteristic polynomial of a linear endomorphism $f \colon V \to V$ is given by \[
        \det (f-t)=\sum_{i=0}^{n} (-1)^i \operatorname{tr}\left( A_{n-i}(f) \right) t ^i .
    \] 
\end{theorem}
\begin{proof}
    {\color{red}todo:algebra} 
\end{proof}


\subsection{The exterior derivative}

Let $U$ denote an open set in $\R^n $, $\{e_1,\cdots ,e_n \} $ the standard basis and $\{\phi_1,\cdots ,\phi_n \} $ the dual basis of $A_1(\R^n )$ (or the basis for the dual space to $\R^n $).

\begin{definition}[]
    A \textbf{differential} $\mathbf p$\textbf{-form} on $U$ is a smooth map $\omega \colon U \to A_p(\R^n )$. The vector space of all such maps is denoted by $\Omega^p (U)$.
\end{definition}
If $p=0$, then $A_0(\R^n )=\R$, and $\Omega^0(U)$ is just the set of smooth functions on $U, C^{\infty}(U,\R)$. The derivative of a smooth map $\omega \colon U \to A_p(\R^n )$ is denoted $D\omega$, and is the linear map \[
    D_x\omega \colon \R^n  \to A_p(\R^n ), \quad e_i  \mapsto \frac{d}{dt}\omega(x+te_i )_{t=0}=\frac{\partial \omega}{\partial x_i }(x).
\] Let $I=(i_1,\cdots ,i_p)$, and write $\phi_I$\footnote{It slightly annoys me that indices aren't in the right place, but I don't want to make any mistakes deviating too far from the book, so they stay at the bottom for covectors.} for $\phi_{i_1}\wedge \cdots \wedge \phi_{i_p}$. Then we have the basis $\phi_I$ for $A_p(\R^n )$ as $I$ runs over all sequences of length $p\leq n$. So every $\omega \in \Omega^p(U)$ can be written in the form $\omega (w)=\sum \omega_I(x)\phi_I$, where the $\omega_I$ are smooth real-valued functions of  $x \in U$. The differential $D_x\omega$ is the linear map \[
D_x\omega(e_j )=\sum_I \frac{\partial \omega_I}{\partial x_j }(x)\phi_I, \quad j=1,\cdots ,n.
\] The function $x \mapsto D_x\omega$ is a smooth map from $U$ to $\operatorname{Hom}(\R^n ,A_p(\R^n ))$. {\color{red}todo:why? how exactly? difference between thsi and derivative?} 
\begin{definition}[]
    The \textbf{exterior differential} $d \colon \Omega^p(U) \to \Omega^{p+1}(U)$ is the linear operator \[
        d_x\omega(\xi_1,\cdots ,\xi_{p+1})=\sum_{\ell=1}^{p+1} (-1)^{\ell-1}D_x\omega(\xi_{\ell})(\xi_1,\cdots ,\hat{\xi}_{\ell},\cdots ,\xi_{p+1})
    \] where $(\xi_1,\cdots ,\hat{\xi}_{\ell},\cdots ,\xi_{p+1})=(\xi_1,\cdots ,\xi_{\ell-1},\xi_{\ell +1},\cdots ,\xi_{p+1})$. {\color{red}todo:what?} 
\end{definition}
The result lies in $\Omega^{p+1}(U)$ by \cref{transp}. If $\xi_i =\xi_{i+1}$, then 
\begin{align*}
    \sum_{\ell=1}^{p+1} &(-1)^{\ell -1}D_x\omega(\xi_{\ell})(\xi_1,\cdots ,\hat{\xi}_{\ell},\cdots ,\xi_{p+1})\\
    =&(-1)^{i-1}D_x\omega(\xi_i )(\xi_1,\cdots ,\hat{\xi}_i ,\cdots ,\xi_{p+1})\\
     &+(-1)^i D_x\omega(\xi_{i+1})(\xi_1,\cdots ,\hat{\xi}_{i+1},\cdots ,\xi_{p+1})\\
    =&0.
\end{align*}
In the second step, the rest of the terms cancel out by properties of the exterior product, since they all contain both $\xi_i $ and $\xi_{i+1}$. The final term also cancels out since $(\xi_1,\cdots ,\hat{\xi}_i ,\cdots ,\xi_{p+1})=(\xi_1,\cdots ,\hat{\xi}_{i+1},\cdots ,\xi_{p+1})$.

\begin{example}
    Let $x_i  \colon U \to \R$ be $i$th projection. Then $dx_1 \in \Omega^1(U)$ is the constant map $dx_i \colon x \to \phi_i $, which follows from the definition of the differential. In general, for $f \in \Omega^0(U)$, we have \[
        d_x f(\zeta)= \frac{\partial f}{\partial x_1}(x)\zeta^1+\cdots + \frac{\partial f}{\partial x_n }(x)\zeta ^n .
    \] 
\end{example}

\begin{lemma}
    If $\omega(x)=f(x)\phi_I$, then $d_x\omega=d_x f \wedge \phi_I$.
\end{lemma}
\begin{proof}
    Note that \[
        D_x\omega(\zeta)=(D_x f)(\zeta)\phi_I =\left( \frac{\partial f}{\partial x_1}\zeta^1+\cdots + \frac{\partial f}{\partial x_n }\zeta^n  \right) \phi_I=d_xf (\zeta) \phi_I.
    \] So by the definition of the exterior derivative, we have 
    \begin{align*}
        d_x\omega(\xi_1,\cdots ,\xi_{p+1})&=\sum_{k=1}^{p+1} (-1)^{k-1}d_x f(\xi_k) \phi_I (\xi_1,\cdots ,\hat{\xi}_k,\cdots ,\xi_{p+1})\\
        &= [d_x f \wedge \phi_I](\xi_1,\cdots ,\xi_{p+1}).\qedhere
    \end{align*}
\end{proof}
{\color{red}todo:this entire proof?} For $\phi_I \in A_p(\R^n )$, we have $\phi_k \wedge \phi_I=0$ if $k \in I$, and $(-1)^r \phi_J$ if $k\notin I$, where $r$ is determined by $i_r < k < i_{r+1}$ and $J=(i_1,\cdots ,i_r,k,\cdots ,i_p)$.

\begin{lemma}
    For $p\geq 0$, the composition $\Omega^p (U) \to \Omega^{p+1}(U)\to \Omega^{p+2}(U)$ is identically zero.
\end{lemma}
\begin{proof}
    Let $\omega=f \phi_I$. Then $d\omega=df\wedge \phi_I=\frac{\partial f}{\partial x_1}\phi_1\wedge \phi_I + \cdots + \frac{\partial f}{\partial x_n }\phi_n \wedge \phi_I.$ {\color{red}todo:alternating  terms?? does $I$ denote one sequence or several?} Since $\phi_i \wedge \phi_i =0 $ and $\phi_i \wedge\phi_j =-\phi_j \wedge \phi_i $, we have 
    \begin{align*}
        d^2\omega&= \sum_{i,j=1}^{n} \frac{\partial ^2 f}{\partial x_i \partial x_j }\phi_i  \wedge (\phi_j \wedge \phi_I)\\
                 &= \sum_{i<j}^{} \left( \frac{\partial ^2 f}{\partial x_i \partial x_j }-\frac{\partial ^2 f}{\partial x_j \partial x_j } \right) \phi_i \wedge \phi_j  \wedge \phi_I=0.\qedhere
    \end{align*}
\end{proof}
The exterior product on $A_*(\R^n )$ induces an exterior product on $\Omega^*(U)$ by defining $(\omega_1\wedge \omega_2)(x)=\omega_1(x)\wedge \omega_2(x)$. The exterior product of a $p$-form and $q$-form is a $(p+q)$-form, so it induces a bilinear map $\wedge \colon \Omega^p (U) \times \Omega^q(U) \to \Omega^{p+q}(U).$ Then for $f \in C^{\infty}(U,\R)$, we have $(f\omega_1)\wedge \omega_2=f(\omega_1\wedge \omega_2)=\omega_1\wedge f\omega_2$. Note that $f\wedge \omega=f\omega$ when $f \in \Omega^0(U)$ and $\omega \in \Omega^p(U)$.

\begin{lemma}\label{product} 
    For $\omega_1 \in \Omega^p(U)$ and $\omega_2 \in \Omega^q(U)$, \[
        d(\omega_1\wedge \omega_2)=d\omega_1 \wedge \omega_2+(-1)^p\omega_1\wedge d\omega_2.
    \] 
\end{lemma}
\begin{proof}
    It suffices to show this holds for $\omega_1=f \phi_I$ and $\omega_2=g \phi_J$. Since $\omega_1\wedge \omega_2= fg \phi_I\wedge \phi_J$, we have 
    \begin{align*}
        d(\omega_1\wedge \omega_2)&=d(fg)\wedge \phi_I\wedge \phi_J=((df)g+fdg)\wedge \phi_I\wedge \phi_J\\
                                  &= df g\wedge \phi_I\wedge \phi_J + fdg \wedge \phi_I \wedge \phi_J \\
                                  &= df \wedge  \phi_I \wedge g\phi_J +(-1)^pf \phi_I \wedge dg\wedge \phi_J\\
                                  &=d\omega_1\wedge\omega_2 +(-1)^p \omega_1 \wedge d\omega_2.\qedhere
    \end{align*}
\end{proof}

\subsection{Finally, de Rham cohomology}
In short, we have a new anti-commutative algebra $\Omega^*(U)$ with a \emph{differential} (or boundary) operator \[
    d \colon \Omega^*(U) \to \Omega ^{*+1}(U),\quad d\circ d=0,
\] and $d$ is a \emph{derivation} (since $d(\omega_1\wedge \omega_2)=d\omega_1\wedge \omega_2+d\omega_2\wedge \omega_1$ by \cref{product} and anticommutativity). Then $(\Omega^*(U),d)$ is an \emph{commutative differential graded algebra}\footnote{Strangely, even though the algebra is anticommutative, we call it commutative graded. This is just convention.}, called the \textbf{de Rham complex} of $U$.

\begin{theorem}
    There is precisely one linear operator $d \colon \Omega^p \to \Omega^{p+1}(U)$, $p=0,1,\cdots ,$ such that 
    \begin{enumerate}[label=(\roman*)]
        \setlength\itemsep{-.2em}
    \item $f \in \Omega^0(U)$, $df=\frac{\partial f}{\partial x_1}\phi_1+ \cdots + \frac{\partial f}{\partial x_n }\phi_n $,
    \item $d \circ d=0$,
    \item $d(\omega_1\wedge \omega_2)=d\omega_1\wedge \omega_2+(-1)^p\omega_1\wedge d\omega_2$ if $\omega_1 \in \Omega^p(U)$.
    \end{enumerate}
\end{theorem}
\begin{proof}
    We know that the exterior differential $d$ satisfies these properties. To show uniqueness, say $d'$ satisfies these properties: we will show it has to be the exterior derivative. (i) tells us that $d=d'$ on $\Omega^0(U)$ (since it characterizes smooth functions on $U$), in particular $d'x_i =dx_i =\phi_i $. Since $d' \circ d'=0$, then $d'\phi_i=d'(d'(x_i ))=0 $, and $d' \phi_I=0$. Let $\omega=f \phi_I=f\wedge \phi_I$ for $f \in C^{\infty}(U,\R)$. Then \[
    d' \omega=d' f\wedge \phi_I+f\wedge d'\phi_I=d'f\wedge \phi_I=df\wedge \phi_I=d\omega.
    \] 
\end{proof}
\begin{example}
    Let us show a concrete example. Let $U \subseteq \R^3$, then $d \colon \Omega^1 (U) \to \Omega^2(U)$ looks like the following:
    \begin{gather*}
        d(f_1\phi_1+f_2\phi_2+f_3\phi_3)=df_1\wedge \phi_1+df_2\wedge \phi_2+df_3\wedge \phi_3=\\
        \left( \frac{\partial f_2}{\partial x_1}-\frac{\partial f_1}{\partial x_2} \right) \phi_1\wedge \phi_2+\left( \frac{\partial f_3}{\partial x_3}-\frac{\partial f_2}{\partial x_3} \right) \phi_2\wedge \phi_3+\left( \frac{\partial f_1}{\partial x_3}-\frac{\partial f_3}{\partial x_1} \right) \phi_3\wedge \phi_1.
    \end{gather*}
\end{example}
{\color{red}todo:at this point, the hard-to-read book was abandoned for bott and tu} 
