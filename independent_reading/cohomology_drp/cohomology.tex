\newcommand\Supp{\ensuremath{\operatorname{Supp}}} 
\section{De Rham Cohomology} 

Switching gears, the reference book has changed to Bott and Tu. 
\subsection{The de Rham Complex on $\R^n $}
   If $x_1,\cdots ,x_n $ are the standard coordinates on $\R^n $, define $\Omega^*$ to be the algebra over $\R$ generated by $dx_1,\cdots ,dx_n $ with the relations
   \[
   \begin{cases}
       (dx_i )^2=0,\\
       dx_i dx_j =-dx_j dx_i ,\ i\neq j.
   \end{cases}
   \] 
   As a real vector space this has basis $1, dx_i ,dx_i dx_j , dx_i dx_j dx_k,\cdots ,dx_1\cdots dx_n $, where $i<j,i<j<k$ (or the $(*,n)$-shuffles). The $C^{\infty}$ \textbf{differential forms} on $\R^n $ are elements of $\Omega^* (\R^n )= \{C^{\infty}\ \text{functions on} \ \R^n \} \bigotimes_{\R}\Omega^*$. Recall that the tensor product of two $R$-algebras $A,B$ has basis $a_i\otimes b_j $, where multiplication is defined by $(a_1\otimes b_1)(a_2\otimes b_2)=a_1b_1\otimes a_2b_2$. So a form $\omega$ can be uniquely written as $\sum f_{i_1\cdots i_q}dx_{i_1}\cdots dx_{i_q}$, where the coefficients $f_{i_1\cdots i_q}$ are smooth functions. The multi-index notation simplifies this to $\omega=\sum f_I dx_I$. The algebra $\Omega^*(\R^n )=\bigoplus _{q=0}^n \Omega^q(\R^n )$ is naturally graded, where $\Omega^q$ is the space of $C^{\infty}$ $q$-forms on $\R^n $. There is a \emph{differential operator} \[
       d \colon \Omega^q(\R^n ) \to \Omega^{q+1}(\R^n )
   \] defined as follows:
   \begin{enumerate}[label=(\roman*)]
   \setlength\itemsep{-.2em}
       \item if $f \in \Omega^0(\R^n )$, then $df=\sum \frac{\partial f}{\partial x^i }dx_i $,
        \item if $\omega=\sum f_I dx_I$, then $d\omega=\sum df_I dx_I$.
   \end{enumerate}We call this differential operator \textbf{exterior differention}.
\begin{example}
    If $\Omega=x\, dy$, then $d\omega=dx\,dy$. On $\R^3$, $\Omega^0(\R^3)$ and $\Omega^3(\R^3)$ are both $1$-dimensional and $\Omega^1(\R^3)$ and $\Omega^2(\R^3)$ are each $3$-dimensional over the $C^{\infty}$ functions, so we identify 
    \begin{gather*}
   \underset{f}{ \{\text{functions} \} } \quad \simeq \quad \underset{f}{\{0\text{-forms} \}}  \quad \simeq \quad \underset{f\,dx\,dy\,dz}{\{3\text{-forms} \ \} 
}  ,   \\
\underset{X=(f_1,f_2,f_3)}{ \{\text{vector fields} \} } \quad \simeq \quad \underset{f_1dx+f_2dy+f_3dz}{\{1\text{-forms} \}}  \quad \simeq \quad \underset{f_1dy\,dz-f_2dx\,dz+f_3dx\,dy}{\{2\text{-forms} \ \} 
}  .
    \end{gather*}So for functions, \[
    df= \frac{\partial f}{\partial x}dx+\frac{\partial f}{\partial y}dy+\frac{\partial f}{\partial z}dz,
    \] for $1$-forms \[
    d(f_1dx+f_2dy+f_3dz)=\left( \frac{\partial f_3}{\partial y}-\frac{\partial f_2}{\partial z} \right) dy\,dz-\left( \frac{\partial f_1}{\partial z}-\frac{\partial f_3}{\partial x} \right) dx\,dz+\left( \frac{\partial f_2}{\partial x}-\frac{\partial f_1}{\partial y} \right) dx\,dy,
    \] and for $2$-forms \[
    d(f_1\,dy\,dz-f_2\,dx\,dz+f_3\,dx\,dy=\left( \frac{\partial f_1}{\partial x}+\frac{\partial f_2}{\partial y}+\frac{\partial f_3}{\partial z} \right) dx\,dy\,dz.
\] So $d(0$-forms$)=$ gradient, $d(1$-forms$)=$ curl, and $d(2$-forms$)=$ divergence.{\color{red}todo:yay worked it out! transcribe} 
\end{example}
\begin{definition}[]
    Define the \textbf{wedge product} of two differential forms, written $\tau \wedge \omega$ for $\tau= \sum f_I dx_I,\ \omega=\sum g_J dx_J$ by \[
    \tau \wedge \omega=\sum f_I g_J \,dx_I \,dx_J.
\] Note that $\tau \wedge \omega=(-1)^{\deg \tau \deg \omega}\omega \wedge \tau$.
\end{definition}
\begin{prop}
    $d$ is an antiderivation, i.e., \[
        d(\tau \wedge \omega)=(d\tau )\wedge \omega +(-1)^{\deg \tau }\tau \wedge d\omega.
    \] 
\end{prop}
\begin{proof}
    By linearity, we check on just the monomials $\tau=f_I\,dx_I,\ \omega=g_J \,dx_J$. Then \[
        d(\tau \wedge \omega)=d(f_1g_J)dx_I\,dx_J=(df_I)g_J\, dx_I\,dx_J+f_I\,dg_J\,dx_I\,dx_J=(d\tau)\wedge \omega+(-1)^{\deg \tau}\tau \wedge d\omega.\qedhere
    \] 
\end{proof}
\begin{prop}
    $d^2=0$.
\end{prop}
\begin{proof}
    On functions, \[
        d^2f=d\left( \sum _i \frac{\partial f}{\partial x_i }dx_i  \right) =\sum _{i,j}\frac{\partial ^2f}{\partial x_j \partial x_i }dx_j dx_i =0,
    \] since the factors $\partial ^2 /\partial x_j  \partial x_i $ are symmetric in $i,j$ (mixed partials commute) while the $dx_j dx_i $ are skew-symmetric in $i,j$, hence $d^2f=0$. On forms $\omega=f_I\,dx_I$, \[
    d^2\omega =d^2(f_I\,dx_I)=d(df_Idx_I)=0.\qedhere
    \] 
\end{proof}
The complex $\Omega^*(\R^n )$ with the differential operator $d$ is the \textbf{de Rham complex} on $\R^n $. The kernel of $d$ are \textbf{closed forms} and the image of $d$ are \textbf{exact forms}. You can view the de Rham complex as a set of differential equations with solutions the closed forms. For example, finding a closed $1$-form $f\, dx+g\,dy$ on $\R^2$ is just like solving the differential equation $\partial g/\partial x-\partial f /\partial y=0$. {\color{red}todo:where did the $dx\, dy$ term go?} 

Exact forms are automatically closed, since composing with $d$ again gives zero. These are the ``trivial'' or ``uninteresting'' solutions: the de Rham cohomology measures the size of the space of ``interesting'' solutions.

\begin{definition}[]
    The $q$-th \textbf{de Rham cohomology} of $\R^n $ is the vector space \[
        H_{DR}^q (\R^n )= \{\text{closed} \ q\text{-forms}\} / \{\text{exact} \ q\text{-forms}  \} .
    \] We often write $H^q(\R^n )$ in place of $H_{DR}^q(\R^n )$. Denote the cohomology class of a form by $[\omega]$.
\end{definition}
All the definitions work just as well with an open subset $U\subseteq \R^n $. For example, define $\Omega^*(U)$ to be the algebra $\{C^{\infty} \ \text{functions on} \ U\} \bigotimes _{\R}\Omega^*$.\footnote{To avoid confusion, when the notation $C^{\infty}$ functions on $\{\text{set} \} $ is used, the codomain of the functions is just the base field, in this case $\R$.}Then we may speak of the de Rham cohomology $H_{DR}^*(U)$ of $U$.
\begin{example}
    If $n=0$, then \[
        H^q \left( \text{point}   \right) =
    \begin{cases}
        \R,\quad &\text{if} \ q=0,\\
        0,& \text{if} \ q>0.
    \end{cases}
\] Since $\ker d \subseteq \Omega^0(\R^1)$ consists of constant functions, $H^0(\R^1)=\R$ as we've seen before, On $\Omega^1(\R^1)$, $\ker d$ consists of all the $1$-forms, since every 1-form $f\,dx$ gets send to $\frac{\partial f}{\partial x}dx\,dx=0$. For $\omega=g(x)\,dx$ a $1$-form, consider $f= \int_{0}^{x} g(u) \, du$, so $df=g(x)\,dx$. Therefore every $1$-form on $\R^1$ is exact and $H^1(\R^1)=0$. If $U$ is a disjoint union of $m$ intervals on $\R$, then $H^0(U)=\R^m$, and $H^1(U)=0$. In general, 
\[
    H^*(\R^n )=
    \begin{cases}
        \R\quad& \text{in dimension } \ 0,\\
        0 & \text{otherwise.} 
    \end{cases}
\] This result is called the \emph{Poincar\'e lemma}.
\end{example}

\subsection{Recap of (co)homological algebra}
A direct sum of vector spaces $C=\bigoplus_{q\in \Z}C^q$ is called a \textbf{differential complex} if there are homomorphisms \[
    \cdots \to C^{q-1}\overset{d}{\to } C^q \overset{d}{\to } C^{q+1}\to \cdots 
\] such that $d^2=0$. We say that $d$ is the \textbf{differential operator} of the complex $C$. Note that the de Rham complex is a differential complex, and the exterior derivative is a differential operator on  $\Omega^q(U)$. The \textbf{cohomology} of $C$ is the direct sum of vector spaces $H(C)=\bigoplus_{q\in \Z}H^q(C)$, where $H^q(C)=(\ker d \cap C^q) / (\im d \cap C^q)$. A map $f \colon A \to B$ between two differential complexes is a \textbf{chain map} if it commutes with the differential operators of $A$ and $B$, that is, $f d_A=d_Bf$. 

A sequence of vector spaces \[
\cdots \longrightarrow V_{i-1}\overset{f_{i-1}}{\longrightarrow} V_i \overset{f_i }{\longrightarrow} V_{i+1}\longrightarrow\cdots 
\] is \textbf{exact} if for all $i$, $\ker f_i =\im f_{i-1}.$ An exact sequence of the form $0 \to A\to B\to C\to 0$ is a \textbf{short exact sequence}. Given a short exact sequence of differential complexes \[
0 \longrightarrow A \overset{f}{\longrightarrow } B \overset{g}{\longrightarrow } C\longrightarrow 0
\] where $f$ and $g$ are chain maps, there is a long exact sequence of cohomology groups \[
\cdots \to H^q(A) \overset{f^*}{\longrightarrow} H^q(B) \overset{g^*}{\longrightarrow} H^q(C) \overset{d^*}{\longrightarrow} H^{q+1}(A)\to \cdots 
\] where $f^*,g^*$ are the induced homomomorphisms on cohomology, and $d^*[c]$ for $c \in C^q$ is obtained as follows:
\begin{figure}[H]
\centering
\begin{tikzcd}
            & {}                                & {}                                & {}                           &   \\
0 \arrow[r] & A^{q+1} \arrow[u] \arrow[r, "f"]  & B^{q+1} \arrow[u] \arrow[r, "g"]  & C^{q+1} \arrow[u] \arrow[r]  & 0 \\
0 \arrow[r] & A^q \arrow[u, "d"] \arrow[r, "f"] & B^q \arrow[u, "d"] \arrow[r, "g"] & C^q \arrow[u, "d"] \arrow[r] & 0 \\
            & {} \arrow[u]                      & {} \arrow[u]                      & {} \arrow[u]                 &  
\end{tikzcd}
\end{figure}
By the fact that $g$ is onto, there exists a $b\in B^q$ such that $g(b)=c$. Since $g(db)=d(gb)=dc=0$ (implying that $db \in \ker g$), we have $db=f(a)$ for some $a \in A^{q+1}$ by exactness. It is easily checked that $a$ is closed. $d^*[c]$ is defined to be the cohomology class $[a]$ in $H^{q+1}(A)$. 

\subsection{Compact supports}
Recall the \textbf{support} of a continuous function $f$ on a topological space $X$ is the closure of the set on which $f$ is nonzero, that is, $\Supp f= \overline{\{p \in X\mid f(p)\neq 0\} }$. If when defining the de Rham complex we only consider the $C^{\infty}$ functions with compact support (equivalent to having bounded support in this case), the resulting complex is called the \textbf{de Rham complex} $\Omega_c^*(\R^n )$ \textbf{with compact supports}: 
\[
    \Omega_c^*(\R^n ):= \{C^{\infty} \ \text{functions on} \ \R^n \ \text{with compact support}  \} \bigotimes _{\R}\Omega^*.
\] Denote the cohomology of this complex by $H_c^*(\R^n )$.

\begin{example}
First note that 
\[
    H_c^*(\text{point} )=
\begin{cases}
    \R\quad & \text{in dimension}\ 0,\\
    0 & \text{elsewhere.} 
\end{cases}
\]The closed 0-forms on $\R^1$ are once again the constant functions, and since no constant function has compact support, $H^0_c(\R^1)=0$ {\color{red}todo:more confusion about domains: doesn't $\R\to 0$ work? is $\O$ compact? also, is compact support like the analogue of reduced homology?}  To compute $H^1_c(\R^1)$ consider the integration map \[
\int _{\R^1} \colon \Omega^1_c(\R^1) \longrightarrow\R^1.
\] {\color{red}todo:were we ever taught how to integrate on forms?} This map is onto, and vanishes on the exact 1-forms $df$ where $f$ has compact support, for if the support of $f$ lies in the interior of $[a,b]$, then \[
\int_{\R^1} \frac{df}{dx}dx=\int_{a}^{b} \frac{df}{dx} \, dx=f(b)-f(a)=0.
\] If $g(x)\,dx \in \Omega_c^1(\R^1) $ is also in $\ker \int_{\R^1}$, then $f(x)=\int_{-\infty}^{x} g(u) \, du$ will have compact support and $df=g(x)\,dx$. Hence $\ker \int _{\R^1}$ consists of precisely the exact forms and \[
H_c^1(\R^1)= \frac{\Omega_c^1(\R^1)}{\ker \int _{\R^1}}=\R^1.
\] To see this, {\color{red}todo:elaborate} 
\begin{remark}
    If $g(x)\, dx \in \Omega^1_c(\R^1)$ does not have total integral 0, then $f(x)= \int_{-\infty}^{x} g(u) \, du$ will not have compact support and $g(x)\,dx$ will not be exact.
\end{remark}In general, \[
H_c^*(\R^n )=
\begin{cases}
    \R\quad & \text{in dimension} \ n,\\
    0& \text{otherwise.} 
\end{cases}
\] This is called the \emph{Poincar\'e lemma for cohomology with compact support}.
\end{example}

\begin{example}
    {\color{red}todo:comptue this with both regular (find the closed forms)+mayer vietoris, ask about summer drp} 
    To compute $H_{DR}^*(\R^2 \setminus \{p,q\} )$ for $p,q \in \R^2$,
\end{example}

If $x_1,\cdots ,x_m$ and $y_1,\cdots ,y_n $ are the standard coordinates on $\R^m$ and $\R^n $, then a smooth map $f \colon \R^m \to \R^n $ incudes a pullback map on $C^{\infty}$ functions $f^* \colon \Omega^0(\R^n ) \to \Omega^0(\R^m)$ defined by $f^*(g)=g \circ f$. To uniquely extend this pullback to all forms $f^* \colon \Omega^* (\R^n ) \to \Omega^*(\R^m),$ consider \[
    f^*\left(\sum g_I dy_{i_1}\cdots dy_{i_q}\right)=\sum (g_I \circ f)df_{i_1}\cdots df_{i_q}, 
\] where $f_i =y_i  \circ f$ is the $i$th component of $f$.
\begin{prop}
   $f^*$ commutes with $d$. 
\end{prop}
\begin{proof}
    \begin{align*}
        df^*(g_I dy_{i_1}\cdots dy_{i_q})&=d((g_I \circ f) df_{i_1}\cdots df_{i_q})=d(g_I \circ f)df_{i_1}\cdots df_{i_q},\\
        f^*d(g_I dy _{i_1}\cdots dy_{i_q})&=f^*\left( \sum_{i=1}^{n} \frac{\partial g_I}{\partial y_i }dy_i dy_{i_1}\cdots dy_{i_q} \right) \\
                                          &=\sum_{i=1}^{n} \left( \left( \frac{\partial g_I}{\partial y_i }\circ f \right) df_i  \right) df_{i_1}\cdots df_{i_q}\\
                                          &=d(g_I \circ f)df_{i_1}\cdots df_{i_q}.\qedhere
    \end{align*}
\end{proof}

