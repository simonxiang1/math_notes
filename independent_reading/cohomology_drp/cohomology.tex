\section{De Rham Cohomology} 
We finally get to the goods. 

\subsection{The exterior derivative}

Let $U$ denote an open set in $\R^n $, $\{e_1,\cdots ,e_n \} $ the standard basis and $\{\phi_1,\cdots ,\phi_n \} $ the dual basis of $A_1(\R^n )$ (or the basis for the dual space to $\R^n $).

\begin{definition}[]
    A \textbf{differential} $\mathbf p$\textbf{-form} on $U$ is a smooth map $\omega \colon U \to A_p(\R^n )$. The vector space of all such maps is denoted by $\Omega^p (U)$.
\end{definition}
If $p=0$, then $A_0(\R^n )=\R$, and $\Omega^0(U)$ is just the set of smooth functions on $U, C^{\infty}(U,\R)$. The derivative of a smooth map $\omega \colon U \to A_p(\R^n )$ is denoted $D\omega$, and is the linear map \[
    D_x\omega \colon \R^n  \to A_p(\R^n ), \quad e_i  \mapsto \frac{d}{dt}\omega(x+te_i )_{t=0}=\frac{\partial \omega}{\partial x_i }(x).
\] Let $I=(i_1,\cdots ,i_p)$, and write $\phi_I$\footnote{It slightly annoys me that indices aren't in the right place, but I don't want to make any mistakes deviating too far from the book, so they stay at the bottom for covectors.} for $\phi_{i_1}\wedge \cdots \wedge \phi_{i_p}$. Then we have the basis $\phi_I$ for $A_p(\R^n )$ as $I$ runs over all sequences of length $p\leq n$. So every $\omega \in \Omega^p(U)$ can be written in the form $\omega (w)=\sum \omega_I(x)\phi_I$, where the $\omega_I$ are smooth real-valued functions of  $x \in U$. The differential $D_x\omega$ is the linear map \[
D_x\omega(e_j )=\sum_I \frac{\partial \omega_I}{\partial x_j }(x)\phi_I, \quad j=1,\cdots ,n.
\] The function $x \mapsto D_x\omega$ is a smooth map from $U$ to $\operatorname{Hom}(\R^n ,A_p(\R^n ))$. {\color{red}todo:why? how exactly? difference between thsi and derivative?} 
\begin{definition}[]
    The \textbf{exterior differential} $d \colon \Omega^p(U) \to \Omega^{p+1}(U)$ is the linear operator \[
        d_x\omega(\xi_1,\cdots ,\xi_{p+1})=\sum_{\ell=1}^{p+1} (-1)^{\ell-1}D_x\omega(\xi_{\ell})(\xi_1,\cdots ,\hat{\xi}_{\ell},\cdots ,\xi_{p+1})
    \] where $(\xi_1,\cdots ,\hat{\xi}_{\ell},\cdots ,\xi_{p+1})=(\xi_1,\cdots ,\xi_{\ell-1},\xi_{\ell +1},\cdots ,\xi_{p+1})$. {\color{red}todo:what?} 
\end{definition}
The result lies in $\Omega^{p+1}(U)$ by \cref{transp}. If $\xi_i =\xi_{i+1}$, then 
\begin{align*}
    \sum_{\ell=1}^{p+1} &(-1)^{\ell -1}D_x\omega(\xi_{\ell})(\xi_1,\cdots ,\hat{\xi}_{\ell},\cdots ,\xi_{p+1})\\
    =&(-1)^{i-1}D_x\omega(\xi_i )(\xi_1,\cdots ,\hat{\xi}_i ,\cdots ,\xi_{p+1})\\
     &+(-1)^i D_x\omega(\xi_{i+1})(\xi_1,\cdots ,\hat{\xi}_{i+1},\cdots ,\xi_{p+1})\\
    =&0.
\end{align*}
In the second step, the rest of the terms cancel out by properties of the exterior product, since they all contain both $\xi_i $ and $\xi_{i+1}$. The final term also cancels out since $(\xi_1,\cdots ,\hat{\xi}_i ,\cdots ,\xi_{p+1})=(\xi_1,\cdots ,\hat{\xi}_{i+1},\cdots ,\xi_{p+1})$.

\begin{example}
    Let $x_i  \colon U \to \R$ be $i$th projection. Then $dx_1 \in \Omega^1(U)$ is the constant map $dx_i \colon x \to \phi_i $, which follows from the definition of the differential. In general, for $f \in \Omega^0(U)$, we have \[
        d_x f(\zeta)= \frac{\partial f}{\partial x_1}(x)\zeta^1+\cdots + \frac{\partial f}{\partial x_n }(x)\zeta ^n .
    \] 
\end{example}

\begin{lemma}
    If $\omega(x)=f(x)\phi_I$, then $d_x\omega=d_x f \wedge \phi_I$.
\end{lemma}
\begin{proof}
    Note that \[
        D_x\omega(\zeta)=(D_x f)(\zeta)\phi_I =\left( \frac{\partial f}{\partial x_1}\zeta^1+\cdots + \frac{\partial f}{\partial x_n }\zeta^n  \right) \phi_I=d_xf (\zeta) \phi_I.
    \] So by the definition of the exterior derivative, we have 
    \begin{align*}
        d_x\omega(\xi_1,\cdots ,\xi_{p+1})&=\sum_{k=1}^{p+1} (-1)^{k-1}d_x f(\xi_k) \phi_I (\xi_1,\cdots ,\hat{\xi}_k,\cdots ,\xi_{p+1})\\
        &= [d_x f \wedge \phi_I](\xi_1,\cdots ,\xi_{p+1}).\qedhere
    \end{align*}
\end{proof}
{\color{red}todo:this entire proof?} For $\phi_I \in A_p(\R^n )$, we have $\phi_k \wedge \phi_I=0$ if $k \in I$, and $(-1)^r \phi_J$ if $k\notin I$, where $r$ is determined by $i_r < k < i_{r+1}$ and $J=(i_1,\cdots ,i_r,k,\cdots ,i_p)$.

\begin{lemma}
    For $p\geq 0$, the composition $\Omega^p (U) \to \Omega^{p+1}(U)\to \Omega^{p+2}(U)$ is identically zero.
\end{lemma}
\begin{proof}
    Let $\omega=f \phi_I$. Then $d\omega=df\wedge \phi_I=\frac{\partial f}{\partial x_1}\phi_1\wedge \phi_I + \cdots + \frac{\partial f}{\partial x_n }\phi_n \wedge \phi_I.$ {\color{red}todo:alternating  terms?? does $I$ denote one sequence or several?} Since $\phi_i \wedge \phi_i =0 $ and $\phi_i \wedge\phi_j =-\phi_j \wedge \phi_i $, we have 
    \begin{align*}
        d^2\omega&= \sum_{i,j=1}^{n} \frac{\partial ^2 f}{\partial x_i \partial x_j }\phi_i  \wedge (\phi_j \wedge \phi_I)\\
                 &= \sum_{i<j}^{} \left( \frac{\partial ^2 f}{\partial x_i \partial x_j }-\frac{\partial ^2 f}{\partial x_j \partial x_j } \right) \phi_i \wedge \phi_j  \wedge \phi_I=0.\qedhere
    \end{align*}
\end{proof}
The exterior product on $A_*(\R^n )$ induces an exterior product on $\Omega^*(U)$ by defining $(\omega_1\wedge \omega_2)(x)=\omega_1(x)\wedge \omega_2(x)$. The exterior product of a $p$-form and $q$-form is a $(p+q)$-form, so it induces a bilinear map $\wedge \colon \Omega^p (U) \times \Omega^q(U) \to \Omega^{p+q}(U).$ Then for $f \in C^{\infty}(U,\R)$, we have $(f\omega_1)\wedge \omega_2=f(\omega_1\wedge \omega_2)=\omega_1\wedge f\omega_2$. Note that $f\wedge \omega=f\omega$ when $f \in \Omega^0(U)$ and $\omega \in \Omega^p(U)$.

\begin{lemma}\label{product} 
    For $\omega_1 \in \Omega^p(U)$ and $\omega_2 \in \Omega^q(U)$, \[
        d(\omega_1\wedge \omega_2)=d\omega_1 \wedge \omega_2+(-1)^p\omega_1\wedge d\omega_2.
    \] 
\end{lemma}
\begin{proof}
    It suffices to show this holds for $\omega_1=f \phi_I$ and $\omega_2=g \phi_J$. Since $\omega_1\wedge \omega_2= fg \phi_I\wedge \phi_J$, we have 
    \begin{align*}
        d(\omega_1\wedge \omega_2)&=d(fg)\wedge \phi_I\wedge \phi_J=((df)g+fdg)\wedge \phi_I\wedge \phi_J\\
                                  &= df g\wedge \phi_I\wedge \phi_J + fdg \wedge \phi_I \wedge \phi_J \\
                                  &= df \wedge  \phi_I \wedge g\phi_J +(-1)^pf \phi_I \wedge dg\wedge \phi_J\\
                                  &=d\omega_1\wedge\omega_2 +(-1)^p \omega_1 \wedge d\omega_2.\qedhere
    \end{align*}
\end{proof}

\subsection{Finally, de Rham cohomology}
In short, we have a new anti-commutative algebra $\Omega^*(U)$ with a \emph{differential} (or boundary) operator \[
    d \colon \Omega^*(U) \to \Omega ^{*+1}(U),\quad d\circ d=0,
\] and $d$ is a \emph{derivation} (since $d(\omega_1\wedge \omega_2)=d\omega_1\wedge \omega_2+d\omega_2\wedge \omega_1$ by \cref{product} and anticommutativity). Then $(\Omega^*(U),d)$ is an \emph{commutative differential graded algebra}\footnote{help}, called the \textbf{de Rham complex} of $U$.

\begin{theorem}
    There is precisely one linear operator $d \colon \Omega^p \to \Omega^{p+1}(U)$, $p=0,1,\cdots ,$ such that 
    \begin{enumerate}[label=(\roman*)]
        \setlength\itemsep{-.2em}
    \item $f \in \Omega^0(U)$, $df=\frac{\partial f}{\partial x_1}\phi_1+ \cdots + \frac{\partial f}{\partial x_n }\phi_n $,
    \item $d \circ d=0$,
    \item $d(\omega_1\wedge \omega_2)=d\omega_1\wedge \omega_2+(-1)^p\omega_1\wedge d\omega_2$ if $\omega_1 \in \Omega^p(U)$.
    \end{enumerate}
\end{theorem}
\begin{proof}
    We know that the exterior differential $d$ satisfies these properties. To show uniqueness, say $d'$ satisfies these properties: we will show it has to be the exterior derivative. (i) tells us that $d=d'$ on $\Omega^0(U)$ (since it characterizes smooth functions on $U$), in particular $d'x_i =dx_i =\phi_i $. Since $d' \circ d'=0$, then $d'\phi_i=d'(d'(x_i ))=0 $.
\end{proof}
