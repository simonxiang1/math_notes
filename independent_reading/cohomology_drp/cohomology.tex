\section{De Rham Cohomology} 

Switching gears, the reference book has changed to Bott and Tu. 
\subsection{The de Rham Complex on $\R^n $}
   If $x_1,\cdots ,x_n $ are the standard coordinates on $\R^n $, define $\Omega^*$ to be the algebra over $\R$ generated by $dx_1,\cdots ,dx_n $ with the relations
   \[
   \begin{cases}
       (dx_i )^2=0,\\
       dx_i dx_j =-dx_j dx_i ,\ i\neq j.
   \end{cases}
   \] 
   As a real vector space this has basis $1, dx_i ,dx_i dx_j , dx_i dx_j dx_k,\cdots ,dx_1\cdots dx_n $, where $i<j,i<j<k$ (or the $(*,n)$-shuffles). The $C^{\infty}$ \textbf{differential forms} on $\R^n $ are elements of $\Omega^* (\R^n )= \{C^{\infty}\ \text{functions on} \ \R^n \} \bigotimes_{\R}\Omega^*$. Recall that the tensor product of two $R$-algebras $A,B$ has basis $a_i\otimes b_j $, where multiplication is defined by $(a_1\otimes b_1)(a_2\otimes b_2)=a_1b_1\otimes a_2b_2$. So a form $\omega$ can be uniquely written as $\sum f_{i_1\cdots i_q}dx_{i_1}\cdots dx_{i_q}$, where the coefficients $f_{i_1\cdots i_q}$ are smooth functions. The multi-index notation simplifies this to $\omega=\sum f_I dx_I$. The algebra $\Omega^*(\R^n )=\bigoplus _{q=0}^n \Omega^q(\R^n )$ is naturally graded, where $\Omega^q$ is the space of $C^{\infty}$ $q$-forms on $\R^n $. There is a \emph{differential operator} \[
       d \colon \Omega^q(\R^n ) \to \Omega^{q+q}(\R^n )
   \] defined as follows:
   \begin{enumerate}[label=(\roman*)]
   \setlength\itemsep{-.2em}
       \item if $f \in \Omega^0(\R^n )$, then $df=\sum \frac{\partial f}{\partial x^i }dx_i $,
        \item if $\omega=\sum f_I dx_I$, then $d\omega=\sum df_I dx_I$.
   \end{enumerate}We call this differential operator \textbf{exterior differention}.
\begin{example}
    If $\Omega=x\, dy$, then $d\omega=dx\,dy$. On $\R^3$, $\Omega^0(\R^3)$ and $\Omega^3(\R^3)$ are both $1$-dimensional and $\Omega^1(\R^3)$ and $\Omega^2(\R^3)$ are each $3$-dimensional over the $C^{\infty}$ functions, so we identify 
    \begin{gather*}
   \underset{f}{ \{\text{functions} \} } \quad \simeq \quad \underset{f}{\{0\text{-forms} \}}  \quad \simeq \quad \underset{f\,dx\,dy\,dz}{\{3\text{-forms} \ \} 
}  ,   \\
\underset{X=(f_1,f_2,f_3)}{ \{\text{vector fields} \} } \quad \simeq \quad \underset{f_1dx+f_2dy+f_3dz}{\{1\text{-forms} \}}  \quad \simeq \quad \underset{f_1dy\,dz-f_2dx\,dz+f_3dx\,dy}{\{2\text{-forms} \ \} 
}  .
    \end{gather*}So for functions, \[
    df= \frac{\partial f}{\partial x}dx+\frac{\partial f}{\partial y}dy+\frac{\partial f}{\partial z}dz,
    \] for $1$-forms \[
    d(f_1dx+f_2dy+f_3dz)=\left( \frac{\partial f_3}{\partial y}-\frac{\partial f_2}{\partial z} \right) dy\,dz-\left( \frac{\partial f_1}{\partial z}-\frac{\partial f_3}{\partial x} \right) dx\,dz+\left( \frac{\partial f_2}{\partial x}-\frac{\partial f_1}{\partial y} \right) dx\,dy,
    \] and for $2$-forms \[
    d(f_1\,dy\,dz-f_2\,dx\,dz+f_3\,dx\,dy=\left( \frac{\partial f_1}{\partial x}+\frac{\partial f_2}{\partial y}+\frac{\partial f_3}{\partial z} \right) dx\,dy\,dz.
\] So $d(0$-forms$)=$ gradient, $d(1$-forms$)=$ curl, and $d(2$-forms$)=$ divergence.{\color{red}todo:yay worked it out! transcribe} 
\end{example}
\begin{definition}[]
    Define the \textbf{wedge product} of two differential forms, written $\tau \wedge \omega$ for $\tau= \sum f_I dx_I,\ \omega=\sum g_J dx_J$ by \[
    \tau \wedge \omega=\sum f_I g_J \,dx_I \,dx_J.
\] Note that $\tau \wedge \omega=(-1)^{\deg \tau \deg \omega}\omega \wedge \tau$.
\end{definition}
\begin{prop}
    $d$ is an antiderivation, i.e., \[
        d(\tau \wedge \omega)=(d\tau )\wedge \omega +(-1)^{\deg \tau }\tau \wedge d\omega.
    \] 
\end{prop}
\begin{proof}
    By linearity, we check on just the monomials $\tau=f_I\,dx_I,\ \omega=g_J \,dx_J$. Then \[
        d(\tau \wedge \omega)=d(f_1g_J)dx_I\,dx_J=(df_I)g_J\, dx_I\,dx_J+f_I\,dg_J\,dx_I\,dx_J=(d\tau)\wedge \omega+(-1)^{\deg \tau}\tau \wedge d\omega.
    \] 
\end{proof}
\begin{prop}
    $d^2=0$.
\end{prop}
\begin{proof}
    On functions, \[
        d^2f=d\left( \sum _i \frac{\partial f}{\partial x_i }dx_i  \right) =\sum _{i,j}\frac{\partial ^2f}{\partial x_j \partial x_i }dx_j dx_i =0,
    \] since the factors $\partial ^2 /\partial x_j  \partial x_i $ are symmetric in $i,j$ (mixed partials commute) while the $dx_j dx_i $ are skew-symmetric in $i,j$, hence $d^2f=0$. On forms $\omega=f_I\,dx_I$, \[
    d^2\omega =d^2(f_I\,dx_I)=d(df_Idx_I)=0.\qedhere
    \] 
\end{proof}
The complex $\Omega^*(\R^n )$ with the differential operator $d$ is the \textbf{de Rham complex} on $\R^n $. The kernel of $d$ are \textbf{closed forms} and the image of $d$ are \textbf{exact forms}. You can view the de Rham complex as a set of differential equations with solutions the closed forms. For example, finding a closed $1$-form $f\, dx+g\,dy$ on $\R^2$ is just like solving the differential equation $\partial g/\partial x-\partial f /\partial y=0$.

Exact forms are automatically closed, since composing with $d$ again gives zero. These are the ``trivial'' or ``uninteresting'' solutions: the de Rham cohomology measures the size of the space of ``interesting'' solutions.

\begin{definition}[]
    The $q$-th \textbf{de Rham cohomology} of $\R^n $ is the vector space \[
        H_{DR}^q (\R^n )= \{\text{closed} \ q\text{-forms}\} / \{\text{exact} \ q\text{-forms}  \} .
    \] We often write $H^q(\R^n )$ in place of $H_{DR}^q(\R^n )$. Denote the cohomology class of a form by $[\omega]$.
\end{definition}
All the definitions work just as well with an open subset $U\subseteq \R^n $. For example, define $\Omega^*(U)$ to be the algebra $\{C^{\infty} \ \text{functions on} \ U\} \bigotimes _{\R}\Omega^*$. Then we may speak of the de Rham cohomology $H_{DR}^*(U)$ of $U$.
\begin{example}
    If $n=0$, then \[
        H^q=
    \begin{cases}
        \R,\quad &\text{if} \ q=0,\\
        0,& \text{if} \ q>0.
    \end{cases}
\] Since $\ker d \cap \Omega^0(\R^1)$ consists of constant functions, $H^0(\R^1)=\R$. 
\end{example}
