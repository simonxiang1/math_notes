\newpage
\part{Euclidian Spaces}
\section{Smooth Functions on a Euclidian Space}
\begin{center}
    \sc{Introduction} 
\end{center}
Calculus talks about differentiation and integration on $\R$, while real analysis extends this to $\R^n $. Vector calculus talks about integrals on curves and surfaces, and now we extend these concepts to higher dimensions, the structures which with we work with are called manifolds. Things become simple: gradient, curl, and divergence are cases of the exterior derivative, and the FTC for line integrals, Green's theorem, Stokes' theorem, and the divergence theorem are manifestations of the generalized Stokes' theorem.

Manifolds arise even when dealing with the space we live in, for example the set of affine motions in $\R^3$ is a $6$-manifold. This is our plan: recast calculus on $\R^n $ so we can generalize it to manifolds by differential forms. Working in $\R^n $ first isn't necessary, but much easier, since the examples are simple. Then, we define a manifold and talk about tangent spaces, working with the idea of approximating nonlinear things with linear things, with Lie groups and Lie algebras as examples. Finally, we do calculus on manifolds, generalizing the theorems of vector calculus, with the de Rham cohomology groups as $C^{\infty}$ and topological invariants.
\orbreak
\subsection{$C^{\infty}$ Versus Analytic Functions}
Let's talk about $C^{\infty}$ functions on $\R^n $. Write a base for $\R^n $ as $x^1,\cdots ,x^n $ and let $p=(p^1,\cdots ,p^n )$ be a point in an open set $U$ in $\R^n $. Differential geometry uses \emph{superscripts}, not subscripts, more on this later.
\begin{definition}[]
    Let $k$ be a nonnegative integer. A function $f \colon U \to \R$ is $C^k$ at $p$ if its partial derivatives $\frac{\partial ^j f}{\partial x^{i_1} \cdots \partial x^{i_j }}$ of all orders $j\leq k$ exist and are continuous at $p$. The function $f \colon U \to \R$ is $C^{\infty}$ at $p$ if it is $C^k$ for all $k\geq 0$, that is, its partial derivatives of all orders \[
        \frac{\partial ^kf}{\partial x^{i_1}\cdots \partial x^{i_k}}
    \] exist and are continuous at $p$. We say $f$ is $C^k$ on $U$ if it is  $C^k$ for all points in $U$, and the concept of $C^{\infty}$ on a set $U$ is defined similarly. When we say ``smooth'', we mean $C^{\infty}$.
\end{definition}
\begin{example}
    \,
    \begin{enumerate}[label=(\roman*)]
        \item We call $C^0$ functions on $U$ continuous on $U$.
        \item Let $f \colon \R \to \R$ be $f(x)=x^{1/3}$. Then $f'(x)$ is $\frac{1}{3}x^{-2/3}$ for $x\neq 0$ and undefined at zero, so $f$ is $C^0$ but not $C^1$ at $x=0$.
        \item Let $g\colon \R \to \R$ be defined by  \[
                g(x)=\int_{0}^{x} f(t) \, dt= \int_{0}^{x} t ^{1/3} \, dt= \frac{3}{4}x^{4/3}.
            \] Then $g'(x)=f(x)=\frac{1}{3}$, so $g(x)$ is $C^1$ but not $C^2$ at $x=0$. In general, we can construct functions that are $C^k$ but not $C^{k+1}$ at a point.
        \item Polynomials, the sine and cosine functions, and the exponential functions on $\R$ are all $C^{\infty}$.
    \end{enumerate}
\end{example}
A function $f$ is \textbf{real-analytic} at $p$ if in some neighborhood of $p$ it is equal to its Taylor series at $p$, that is, \[
    f(x)=f(p)+\sum_{i}^{} \frac{\partial f}{\partial x^i }(p)(x^i -p^i )+\frac{1}{2!}\sum_{i,j}^{} \frac{\partial ^2f}{\partial x^i \partial x^j}(p) (x^i -p^i )(x^j-p^j)+\cdots 
\] Real-analytic functions are $C^{\infty}$ because you can differentiate them termwise in their region of convergence. The converse does not hold: define \[
f(x)=
\begin{cases}
    e^{-1/x} \quad & \text{for} \ x>0;\\
    0 & \text{for} \ x\leq 0.
\end{cases}
\] We can show $f$ is $C^{\infty}$ on $\R$ and the derivatives $f^{(k)}(0)=0$ for all $k\geq 0$ by induction, then the Taylor series must be zero in any neighborhood of the origin, but $f$ is not. Then $f$ isn't equal to its Taylor series, and we have a smooth non-analytic function.
\subsection{Taylor's Theorem with Remainder}
However, we have a Taylor's theorem with remainder for $C^{\infty}$ functions that's good enough. Say a subset $S$ of $\R^n $ is \textbf{star-shaped} with respect to a point $p$ in $S$ if for every $x\in S$, the line segment from $p$ to $x$ lies in $S$.
\begin{lemma}[Taylor's theorem with remainder]
    Let $f$ be a $C^{\infty}$ function on an open subset $U$ of $\R^n $ star-shaped with respect to a point $p=(p^1,\cdots ,p^n )$ in $U$. Then there are $C^{\infty}$ functions $g_1(x),\cdots ,g_n (x)$ on $U$ such that \[
        f(x)=f(p)+\sum_{i=1}^{n} (x^i -p^i )g_i (x), \quad g_i (p)=\frac{\partial f}{\partial x^i }(p).
    \]  
\end{lemma}
\begin{proof}
    For any $x\in U$ the line segment $p+t(x-p), \ 0\leq t \leq 1$ lies in $U$. So $f(p+t(x-p))$ is defined, and by the chain rule we have \[
        \frac{d}{dt}f(p+t(x-p))=\sum_{}^{} (x^i -p^i ) \frac{\partial f}{\partial x^i }(p+t(x-p)).
    \] Integrating both sides with respect to $t\in [0,1]$ we have \[
    \Big. f(p+t(x-p)) \Big| _0^1 = \sum_{}^{} (x^i -p^i )\int_{0}^{1} \frac{\partial f}{\partial x^i }(p+t(x-p)) \, dt.
\] Let $g_i (x)=\int_{0}^{1} \frac{\partial f}{\partial x^i }(p+t(x-p)) \, dt$. Then $g_i (x)$ is $C^{\infty}$ and the above expression simplifies to \[
f(x)-f(p)=\sum_{}^{} (x^i -p^i )g_i (x).
\] Furthermore, $g_i (p)=\int_{0}^{1} \frac{\partial f}{\partial x^i }(p) \, dt=\frac{\partial f}{\partial x^i }(p)$. 
\end{proof}
If $n=1$ and $p=0$, this lemma says that $f(x)=f(0)+xf_1(x)$ for a $C^{\infty}$ function $f_1(x)$. Applying repeatedly gives $f_i (x)=f_i (0)+xf_{i+1}(x)$, where $f_i ,f_{i+1}$ are $C^{\infty}$ functions. So 
\begin{align*}
    f(x)&=f(0)+x(f_1(0)+xf_2(x))\\
        &=f(x)+xf_1(x)+x^2(f_2(0)+xf_3(x))\\
        &\quad \vdots \\
        &=f(0)+f_1(0)x+f_2(0)x^2+\cdots +f_i (0)x^i +f_{i+1}(x)x^{i+1}.
\end{align*} If we differentiate this expression $k$ times, we get $f^{(k)}(0)=k!f_k(x)$, which simplifies to $f_k(0)=\frac{1}{k!}f^{(k)}(0)$ for $k=1,2,\cdots ,i$. Note that balls are star-shaped, and since $U$ is open there exists an $\varepsilon >0$ such that $p\in B(p,\varepsilon )\subseteq U$. So when a function's domain is restricted to $B(p,\varepsilon )$, $f$ is defined on a star-shaped neighborhood of $p$ and Taylor's theorem with remainder applies.

\section{Tangent Vectors in $\R^n $ as Derivations}
Vectors at a point $p$ are usually represented by columns of points or arrows stemming from $p$. A vector at $p$ is tangent to a surface if it lies in the tangent plane at $p$, the limiting position of the secant planes through $p$. This kind of definition assumes we live in $\R^n $ and would not work for a large class of manifolds. We will find a generalization that works for manifolds.
\subsection{The Directional Derivative}
We usually visualize the tangent space $T_p (\R^n )$ at $p\in \R^n $ as the vector space of all arrows emanating from $p$. This can be identified with the vector space $\R^n $. We write points as $p=(p^1,\cdots ,p^n )$ and vectors $v$ in $T_p(\R^n )$ as \[
v= 
\begin{bmatrix}
    v^1 \\ \vdots \\ v^n 
\end{bmatrix} \quad \text{or} \quad \langle v^1,\cdots ,v^n  \rangle .
\] We denote the standard basis for $\R^n $ or $T_p (\R^n )$ by $\{e_1,\cdots ,e_n \} $. Then $v=\sum_{}^{} v^i e_i .$ Sometimes we denote $T_p(\R^n )$ by $T_p\R^n $. Elements of $T_p (\R^n )$ are called \textbf{tangent vectors} (or simply \textbf{vectors}) at $p$ in $\R^n $.

The line through a point $p=(p^1,\cdots ,p^n )$ with direction $\langle v_1,\cdots ,v_n  \rangle $ in $\R^n $ has parametrization $c(t)=(p^1+tv^1,\cdots ,p^n +tv^n )$, with $i$th component $c^i (t)=p^i +tv^i $. If $f$ is $C^{\infty}$ in a neighborhood of $p$ in $\R^n $ and $v$ is a tangent vector at $p$, the \textbf{directional derivative} of $f$ in the direction $v$ at $p$ is defined to be \[
    D_v f = \lim_{t \to 0} \frac{f(c(t))-f(p)}{t}= \left. \frac{d}{dt} \right| _{t=0}f(c(t)).
\] By the chain rule, \[
D_v f = \sum_{i=1}^{n} \frac{dc^i }{dt}(0)\frac{\partial f}{\partial x^i }(p)=\sum_{i=1}^{n} v^i \frac{\partial f}{\partial x^i }(p).
\] Note that $D_vf$ is a number, since we evaluate partial derivatives at a point $p$. We write $D_v = \sum_{}^{} v^i  \left. \frac{\partial }{\partial x^i } \right| _p$ for the operator that sends a function $f$ to the number $D_v f$. Often times we omit the subscript $p$.
    \subsection{Germs of Functions}
    If two functions agree on some neighborhood of a point $p$, they will have the same directional derivatives at $p$. This suggests introducing an equivalence relation on the $C^{\infty}$ functions defined in some neighborhood of $p$. Consider the set of pairs $(f,U)$, where $U$ is a neighborhood of $p$ and $f \colon U \to \R$ is a $C^{\infty}$ function. We say $(f,U)$ is equivalent to $(g,V)$ if there is an open set  $W\subseteq U\cap V$ containing $p$ such that $f=g$ when restricted to $W$. The equivalence class of $(f,U)$ is called the \textbf{germ} of $f$ at $p$. We write $C_p ^{\infty}(\R^n )$ or simply $C_p^{\infty}$ if there is no possibility of confusion, for the set of all germs of $C^{\infty}$ functions on $\R^n $ at $p$.
    \begin{example}
        The functions $f(x)=\frac{1}{1-x}$ with domain $\R \setminus \{1\} $ and $g(x)=\sum_{n=0}^{\infty} x^n$ with domain $(-1,1)$ have the same germ at any point $p$ in the open inverval $(-1,1)$.
    \end{example}
    An \textbf{algebra} over a field $K$ is a vector space $A$ over $K$ with a multiplication map $\mu \colon A\times A \to A$, usually written $\mu(a,b)=a\times b$, such that for all $a,b,c\in A$ and $r\in K$, 
    \begin{enumerate}[label=(\roman*)]
        \item $(a\times b)\times c=a\times (b\times c)$ (associativity),
        \item $(a+b)\times c=a\times c+b\times c$ and $a\times (b+c)=a\times b+a\times c$ (distributivity),
        \item $r(a\times b)=(ra)\times b=a\times (rb)$ (homogeneity).
    \end{enumerate}
    Equivalently, an algebra over a field $K$ is a ring $A$ also a $K$-vector space such that ring multiplication satisfies the homogeneity condition. So an algebra has three operations: the addition and multiplication of a ring, and the scalar multiplication of a vector space. Usually we write $ab$ in place of $a\times b$.

    Addition and multiplication of functions induce operations on $C_p^{\infty}$, making it into an algebra over $\R$.

    \subsection{Derivations at a Point}
    A map $L \colon V \to W$ between vector spaces over a field $K$ is called a \textbf{linear map} or a \textbf{linear operator} if for any $r\in K$ and $u,v\in V$,
    \begin{enumerate}[label=(\roman*)]
        \item $L(u+v)=L(u)+L(v);$
        \item $L(rv)=rL(v).$
    \end{enumerate}
    Such a map can also be called $K$\emph{-linear}.

    For every tangent vector $v$ at a point $p$ in $\R^n $, the directional derivative at $p$ gives a map of real vector spaces $D_v \colon C_p^{\infty} \to \R$. Since $D_v f= \sum_{i=1}^{n} v^i  \frac{\partial f}{\partial x^i }(p),$ we have that $D_v$ is $\R$-linear and satisfies the Leibniz rule \[
        D_v(fg)=(D_vf)g(p)+f(p)D_vg,
    \] since the partial derivatives $\partial  / \partial x^i \big|_p$ have these properties. In general, a linear map $D \colon C_p^{\infty} \to \R$ satisfying the Leibniz rule is called a \textbf{derivation at} $\mathbf p$ or a \textbf{point-derivation} of $C_p^{\infty}$. Denote the set of all derivations at $p$ by $\mathcal{D} _p(\R^n )$. This set is a real vector space, since the sum of two derivations at $p$ and a scalar multiple of a derivation at $p$ are again derivations at $p$. We know that directional derivatives at $p$ are all derivations at $p$, so we have a map \[
    \phi \colon T_p(\R^n ) \to \mathcal{D} _p(\R^n ) , \quad v\mapsto D_v=\sum_{}^{} v^i  \left. \frac{\partial }{\partial x^i } \right| _p.
    \] Since $D_v$ is linear in $v$, the map $\phi$ is a linear operator of vector spaces.
    \begin{lemma}
        If $D$ is a point-derivation of $C_p^{\infty}$, then $D(c)=0$ for any constant function $c$.
    \end{lemma}
    \begin{proof}
        By $\R$-linearity, $D(c)=cD(1)$. To show $D(1)=0$, we have \[
            D(1)=D(1\times 1)=D(1)\times 1+1\times D(1)=2D(1)
        \] by the Leibniz rule. Subtracting $D(1)$ from both sides gives $D(1)=0$.
    \end{proof}
    \begin{theorem}
        The linear map $\phi \colon T_p(\R^n ) \to \mathcal{D} _p(\R^n )$ is an isomorphism of vector spaces.
    \end{theorem}
    \begin{proof}
        To show injectivity, assume $D_v=0$ for $v\in T_p(\R^n )$. Applying $D_v$ to the coordinate function $x^j$ (that sends $x\mapsto x^j$) gives \[
            0=D_v(x^j)=\sum_{i}^{} v^i \left. \frac{\partial }{\partial x^i } \right| _p x^j=\sum_{i}^{} v^i  \delta_i ^j=v^j.
            \] \footnote{I think $\delta_i ^j$ refers to the function that is one if $i=j$ and zero otherwise.} To prove surjectivity, let $D$ be a derivation at $p$ and let $(f,V)$ be a representative of a germ in $C_p^{\infty}$. Making $V$ smaller if necessary, assume that $V $ is an open (star-shaped) ball. By Taylor's theorem with remainder there are $C^{\infty}$ functions $g_i (x)$ in a neighborhood of $p$ such that \[
            f(x)=f(p)+\sum_{}^{} (x^i -p^i )g_i (x), \quad g_i (p)=\frac{\partial f}{\partial x^i }(p).
        \] Note that $D(f(p))$ and $D(p^i )$ equal zero since $f(p)$ and $p^i $ are constant. Applying $D$ to both sides, we get by the Leibniz rule \[
        D f(x)=\sum_{}^{} (Dx^i )g_i (p)+\sum (p^i -p^i ) Dg_i (x)=\sum (Dx^i ) \frac{\partial f}{\partial x^i }(p).
        \] So $D=D_v$ for $v=\langle Dx^1,\cdots ,Dx^n  \rangle $.
    \end{proof}
    This theorem shows that we can identify tangent vectors at $p$ with derivations at $p$. Under the identification $T_p(\R^n )\simeq \mathcal{D} _p (\R^n )$, the standard basis $\{e_1,\cdots ,e_n \} $ for $T_p(\R^n )$ corresponds to the set $\{\partial /\partial x^1|_p,\cdots ,\partial /\partial x^n |_p\}$ of partial derivatives. From now, we write a tangent vector $\langle v^1,\cdots ,v^n  \rangle =\sum_{}^{} v^i e_i $ as $v=\sum_{}^{} v^i  \left. \frac{\partial }{\partial x^i } \right| _p$. Although the vector space $\mathcal{D} _p(\R^n )$ of derivations is not as intuitive, they turn out to be more suitable for generalization to manifolds.

\subsection{Vector Fields}
A \textbf{vector field} $X$ on an open subset $U$ of $\R^n $ is a function that assigns to each point $p$ in $U$ a tangent vector $X_p$ in $T_p(\R^n )$. Since $T_p(\R^n )$ has basis $\{\partial /\partial x^i |_p\} $, the vector $X_p$ is a linear combination $
    X_p = \sum_{}^{} a^i (p) \left. \frac{\partial }{\partial x^i } \right| _p, \ p \in U.$ We say that the vector field $X$ is $C^{\infty}$ on $U$ if the coefficient functions $a^i $ are all $C^{\infty}$ on $U$.
We can identify vector fields on $U$ with column vectors of $C^{\infty}$ functions on $U$: \[
X=\sum a^i  \frac{\partial }{\partial x^i }\quad \longleftrightarrow \quad
\begin{bmatrix}
    a^1\\ \vdots \\ a^n 
\end{bmatrix}.
\] The ring of $C^{\infty}$ functions on $U$ is commonly denoted $C^{\infty}(U)$ or $\mathcal{F} (U)$. Since one can multiply a $C^{\infty}$ vector field by a $C^{\infty}$ function and still get a $C^{\infty}$ vector field, the set of all $C^{\infty}$ vector fields on $U$, denoted $\mathfrak X (U)$, is not only a vector space over $\R$ but also a $C^{\infty}(U)$-module. 
\subsection{Vector Fields as Derivations}
If $X$ is a $C^{\infty}$ vector field on an open subset $U$ of $\R^n $ and $f$ is a $C^{\infty}$ function on $U$, we define a new function $Xf$ on $U$ by $(Xf)(p)=X_pf$ for any $p\in U$. Writing $X=\sum a^i  \partial /\partial x^i $, we get \[
    (Xf)(p)=\sum a^i (p) \frac{\partial f}{\partial x^i }(p) \quad \text{or} \quad Xf=\sum a^i  \frac{\partial f}{\partial x^i },
\] which shows $Xf$ is a $C^{\infty}$ function on $U$. So a $C^{\infty}$ vector field gives rise to an $\R$-linear map $C^{\infty}(U)\to C^{\infty}(U)$, $f\mapsto Xf$.
\begin{prop}
    If $X$ is a $C^{\infty}$ vector field and $f$ and $g$ are $C^{\infty}$ functions on an open subset $U$ of $\R^n $, then $X(fg)$ satisfies the product rule (Leibniz rule) $X(fg)=(Xf)g+fXg$.
\end{prop}
\begin{proof}
    At each point $p \in U$, the vector $X_p$ satisfies the Leibniz rule $X_p(fg)=(X_pf)g(p)+f(p)X_pg$. As $p$ varies over $U$, this becomes an equality of functions $X(fg)=(Xf)g+fXg$.
\end{proof}
If $A$ is an algebra over a field $K$, a \textbf{derivation} of $A$ is a $K$-linear map $D \colon A \to A$ such that \[
    D(ab)=(Da)b+aDb \quad \text{for all} \ a,b\in A.
\] The set of all derivations of $A$ is closed under addition and scalar multiplication and forms a vector space, denoted $\operatorname{Der}(A)$. As noted above, a $C^{\infty}$ vector field on an open set $U$ gives rise to a derivation of the algebra $C^{\infty}(U)$. We therefore have a map $\varphi \colon \mathfrak X(U) \to \operatorname{Der}(C^{\infty}(U))$, $X\mapsto (f\mapsto Xf)$. Similar to how tangent vectors at a point $p$ can be identified with the point-derivations of $C_p^{\infty}$, so the vector fields on an open set $U$ can be identified with the derivations of the algebra $C^{\infty}(U)$, i.e., the map $\varphi $ is an isomorphism of vector spaces. Note that a derivation at $p$ is not a derivation of the algebra $C_p^{\infty}$. A derivation at $p$ is a map from $C_p^{\infty}$ to $\R$, while a derivation of the algebra $C_p^{\infty}$ is a map from $C_p^{\infty}\to C_p^{\infty}$.

\section{Alternating $k$-Linear Functions}
\subsection{Dual Space}
If $V$ and $W$ are real vector spaces, we denote the vector space of all linear maps $f \colon V \to W$ by $\operatorname{Hom}(V,W)$. Define the \textbf{dual space} $V^*$ to be the set of all real valued linear functions on $V$, denoted  $V^*:= \operatorname{Hom}(V,\R)$. Elements of $V^*$ are called \textbf{covectors} or $1$-\textbf{covectors} on $V$. Assume $V$ is finite-dimensional, and let $\{e_1,\cdots ,e_n \} $ be a basis for $V$. Then every $v\in V$ is uniquely a linear combination $v=\sum v^i e_i $ with $v^i \in \R$. Let $\alpha ^i \colon V \to \R$ be the linear function that picks out the $i$th coordinate, given by  $\alpha ^i (v)=v^i $. Note that $\alpha ^i (e_j )=\delta_j ^i $.
\begin{prop}
    The functions $\alpha ^1,\cdots ,\alpha ^n $ form a basis for $V^*$.
\end{prop}
\begin{proof}
    Let $f\in V^*$ and $v=\sum v^i e_i \in V$. Then $f(v)=\sum v^i f(e_i )=\sum f(e_i )\alpha ^i (v)$. So $f=\sum f(e_i )\alpha ^i $, and so the $\alpha ^i $ span $V^*$. Now suppose $\sum c_i \alpha ^i =0$ for some $c_i \in \R$. Then $0=\sum c_i \alpha ^i (e_j )=\sum c_i \delta_j ^i =c_j $ for $j=1,\cdots ,n$. So the $\alpha ^i $ are LI.
\end{proof}
This basis $\{\alpha ^1,\cdots ,\alpha ^n \} $ is said to be \emph{dual} to the basis $\{e_1,\cdots ,e_n \} $ for $V$.
\begin{cor}
    The dual space $V^*$ of a finite-dimensional vector space $V$ has the same dimension as $V$.
\end{cor}
\begin{example}
    If $e_1,\cdots ,e_n $ is a basis for a vector space $V$, every $v\in V$ can be uniquely written as a linear combo $v=\sum b^i (v)e_i $, where $b_i (v)\in \R$. Let $\alpha^1,\cdots ,\alpha ^n $ be the basis of $V^*$ dual to $e_1,\cdots ,e_n $. Then \[
        \alpha ^i (v)= \alpha ^i  \left( \sum_j b^j(v)e_j  \right) =\sum_j b^j (v)\alpha ^i (e_j )=\sum_j b^j(v)\delta_j ^i =b^i (v).
    \] So the set of coordinate functions $b^1,\cdots ,b^n $ WRT the basis $e_1,\cdots ,e_n $ is precisely the dual basis.
\end{example}
\subsection{Permutations}
Quick review since you know what permutations are. They're self-bijections, or elements of the symmetric group on such set. You use cycle notation to denote them, and a transposition is a $2$-cycle. Recall that the sign of a permutation denoted $\operatorname{sgn}(\sigma)$ is $\pm 1$ depending on whether the permutation is even or odd. Since $\operatorname{sgn}(\sigma\tau)=\operatorname{sgn}(\sigma)\operatorname{sgn}(\tau)$, one way to compute signs is to decompose and count (eg if its a product of odd and even cycles it must be odd).

An \textbf{inversion} in a permutation $\sigma$ is an ordered pair $(\sigma(i),\sigma(j))$ such that $i<j$ but $\sigma(i)>\sigma(j)$. For example, the permutation $(124)(35)$ has the inversions $(21),(41),(51),(43),(53)$. Another way to compute the sign of a permutation is to count the number of inversions:
\begin{prop}
    A permutation is even iff it has an even number of inversions.
\end{prop}
\begin{proof}
    We can multiply by transpositions corresponding to inversions and recover our original list. 
    \begin{enumerate}[label=(\arabic*)]
        \item Find $1$ in the list $\sigma(1),\sigma(2),\cdots ,\sigma(k)$, then every number before $1$ gives rise to an inversion. Say $\sigma(i)=1$, then $(\sigma(1),1),\cdots ,(\sigma(i-1),1)$ are all inversions. Apply the $i-1$ transpositions to move $1$ to the front, the number of inversions ending in $1$.
        \item Now find $2$ in the list $1,\sigma(1),\cdots ,\sigma(\hat{i}),\cdots ,\sigma(k)$ (using deletion notation, note that we moved $1$ to the front). Every number (besides 1) preceding $2$ gives rise to an inversion $(\sigma(m),2)$, suppose we have $i_2$ such inversions. Applying $i_2$ transformations, we move $2$ to the front.
    \end{enumerate}
If you continue to the sort, you'll see the number of transpositions required to order the list is the same as the number of inversions. So the transposition decomposition is the number of inverstions, and $\operatorname{sgn}(\sigma)=(-1)^{\text{\# of inversions in} \,\sigma}$.
\end{proof}
\subsection{Multilinear Functions}
Let $V^k=\overset{\text{k times}  }{\overbrace{V\times \cdots \times V}} $ for $V$ a real vector space. A function $f \colon V^k \to \R$ is $\mathbf k$\textbf{-linear} if it is linear in each of its $k$ arguments, that is, $f(\cdots ,av+bw,\cdots )=af(\cdots ,v,\cdots )+b f(\cdots ,w,\cdots )$ for $a,b\in \R$, $v,w\in V$. Usually we say ``bilinear'' and ``trilinear'' instead of $2$-linear and $3$-linear. A $k$-linear function on $V$ is also called a $\mathbf k$\textbf{-tensor} on $V$. We denote the vector space of $k$-tensors on $V$ by $L_k(V)$. If $f$ is a $k$-tensor on $V$, we call $k$ the \textbf{degree} of $f$.
\begin{example}
    The dot product $f(v,w)=v\cdot w$ on $\R^n $ is bilinear. 
\end{example}
\begin{example}
    If we view the determinant $f(v_1,\cdots ,v_n )=\det[v_1\cdots v_n ]$ as a function on the $n$ column vectors in $\R^n $, then the determinant is $n$-linear.
\end{example}
\begin{definition}[]
    A $k$-linear function $f \colon V^k \to \R$ is \textbf{symmetric} if $f(v_{\sigma(1)},\cdots ,v_{\sigma(k)})=f(v_1,\cdots ,v_k)$ for all permutations $\sigma \in S_k$; it is \textbf{alternating} if $f(v_{\sigma(1)},\cdots ,v_{\sigma(k)})=(\operatorname{sgn}\sigma)f(v_1,\cdots ,v_k)$ for all $\sigma\in S_k$. 
\end{definition}
\begin{example}\,
    \begin{enumerate}[label=(\roman*)]
        \item The dot product $f(v,w)=v\cdot w$ on $\R^n $ is symmetric.
        \item The determinant $f(v_1,\cdots ,v_n )=\det[v_1\cdots v_n ]$ is alternating.
    \end{enumerate}
    Intuitively, symmetric $k$-linear functions don't care which order you input the variables in, while alternating multilinear maps preserve even orientation, but get flipped with an odd number of shuffles.
\end{example}
We are especially interested in the space $A_k(V)$ of all alternating $k$-linear functions on a vector space $V$ for $k>0$. These are called \textbf{alternating} $\mathbf k$\textbf{-tensors}, $\mathbf k$\textbf{-covectors}, or \textbf{multicovectors} on $V$. For $k=0$, we define a $0$-covector to be a constant so that $A_0(V)$ is the vector space $\R$. A $1$-covector is just a covector.
\subsection{Permutation Action on $k$-Linear Functions}
If $f$ is a $k$-linear function on a vector space $V$ and $\sigma\in S_k$, we define a new $k$-linear function by $(\sigma f)(v_1,\cdots ,v_k)=f(v_{\sigma(1)},\cdots ,v_{\sigma(k)})$. So $f$ is symmetric iff $\sigma f=f$ and alternating iff $\sigma f=(\operatorname{sgn}\sigma)f$ for all $\sigma \in S_k$. In the trivial case, $S_1 $ is the identity group, and a $1$-linear function is both symmetric and alternating, in particular, $A_1(V)=L_1(V)=V^*$.
\begin{lemma}
    If $\sigma,\tau \in S_k$ and $f$ is a $k$-linear function on $V$, then $\tau(\sigma f)=(\tau\sigma) f$. In other words, $S_k$ acts (from the left) on $L_k(V)$, the space of $k$-linear functions on $V$.
\end{lemma}
\begin{proof}
    Exercise. 
\end{proof}
\subsection{The Symmetrizing and Alternating Operators}
We can turn a $k$-linear function $f$ on a vector space $V$ into a symmetric $k$-linear function $Sf$ by defining \[
    (Sf)(v_1,\cdots ,v_k)= \sum_{\sigma \in S_k}^{} f(v_{\sigma(1)},\cdots ,v_{\sigma(k)})
\] or with our new notation, $Sf=\sum _{\sigma\in S_k}\sigma f$. Similarly, we the define the alternization $Af=\sum_{\sigma\in S_k}(\operatorname{sgn}\sigma)\sigma f$.
\begin{prop}\,
    \begin{enumerate}[label=(\roman*)]
        \item The $k$-linear function $Sf$ is symmetric.
        \item The $k$-linear function $Af$ is alternating.
    \end{enumerate}
\end{prop}
\begin{proof}\,
    \begin{enumerate}[label=(\roman*)]
        \item If $\tau \in S_k$, we have $\tau(Sf)=\sum_{\sigma\in S_k}\tau(\sigma f)=\sum_{\sigma\in S_k}(\tau\sigma)f=Sf$, since by applying $\tau^{-1}$ we can obtain every permutation in $S_k$.
        \item If $\tau \in S_k$, we have $\tau(Af)=\sum_{\sigma \in S_k}(\operatorname{sgn}\sigma)\tau (\sigma f)=\sum_{\sigma \in S_k}(\operatorname{sgn}\sigma)(\tau\sigma)f$, since $S_k$ acts on $L_k(V)$ from the left. Since $(\operatorname{sgn}\tau)(\operatorname{sgn}\tau)=1$, this expression is equal to $(\operatorname{sgn}\tau)\sum_{\sigma\in S_k}(\operatorname{sgn}\tau\sigma)(\tau\sigma)f=(\operatorname{sgn}\tau)Af$.\qedhere
    \end{enumerate}
\end{proof}
\begin{lemma}
    If $f$ is an alternating $k$-linear function on a vector space $V$, then $Af=(k!)f$.
\end{lemma}
\begin{proof}
    \[
        Af=\sum_{\sigma\in S_k}(\operatorname{sgn}\sigma)\sigma f=\sum_{\sigma\in S_k}(\operatorname{sgn}\sigma)(\operatorname{sgn}\sigma)f=\sum_{\sigma\in S_k}f=k!f.\qedhere
    \] 
\end{proof}
\subsection{The Tensor Product}
Let $f$ be a $k$-linear function and $g$ be an $\ell$-linear function on a vector space $V$. Their \textbf{tensor product} is the $(k+\ell)$-linear function $f \otimes g$ defined by \[
    (f \otimes g)(v_1,\cdots ,v_{k+\ell})=f(v_1,\cdots ,v_k)g(v_{k+1},\cdots ,v_{k+\ell}).
\] 
\begin{example}
    Let $e_1,\cdots ,e_n $ be the standard basis for $\R^n $ and let $\alpha ^1,\cdots ,\alpha ^n $ be its dual basis. The Euclidian inner product on $\R^n $ is the bilinear function $\langle , \rangle  \colon \R^n  \times \R^n \to \R$ defined by $\langle v,w \rangle =\sum v^i w^i $ for $v=\sum v^i e_j $ and $w=\sum w^i e_i $. We can express $\langle , \rangle $ in terms of the tensor product: 
    \begin{align*}
        \langle v,w \rangle &=\sum_{i}^{} v^i w^i =\sum_i \alpha ^i (v)\alpha ^i (w)\\
                            &=(\alpha ^i \otimes \alpha ^i )(v,w).
    \end{align*}
    So $\langle , \rangle =\sum_i \alpha ^i \otimes \alpha ^i $. This notation is often used in differential geometry to describe an inner product on a vector space.
\end{example}
\subsection{The Wedge Product}
We would like for the product of two alternating multilinear functions to also be alternating. This motivates the \textbf{wedge product}: for $f\in A_k(V)$ and $g\in A_{\ell}(V)$, $f\wedge g = \frac{1}{k!\ell!}A(f\otimes g);$ or explicitly, \[
    (f\wedge g)(v_1,\cdots ,v_{k+\ell})=\frac{1}{k!\ell!}\sum_{\sigma\in S_{k+\ell}}(\operatorname{sgn}\sigma) f(v_{\sigma(1)},\cdots ,v_{\sigma(k)})g(v_{\sigma(k+1)},\cdots ,v_{\sigma(k+\ell)}).
\] Since the definition involves the alternization, $f\wedge g$ is alternating.

When $k=0$, the element $f\in A_0(V)$ is a constant $c$: then the wedge product $c\wedge g$ is scalar multiplication, since $\frac{1}{\ell!}\sum_{\sigma \in S_{\ell}}(\operatorname{sgn}\sigma)cg(v_{\sigma(1)},\cdots ,c_{\sigma(\ell)})=cg(v_1,\cdots ,v_{\ell})$. So $c\wedge g=cg$ for $c\in \R$ and $g\in A_{\ell}(V)$.

The coefficient $1 /(k!\ell!)$ compensates for the repetitions in the sum: there are $k!$ permutations that permute the arguments of $f$ and similarly $\ell!$ permutations of the arguments of $g$. So we divide to get rid of repeating terms.
\begin{example}
    For $f\in A_2(V)$ and $g\in A_1(V)$, $A(f\otimes g)(v_1,v_2,v_3)=f(v_1,v_2)g(v_3)-f(v_1,v_3)g(v_2)+f(v_2,v_3)g(v_1)-f(v_2,v_1)g(v_3)+f(v_3,v_1)g(v_2)-f(v_3,v_2)g(v_1)$. There are three pairs of equal terms, for example $f(v_1,v_2)g(v_3)=-f(v_2,v_1)g(v_3)$ (since $f$ is alternating) and so on. So after dividing by two, $(f\wedge g)(v_1,v_2,v_3)=f(v_1,v_2)g(v_3)-f(v_1,v_3)g(v_2)+f(v_2,v_3)g(v_1)$.
\end{example}
A way to avoid such redundancies in the definition of $f\wedge g$ is to stipulate that $\sigma(1),\cdots ,\sigma(k)$ be in ascending order and $\sigma(k+1),\cdots ,\sigma(k+\ell)$ also be in ascending order in the sum of the wedge product. We call a permutation $\sigma \in S_{k+\ell}$ a $\mathbf {(k,\ell)}$\textbf{-shuffle} if $\sigma(1)< \cdots < \sigma(k)$ and $\sigma(k+1)< \cdots  < \sigma(k+\ell)$. Then we can rewrite the definition of the wedge product as \[
    (f\wedge g)(v_1,\cdots ,v_{k+\ell}) = \sum_{(k,\ell)\text{-shuffles} \ \sigma}(\operatorname{sgn}\sigma )f (v_{\sigma(1)},\cdots ,v_{\sigma(k)})g(v_{\sigma(k+1)},\cdots ,v_{\sigma(k+\ell)}).
\] Written this way, the wedge $f\wedge g$ is a sum of ${k+\ell \choose k}$ terms, rather than $(k+\ell)!$ terms.
\begin{example}
    If $f$ and $g$ are covectors on a vector space $V$ and $v_1,v_2\in V$, then $(f\wedge g)(v_1,v_2)=f(v_1)g(v_2)-f(v_2)g(v_1)$.
\end{example}
\subsection{Anticommutativity of the Wedge Product}
It follows from the definition of the wedge product that $f\wedge g$ is bilinear in $f$ and $g$.
\begin{prop}
    The wedge product is anticommutative: if $f\in A_k(V)$ and $g\in A_{\ell}(V)$, then $f\wedge g=(-1)^{k\ell}g\wedge f$.
\end{prop}
\begin{proof}
    Define $\tau \in S_{k+\ell}$ to be the permutation \[
    \tau=
    \begin{pmatrix}
        1 & \cdots  & \ell & \ell+1 & \cdots  & \ell +k\\
        k+1 & \cdots  & k+\ell & 1 & \cdots  & k
    \end{pmatrix}.
\] Then $\sigma(1)=\sigma\tau(\ell+1),\cdots ,\sigma(k)=\sigma\tau(\ell+k),\ \sigma(k+1)=\sigma\tau(1),\cdots ,\sigma(k+\ell)=\sigma\tau(\ell)$. For any $v_1,\cdots ,v_{k+\ell}\in V$,
\begin{align*}
    A(f\otimes g)(v_1,\cdots ,v_{k+\ell})&=\sum _{\sigma \in S_{k+\ell}}(\operatorname{sgn}\sigma)f(v_{\sigma(1)},\cdots ,v_{\sigma(k)})g(v_{\sigma(k+1)},\cdots ,v_{k\sigma(k+\ell)})\\
                                         &= \sum _{\sigma \in S_{k+\ell}}(\operatorname{sgn}\sigma)f (v_{\sigma\tau(\ell+1)},\cdots , v _{\sigma\tau(\ell+k)})g(v_{\sigma\tau(1)},\cdots ,v_{\sigma\tau(\ell)})\\
                                         &=(\operatorname{sgn}\tau) \sum _{\sigma\in S_{k+\ell}}(\operatorname{sgn}\sigma\tau) g (v_{\sigma\tau(1)},\cdots , v_{\sigma\tau(\ell)})f(v_{\sigma\tau(\ell+1)},\cdots ,v_{\sigma\tau(\ell+k)})\\
                                         &=(\operatorname{sgn}\tau)A (g\otimes f)(v_1,\cdots ,v_{k+\ell}).
\end{align*}So $A(f\otimes g)=(\operatorname{sgn}\tau )A(g\otimes f)=(-1)^{k\ell}A(g\otimes f)$. Dividing by $k!\ell!$ gives $f\wedge g=(-1)^{k\ell}g\wedge f$.
\end{proof}
\begin{cor}
    If $f$ is a $k$-covector on $V$ and $k$ is odd, then $f\wedge f=0$. 
\end{cor}
\begin{proof}
    We have $f\wedge f = (-1)^{k^2}f\wedge f=-f \wedge f$ by anticommutativity, so $f\wedge f=0$.
\end{proof}
\subsection{Associativity of the Wedge Product}
\begin{lemma}
    Suppose $f$ is a $k$-linear function and $g$ is an $\ell$-linear function on a vector space $V$. Then
    \begin{enumerate}[label=(\roman*)]
        \item $A(A(f)\otimes g)=k!A(f\otimes g)$, and 
        \item $A(f\otimes A(g))=\ell! A(f\otimes g)$.
    \end{enumerate}
\end{lemma}
\begin{proof}
    We have \[
        A(A(f)\otimes g)= \sum _{\sigma \in S_{k+\ell}}(\operatorname{sgn}\sigma) \sigma \left( \sum _{\tau\in S_k}(\operatorname{sgn}\tau)(\tau f)\otimes g \right) .
    \] We can view $\tau \in S_k$ as a permutation in $S_{k+\ell}$ such that $\tau(i)=i$ for $i=k+1,\cdots ,k+\ell$. For such a $\tau$, $(\tau f)\otimes g=\tau (f\otimes g)$. So \[
    A(A(f)\otimes g)=\sum _{\sigma \in S_{k+\ell}}\sum _{\tau \in {S_k}}(\operatorname{sgn}\sigma)(\operatorname{sgn}\tau) (\sigma \tau)(f\otimes g).
\] Let $\mu =\sigma\tau \in S_{k+\ell}$, there are $k!$ ways to write $\mu=\sigma\tau$ with $\sigma \in S_{k+\ell}$ and $\tau\in S_k$, because each $\tau \in S_k$ determines a unique $\sigma$ by the formula $\sigma=\mu\tau^{-1}$. So the double sum above can be rewritten as \[
A(A(f)\otimes g)=k! \sum _{\mu \in S_{k+\ell}}(\operatorname{sgn}\mu)\mu (f\otimes g).
\]  The proof for (ii) is similar.
\end{proof}
\begin{prop}
    Let $V$ be a real vector space and $f,g,h$ alternating multilinear functions on $V$ of degrees $k,\ell,m$ respectively. Then \[
        (f\wedge g)\wedge h = f\wedge (g\wedge h).
    \] 
\end{prop}
\begin{proof}
    By the definition of the wedge product, we have 
    \begin{align*}
        (f\wedge g)\wedge h &= \frac{1}{(k+\ell)!m!}A((f\wedge g)\otimes h)\\
                            &=\frac{1}{(k_\ell)!m!}\frac{1}{m!\ell!}A(A(f\otimes g)\otimes h)\\
                            &=\frac{(k+\ell)!}{(k+\ell)!m!k!\ell!}A((f\otimes g)\otimes h)\\
                            &=\frac{1}{k!\ell!m!}A((f\otimes g)\otimes h).
    \end{align*}Similarly, since the tensor product is associative, we conclude that $(f\wedge g)\wedge h=f\wedge (g\wedge h)$.
\end{proof}
\begin{cor}
    Let $V$ be a real vector space and $f,g,h$ be alternating multilinear functions on $V$ of degrees $k,\ell,m$ respectively. Then \[
        f\wedge g\wedge h = \frac{1}{k!\ell!m!}A(f\otimes g \otimes h).
    \] More generally, if $f_i \in A_{d_i }(V)$, then \[
    f_1\wedge \cdots \wedge f_r = \frac{1}{(d_1)!\cdots (d_r)!}A(f_1\otimes \cdots \otimes f_r).
\] 
\end{cor}
\begin{prop}
Let $[b^i _j ]$ denote the matrix whose $(i,j)$-entry is $b^i _j $. If $\alpha ^1,\cdots ,\alpha ^k$ are linear functions on a vector space $V$ and $v_1,\cdots ,v_k\in V$, then \[
        (\alpha ^1\wedge \cdots \wedge \alpha ^k)(v_1,\cdots ,v_k)=\det [\alpha ^i (v_j )].
    \] 
\end{prop}
\begin{proof}
    \begin{align*}
        (\alpha ^1\wedge \cdots \wedge \alpha ^k)(v_1,\cdots ,v_k)&= A(\alpha ^1 \otimes \cdots \otimes \alpha ^k)(v_1,\cdots ,v_k)\\
                                                                  &=\sum _{\sigma \in S_k}(\operatorname{sgn}\sigma)\alpha ^1(v_{\sigma(1)})\cdots \alpha ^k(v_{\sigma(k)})\\
                                                                  &=\det [\alpha ^i (v_j )].\qedhere
    \end{align*}
\end{proof}
\subsection{A Basis for $k$-Covectors}
Let $e_1,\cdots ,e_n $ be a basis for a real vector space $V$, and let $\alpha ^1,\cdots ,\alpha ^n $ be the dual basis for $V^*$. We introduce the multi-index notation $I=(i_i ,\cdots ,i_k)$ and write $e_I$ for $(e_{i_1},\cdots ,e_{i_k})$ and $\alpha ^I$ for $\alpha ^{i_1}\wedge \cdots \wedge \alpha ^{i_k}$. A $k$-linear function $f$ on $V$ is completely determined by its values on all $k$-tuples $e_I$. If $f$ is alternating, then it is completely determined by its values on $e_I$ with $1\leq i_1< \cdots  < i_k \leq n$; that is, we can just talk about $e_I$ with $I$ in ascending order. Suppose $I,J$ are ascending, then since the wedge of covectors forms the determinant, we have \[
    \alpha ^I (e_J) = 
    \begin{cases}
        1 \quad & \text{if} \ I=J;\\
        0& \text{if} \ I\neq J.

    \end{cases}
\] 
\begin{prop}
    The alternating $k$-linear functions $\alpha ^I$ where $I=(i_1<\cdots < i_k)$ form a basis for the space $A_k(V)$ of alternating $k$-linear functions on $V$.
\end{prop}
\begin{proof}
    First we show linear independence. Suppose $\sum c_I \alpha ^I=0$, where $c_I\in \R$ and $I$ runs over ascending multi-indices of length $k$. Applying both sides to $e_J$, we have $0=\sum c_I\alpha ^I(e_J)=c_J$, since among all the ascending multi-indices $I$ of length $k$ there is only one equal to $J$. So the $\alpha ^I$ are LI.

    To show that the $\alpha ^I$ span $A_k(V)$, let $f\in A_k(V)$. We claim that $f=\sum f(e_I)\alpha ^I$, where $I$ runs over all ascending multi-indices of length $k$. Let $g=\sum f(e_I)\alpha ^I$. By $k$-linearity and the alternating property, if two covectors agree on all $e_J$, then they are equal. But \[
        g(e_J)=\sum f(e_I)\alpha ^I(e_J)=\sum f(e_I)\delta^I_J=f(e_J).
    \] So $f=g=\sum f(e_I)\alpha ^I$.
\end{proof}
\begin{cor}
    If a vector space $V$ is of dimension $n$, then $A_k(V)$ has dimension ${n \choose k}$.
\end{cor}
\begin{proof}
    We obtain an ascending multi-index $I=(i_1<\cdots < i_k)$ by choosing a subset of $k$ numbers from $1,\cdots ,n$: This can be done in ${n \choose k}$ ways.
\end{proof}
\begin{cor}
    If $k> \dim V$, then $A_k(V)=0$.
\end{cor}
\begin{proof}
    In $\alpha ^{i_1}\wedge \cdots \wedge \alpha ^{i_k}$, at least two of the factors must be the same, say $\alpha $. Because $\alpha $ is a $1$-covector, $\alpha \wedge \alpha =0$, so $\alpha ^{i_1}\wedge \cdots \wedge \alpha ^{i_k}=0$.
\end{proof}

\section{Differential Forms on $\R^n $}
\subsection{Differential $1$-Forms and the Differential of a Function}
The \textbf{cotangent space} to $\R^n $ at $p$, denoted by $T_p^*(\R^n )$ or $T_p^* \R^n $, is defined as the dual space of $T_p(\R^n )$. So elements are covectors or linear functionals on $T_p(\R^n )$. Similarly, a \textbf{covector field} or \textbf{differential} $\mathbf 1$\textbf{-form} ($1$-form for short) $\omega$ on an open subset $U$ of $\R^n $ is a function that assigns to each point $U$ a covector $\omega_p \in T_p^*(\R^n )$.

From any $C^{\infty}$ function $f\colon U \to \R$, we can construct a $1$-form $df$, called the \textbf{differential} of $f$ as follows. For $p\in U$ and $X_p \in T_p U$, define $(df)_p(X_p)=X_pf.$ Let $x^1,\cdots ,x^n $ be the standard coordinates on $\R^n $. Recall that $\{\partial /\partial x^1 |_p,\cdots ,\partial /\partial x^n |_p\} $ is a basis for the tangent space $T_p(\R^n )$.
\begin{prop}
If $x^1,\cdots ,x^n $ are the standard coordinates on $\R^n $, then at each point $p \in \R^n $, $\{(dx^1)_p,\cdots ,(dx^n )_p\} $ is the basis for the cotangent space $T_p^*(\R^n )$ dual to the basis $\{\partial /\partial x^1|_p,\cdots ,\partial /\partial x^n |_p\}$.
\end{prop}
\begin{proof}
    \[
        (dx^i )_p\left( \left. \frac{\partial }{\partial x^j} \right|_p  \right) = \left. \frac{\partial }{\partial x^j} \right| _p x^i =\delta_j ^i .\qedhere
    \] 
\end{proof}
If $\omega$ is a $1$-form on an open $U \subseteq \R^n $, then at each point $p \in U$ we have a linear combination $\omega_p=\sum a_i (p)(dx^i )_p$ for some $a_i (p)\in \R$, since the $(dx^i )_p$ form a basis for the cotangent space $T_p^*(\R^n )$. As $p$ varies over $U$, the coefficients $a_i $ become functions on $U$, and we write $\omega = \sum a_i dx^i $. The covector field $\omega$ is said to be $C^{\infty}$ on $U$ if the coefficient functions $a_i $ are all $C^{\infty}$ on $U$. If $x,y,$ and $z$ are coordinates on $\R^3$, then $dx,dy,$ and $dz$ are $1$-forms on $\R^3$.
\begin{prop}
    If $f \colon U \to \R$ is a $C^{\infty}$ function on an open $U\subseteq \R^n $, then $df = \sum \frac{\partial f}{\partial x^i }dx^i $.
\end{prop}
\begin{proof}
    At each point $p \in U$, we have $(df)_p=\sum a_i (p)(dx^i )_p$ for some constants $a_i (p)$ depending on $p$. So $df=\sum a_i dx^i $ for some functions $a_i $ on $U$. To evaluate $a_j $, apply both sides of the first expression to the coordinate vector field $\partial /\partial x^j$ ; that is, \[
        df\left( \frac{\partial f}{\partial x^j} \right) =\sum_i a_i dx^i \left( \frac{\partial }{\partial x^j} \right) =\sum_i a_i \delta_j ^i =a_j .
    \] On the other hand, $df \left( \frac{\partial }{\partial x^j} \right) =\frac{\partial f}{\partial x^i }$ by the definition of the differential.
\end{proof}
So if $f$ is $C^{\infty}$, $df$ is also $C^{\infty}$.
\begin{example}
    Every tangent vector $X_p \in T_p(\R^n )$is a linear combination of the standard basis vectors: $X_p=\sum _i b^i (X_p)\left. \frac{\partial }{\partial x^i } \right| _p$.
\end{example}
We have seen that at each $p \in \R^n $, we have $b^i (X_p)=(dx^i )_p (X_p)$. So the coefficient $b^i $ of a vector WRT the standard basis $\frac{\partial }{\partial x^1},\cdots ,\frac{\partial }{\partial x^n}$ is none other than the dual form $dx^i $ on $\R^n .$
\subsection{Differential $k$-Forms}

In general, a \textbf{differential form} $\omega$ \textbf{of degree} $\mathbf k$ or a $\mathbf k$\textbf{-form} on an open $U\subseteq \R^n $ is a function that assigns to each $p \in U$ an alternating $k$-linear function on $T_p(\R^n )$, i.e., $\omega_p \in A_k(T_p\R^n )$. Since $A_1(T_p\R^n )=T_p^*(\R^n )$, this generalizes the idea of a $1$-form.
