\subsection{Alternating $k$-Linear Functions}
\begin{prob}
    If $f$ is a trilinear function on a vector space $V$, what is $(Af)(v_1,v_2,v_3)$, where $v_1,v_2,v_3\in V$?
\end{prob}
\begin{solution}
    Recall the six elements of $S_3$ are $1,(12),(23),(13),(123),(132)$. So
    \begin{align*}
        (Af)(v_1,v_2,v_3)=&\sum _{\sigma \in S_3}(\operatorname{sgn}\sigma)\sigma f\\
                         =&f(v_1,v_2,v_3)-\\
                                                                                   &f(v_2,v_1,v_3)-\\
                                                                                   &f(v_1,v_3,v_2)-\\
                                                                                   &f(v_3,v_2,v_1)+\\
                                                                                    &f(v_2,v_3,v_1)+\\
                                                                                    &f(v_3,v_1,v_2).
    \end{align*}
\end{solution}
\begin{prob}
Show that if $f,g,h$ are multilinear functions on $V$, then $
    (f\otimes g)\otimes h=f\otimes (g\otimes h).$
\end{prob}
\begin{solution}
    Let $f,g,h$ be $j,k,$ and $\ell$-linear functions respectively. Then 
    \begin{align*}
        (f\otimes g)\otimes h&=\big(f(v_1,\cdots ,v_j )g(v_{j+1},\cdots ,v_{j+k})\big)\otimes h\\
                             &=f(v_1,\cdots ,v_j )g(v_{j+1},\cdots ,v_{j+k})h(v_{j+k+1},\cdots ,v_{j+k+\ell})\\
                             &=f(v_1,\cdots ,v_j )(g\otimes h)\\
                             &=f\otimes (g\otimes h).
    \end{align*}
\end{solution}
\begin{prob}
    For $f,g\in A_2(V)$, write out the definition of $f\wedge g$ using $(2,2)$-shuffles.    
    %gotta find all the (2,2)-shuffles in S^4 scree
\end{prob}
\begin{solution}
    Note that the images of $\{1,2,3,4\} $ under the  $(2,2)$-shuffles of $S_4$ are $1234,1324,1423,2314,2413,3412$. So the $(2,2)$-shuffles are $1,(23),(243),(123),(1243)$ and $(13)(24)$.
    \begin{align*}
        f\wedge g=&\sum _{(2,2)\text{-shuffles} \ \sigma}(\operatorname{sgn}\sigma)f(v_{\sigma(1)},v_{\sigma(2)})g(v_{\sigma(3)},v_{\sigma(4)})\\
                 =&f(v_1,v_2)g(v_3,v_4)-\\
                 &f(v_1,v_3)g(v_2,v_4)+\\
                 &f(v_1,v_4)g(v_2,v_3)+\\
                 &f(v_2,v_3)g(v_1,v_4)-\\
                 &f(v_2,v_4)g(v_1,v_3)+\\
                 &f(v_3,v_4)g(v_1,v_2).
    \end{align*}
\end{solution}

