\subsection{Smooth Functions on Euclidian Space}
\begin{prob}
Find a function $h \colon \R \to \R$ that is $C^2$ but not $C^3$ at $x=0$.    
\end{prob}
\begin{solution}
    Take $h(x)=x^{5/2}$. Then $h''(x)=\frac{15}{4}\sqrt{x} $, but $h'''(x)=\frac{15}{8}x^{-1/2}$ for $x\neq 0$ and undefined at zero. Therefore $h$ is $C^2$ but not $C^3$.
\end{solution}
\begin{prob}
    Define $f(x)$ on $\R$ by \[
        f(x)=
        \begin{cases}
            e^{-1/x} \quad & \text{for} \ x>0;\\
            0 & \text{for} \ x \leq 0.
        \end{cases}
    \] 
    \begin{enumerate}[label=(\alph*)]
        \item Show by induction that for $x>0$ and $k\geq 0$, the $k$th derivative $f^{(k)}(x)$ is of the form $p_{2k}(1/x)e^{-1/x}$ for some polynomial $p_{2k}(y)$ of degree $2k$ in $y$.
        \item Prove that $f$ is $C^{\infty}$ on $\R$ and that $f^{(k)}(0)=0$ for all $k\geq 0$.
    \end{enumerate}
\end{prob}
\begin{solution}
    \,
    \begin{enumerate}[label=(\alph*)]
        \item We use induction on $k$. For $k=1$, we have $f'(x)=\frac{1}{x^2}e^{-1/x}$. In this case, $p_2(y)=y^2$, and so $p_2(1/x)=\frac{1}{x^2}$. Now assume $f^{(k)}(x)$ is of the form $p_{2k}(1/x)e^{-1/x}$ for some polynomial $p_{2k}(y)$ of degree $2k$ in $y.$ Then by the product rule,\[
                f^{(k+1)}(x)=p_{2k}'(1/x)e^{-1/x}+\frac{1}{x^2}e^{-1/x}p_{2k}(1/x)=e^{-1/x}\left( p'_{2k}(1/x) +\frac{1}{x^2}p_{2k}(1/x)\right) .
            \] For the sum in the right expression, $p_{2k}'(1/x)+\frac{1}{x^2}p_{2k}(1/x)$ has degree $2(k+1)$: to see this, note that $p_{2k}'$ has degree $2k-1$, so we can forget about it. If $p_{2k}(1/x)=a_{2k}\left(\frac{1}{x}\right)^{2k}+b_{2k-1}\left( \frac{1}{x} \right) ^{2k-1}+\cdots $ for constants $a_{2k},b_{2k},\cdots $, we have $\frac{1}{x^2}p_{2k}=a_{2k}\frac{1}{x^{2k+2}}+b_{2k}\frac{1}{x^{2k+1}}+\cdots =a_{2k}\frac{1}{x^{2(k+1)}}+\cdots $. So this polynomial has degree $2(k+1)$. Therefore $f^{(k)}(x)$ is of the form $p_{2(k+1)}(1/x)e^{-1/x}$ for some polynomial $p_{2(k+1)}$ of degree $2(k+1)$, and we are done.
        \item Our strategy is to show that $f^{(k)}(x)=0$ for $x<0$, $f^{(k)}=0$, and $\lim_{x\to 0}f^{(k)}(x)=0$ for $x>0$. These conditions ensure that $f$ is smooth and the $k$th derivative vanishes at zero. First, note that $f^{(k)}(x)=0$ for $x<0$ by definition. To show $\lim_{x\to 0}f^{(k)}(x)=0$ for $x>0$, recall that $f^{(k)}(x)=p_{2k}(1/x)e^{-\frac{1}{x}}$. Using the genius substitution $u=\frac{1}{x}$, we can rewrite this limit as $\lim_{u\to \infty}\frac{p_{2k}}{e^u}$. From here, apply L'\,H\^opital's rule $2k$ times to get our desired result.

            Finally, we show $f^{(k)}(0)=0$. We do this by induction on $k$. The base case is true by definition. Assume $f^{(k)}(0)=0$. Then $f^{(k+1)}(0)=\lim_{h\to 0}\frac{f^{(k)}(h)-f^{(k)}(0)}{h-0}=\lim_{h\to 0}\frac{f^{(k)}(h)}{h}=\lim_{h\to 0}\frac{p_{2k}(1/h)e^{-1/h}}{h}$. Once again, make the substitution $u=1/h$ to get $f^{(k+1)}(0)=\lim_{u\to \infty}\frac{u p_{2k}(u)}{e^u}=0$ by $2k+1$ applications of L'\,H\^opital's rule.\qedhere
    \end{enumerate}
\end{solution}
\begin{prob}
Let $U\subseteq \R^n $ and $V\subseteq \R^n $ be open subsets. A $C^{\infty}$ map $F \colon U \to V$ is called a \textbf{diffeomorphism} if it is bijective and has a $C^{\infty}$ inverse $F^{-1} \colon V \to U$.    
\begin{enumerate}[label=(\alph*)]
    \item Show that the function $f\colon (-\pi /2, \pi /2) \to \R, \ f(x)=\tan x$ is a diffeomorphism.
    \item Find a linear function $h \colon (a,b) \to (-1,1)$, thus proving that any two finite open intervals are diffeomorphic.
\end{enumerate}
Then the composition $f \circ h \colon (a,b) \to \R$ is then a diffeomorphism of an open interval to $\R$.
\end{prob}
\begin{solution}\,
\begin{enumerate}[label=(\alph*)]
    \item We want to show that $\tan x$ is a smooth bijection and has a smooth inverse. Let $\tan(a)=\tan(b)$, then these numbers are associated to the same angle in $(-\pi /2, \pi /2),$ similarly, every real number is mapped onto by an angle in $(- \pi/2, \pi /2)$. For smoothness, note that $\tan'(x)=\sec^2(x)$, $\tan ''(x)=2\sec ^2 (x)\tan(x)$. From here you can see that the remaining derivatives are all products of $\sec$ and $\tan$, which are both defined on $(-\pi /2, \pi /2)$ (since $\cos$ never hits zero on this interval). So $\tan x$ is smooth.

        The $C^{\infty}$ inverse has to be $\arctan \colon (-\pi /2, \pi /2)\to \R$, there are no better candidates. We have $\arctan \circ \tan (x)= \operatorname{id}_{\R}$ by definition, so $\arctan$ is an inverse: to see smoothness, note that $\arctan'(x)=\frac{1}{1+x^2}$, $\arctan''(x)=-\frac{2x}{(1+x^2)^2}$, and so on. These functions are all continuous on $(-\pi /2, \pi /2)$, and so $\arctan $ is a smooth inverse for $\tan$. Therefore $\tan \colon (- \pi /2, \pi /2) \to \R$ is a diffeomorphism.
    \item Consider the function with its graph being a line segment joining $(a,1)$ to $(b,-1)$.
\end{enumerate}
\end{solution}



