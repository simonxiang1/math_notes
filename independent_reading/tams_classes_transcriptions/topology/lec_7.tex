\section{September 16, 2021} 
Welcome to algebraic geometry! Today we start talking about algebraic geometry.

\subsection{Motivating schemes}
What's the idea behind schemes? We want to take the theory of rings and embed them in a contravariant fashion, identifying them with something called affine schemes by $\mathrm{Spec}$ ( $\mathsf{Rings} ^{\mathrm{op}}\underset{\mathrm{Spec}}{\xrightarrow{\sim}} \mathsf{Aff}  $). It turns out affine schemes are equivalent to rings as categories, so sometimes people define affine schemes as $\mathsf{Rings} ^{\mathrm{op}}, $ which is kind of cheating, since we want affine schemes to be some kind of geometric object. We think of \textbf{affine schemes} as a subcategory of a bigger class of geometric objects, which we call \textbf{schemes} $\mathsf{Sch} $. One can make the comparison that affine schemes are to schemes as Euclidian spaces are to manifolds. We build schemes by gluing together affine schemes, but unlike the theory of smooth manifolds where Euclidian spaces are all isomorphic to $\R^n $, the theory of affine schemes is extremely rich (equivalent to the theory of rings).

What do we mean by a ``geometric object''? Let LRS denote the notion of a \textbf{locally ringed space}, which is a set plus a topology, and a notion of a function. (We postpone this idea for a little while.) Schemes are going to be sets, with a topology and a function, and so also locally ringed spaces. Given a LRS, we can attach its ring of (global) funtions which defines a functor $\mathrm{LRS}\xrightarrow{\text{global functions} } \mathsf{Rings} ^{\mathrm{op}} $. Then the functor $\mathrm{Spec}$ is going to be the right adjoint to the functor of global functions.\[
\begin{tikzcd}
    \mathsf{Rings}^{\mathrm{op}} \arrow[r, "\sim"] \arrow[r, "\mathrm{Spec}"'] \arrow[rrr, "\mathrm{Spec}"', bend right, shift right] & \mathsf{Aff} \arrow[r, "\subset", phantom] & \mathsf{Sch} \arrow[r, "\subset", phantom] & \mathrm{LRS} \arrow[lll, "\text{global functions}"', bend right, shift right]
\end{tikzcd}
\] 

\subsection{The spectrum of a ring}
Our goal for today is to define $\mathrm{Spec}(R)$ as a set. What should points of $\mathrm{Spec}(R)$ be? Let's start with some classical algebraic geometry. In an ideal world, a prerequisite for this class would be a course in classical algebraic geometry, the theory of varieties (Harris). Harris' book is extremely classical and heavy with examples (the foil to this class). Classically, what people mean by ``affine variety'' is a subset of $n$-dimensional affine space over an algebraically closed field of characteristic 0 (say $\C$). Look at a subset of $\C^n \supset V$ which is cut out by polynomial equations, look at their common zero locus, and that is a variety. Consider the quotient of a polynomial ring $\C[x_1 ,\cdots ,x_n ] / I$ where $I$ is the ideal of functions that vanish on $V$- this is the setting of classical algebraic geometry.

In classical algebraic geometry, there is a natural idea of what points should do. There is no ambiguity about what a point is, since they're subsets of $\C^n $ that solve a given set of equations. The big idea is Hilbert's \textbf{Nullstellensatz} (the zero space theorem in German), which says that points are given by maximal ideals. If $\mathfrak m \subseteq R$ is maximal, then $R /M$ is a field. So $\mathfrak m \subseteq R$ is equivalent to a surjective homomorphism $R \twoheadrightarrow k$. Why is this relevant to a point? Points give maps to fields, where we evaluate functions at points (the evaluation homomorphism) giving a map $R \twoheadrightarrow \C$ in this case. We need to know the following: are there other weird maximal ideals we forgot to consider?

\begin{namedthm}{Hilbert's Nullstellensatz} 
    If $k$ is a field, maximal ideals $\mathfrak m $ in $R=k[x_1, \cdots ,x_n ]$ have residue field $R / \mathfrak m$ a finite extension of $k$.
\end{namedthm}

