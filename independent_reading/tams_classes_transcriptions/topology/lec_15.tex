\section{October 5, 2021} 
Our first in person lecture! What are sheaves? They are 
\begin{itemize}
\setlength\itemsep{-.2em}
    \item substitutes for function spaces,
    \item a geometric inerpretation for modules.
\end{itemize}
For $R \mapsto \mathrm{Spec}R \ni \{\lambda \colon R \to k\} $ equal to  eigenvalues for (commutry?) operation on $V / k$. $r \in R \mapsto  r \cdot v = \lambda(r) v$, so eigenspaces are sheaves on $\mathrm{Spec}R$.

To define sheaves, we need $X$ a topological space, $\mathcal{C} $ a category (usually $\mathsf{Set} , \mathsf{Ab} , $ or $\mathsf{Ring} $). A $\mathcal{C} $-valued sheaf on $X$ is a \emph{local} expression on $X$ generalizing propereties of functions. For $U \subseteq X$ open, this leads to $F(U) \in  \mathcal{C} $ ``functions on $U$''.

\subsection{Presheaves}


\begin{definition}[]
    For $X$ a topological space, let $\mathrm{Top}(X)= \mathrm{Open}(X)$ be the poset of open sets in $X$ under inclusion.
\end{definition}
\begin{definition}[]
    A ($\mathcal{C} $-valued) \textbf{presheaf}  is a functor $F \colon \mathrm{Open}(X)^{\mathrm{op}} \to \mathcal{C} $, i.e., $U \subseteq X \mapsto F(U) \in \mathcal{C} $, and $V \subseteq U \mapsto  \mathrm{res}_U^V \colon F(U) \to F(V)$, where $\mathrm{res}$ is the restriction.
\end{definition}
For $F,G$, $\mathcal{C} $-valued presheaves, $\varphi \colon F \to G$ means for $V \subseteq U$, the following diagram commutes:
\[
    \begin{tikzcd}
\varphi_U \colon F(U) \arrow[r] \arrow[d, "\mathrm{res}"] & G(U) \arrow[d, "\mathrm{res}"] \\
\varphi_V \colon F(V) \arrow[r]                           & G(V)                          
\end{tikzcd}
\] 
In other words, a natural transformation. Denote the category of presheaves by $\mathcal{C} _X ^{\mathrm{pre}}$.
\begin{example}
    We have the \emph{constant presheaf}, where for $A \in \mathcal{C} $, $A(U)=A$, and $\mathrm{res}=\id$. We also have the \emph{skyscraper} presheaf, where for $x \in X$, $\mathcal{C} =\mathsf{Ab} $,\[
        \delta _{A,x}(U) = 
        \begin{cases}
            A & \text{for} \ x \in U,\\
            0 & \text{for} \ x \notin U.
        \end{cases}
    \] 
\end{example}
\begin{example}
    The important examples of \emph{functions} of any kind. For $X= \R^n $, $\mathcal{C} =\mathsf{Set} , \mathsf{Ab}, \mathsf{Ring} $, then $C(U)$ is the space of continuous $\R$-valued functions, $C^{\infty}$ means smooth, $C^{k}$ means $k$-differentiable. The complex analytic functions on $\mathrm{C}\mathrm{P} ^1$ are all constant-- in general functions don't extend. So instead of functions on the whole space, we keep track of functions on smaller spaces. 
\end{example}
\begin{example}
    Fix a topological space $Y$ (for example $\R$), and consider $\mathrm{Map}_{\mathrm{cont}}(X,Y)$, the continuous maps $X \to Y$. This forms a presheaf by $U \mapsto \mathrm{Map}_{\mathrm{cont}(U,Y)}$.
\end{example}
Let's formalize the global and local aspects of a presheaf.
\begin{figure}[H]
    \centering
    \begin{tabular}{l|l} 
        \textbf{Global}  & \textbf{Local}  \\ \hline
        Global sections of $F$, $\Gamma (F):= F(X)$. & Stalks for $x \in X$ \\
        Here $F(U):=$ sections on $U$. & Germs 
    \end{tabular}
    \label{global_local} 
    \caption{Global and local aspects of presheaves.} 
\end{figure}
\begin{definition}[]
    Define the \textbf{stalk of} $x$ by $F_x := \varinjlim _{x \in U} F(U)$. We can think of this by the value of $F$ on the infinite intersection of opens $F\left( \bigcap U \right)  $, but infinite intersection of opens don't make sense, hence our notion. 
\end{definition} For $\mathcal{C} =\mathsf{Ab} , \mathsf{Set} $, we identify this with the set $\{f_U \in F(U), x\in U\} / f_U \sim f_V $ if they have the same restriction to $W \subseteq  U \cap V$. This equivalence relation is the same as modding by germs of functions. Stalks and germs are talking about the same thing, we usually talk about the stalk of a sheaf and the germ of a function. {\color{red}todo:something about presheavse as taylor series with nonzero radius of convergence on $C^{\infty}$?} 

\subsection{Sheaves}

Now we want to introduce a notion of gluing/descent/patching to make sheaves. Let $U = \bigcup_{i \in  I} U_i $ be a covering of $U$, then functions on $U$ correspond to functions on $U _i $ which match on overlaps. For $f $ on $U$, we have  $f _n $ on $U$, where $\mathrm{res}_U ^{U_i }= \left. f \right| _{U_i }$. If $U_{ij}=U_i  \cap U_j $, then $\left. \left. f \right| _{U_i }   \right| _{U_i  \cap U_j}= \left. \left. f \right| _{U_j } \right| _{U_i  \cap U_j }$.

\begin{definition}[First definition of a sheaf]
For $F \in \mathcal{C} _X ^{\mathrm{pre}},$ we say $F$ is a \textbf{sheaf} if 
\begin{itemize}
\setlength\itemsep{-.2em}
    \item \emph{identity}: For a cover $U = \bigcup_{i \in  I} U_i $, $F(U)\xrightarrow{\prod \mathrm{res}_{U}^{U_i }}\prod F(U_i ) $ is a monomorphism (or injective).
    \item \emph{gluing}: Given sections $\{f _i  \in F(U_i )\} $ agreeing on overlaps, $\left. f_i  \right| _{U_{ij}} =\left. f_j  \right| _{U_{ij}}$ implies there exists an $f \in F(U)$ with $f=\left. f \right| _{U_i }$.
\end{itemize}
\end{definition}

We give a little bit more formal definition for those who are into that sort of thing.

\begin{definition}[]
    We say $F$ is a \textbf{sheaf} if for $U = \bigcup_{i \in  I} U_i $, 
    \[
    F(U) \to \prod_{i \in I} F(U_i )\rightrightarrows \prod _{i,j \in I}F(U _{ij})
    \] $\{f _i \} _{i \in I}, \ \left. f_i  \right| _{U_{ij}}=\left. f_j \right| _{U_{ij}} $ is an \emph{equalizer}.
\end{definition}
This is a lot more compact than the previous definition.

\begin{example}
    We have $F( U \amalg V) \cong F(U) \times F(V)$ for sheaves. 
    \begin{itemize}
    \setlength\itemsep{-.2em}
        \item Functions of any kind form sheaves.
        \item Maps to a fixed target form a sheaf.
        \item Let $\mathcal{C} =\mathsf{Set} $, $X=$ a topological space (say $S^1 $), and $F \in $ sheaves of sets on $X$. Let $F(U)$ be the sectionsof $\pi \colon Z \to X$ over $U$. For $Z \xrightarrow{\pi}  X$ (eg the universal cover $\R \to S^1 $), we look at the \textbf{sheaf of sections} of $\pi \colon F(U)=$ sections of $\pi ^{-1} (U) \to U$, where a section is a map $s \colon U \to \pi ^{-1}(U)$ such that $\pi s= \id$. These form a presheaf, and we need to check that these form a sheaf. This is very much like a map since we can glue in the same way, in fact sections of $\pi$ are equivalent to maps $X \to Y$. This is the notion of a \emph{graph} of a function, a section of the projection $X \times Y\to X$. 

            Here $F(S^1 )= \O$, but $F(U) \simeq  \Z$ for $U$ an interval. There are no global sections ($\Gamma(F)=\O$), but there are tons of local sections. This is why sheaves are useful. 
        \item Recall the constant presheaf, $A = \Z \in \mathsf{Ab} $, $\Z(U)= \Z$. Something is off-- the constant presheaf is not actually a sheaf. Given a disjoint union, if the constant presheaf were a sheaf by the sheaf axiom we would have $U \amalg V \mapsto A\times A$ for $U \mapsto A, V\mapsto A$, however for $U,V$ disconnected this would fail. 

            Define the \textbf{constant sheaf} $A^{\mathrm{sheaf}} \colon U \mapsto A $ if $U$ is connected. Then $U \mapsto \prod A$ where each $A$ is a component of $U$. Think of $A$ as a set, and $\mathrm{Map}_{\mathrm{cont}}(U,A)$. We'll see next time there is a way to correct every presheaf.
    \end{itemize}
        Our example $\R \to S^1 $ earlier is an example of a \textbf{locally constant} sheaf, which looks like $\Z$ on small enough open sets. The fact that the circle has an interesting topology is captured by the fact that locally it's doing nothing but has some sort of global structure.
\end{example}
