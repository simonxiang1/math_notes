\section{October 12, 2021} 
Our plan for today is to try to finish off sheafs, then next time we'll get back to schemes. For $\mathcal{C} =\mathsf{Set} ,\mathsf{Ab} ,\mathsf{Ring} , . . .$, and $X$ a topological space, we consider $\mathcal{C} _X$ the $\mathcal{C} $-valued sheaves on $X$. Then $x \in X \leadsto F_x \in \mathcal{C} $. So presheaves give $x \mapsto F_x$ a $\mathcal{C} $-valued function on $X$. A sheaf is a continuous/locally constant $\mathcal{C} $-valued function, so a sheaf is determined by its stalks. We showed last time that for a sheaf $F$, $F(U) \hookrightarrow  \prod _{x \in U}F_x$.

For $F$ a presheaf, we attached the following thing called $F ^{\mathrm{sh}}(U)$, called the \textbf{sheafification} of $F$. Define \[
    F ^{\mathrm{sh}}(U) := \left\{ (s_x) \in \prod F_x \mid \forall x \in U, \ \exists \text{open nbd} \ x \in V \subseteq U \ \text{and} \ s_v \in F(V) \ni \forall y \in V,s_y= \text{germ of } s_V \ \text{at} \ y\right\} .
\] This is some kind of continuity condition.
\begin{theorem}
    The map $F \mapsto  F^{\mathrm{sh}}$ defines a functor $\mathcal{C} _X ^{\mathrm{pre}}\to \mathcal{C} _X$ from presheaves to sheaves, which is left adjoint to the inclusion $\mathcal{C} _X \hookrightarrow  \mathcal{C} _X ^{\mathrm{pre}}$.
\end{theorem}
\begin{proof}
If $\mathcal{G} $ is a sheaf rather than a presheaf, then $G \xrightarrow{\sim} \mathcal{G} ^{\mathrm{sh}}$ canonically. There is a natural map $F(U) \to F^{\mathrm{sh}}(U)$ by attaching honest sections, which is a map $F \to F ^{\mathrm{sh}}$. Then the adjunction statement makes the following diagram commute: \[
\begin{tikzcd}
F \arrow[rr, "\varphi"] \arrow[rd] &                                                  & G \\
                                   & F^{\mathrm{sh}} \arrow[ru, "\exists !"', dotted] &  
\end{tikzcd}
\] To see this, consider the following diagram: \[
\begin{tikzcd}
F(U) \arrow[rr, "\varphi"] \arrow[rd] \arrow[dd] &                                                     & G(U) \arrow[rd, "s"] \arrow[dd] &                               \\
                                                 & F^{\mathrm{sh}}(U) \arrow[ld] \arrow[ru] \arrow[rr, crossing over] &                                 & G^{\mathrm{sh}}(U) \arrow[ld] \\
\prod_{x\in U} F_x \arrow[rr]                    &                                                     & \prod_{x\in U}G_x               &                              
\end{tikzcd}
\] 
\end{proof}
\begin{example}
    We have $($constant presheaf$)^{ \mathrm{sh}}=$constant sheaf. For $A$ a set, $A^{\mathrm{pre}}(U)=A$ for all $U$. The constant sheaf $A(U)$ is equal to locally constant $A$-valued functions.
\end{example}
The map $F \mapsto \{F_x\} _{x \in X}$ is a functor from sheaves of sets to products of sets over $x \in X$. The claim is that this functor is \emph{faithful}, or injective over $\Hom$'s. So a map of sheaves $\varphi  \colon F \to G$ is determined by what it does on stalks $\{\varphi _x \colon F_x \to G_x\} $. This is completely false for presheaves, going to stalks loses a lot of information. \[
\begin{tikzcd}
F(U) \arrow[r, "\varphi", shift left] \arrow[r, "\psi"', shift right] \arrow[d, hook] & G(U) \arrow[d, hook] \\
\prod F_x \arrow[r, "\varphi_x", shift left] \arrow[r, "\psi"', shift right]          & \prod G_x           
\end{tikzcd}
\] Passing to stalks reflects monomorphisms and epimorphisms, i.e., $\varphi  \colon F \to G$ is monic/epic if $\varphi _x \colon  F_x\to G_x$ is for all $x$. In fact, $F \xrightarrow{\varphi } G$ is monic/epic/an isomorphism iff the $\varphi _x$ for all $x \in X$. So there is a tight categorical connection between properties of sheaves and on their stalks. Things get interesting in the middle-- what do we mean by that? We have maps on sheaves $F \to G$, then maps on stalks $F_x \to G_x$, and in the middle maps on opens $F(U) \to G(U)$. {\color{red}todo:?} 

\begin{example}
    Consider $\mathcal{C} =\mathsf{Ab} $, and write down some short exact sequences $0 \to A \xrightarrow{\varphi }  B \xrightarrow{\psi}  C \to 0$ where $\varphi $ is injective, $\psi$ is surjective, and $C = B / A$. We will try to define this in the world of sheaves: \[
    \begin{tikzcd}
\ker\varphi \arrow[r] \arrow[d] & B \arrow[d, "\varphi"] & A \arrow[d, "\varphi"] \arrow[r] & 0 \arrow[d]                 \\
0 \arrow[r]                     & C                      & B \arrow[r]                      & \operatorname{coker}\varphi
\end{tikzcd}
    \] For $O \in \mathsf{Ab} _X$, $O$ a sheaf, $O(U)=O$ for all $U$. Consider sheaves $\mathsf{Ab}_{\C}$ on $\C$ or $\mathsf{Ab} _{\C^*}$ on $\C^*$. Here $\Z \to \mathcal{O} $, where $\mathcal{O} $ is the sheaf of holomorphic/analytic (or smooth) functions. Constant functions are nice, so we have an inclusion $0 \to \Z \to \mathcal{O} $. What is the cokernel? Given $f$ a function, attach to it $e ^{ 2 \pi i f}$ which is a nowhere vanishing function, where for $f$ integer valued $e^{2\pi i f}=1$. Then \[
    0 \longrightarrow \Z \longrightarrow \mathcal{O} \xrightarrow{e ^{2\pi i (-)}} ? \longrightarrow 0
\]Here ? is the abelian group of nonvanishing functions. Define $G(U) - \{ e^{2 \pi  i f }, f \in \mathcal{O} (U) = \im \left( e^{2\pi i -} \right) \} $. The claim is that this is a presheaf but not a sheaf-- why? For $\mathsf{Ab} _{\C^*}$, consider the function $g=z$ nowhere vanishing but is \emph{not} of the form $e^{2\pi i f}$ for $f \in \mathcal{O} (\C^*)$, which just means we can't find a branch of the logarithm for it. Every nowhere vanishing function is locally in the image of $e^{2\pi i -}$. Analogously, \[
0 \longrightarrow \Z_x \longrightarrow C^{\infty}_X \xrightarrow{e ^{2\pi ix}} \{\text{smooth} \ S^1\text{-valued functions} \} _X \longrightarrow 0
\] where $X=S^1 $,  which reflects the  interesting topology of the circle.
Now consider \[
    \pm 1 \longrightarrow \{ \text{nowhere vanishing functions} \} \xrightarrow{(-)^2} \{\text{nowhere vanishing square functions} \} ,
\] which wants to look like this: \[
\Z / n \hookrightarrow \mathcal{O} ^*  \twoheadrightarrow \mathcal{O} ^*.
\] On $\C^*$, $\sqrt{z} $ does not exist, since the sheafification $z \in \mathcal{O} ^*$ is not in the image of $(-)^2$. Topologically, this says we have a double cover of the circle.
Now for $X=S^1 $, consider \[
0 \to \R \to C^{\infty}\xrightarrow{d} \Omega _{\mathrm{cl}}^1 \to 0
\] a short exact sequence of sheaves, where $\Omega_{\mathrm{cl}}^1$ is the set of closed 1-forms. The derivative $df =0$ implies $f$ is locally constant. This is precisely the Poincar\'e lemma, which says that locally for any $\omega$ with $d\omega =0$, $\omega=df$ for some $f$. In other words, $\Omega_{\mathrm{cl}}^1= \Omega_{\mathrm{ex}}^1$ on small enough opens, so there is a presheaf $\Omega_{\mathrm{ex}}^1 = \im (d)$ since exactness is not a local condition. The discrepancy between the sheaf $\Omega_{\mathrm{cl}}$ and the presheaf $\Omega_{\mathrm{ex}}$ is de Rham cohomology. 
\end{example}
