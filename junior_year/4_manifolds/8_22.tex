\section{Classification problems in differential topology}
The central problem of differential topology is the classification up to diffeomorphism of smooth manifolds. An ideal solution might have the following aspects:
\begin{itemize}
\setlength\itemsep{-.2em}
    \item \textbf{Models.} A collection $\{X_i \} _{i \in  I}$ of smooth connected manifolds of a particular dimension---maybe with some other traits like simple connectivity or compactness, representing all diffeomorphism types without redundancy.
    \item \textbf{Which manifold do I have in my hand?} When given some manifold $M$, we can decide which $X_i $ it's diffeomorphic to, perhaps by computing invariants. If  $M$ is described by a finite set of data we can ask for an algorithm for this determination.
    \item \textbf{Are these two manifolds diffeomorphic?} We can compute invariants to decide this, or use an algorithm if the manifolds are presented in a finite fashion.
    \item \textbf{Families.} We want to understand smooth families, say $\{M_b\} _{ b \in B}$ where $B$ is a family. In other words we want to think about smooth fiber bundles $M_b \hookrightarrow \mathcal{M} \to B$, including understanding the homotopy type of $\mathrm{Diff}(M)$.
\end{itemize}
\subsection{Dimension 1}
We can say a smooth compact connected 1-manifold $M$ is diffeomorphic to $S^1 $. There is something interesting to ask about smooth families of 1-dimensional manifolds. There is a deformation retraction from $\mathrm{Diff}(S^1 )$ to its subgroup $S^1 $, which you can prove. So we have a homotopy equivalence $\operatorname{Diff}S^1  \simeq S^1 $, and we can use this to show that families of copies of the circle are really interchangeable with unit circle bundles of rank 2 vector bundles. Using this (module an orientability issue), we are looking at $H^2(\mathrm{base})$.

\subsection{Dimensions 2 and 3}
In dimension 2, we have our conditions being compact, oriented, and connected surfaces for simplicity.
\begin{enumerate}[label=(\alph*)]
\setlength\itemsep{-.2em}
    \item We have standard surfaces $\Sigma_g$, the surfaces of genus $g$.
    \item We can use the Euler characteristic $\chi(\Sigma) = 2 - 2g$ for $\Sigma_g$. 
    \item The algorithmic aspect is also satisfied by the Euler characteristic.
    \item $\mathrm{Diff}^+ (\Sigma)$ is the group of orientation preserving diffeomorphism. The mapping class group $\pi_0\operatorname{Diff}^+\Sigma$ (which is complicated). However, $\mathrm{Diff}_0(\Sigma)$ has diffeomorphisms isotopic to $\id.$ 
        \begin{itemize}
        \setlength\itemsep{-.2em}
    \item The inclusion $\mathrm{SO}(3) \hookrightarrow \mathrm{Diff}^+(S^2)$ is a homotopy equivalence.
    \item Writing  $T^2 = \R^2 / \Z^2$, the inclusion $T^2 \to  \mathrm{Diff}_0(T^2)$ (where $T^2$ acts on itself by translations) is a homotopy equivalence, while $\pi_0 \mathrm{Diff}^+(T^2)\cong \mathrm{SL}_2(\Z)$ (via the action of the mapping class group on $H_1(T^2; \Z) = \Z^2$).
    \item For $g(\Sigma) >1 $, $\mathrm{Diff}_0(\Sigma)$ is contractible.
        \end{itemize}
\end{enumerate}
All of this ``generalizes'' to dimension 3 through Thurston's geometrization. The basic point is that for $M^3$ a closed 3-manifold, $\pi_1 M$ is (nearly) a complete invariant.\footnote{Minus the lens spaces, but those are truly exceptions.}

\subsection{Higher dimensions}
The desired requirements are ambitious in higher dimensions.
    One issue is that arbitrary finitely presented groups can be represented by a manifold, and there are algorithmic problems with such groups (the word problem). Restricting to $\pi_1 = 1$, there are uncountably many 4-manifolds diffeomorphic to $\R^4$. But there are countably many (connected) compact manifolds. 
\emph{Surgery theory} tells us there are excellent conceptual answers to classify of simply connected compact $(n \geq 5)$-manifolds. We need Poincar\'e duality, a tangent bundle, and ``surgery obstruction''. A candidate tangent bundle $T$ must obey a constraint given by the \emph{Hirzebruch index theorem}. Surgery theory says this is sufficient to produce a manifold, and how many.

All of this breaks down in dimension 4.

