\section{Intersection forms} 
Tim forgot to show up to class today. Take $M$ to be a closed oriented $2n$-manifold. Then we have the cup product pairing on $H^n (M)$, where $x,y \in H^n M,\ \Z\ni x \cdot y = \langle (x \smile y ) \in H^{2n}(M), [M] \in H_{2n}(M) \rangle $. This pairing is skew-symmetric when $n$ is odd, and symmetric when $n$ is even, i.e. when $\dim M$ is a multiple of 4. It is closely related to Poincar\'e duality, namely: \[
    \underset{\in  \Z}{\underbrace{(x \cdot y)}}  \underset{\in H_0M}{[\mathrm{pt}]}  = \langle x \smile y, [M] \rangle [\mathrm{pt}] =\underset{H_0M}{ \underbrace{(x \smile y) \frown [M]}} = x \frown (y \frown [M]) = x\frown D_M y.
\] So the cup product makes $H_{-*}$ a graded module over $H^*$, where $(x \cdot y) = \langle x \in H^n ,D_My \in H_n  \rangle \in \Z$. We discussed this last time, the cup product in $H^*$ leads to Poincar\'e duality for intersections in $H_n $. In dimension 4, we have $x \in H^2(M), y \in H^2(M)$ Poincar\'e dual to $[\Sigma],[\Sigma'] \in H_2(M)$. Here $\Sigma,\Sigma' \subset M$ are oriented compact embedded surfaces. The claim is that $x\cdot y$ is given by the intersection number $\# \Sigma \cap \Sigma '$, provided that $\Sigma \transv \Sigma'$.

At $x \in \Sigma \cap \Sigma'$, we have $T_x M= T_x \Sigma \oplus T_x \Sigma'$. Compare orientations $o_{M,x}$ and $o_{\Sigma,x}\oplus o_{\Sigma', x}$. Attach a sign $\varepsilon _x = \pm 1$ according to whether orientations match or not, so $\# \Sigma \cap \Sigma' = \Sigma _{x \in \Sigma \cap \Sigma'}\varepsilon _x$. Can we always find such represented surfaces? The answer is yes, but we will not discuss this today.
\begin{example}
    Let us examine $\C \mathrm{P}^2$ as a complex surface. We have $T_z \C \mathrm{P}^2$ a complex vector space. Any finite dimensional complex vector space $V$ is \emph{oriented} as a real vector space. If $(e_1, \cdots ,e_d)$ is a complex basis, we have an oriented real basis $(e_1,ie_1, e_2, ie_2, \cdots ,e_d, ie_d)$. We also have $H_2(\C \mathrm{P}^2) \cong H^2(\C \mathrm{P}^2) \cong \Z$, generated by $[\C \mathrm{P}^1] = [L]$ for any (projective) line $L \subset \C\mathrm{P}^2$ (since $\C \mathrm{P}^2 = \P(\C^3), L = \P(V)$ for $V$ a dimension 2 complex subspace).

    When trying to apply the intersection number method, the immediate problem is that $\C \mathrm{P}^1 = \ell$ is not transverse to itself. This is easy to fix, just take a second representative $\ell \cdot  \ell = [L] \cdot [L']=\pm 1$ for lines $L \transv L'$.
\begin{namedthing}{Notation} 
    Let $Q_M$ denote the intersection (cup product) pairing on $H_2(M) = H^2(M)$ for $M^4$.
\end{namedthing}
So $(H_2(\C \mathrm{P^2}), Q_{\C \mathrm{P^2}})\cong (\Z, \times) $, the matrix in basis $1 \in \Z$, or $[1]$.
\end{example}
\begin{example}
    For $\overline{\C \mathrm{P^2}}$, we have $Q _{\overline{\C \mathrm{P^2}}} = (\Z, [-1])$. 
\end{example}
\begin{example}
    For $S^2 \times S^2$, a basis for $H_2$ is $h = [S^2 \times   \mathrm{pt}], v = [\mathrm{pt} \times  S^2]$. They intersect transversely at one place, so $h \cdot v = \pm 1$. These are complex submanifolds, so their tangent spaces are invariant by multiplication by $i$. So $h\cdot v = +1$. What about $h \cdot h$? Resolving the lack of transversality is done by perturbing it to be $h \cdot h = \# (S^2 \times  {x} ) \cap  (S^2 \times  {y} )$ for $x\neq y$. This intersection is empty and tranverse, and is 0. Similarly  $v \cdot v=0$. So the matrix for $Q _{S^2 \times S^2}$ in basis $(h,v)$ is $\left[ 
    \begin{smallmatrix}
        0 & 1 \\ 1 & 0
\end{smallmatrix}\right] $, which is a basic example of a hyperbolic quadratic form. This is an \emph{even} form, where $Q _{S^2 \times S^2}(c, c) \in 2 \Z$. To see this, note that $Q(ab + bv, ab + bv) = 2ab \in 2 \Z$. In contrast, $Q _{\C \mathrm{P^2}}$ is \emph{odd} (i.e. not even), since the self intersection of the line is 1.
\end{example}
\begin{note}
    $H_2 (M_1 \# M_2) = H_2 (M_1)\oplus H_2(M_2)$.
\end{note}
We deduce that $Q _{M_1\# M_2}= Q _{M_1}\oplus Q _{M_2}$. This becomes clear when you know all $H_2$-classes are representable by surfaces. For example, for $\C \mathrm{P^2} \# \overline{\C \mathrm{P^2}}$ we have $Q = \left[ 
\begin{smallmatrix}
    1 & 0 \\ 0 & -1
\end{smallmatrix}\right] $, which is odd. So there exists no (oriented)\footnote{For $M_1,M_2 $ closed oriented $n$-manifolds, an \textbf{oriented homotopy equivalence} is a homotopy equivalence $h \colon M_1 \to M_2$ with the property that the pushforward $h_*[M_1]=[M_2]$.} homotopy equivalence $S^2 \times  S^2 \leftrightarrow \C \mathrm{P^2}\# \overline{\C \mathrm{P^2}}$. So the intersection form is an invariant of oriented manifolds modulo oriented homotopy equivalence.
\subsection{Non-degeneracy}
For $A$ an abelian group, write $A' = A / A _{\mathrm{tors}}$. The assertion is that for $M^4$, the intersection pairing $O_M$ is a non-degenerate pairing on the second homology modulo torsion $H_2(M)'$. That is to say, $Q_M \colon H_2(M)' \times H_2(M)' \to \Z$, and the map $H_2(M)' \to  \Hom((H_2M)', \Z), h \mapsto  Q_M(h, -)$ is an isomorphism of abelian groups, which follows readily from Poincar\'e duality. In the prescence of a fundamental group we may have to kill torsion, but as we discussed, when $\pi_1 M = 0$, the torsion subgroup of $H_2$ is 0. So $Q_M$ is non-degenerate on $H_2 M$ itself.

