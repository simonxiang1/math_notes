\section{Classifying 4-manifolds} 
This is arguably the biggest mystery in geometric topology. 
\begin{namedthing}{Question} 
   What is the classification of closed, simply connected, smooth 4-manifolds up to diffeomorphism?
\end{namedthing}
Why do we say simply connected? We want to avoid algorithmic decidability issues involving $\pi_1$ of multiply connected manifolds. The question of whether $\langle g_1, \cdots ,g_{\varphi } \mid  r_1 ,\cdots , r_q\rangle $ is the trivial group is an algorithmically unsolvable problem. There are ways to build a 4-manifold with fundamental group isomorphic group, so a classification scheme runs into a lot of danger of being a solution to this algorithmically unsolvable problem.

It turns out that the oriented homotopy type $ X$ of such a manifold (smooth or not) can be encoded in a \emph{unimodular} symmetric bilinear form $\sigma_X$ over $\Z$. What is a symmetric bilinear form over $\Z$? We need
\begin{itemize}
\setlength\itemsep{-.2em}
    \item an abelian group $H$,
    \item a function $\sigma \colon H \times H \to \Z$ that is linear in each variable, and $\sigma(x,y) = \sigma(y, x)$.
\end{itemize}
Unimodular means that the map $H \to \Hom(H, \Z), x \mapsto  \sigma(x, -)$ to its dual is an isomorphic. This implies that $H$ is free abelian of finite rank. The claim is that attaching such a form knows the homotopy type of $X$. Namely, $H_X = H^2(X,\Z)$ and $\sigma_X(x,y)= \langle (x\smile y) \in H^4(X, \Z), [X] \in H_4(X,\Z) \rangle $ where $[X]$ is the fundamental class for the orientation and $\langle  \rangle $ is the evaluation pairing. This actually makes sense for a simply connected 4-dimensional Poincar\'e complex, which is a 4-dimensional CW complex that obeys Poincar\'e duality.
Part of the story of surgery theory is that we start off with a blob that becomes a 4-manifold, and it turns out that blob needs to be a Poincar\'e complex. It follows from Poincar\'e duality and the universal coefficients theorem that this pairing is unimodular.

One way of getting at it is to say that $H^2(X) \cong_{\mathrm{PD}} H_2(X)$, so $\sigma_X$ is a pairing on second homology. In these terms, $\sigma_X$ is the \emph{intersection form}. If $\Sigma_1$ and $\Sigma_2$ are closed oriented surfaces embedded in $X$ (smooth), then $\sigma_X (\mathrm{PD}[\Sigma_1],\mathrm{PD}[\Sigma_2]) = \Sigma_1 \cdot \Sigma_2$ (where $\cdot $ is the oriented intersection number). To make things concrete, we can express a unimodular symmetric bilinear form $(H, \sigma)$ in terms of a symmetric matrix $Q$ over $\Z$. We pick an integer basis $(e_1, \cdots ,e_j )$ for $H$, then $Q_{ij}=\sigma(e_i ,e_j )$. Unimodularity implise that $\det Q = \pm 1$, and this pairing is a reversible process. Since we can change the basis, a unimodular form gives us a matrix $Q$ as above modulo  $\Z$-equivalence, where $Q \sim M^T Q M$ where $M \in  \mathrm{GL}(\Z^b) $.

A more precise version of the question is this:
\begin{namedthing}{Question} 
    \begin{enumerate}[label=(\roman*)]
    \setlength\itemsep{-.2em}
        \item Which unimodular forms $(H,\sigma)$ arise as intersection forms of simply connected smooth closed oriented 4-manifolds?
        \item What are the possible diffeomorphism types of 4-manifolds representing a given form?
    \end{enumerate}
\end{namedthing}
Working with topological simply connected, oriented, closed 4-manifolds up to homeomorphism, Mike Freedman from the early 80s showed that all unimodular forms arise. For example, the $E_8$ lattice and the Leech lattice all arise as topological 4-manifolds. When smoothable, 4-manifolds are homeomorphic iff they have isomorphic intersection forms. Essentially, Freedman showed that the theory of surgery which breaks down in 4 dimensions, can be amended when working with topological manifolds up to \emph{homeo}morphism rather than smooth manifolds up to \emph{diffeo}morphism.

We have formed a question and solved it in the topological category so to speak. Let's see what's different in the smooth category. Not all unimodular forms arise from \emph{smooth} manifolds.
\begin{theorem}[Rokhlin, 1952]
    Suppose that $X^4$ is a smooth, closed, oriented, simply connected 4-manifold with \emph{even} intersection form ($\sigma_X (x,x) \in 2\Z$). Then the \emph{signature} (number of positive eigenvalues of $Q$ minus the number of eeigenvalues) of $\sigma_X$ is divisible by 16. 
\end{theorem}
If we take the signature of an even unimodular form, this is always divisible by 8 (by algebra). For example the $E_8$ form is positive definite, so the signature and rank are both 8. Rokhlin says that the for the form coming from a 4-manifold, looking at its signature and dividing by 8 is not merely an integer but an \emph{even} integer. So $E_8$ cannot arise, but $E_8\oplus E_8$ possibly could. There are many proofs but none are easy; some use the Pontryagin-Thom construction to make a connection with the stable homotopy group $\pi_{3+n}(S^n ) \cong \Z /24$. We will prove this theorem but not by that route.

Matters stood here for more or less three decades. In 1982, this graduate student Simon Donaldson came up with an unbelievable beautiful proof of an unbelievable beautiful theorem.
\begin{namedthm}{Donaldson's Diagonalizability Theorem} 
    If a positive definite ($\sigma(x,x) > 0$ for all $x \neq 0$) unimodular form $\sigma$ arises as the intersection form of a smooth closed oriented simply connected 4-manifold, then $\sigma$ can be represented by the identity matrix.
\end{namedthm}
The number of isomorphism classes of unimodular positive definite forms of rank 32 is greater than $10 ^{16}$. Donaldson says that precisely one of those come up as a 4-manifold. The proof uses \textbf{gauge theory}, the geometric analysis of a PDE with gauge symmetry, in particular the Yangs-Mills instanton. We will prove this but not using the Yangs-Mills instanton, but an alternative method brought in by Witten in 1994, the \textbf{Seiberg-Witten equations}, which arose by the work of Seiberg and Witten in string theory.
These new equations bypassed many of the difficulties of instantons.

In summary:
\begin{enumerate}[label=(\roman*)]
\setlength\itemsep{-.2em}
    \item Nearly complete answers arise via SW theory.
    \item SW \emph{invariants} distinguish many diffeomorphism types.
\end{enumerate}
A complete answer is not known and will require completely new ideas.

