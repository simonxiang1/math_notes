\section{Basics of 4-manifold topology} 
\begin{example}
Let's get going with some first examples of closed simply connected 4 manifolds.
\begin{itemize}
\setlength\itemsep{-.2em}
    \item Our first example is $S^4$. It's oriented as a hypersurface in $\R^5$.
    \item Our next example is $\C\mathrm{P}^2 = (\C^3 \setminus \{0\} ) / \C^{\times }= S^5 / \mathrm{U}(1)= \{\text{lines in } \ \C^3\} $. It has a CW structure wherer $\C \mathrm{P}^2 = e_0 \cup  e_2 \cup  e_4$, where the 0-cell and 2-cell make up a copy of $\C\mathrm{P}^1$ sitting in $\C \mathrm{P}^2$, which makes clear that $\pi_1(\C \mathrm{P}^2)$ is trivial (there are no 1-cells). It's canonically oriented, since its tangent spaces are \emph{complex} (2-dimensional) vector spaces.
    \item Recall that $\C \mathrm{P}^2$ comes with a canonical orientation, so we regard it as an oriented manifold. Then there is a manifold $\overline{\C \mathrm{P}^2}$ which is $\C\mathrm{P}^2$ with the ``wrong'' orientation. It turns out not to be oriented homotopy equivalent to $\C \mathrm{P}^2$.
    \item Products of spheres $S^2 \times S^2$ are 4-manifolds, which can be oriented as a compact surface. There also exists $\overline{S^2 \times S^2}$, but there is an orientation reversing diffeomorphism of the 2-sphere---the antipodal map. So these two manifolds are actually diffeomorphic through $\mathrm{a} \times  \id$.
    \item There is also $S^1  \times S^3$, which is not simply connected since $\pi_1(S^1  \times  S^3) = \Z$.
\end{itemize}
If $X_1,X_2$ are connected \emph{oriented} smooth $n$-manifolds, we can form a new smooth $n$-manifold called the \textbf{connected sum} $X_1 \# X_2$ defined up to diffeomorphism. In dimension 2, we remove a coordinate disk from each, attached to it a cylinder, and identify the cylinders. More precisely, take oriented charts $\chi_i\colon D^n  \hookrightarrow  X_i $ ($i=1,2$). Pick $\rho \in \mathrm{O}(n)$ where $\rho \colon \R^n  \to \R^n $, $\det(\rho) = -1$, e.g. $\rho(x_1, \cdots ,x_n ) = (-x_1, x_2, \cdots ,x_n )$. Let $X_i  ^{\circ}= X_i  \setminus \chi_i( \frac{1}{2}D^n ).$ So we throw out a smaller disk, and set \[
    X_1 \# X_2 = X_1 ^{\circ }\cup  X_2 ^{\circ } / \sim, \quad \chi_1(x) \sim \chi_2(\rho x),\quad x \in D^n  \setminus \frac{1}{2}D^n .
\] The tricky thing is that there is an orientation reversing map built into this gluing. Well-definedness up to oriented diffeomorphism is not obvious. The fact that the connected sum doesn't depend on charts is not a triviality by any means.
\end{example}

\subsection{(Co)homology}
Now that we are confident that simply connected 4-manifolds exist, let us proceed. Let's start talking about homology and cohomology. We begin by talking about singular homology $H_*(X) = H_*( \to \cdots \to S_2(X) \to  S_1(X) \to S_0(X) \to 0)$, where singular chains are maps $\Delta ^n  \to X$, and cohomology $H^*(X)$, the homology of the dual complex $H^*(\Hom(S_*X, \Z))$.

Recall the notion of an orientation for a topological $n$-manifold $X$ (usually we think of orientations on tangent spaces). For $K \subset K$, write $H_*( X \mid K) = H_*(X, X \setminus K)$ for the relative homology of $X$ with respect to $K$. We have local homology defined as $H_n (X \mid x)$, where $x \in X$. Cycles for this homology group are $n$-simplices in $X$ whose boundary lies in the complement of $x$. If $x \in  U \subset X$ where $U \cong \R^n $ is open, then we have \[
    H_n (X \mid x) \underset{\text{excision} }{\xleftarrow{\cong} } H_n (U \mid x) \cong H_n (\R^n  \mid 0) \cong \Z,
\] where the generator is a cycle around the origin. Then $X$ leads to a family of copies of $\Z$ given by $\left\{H_n (X \mid x) \, \big| \, x \in X\right\} $, and an \textbf{orientation} is given by a \emph{coherent} choice of generators for these groups. {\color{red}todo:details} 
Closely related, a \textbf{fundamental class} $[X]$ is an element $[X] \in  H_n (X)$ whose images in $H_n (X \mid x) \cong \Z$ are generators. Then it is clear that a fundamental class leads to an orientation, and a fundamental class exists only if $X$ is compact. 
\begin{theorem}
    Let $X$ be a connected topological $n$-manifold. Then
    \begin{itemize}
    \setlength\itemsep{-.2em}
        \item $H_i (X) = 0$ for all $i >n$.
        \item If $X$ is non-orientable or non-compact, then $H_n (X) = 0$.
        \item If $X$ is orientable and compact, then $H_n (X) \cong \Z$ generated by a fundamental class.
    \end{itemize}
\end{theorem}
\begin{namedthm}{Poincar\'e Duality Theorem} 
    For $X$ a compact $n$-manifold, an orientation determines an isomorphism  $D_X \colon H^k (X) \xrightarrow{\cong}  H_{n-k} (X)$, where $D _{\overline{X}}= -D_X$.
\end{namedthm}
We will not name the isomorphism now since we don't have the tools to set it up yet. For $X^4$ a closed, connected, oriented 4-manifold, where does it have interesting homology and cohomology potentially?
\begin{itemize}
\setlength\itemsep{-.2em}
    \item $H^0(X) \cong H_4(X) \cong \Z [X]$
    \item $H^1(X) \cong H_3(X)$
    \item $H^2(X) \cong H_2(X)$---note that these two are isomorphic. 
    \item $H^3(X) \cong H_1(X)$
    \item $H^4(X) \cong H_0(X) = \Z [ \text{pt.}] $
\end{itemize}
If $X$ is simply connected, then $H_1(X) \cong H^3(X)$ becomes trivial, as well as $H^1 (X) \cong H_3(X)$. So all the juice is in the $H_2^2$ example.
