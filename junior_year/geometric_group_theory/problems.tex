\newpage
\section{Problems} 
\subsection{Groups}

Here we write up some problems.
%\begin{ex}
    %If $H,K$ are subgroups, then $H \cap K$ is a subgroup since $1_G \in H \cap K, $ and $H \cap K$ is closed under multiplication and inverses. However $H \cup K$ is not a subgroup; take $h \in H \setminus K$ and $k \in K \setminus H$; then if $hk \in H \cup K$, it must be true that $hk \in (H \cup K) \setminus (H \cap K)$, since $H$ and $K$ are closed under multiplication. WLOG if $hk \in H \setminus (H \cap K)$, this is a contradiction since $(H \setminus (H \cap K)) \subset H$ which is closed under multiplication, but $k\notin (H \setminus (H \cap K))$.
%\end{ex}
%\begin{ex}[Squares of groups]
%$\Z$ is cyclic but $\Z^2$ is not; consider $\langle m,n \rangle =\Z$. If this were true then $\langle m,n \rangle $ generates $\langle k,k \rangle  $ or $\langle k, -k \rangle $, which are not $\Z \times \Z$.

%$\R^2$ is a vector space over $\Q$ and so is $\R$ (AOC?)
%\end{ex}
\begin{ex}[Rationals are not finitely generated]
    To see that $\Q$ is not finitely generated, assume that $\{a_i  /b_i \} _{i \in \{1,\cdots ,n\} }$ generates $\Q$ for $n \in \N$, $a_i ,b_i  \in \Z$. Let $q \in \Q$, then $q= \sum _{i \in \{1,\cdots ,n\} }c_i \left( \frac{a_i }{b_i } \right)= \frac{\ell}{\prod _{i \in \{1,\cdots ,n\} }b_i }  $ for coefficients $c_i \in \Z$, $\ell \in \Z$ (the value of $\ell$ is irrelevant). Choose $p$ a prime number such that $p \nmid \prod _{i \in \{1,\cdots ,n\} }b_i $, and set $q=1/p$. This implies that \[
\ell p = \prod _{i \in \{1,\cdots ,n\} }b_i ,
    \] which contradicts the fact that $p\nmid \prod _{i \in \{1,\cdots ,n\} }b_i $.
\end{ex}
\begin{ex}[Outer and inner automorphism groups]
    To show that $\mathrm{Inn}(G)$ is a normal subgroup of $\Aut (G)$, let  $\varphi _g \in \mathrm{Inn}(G)$ denote conjugation by $g$. Then for $\psi \in \Aut(G), \psi \varphi _g \psi ^{-1}$ is conjugation by $\psi(g)$, since \[
        \psi \varphi _g\psi ^{-1}(g')=\psi\varphi _g (\psi ^{-1}(g'))= \psi\left( g \psi ^{-1}(g') g^{-1} \right) =\psi(g) g' \psi (g^{-1}).
    \] for $g' \in G$. Therefore $\psi \varphi _g \psi ^{-1} = \varphi  _{\psi(g)} \in \mathrm{Inn}(G)$, and we are done. 

    To determine $\mathrm{Out}(\Z)$, there are only two automorphisms of $\Z; 1 \mapsto 1$ and $1 \mapsto  -1$. The automorphism $1 \mapsto -1$ is not inner, so $\mathrm{Out}(\Z) \cong  \Z /2$.

For $\Z / n$ in general, $\Aut(\Z / n) \cong (\Z / n ) ^{\times }$ through the automorphisms $x \mapsto  x^a$ for $x \in \Z / n, a \in (\Z /n) ^{\times }$. We specify units because if $a$ and $n$ are not relatively prime, $x \mapsto x^a$ fails to be an automorphism since there exists a prime divisor $p$ of $n$, implying the existence of an order $p$ element, which is not allowed. There are no inner automorphisms so $\mathrm{Out}(\Z /2016)\cong \Z/2 \oplus \Z/2 \oplus \Z/6$, and $\mathrm{Out}(\Z / 2017) \cong \Z/2016$.
\end{ex}

\begin{ex}[Isometry groups of the unit square]
    Let $\rho$ denote rotation by 90 degrees, $\mu$ denote reflection over the $x$-axis. Then $D_4 = \{\rho, \rho^2, \rho ^3,\mu,\rho\mu,\rho^2 \mu,\rho^3\mu,\id\} $. Equivalently we have $D_4$ generated by $\{\rho,\mu \mid \rho^4=1,\mu^2=1,\mu\rho\mu^{-1}=\rho ^{-1}\} $. This is not abelian while $\Z/8$ is.
\end{ex}
\begin{ex}[More isometry groups]
    Yes. For $n=2$, consider the isosceles triangle. For $n=3$ and above, consider the $n$-gon divided into $n$ isosceles triangles with vertex at the center. Cut out an arrow out of each all with the same orientation. Then these subsets of $\R^2$ have rotational symmetry with isometry group $\Z /n$, but no reflectional or translational symmetry. 
\end{ex}
\begin{ex}[Even more isometry groups] 
    No. We have that $\mathrm{Isom}(\R^n ) \cong \R^n  \rtimes O(n)$; $\R^n $ represents the translations, and $O(n)$ is the set of distance preserving transformations (reflections and rotations) of $\R^n $ fixing the origin. Let $\rho$ be a transfomration and $\mu$ be a translation. Then $\rho ^{-1} \mu \rho $ is a translation. Here $O(n)$ acts on $\R^n $ in the natural way (as linear transformations are automorphisms). When composing isometries, the translation component should first apply the rotation of the first component before applying the translation of the second, encoded in the relation \[
        (v_1 ^n , o_1) \cdot (v^n_2, o_2) = (v_1 ^n o_1(v_2^n )), o_1o_2)
    \] which implies that $\mathrm{Isom}(\R^n ) \cong  \R^n \rtimes O(n)$.
    {\color{red}todo:} $\Z$ acts on $\Q$? $\Q / \Z$? Assume 
\end{ex}
\begin{ex}[Unfree groups]
    Finite groups cannot be free. $\Z^2$ cannot be free as it has the relation $[x,y]$.

    Using the universal properties, if $\Z /2017$ were free with generating set $S$, let $\varphi  \colon S \to \Z /2$ map an element to $[1] \in \Z /2$ and the rest to $[0]$. There is no non-trivial homomorphism $\varphi  \colon \Z /2017 \to \Z /2$, so $\Z /2017$ cannot be free. An analogous argument applies with $\Z ^2$ and $\R$.
\end{ex}
\begin{ex}[Rank of free groups]
    In the infinite case: say $|S'| < |S|$, where $S'$ generates (not freely) $F$.  Then we have a surjection $F_{S'} \to F_S$, and passing to $\left(F_{S'}\right)_{\mathrm{ab}}\otimes _{\Z}\Q$ and $\left(F\right)_{\mathrm{ab}}\otimes _{\Z}\Q$ gives a surjection from a vector space of rank $|S'|$ onto one of rank $|S|$, a contradiction.
    
    Now let $S,S'$ be free generating sets for $F$. Since free generating sets are also generating sets, we have $|S| \geq |S'|$ and $|S'| \geq S$, which implies $|S|=|S'|$. From this we conclude that all free generating sets of a free group have the same cardinality. 

    To show that the free group generated by two elements contains a subgroup not generated by two elements, consider the subgroup generated by elements $\{xyx ^{-1}, x^2y xy^{-2}, x^3yx ^{-3}\} $. This group has rank 3 by construction and elements that compose non-trivially, and we are done.
\end{ex}
\begin{ex}[Generators and relations, examples]
    \begin{enumerate}[label=(\arabic*)]
    \setlength\itemsep{-.2em}
        \item We have $\langle x,y\mid  xyx^{-1}y^{-1}\rangle $ isomorphic to $\Z^2$, since the commutator relation makes the generators commute and words reduce to $x^n y^m$ for $n,m \in \Z$, precisely $\Z^2$.
        \item These groups are isomorphic since $t = t^{-1}$ by the relation $t^2=0$.
        \item Abelianize to get $\langle x,y \mid x^{2014}=y ^{2014} \rangle $, which is a non-trivial relation. So this group is non-trivial.
        \item For $\langle x,y \mid xyx=yxy \rangle $, note that $xyxy=x^2yx=yxy^2$, and abelianizing this relation $x^2yx=yxy^2$ gives a non-trivial relation $x^2=y^2$ (abelianizing the original group also leads to $\langle x,y \mid x=y \rangle \cong \Z$).
    \end{enumerate}
\end{ex}
\begin{ex}[Positive relations]
    We want to show that for every presentation $\langle S\mid R \rangle $ there is a positive relation $\langle S' \mid R' \rangle $ with $\langle S' \mid R' \rangle \cong  \langle S \mid R \rangle $ and $|S'| \leq |S|+1$, $|R'| \leq |R| +1$.
     \[
         \langle a,b,c,d \mid ab ^{-1} c ^{-1} d ab, b a^{-1} b ^{-1} cda \rangle 
     \] Add a new generator $s'$ and the relation that (deemed unimportant exercise)
\end{ex}
\begin{ex}[Abelianization]
    Some problems about abelianization:
    \begin{enumerate}[label=(\arabic*)]
    \setlength\itemsep{-.2em}
\item To see that $[G,G]$ is normal, let $[g,h] \in [G,G]$, where $[g,h]:= gh g^{-1}h^{-1}$. Let $\ell \in G$. Then 
    \[
        \ell[g,h]\ell ^{-1}= \ell [g,h]\ell ^{-1} [g,h]^{-1}[g,h]=(\ell, [g,h])[g,h] 
    \] which lies in $[G,G]$ as a product of commutators. So $[G,G]$ is normal. The quotient $G _{\mathrm{ab}}:= G / [G,G]$ is abelian because setting a commutator $[g,h]=0$ makes the elements  $g,h$ commute, e.g. $gh=hg$. Since the commutator subgroup contains commutators spanning every pair $(g,h)$ for $g,h \in G$, every element of $G$ commutes with every other element and $G _{\mathrm{ab}}$ is indeed abelian.
\item Let $H$ be abelian and $\varphi \colon G \to H$ be a homomorphism. Then $\varphi (gh)=\varphi (g)\varphi (h)=\varphi (h)\varphi (g)=\varphi (hg)$, so $[g,h] \in \ker \varphi $ for all $g,h \in G$. This implies that $[G,G] \subseteq \ker f$, and from here apply the fundamental homomorphism theorem to get our universal property.
\item To view $\left( \cdot  \right) _{\mathrm{ab}} \colon \mathsf{Grp}  \to \mathsf{Ab} $ as a functor, let $(G)_{\mathrm{ab}}=G _{\mathrm{ab}} \in \mathrm{Ob}(\mathsf{Ab}) $ for $G \in  \mathrm{Ob}(\mathsf{Grp} )$. For $\varphi \colon G \to H , \varphi  \in  \mathrm{Mor}(\mathsf{Grp} )$, compose with the projection $\pi_H$ to get $\overline{\varphi }\colon G \to H _{\mathrm{ab}}, \overline{\varphi }:= \pi_H \circ \varphi $. This yields a homomorphism $G \to H_{\mathrm{ab}}$, and lift this by the universal property of $G_{\mathrm{ab}}$ to a map $\varphi  _{\mathrm{ab}}\colon G_{\mathrm{ab}} \to H_{\mathrm{ab}}$. 
\[
\begin{tikzcd}
                                                                           & H \arrow[d, "\pi_H"] \\
G \arrow[d, "\pi_G"] \arrow[r, "\overline{\varphi}"] \arrow[ru, "\varphi"] & H_{\mathrm{ab}}      \\
G_{\mathrm{ab}} \arrow[ru, "\varphi_{\mathrm{ab}}"', dotted]               &                     
\end{tikzcd}
\] 
Then define $(\varphi )_{\mathrm{ab}}:=\varphi _{\mathrm{ab}}$ as in the above construction. To show that this is indeed a functor, it is clear that $(1_{\mathsf{Grp} })_{\mathrm{ab}}=1_{\mathsf{Ab} }$ since $\mathsf{Grp} $ and $\mathsf{Ab} $ have the same unit, which is $1=1_{\mathrm{ab}}$. For $\theta \colon G \to H, \varphi \colon H \to K$, the fact that $\left( \varphi  \circ \theta \right) _{\mathrm{ab}}=\varphi  _{\mathrm{ab}}\circ \theta _{\mathrm{ab}}$ is a natural consequence of the universal property. \[
\begin{tikzcd}
                                                                         &                                                                            & K \arrow[d, "\pi_K"] \\
                                                                         & H \arrow[d, "\pi_H"] \arrow[ru, "\varphi"] \arrow[r, "\overline{\varphi}"] & K_{\mathrm{ab}}      \\
G \arrow[d, "\pi_G"] \arrow[r, "\overline{\theta}"] \arrow[ru, "\theta"] & H_{\mathrm{ab}} \arrow[ru, "\varphi_{\mathrm{ab}}"', dotted]               &                      \\
G_{\mathrm{ab}} \arrow[ru, "\theta_{\mathrm{ab}}"', dotted]              &                                                                            &                     
\end{tikzcd}
\] This shows that $\left( \cdot  \right) _{\mathrm{ab}}$ is indeed a functor $\mathsf{Grp} \to \mathsf{Ab} $.
\item $F_{\mathrm{ab}}$ is the free abelian group inheriting the same basis as the free group, since we mod out by all commutators.
\item Abelianization mods out all commutators so this is true as well.
    \end{enumerate}
\end{ex}
\begin{ex}[The infinite dihedral group]
    An isometry of $\Z$ is multiplication by $-1$ which squares to itself. Another isometry is addition by one, and composing these isometries under the rule that multiplying by $-1$, adding, and multipliying again by $-1$ is the same thing as subtraction gives all the isometries of $\Z$. This data is precisely encoded in the group $D_{\infty}:= \langle s,t \mid t^2,tst=s ^{-1} \rangle \cong \mathrm{Isom}(\Z)$.
\end{ex}
\begin{ex}[Baumslag-Solitar groups]
    \begin{enumerate}[label=(\arabic*)]
    \setlength\itemsep{-.2em}
\item We have $\mathrm{BS}(1,1)=\langle  a,b\mid bab ^{-1} = a\rangle =\langle a,b \mid [b,a] \rangle \cong \Z^2$.
\item To see that $\mathrm{BS}(m,n)$ is infinite, consider the matrices $\left( 
    \begin{smallmatrix}
        1 & 1 \\ 0 & 1
\end{smallmatrix}\right) $ and $\left( 
\begin{smallmatrix}
    \frac{n}{m} & 0 \\ 0 & 1
\end{smallmatrix}\right) $. Note that $\left( 
    \begin{smallmatrix}
        1 & 1 \\ 0 & 1
\end{smallmatrix}\right) ^n =\left( 
    \begin{smallmatrix}
        1 & n \\ 0 & 1
\end{smallmatrix}\right) $, and $\left( 
\begin{smallmatrix}
    \frac{n}{m} & 0 \\ 0 & 1
\end{smallmatrix}\right) ^{-1}=\left( 
\begin{smallmatrix}
    \frac{m}{n} & 0 \\ 0 & 1
\end{smallmatrix}\right) $.
Then if we consider the map $a \mapsto \left( 
    \begin{smallmatrix}
        1 & 1 \\ 0 & 1
\end{smallmatrix}\right) $ and $b\mapsto \left( 
\begin{smallmatrix}
    \frac{n}{m} & 0 \\ 0 & 1
\end{smallmatrix}\right) $, multiplying matrices gives us \[
\begin{pmatrix}
    \frac{n}{m} & 0 \\ 0 & 1
\end{pmatrix} 
\begin{pmatrix}
    1 & 1 \\ 0 & 1
\end{pmatrix}^m 
\begin{pmatrix}
    \frac{n}{m} & 0 \\ 0 & 1 
\end{pmatrix}^{-1}=
\begin{pmatrix}
    \frac{n}{m} & 0 \\ 0 & 1
\end{pmatrix} 
\begin{pmatrix}
    1 & m \\ 0 & 1
\end{pmatrix} 
\begin{pmatrix}
    \frac{m}{n} & 0 \\ 0 & 1 
\end{pmatrix}=
\begin{pmatrix}
    1 & n \\ 0 & 1
\end{pmatrix}, 
\] precisely encoding the Baumslag-Solitar relation that $b a^m b^{-1}=a^n $. There are an infinite amount of matrices generated from these two matrices by repeated multiplication of $\left( 
\begin{smallmatrix}
    1 & 1 \\ 0 & 1
\end{smallmatrix}\right) $, in other words, the infinite set $\left\{\left( 
\begin{smallmatrix}
    1 & i \\ 0 & 1
\end{smallmatrix}\right) \right\}_{i \in \N \setminus \{0\} } $ is a subset of $\mathrm{BS}(m,n)$.
\item To see that $\mathrm{BS}(m,n)$ is not cyclic, abelianize to get $\mathrm{BS}(m,n)_{\mathrm{ab}}\cong \langle a,b \mid a ^{m-n} \rangle $. Here $b$ freely generates words with no relations, so there is no single generator for $\mathrm{BS}(m,n) _{\mathrm{ab}}$, and therefore $\mathrm{BS}(m,n)$. We conclude that $\mathrm{BS}(m,n)$ is not cyclic.
\item We want to show that \[
        \varphi  \colon \mathrm{BS}(2,3) \to \mathrm{BS}(2,3),\quad a\mapsto a^2,b\mapsto b
    \] defines a well-defined surjective group homomorphism. Under $\varphi $ the relation becomes $ba^4 b^{-1}=a^6$, which adds no additional data. The map $\varphi $ is surjective because 2 divides 2 but not 3, so any $a^2 \in \im(\varphi )$ gets represented (doesn't get cancelled by $a^3$ relation). Now to see that $[ba b^{-1},a] \in \ker \varphi $, write $[bab^{-1},a]=bab^{-1}aba^{-1}b^{-1}a^{-1}$, then \[
    \varphi (bab^{-1}aba^{-1}b^{-1}a^{-1})=(ba^2b^{-1})a^2(ba^{-2}b^{-1})a^{-2}=a^3a^2a^{-3}a^{-2}=1
\] and $[bab^{-1},a] \in \ker \varphi $. So $\varphi $ is a non-injective surjection $\mathrm{BS}(2,3) \to \mathrm{BS}(2,3)$, and $\mathrm{BS}(2,3)$ is non-Hopfian.
    \end{enumerate}
\end{ex}
\begin{ex}[A normal form for $\mathrm{BS} (1,2)$]
    {\color{red}todo: arbitrary word?} 
\end{ex}
\begin{ex}[Surface groups]
    First let $n=m$, then the groups $G_n $ and $G_m$ have the same generators and relations, and are therefore isomorphic. Now suppose $n\neq m$. Abelianization cancels all the relations, and we are left with two free groups of different sizes ($G_n =F _{2n}, G_m = F_{2m}$) which are not isomorphic by assumption. So $G_n  \cong  G_m $ iff $n=m$. $G_n $ is only abelian for $n=1$, since for $n=2$ the pair of generators $a_2,b_1$ do not commute with each other (since $[a_2,b_1]$ is not a relation){\color{red}todo:fix this part} .
\end{ex}
\begin{ex}[Coxeter groups]
    \begin{enumerate}[label=(\arabic*)]
    \setlength\itemsep{-.2em}
\item If $j,k \in \{1,\cdots ,n\} $ with $j\neq k$ and $m_{jk}=2$, we want to show that the corresponding elements $\overline{s}_j ,\overline{s}_k$ commute in $W$. We have the relations \[
        \overline{s}_j ^2=1,\quad \overline{s}_k^2=1,\quad (\overline{s}_j \overline{s}_k)^2=1,
\] which implies that $\overline{s}_j =\overline{s}_j ^{-1}, \overline{s}_k=\overline{s}_k^{-1}$, and so \[
(\overline{s}_j \overline{s}_k)^2=\overline{s}_j \overline{s}_k\overline{s}_j \overline{s}_k=\overline{s}_j \overline{s}_k\overline{s}_j ^{-1}\overline{s}_k^{-1}=[\overline{s}_j ,\overline{s}_k]=1.
\] This implies that $\overline{s}_j $ and $\overline{s}_k$ commute.
\item To see that $\langle s_1,s_2,s_3\mid (s_1s_2)^2,(s_1s_3)^2,(s_2s_3)^2 \rangle \cong (\Z /2)^3$, map $s_1 \mapsto (s_1s_2), s_2 \mapsto (s_2s_3)$, and $s_3 \mapsto (s_1s_3)$.
\item The isometry group of a regular $n$-gon $D_n $ can be viewed as a Coxeter group as follows; consider the presentation $D_n =\langle s,t \mid s^n,t^2, (st)^2 \rangle $. Send $s \mapsto st$ and $t \mapsto t$, then this presentation becomes $\langle st, t \mid (st)^2,t^2, (t(st))^{-n} \rangle $ since $(st)^2=1$ implies 
    \[
        s=t^{-1}s^{-1}t^{-1}=(tst)^{-1}=(t(st))^{-1},\quad s^n =(t(st))^{-n}.
    \] 
    Then this is a Coxeter group of rank two with Coxeter matrix $\left( 
    \begin{smallmatrix}
        1 & -n \\
        -n & 1
    \end{smallmatrix}\right) $.
    \end{enumerate}
\end{ex}
\begin{ex}
    braid
\end{ex}
\begin{ex}
a finitely generated group that is not finitely presented
\end{ex}
\begin{ex}[Special pushouts]
    \begin{enumerate}[label=(\arabic*)]
    \setlength\itemsep{-.2em}
        \item Yes, remove the relations of $A$ to get $G * 1 \subseteq G *A$.
        \item For $\varphi  \colon A \to G$, $G *_A 1$ is just $G /\im \varphi $ since we adjoin relations $\varphi (a)=1$ for $a \in A$.
        \item For $\varphi  \colon A \to G$, $\id_A \colon A \to A$, we have $G *_A A \cong (G * A) / \langle \varphi (a)=a \rangle $.
    \end{enumerate}
\end{ex}
\begin{ex}[The infinite dihedral group strikes back]
    To see that $D_{\infty}\cong  \Z /2 * \Z /2$, using the presentations $\langle s,t \mid t^2, tst= s ^{-1} \rangle $ for $D _{\infty}$ and $\langle a,b \mid a^2,b^2 \rangle $ for $\Z /2 * \Z /2$, consider the map $(ts) \mapsto a, t \mapsto b$. Since $tst = s ^{-1}$ is the same relation as $(ts)^2$, this map yields an isomorphism $D _{\infty} \cong \Z /2 * \Z /2$.

    Now we want to show that $\mathrm{Isom}(\Z) \cong \Z \rtimes _{\varphi }\Z /2$, where $\varphi  \colon \Z /2 \to \Aut(\Z)$ is multiplication by $-1$. 
    Consider the exact sequence \[
    1 \to \Z \to  D _{\infty}\to \Z /2 \to 1,
\] this splits by $\Z /2 \to D _{\infty}$ where $[1] \mapsto ([0], [1])$. So $ D_{\infty}\cong \Z \rtimes _{\varphi }\Z /2$ by $\varphi  \colon \Z /2 \to \Aut(\Z )$, $[1] \mapsto ( 1 \mapsto 1 \cdot  n \cdot  -1=-n)$, which is precisely the automorphism given.
\end{ex}
\begin{ex}[Heisenberg group]
    Say the extension \[
    1 \to  \Z \xrightarrow iH \xrightarrow{\pi} \Z^2\to 1
    \] splits, that is, there exists a map $\varphi $ \[
    (0,1) \mapsto \begin{pmatrix}
       1 & x & z \\
       0 & 1 & y \\
       0 & 0 & 1
   \end{pmatrix} , \quad (1,0)\mapsto 
   \begin{pmatrix}
       1 & x' & z' \\
       0 & 1 & y' \\
       0 & 0 & 1
   \end{pmatrix}
   \] such that projecting gives back the generators of  $\Z^2$. Then we must have $x=0,y=1$ and  $x'=1,y'=0$, and $z,z'=0$ (maps take identities to identities). So \[
    \varphi ((0,1))=
    \begin{pmatrix}
        1 & 0 & 0 \\ 0 & 1 & 1 \\ 0 & 0 & 1
    \end{pmatrix}, \quad \varphi ((1,0))=
    \begin{pmatrix}
        1 & 1 & 0 \\ 0 & 1 & 0 \\ 0 & 0 & 1
    \end{pmatrix}.
    \] It should be the case that $\varphi (1,1)=\varphi ((0,1)+(1,0))=\varphi ((1,0)+(0,1))=\varphi ((0,1))\varphi ((1,0)) = \varphi ((1,0))\varphi ((0,1))$. {\color{red}todo:}  However, \[
    \varphi ((0,1))\varphi ((1,0))=
    \begin{pmatrix}
        1 & 0 & 0 \\ 0 & 1 & 1 \\ 0 & 0 & 1
    \end{pmatrix}
    \begin{pmatrix}
        1 & 1 & 0 \\ 0 & 1 & 0 \\ 0 & 0 & 1
    \end{pmatrix}=
    \begin{pmatrix}
        1 & 1 & 0 \\ 0 & 1 & 1 \\ 0 & 0 & 1
    \end{pmatrix},
    \] while \[
    \varphi ((1,0))\varphi ((0,1))= 
    \begin{pmatrix}
        1 & 1 & 0 \\ 0 & 1 & 0 \\ 0 & 0 & 1
    \end{pmatrix}
    \begin{pmatrix}
        1 & 0 & 0 \\ 0 & 1 & 1 \\ 0 & 0 & 1
    \end{pmatrix}=
    \begin{pmatrix}
        1 & 1 & 1 \\ 0 & 1 & 1 \\ 0 & 0 & 1 
    \end{pmatrix}.
    \] These two matrices are not equal, and we conclude that $\varphi $ must not exist and the extension $1 \to  \Z \xrightarrow iH \xrightarrow{\pi} \Z^2 \to 1$ does not split.
    Now to see that $\langle x,y,z \mid [x,z],[y,z],[x,y]=z \rangle $ is a presentation of $H$, consider the maps \[
    x \mapsto 
    \begin{pmatrix}
        1 & 1 & 0 \\ 0 & 1 & 0 \\ 0 & 0 & 1 
    \end{pmatrix},\quad
    y \mapsto 
    \begin{pmatrix}
        1 & 0 & 0 \\ 0 & 1 & 1 \\ 0 & 0 & 1 
    \end{pmatrix},\quad
    z \mapsto 
    \begin{pmatrix}
        1 & 0 & 1 \\ 0 & 1 & 0 \\ 0 & 0 & 1 
    \end{pmatrix}.
    \] We can verify these satisfy the commutator relations using a computer or by hand.
\end{ex}
\begin{ex}[Equivalence of extensions]
    We say two extensions $1 \to K \xrightarrow iG \xrightarrow{\pi} H \to 1,1 \to K \xrightarrow{i'} G' \xrightarrow{\pi'} H \to 1,$ are equivalent if there exists an isomorphism $T \colon G \to G'$ making the following diagram commute:
\[
\begin{tikzcd}
            &                                     & G \arrow[rd, "\pi"] \arrow[dd, "T"] &             &   \\
1 \arrow[r] & K \arrow[ru, "i"] \arrow[rd, "i'"'] &                                     & H \arrow[r] & 1 \\
            &                                     & G' \arrow[ru, "\pi'"']              &             &  
\end{tikzcd}
\] 
    The short five lemma tells us that in fact, a homomorphism is enough to make the diagram commute. Let us give three pairwise non-equivalent group extensions of the following type: \[
    1 \to  \Z \xrightarrow?  ? \xrightarrow? \Z/3 \to 1.
\] First consider the extension \[
1 \to  \Z \xrightarrow 3 \Z \xrightarrow{\pi} \Z /3 \to 1.
\] Then consider the extension \[
1 \to \Z \xrightarrow i \Z \rtimes_{\varphi _n } \Z /3 \xrightarrow{\pi} \Z /3 \to 1.
\] Finally consider the extension \[
1 \to \Z \xrightarrow i \Z \oplus \Z /3 \xrightarrow{\pi} \Z /3 \to 1.
\] 
\end{ex}
\begin{ex}
    lamplighter
\end{ex}
\begin{ex}
    big centre
\end{ex}
\begin{ex}[Baumslag-Solitar groups and HNN-extensions]
    We can view Baumslag-Solitar groups $\mathrm{BS}(m,n)$ as the HNN-extension of $\Z$ with $A=m\Z, B= n \Z$, with $\vartheta \colon A \to B $ defined by $\vartheta (m)=n$. Then in the definition of HNN-extensions
    \[
        \Z * _{\vartheta} := \left\langle x \amalg \{t\} \mid  \{t ^{-1} x_a t = x _{\vartheta(a)}\mid a \in A\} \cup R_G  \right\rangle 
    \] where \[
    R_G := \{x_g x_h x_k ^{-1} \mid g,h ,k \in \Z \text{ with } g \cdot h =k \text{ in } \Z\} ,
    \]the relations $R_G$ all cancel since $\Z$ has no relations. This leaves us with the group \[
    \Z * _{\vartheta} = \langle x,t \mid t ^{-1} x^m t = x^n \rangle ,
\] which looks like $\mathrm{BS}(m,n)$ when you send $x \mapsto a, t \mapsto  b$ given the following definition of Baumslag-Solitar groups:
    \[
        \mathrm{BS}(m,n) := \langle a,b \mid  b a^m b^{-1} = a^n  \rangle .
    \]
\end{ex}
\begin{ex}[Ascending HNN-extensions]
    If $\vartheta=\id_G$, then the relation  $t^{-1} x_at=x_{\vartheta(a)}$ for $a \in G$ becomes $x_at=x_at$, which is redundant. So $G *_{\vartheta}\cong G * \Z$ generated by the extra generator $t$.

    Now let $\vartheta \in \Aut(G)$, the twisting of $\varphi \colon G \to \Aut(G)$, $1 \mapsto \vartheta$ is represented by the relation $t ^{-1} x_a t = x _{\vartheta(a)}$, so the semidirect has the relations of $G$ as well as this extra generator and twisting relation. This is precisely $G *_{\vartheta}$.
\end{ex}

\subsection{Cayley graphs}
\begin{ex}[Petersen graph]
    
\end{ex}

