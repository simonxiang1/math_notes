\section{Groups} 
\subsection{An introduction}
Geometric group theory is about the relation between algebraic and geometric properties of a group. Specifically:
\begin{itemize}
\setlength\itemsep{-.2em}
    \item Can groups be viewed as geometric objects?
    \item If so, are the geometric and algebraic properties related?
    \item Which geometric objects can a group act on and how does the geometry relate to the algebra?
\end{itemize}
We do this by associating group-valued invariants with geometrical objects like isometry groups or the fundamental group. One of the central insights is that this process can be somewhat reversed.
\begin{itemize}
\setlength\itemsep{-.2em}
    \item Associate a geometric object with the group inquestion.
    \item Take geometric invariants and apply these to the aforementioned geometric objects.
\end{itemize}

\subsection{Group theory review}
\begin{definition}[Group]
    A \textbf{group} is a set $G$ together with a binary operation $G \times G \to G$ satisfying the following axioms:
    \begin{enumerate}[label=(\alph*)]
    \setlength\itemsep{-.2em}
        \item \emph{Associativity:} For all $g_1,g_2,g_3 \in G$, we have \[
                g_1 \cdot (g_2 \cdot g_3) = (g_1 \cdot g_2) \cdot g_3.
        \] 
            \item \emph{Identity}: There exists an \emph{identity} $e \in G$ for $\cdot $ such that for every $g \in G$, \[
            e \cdot g = g = g \cdot e.
            \] It follows that identities are necessarily unique.
        \item \emph{Inverses:} For every $g \in G$ there exists an \emph{inverse element} $g^{-1} \in G$ wrt $\cdot $ such that \[
        g \cdot  g^{-1} = e = g^{-1} \cdot  g.
        \] 
    \end{enumerate}
    A group is \textbf{abelian} if $g_1 \cdot g_2 = g_2 \cdot g_1$ for all $g_1,g_2 \in G$.
\end{definition}
\begin{definition}[Subgroup]
    Let $G$ be a group with respect to $\cdot $. A subset $H \subseteq G$ is a \textbf{subgroup} if $H$ is a group wrt the restriction of $\cdot $ to $H \times  H \subseteq G \times G$. The \textbf{index} $[G : H]$ of a subgroup is the cardinality of $\{g \cdot H \mid g \in G\} $; here $g \cdot  H = \{g \cdot h \mid h \in H\} $.
\end{definition}
\begin{definition}[Group homomorphism/isomorphism]
   Let $G,H$ be groups.
   \begin{itemize}
   \setlength\itemsep{-.2em}
       \item A map $\varphi \colon G \to H$ is a \textbf{group homomorphism} if $\varphi (g_1 \cdot g_2) = \varphi (g_1) \cdot \varphi (g_2)$ for all $g_1,g_2 \in G$. It follows that homomorphisms maps identities to identities.
       \item A homomorphism is an \textbf{isomorphism} if there exists some homomorphism $\psi \colon H \to G$ such that $\varphi  \cdot \psi = \id_H$ and $\psi \circ \varphi = \id_G$.
   \end{itemize}
\end{definition}
\subsection{Automorphism groups}
\begin{example}
    Let $X$ be a set. Then the set $S_X$ of all bijections $X \to X$ is a group wrt composition, the \textbf{symmetric group} over $X$. If $n \in \N$, then we abbreviate $S_n  := S _{\{1,\cdots ,n\} }$. In general, $S_X$ is nonabelian.
\end{example}
\begin{prop}[Cayley's theorem]
   Every group is isomorphic to a subgroup of some symmetric group. 
\end{prop}
\begin{proof}
    Let $G$ be a group. For $g \in G$ we define the map $f_g \colon G \to G$, $x \mapsto g \cdot x$. For all $g, h \in G$ we have $f_g \circ f_h = f _{g \cdot h}$. Therefore, looking at $f _{g ^{-1}}$ shows that $f_g \colon G \to G$ is a bijection for all $g \in G$. It follows that \[
    f \colon G \to S_G, \quad g \mapsto f_g
\] is a group homomorphism, which is injective (trivial kernel or cancellation). So $f$ induces an isomorphism $G\cong \im f \subset S_G$.
\end{proof}
\begin{example}[Automorphism groups]
    Let $G$ be a group. Then the set $\Aut(G)$ of group isomorphisms $G \to G$ is a group wrt composition, the \textbf{automorphism group} of $G$. Clearly $\Aut(G)$ is a subgroup of $S_G$. 
\end{example}
\begin{example}[Isometry groups/Symmetry groups]
    Let $X$ be a metric space. The set $\mathrm{Isom}(X)$ of all isometries of type $X \to X$ forms a group wrt composition (a subgroup of $S_X$). For example, the dihedral groups naturally occur as symmetry groups of regular polygons.
\end{example}
\begin{example}[Matrix groups]
    Let $k$ be a commutative ring with unit, and let $V$ be a $k$-module. Then the set $\Aut(V)$ of all $k$-linear isomorphisms forms a group with respect to composition.  In particular, the set $\mathrm{GL}(n,k)$ of invertible $(n \times n)$-matrices over $k$ is a group for every $n \in \N$. Similarly, $\mathrm{SL}(n,k)$ is also a group.
\end{example}
\begin{example}[Galois groups]
    Let $K \subset L$ be a Galois extension of fields. Then the set \[
        \mathrm{Gal}(L / K) := \{\sigma \in \Aut(L) \, \big| \, \left. \sigma \right|_K = \id_K  \} 
        \]  of field automorphisms of $L$ fixing $K$ is a group wrt composition of the so-called \textbf{Galois group} of the extension $L /K$.
\end{example}
\begin{example}[Deck transformation groups]
   Let $\pi \colon X \to Y$ be a covering map of topological spaces. Then the set \[
       \{f \in  \mathrm{map}(X,X) \mid  f \text{ is a homeomorphism with } \pi \circ f = \pi\} 
   \]  of \textbf{deck transformations} forms a group wrt composition.
\end{example}
These are all examples of the fact that if $X$ is an object in a category $C$, then the set $\Aut_C (X)$ of $C$-isomorphisms of type $X \to X$ is a group with respect to composition in $C$. It should also be easy to see that if $G$ is a group, there exists a category $C$ and an object $X \in \mathrm{Ob}(C)$ such that $\Aut_C(X) \cong G$ by considering the category with one object.
 \begin{example}
     Using categorical language, for $X$ a set we have $S_X \cong  \Aut _{\mathsf{Set} }(X)$. In the algebraic categories $\mathsf{Grp,Ab,Vect_{\R},}{}_R \mathsf{Mod}   $, objects are isomorphic categorically iff they are isomorphic algebraically, and the definitions of automorphism groups coincide. In $\mathsf{Met_{Isom}} $, objects are isomorphic iff they're isometric, and automorphism groups are isometry groups. In $\mathsf{Top} $ isomorphisms are homeomorphisms, and automorphism groups are self-homeomorphisms.
\end{example}
\subsection{Normal subgroups and quotients}
\begin{definition}[Normal subgroup]
    Let $G$ be a group. A subgroup $N$ of $G$ is \textbf{normal} if it is conjugation invariant, i.e., if $g \cdot n \cdot  g^{-1} \in N$ holds for all $n \in N$and all $g \in G$. If $N$ is a normal subgroup of $G$, then we write $N \trianglelefteq G$.
\end{definition}
\begin{example}
    \begin{itemize}
    \setlength\itemsep{-.2em}
        \item All subgroups of abelian groups are abelian and normal.
        \item Let $\tau = (1\ 2)\in S_3$, then $\{\id, \tau\} $ is a subgroup of $S_3$ but not normal. OTOH, $\{\id, \sigma,\sigma^2\} $ generated by $\sigma = (1\ 2\ 3)$ is normal.
        \item Kernels of group homomorphisms are normal in the domain group, conversely, every normal subgroup is the kernel of a certain group homomorphism (namely the canonical projection to the quotient).
    \end{itemize}
\end{example}
Normalness implies well definedness because \[
    (aN)(bN) = a(Nb)N=\underset{\text{normal} }{a(bN)N} =(ab)NN = (ab) N.
\] Alternatively, say $a,\overline{a},b,\overline{b} \in G$ with $aN= \overline{a}N, bN=\overline{b}N$, $n,m \in N$ with $\overline{a}=an, \overline{b}=bm$. Then \[
(\overline{a}\overline{b})N= (an\cdot bm)N=\underset{\text{normal} }{(a(bnb^{-1})bm)N} =(abnm)N=(ab)N.
\] 
\begin{example}
    Let $G$ be a group. An automorphism $\varphi  \colon G \to G$ is an \textbf{inner automorphism} of $G$ if $\varphi $ is given by conjugation by an element of $G$, i.e., if there is an element $g \in G$ such that for all $h \in G$, $\varphi (h) = g h g^{-1}$. Then $\mathrm{Inn}(G)$ is a normal subgroup of $\Aut(G)$. To see this, let  $\varphi _g \in \mathrm{Inn}(G)$ denote conjugation by $g$. Then for $\psi \in \Aut(G), \psi \varphi _g \psi ^{-1}$ is conjugation by $\psi(g)$, since \[
        \psi \varphi _g\psi ^{-1}(g')=\psi\varphi _g (\psi ^{-1}(g'))= \psi\left( g \psi ^{-1}(g') g^{-1} \right) =\psi(g) g' \psi (g^{-1}).
    \] Therefore $\psi \varphi _g \psi ^{-1} = \varphi  _{\psi(g)} \in \mathrm{Inn}(G)$, and we are done. Define the \textbf{outer automorphism group} as $\mathrm{Out}(G) := \Aut (G) / \mathrm{Inn}(G)$.
\end{example}

\subsection{Generating sets}
\begin{definition}[]
    Let $G$ be a group and $S \subseteq G$. The subgroup \textbf{generated} by $S$ in $G$ is the smallest subgroup $\langle S \rangle _G$ of $G$ containing $S$. A group is \textbf{finitely generated} if it contains a finite subset that generates it. Explicitly, we have \[
    \langle S \rangle _G = \bigcap_{ } \{H \mid  H \subset  G \text{ is a subgroup with } S \subset H\} = \{s_1 ^{\varepsilon 1}\cdots s_n  ^{\varepsilon _n }\mid  n \in \N, s_1, \cdots ,s_n  \in S, \varepsilon_1, \cdots ,\varepsilon _n \in \{\pm 1\} \} 
    \] 
\end{definition}
\begin{example}
    If $G$ is a group, then $G$ generates $G$. The trivial group is generated by $\O$. $\{1\} $ generates $\Z$ as well as $\{2,3\} $, but $\{2\} , \{3\} $ don't. $S_X$ is finitely generated iff $X$ is finite.
\end{example}

\subsection{Free groups}
Every vector space admits LI generating sets that are as free as possible, i.e. no relations between them. However most group do not.
\begin{definition}[Free groups, universal property]
    Let $S$ be a set. A group $F$ is \textbf{freely generated} by $S$ if $F$ has the following universal property: For any group $G$ and any map $\varphi  \colon S \to G$ there is a unique group homomorphism $\overline{\varphi }\colon F \to G$ extending $\varphi $:
    \[
\begin{tikzcd}
S \arrow[r, "\varphi"] \arrow[d, hook]      & G \\
F \arrow[ru, "\overline{\varphi}"', dotted] &  
\end{tikzcd}
    \] A group is \textbf{free} if it contains a free generating set.
\end{definition}
\begin{example}
    $\Z$ is freely generated by $\{1\} $, but is not freely generated by $\{2,3\} $ or $\{2\} $ or $\{3\} $.
\end{example}
\begin{prop}
   Let $S$ be a set. Then, up to canonical isomorphism, there is at most one group freely generated by $S$. 
\end{prop}
\begin{proof}
    Let $F,F'$ be two groups freely generated by $S$, $\varphi  \colon S \hookrightarrow F, \varphi ' \colon S \hookrightarrow F'  $. Because $F$ is freely generated by $S$, the existence part of the universal property guaranteed the existence of a group homomorphism $\overline{\varphi }' \colon F \to F'$ such that $\overline{\varphi }' \circ \varphi  = \varphi '$. Analogously, there is a group homomorphism $\overline{\varphi }\colon F' \to F$ satisfying  $\overline{\varphi }\circ \varphi ' = \varphi $.
    \[
    \begin{tikzcd}
S \arrow[r, "\varphi'"] \arrow[d, "\varphi"', hook] & F' \\
F \arrow[ru, "\overline{\varphi}'"', dotted]        &   
\end{tikzcd}\quad
    \begin{tikzcd}
S \arrow[r, "\varphi"] \arrow[d, "\varphi'"', hook] & F \\
F' \arrow[ru, "\overline{\varphi}'", dotted]        &   
\end{tikzcd}
    \] Composing $\overline{\varphi } \circ \overline{\varphi }' \colon F \to F$, the universal property tells us that this must be $\id_F$ since it also fits into the diagram. These isomorphism are canonical since they induce the identity on $S$, and are the only isomorphisms extending the identity on $S$.
\end{proof}
\begin{theorem}
    Let $S$ be a set. Then there exists a group freely generated by $S$.
\end{theorem}
\begin{proof}
    {\color{red}todo:} 
\end{proof}
\begin{cor}
    Let $F$ be a free group, and let $S$ be a free generating set of $F$. Then $S$ generates  $F$.
\end{cor}
\begin{prop}
    Let $F$ be a free group.
    \begin{enumerate}[label=(\arabic*)]
    \setlength\itemsep{-.2em}
        \item Let $S \subseteq F$ be a free generating set of $F$ and let $S'$ be a generating set of $F$. Then $|S'| \geq |S|$.
        \item In particular, all generating sets of $F$ have the same cardinality, called the \emph{rank} of $F$.
    \end{enumerate}
    \begin{definition}[]
        Let $n \in \N$ and let $S = \{x_1, \cdots ,x_n \} $. We write $F_n $ for ``the'' group freely generated by $S$, and call $F_n $ the \textbf{free group of rank} $n$.
    \end{definition}
\end{prop}
Note that while subspaces of vector spaces cannot have bigger dimension than the ambient space, free groups of rank 2 contain subgroups isomorphic to free groups of higher rank, even free subgroups of (countably) infinite rank. 
\begin{cor}
    A group is finitely generated iff it is the quotient of a finitely generated free group, i.e., a group $G$ is finitely generated iff there exists a finitely generated free group $F$ and a surjection $F \to G$.
\end{cor}
\begin{proof}
    Quotients of finitely generated groups are finitely generated. Conversely, let $G$ be finitely generated by $S \subseteq G$, and let $F$ be the free group generated by $S$. Using the universal property of $F$ we have a group homomorphism $\pi \colon F\to G$ that is the identity on $S$. Because $S$ generates $G$ and $S$ lies in the image of $\pi$, it follows that $\im \pi = G$.
\end{proof}
This essentially tells us that all groups can be written in presentation form.

\subsection{Generators and relations}
\begin{definition}[]
    Let $G$ be a group and let $S \subseteq G$ be a subet. The \textbf{normal subgroup of} $G$ \textbf{generated by} $S$ is the smallest (wrt inclusion) normal subgroup of $G$ containing $S$, denoted by $\langle S \rangle ^{\trianglelefteq }_G$.
\end{definition}
\begin{remark}
    Explicitly, let $G$ be a group and $S \subseteq G$. Then $\langle S \rangle ^{\trianglelefteq }_G$ always exists and can be described as follows: \[
    \langle S \rangle ^{\trianglelefteq }_G = \bigcap \{H \mid H \subseteq G  \text{ is a normal subgroup with } S \subseteq H\} = \{g_1 s_1 ^{\varepsilon 1}g_1 ^{-1} \cdots g_n s_n  ^{\varepsilon _n }g_n^{-1} \mid  n \in \N, s_i  \in S, \varepsilon  \in \{\pm 1\} , g_i  \in G\} .
    \] 
\end{remark}
\begin{example}
    All subgroups of abelian groups are normal and $\langle S \rangle _G ^{\trianglelefteq }= \langle S \rangle _G$. Considering $S_3$ and $\tau = (1\ 2)\in S_3$, we have $\langle \tau \rangle _{S_3}= \{\id _{1,2,3},\tau\} $ and $\langle \tau \rangle _{S_3}^{\trianglelefteq }= S_3$.
\end{example}
If $G$ is a group and $N \trianglelefteq G$, in general it is difficult to determine the minimal number of elements of $S \subseteq G$ that satisfies $\langle S \rangle _G ^{\trianglelefteq }= N$. Let $A^*$ denote the set of words.

\begin{definition}[]
    Let $S$ be a set, and let $R \subseteq (S \cup  S^{-1})^*$ be a subset; let $F(S)$ be the free group generated by $S$. Then the group \[
        \langle S \mid R \rangle := F(S) / \langle R \rangle ^{\trianglelefteq }_{F(S)}
    \] is said to be \textbf{generated by} $S$ \textbf{with the relations} $R$; if $G$ is a group with $G \cong  \langle S\mid R \rangle $, then $\langle S \mid R \rangle $ is a \textbf{presentation} of $G$.
\end{definition}
\begin{example}
    For all $n \in \N$, we have $\langle x\mid x^n  \rangle \cong \Z /n$ (universal property or explicit construction). We have $\langle x,y \mid |xyx^{-1}y^{-1} \rangle \cong \Z^2$ (universal property).
\end{example}
\begin{example}
    Let $n \in \N _{\geq 3}$ and let $X_n  \subseteq \R^2$ be a regular $n$-gon (inheriting the induced Euclidian metric). Then \[
        \mathrm{Isom}(X_n ) \cong \langle s,t \mid s^n , t^2, tst^{-1}=s^{-1} \rangle =: D_n .
    \] Geometrically $s$ correspondings to a  $2\pi / n$ rotation about the center and $t$ reflects across the diameter passing through one of the vertices.
\end{example}
{\color{red}todo:thompson's group} 
\begin{example}
    For $m,n \in \N _{>0}$ , the \textbf{Baumslag-Solitar group} $\mathrm{BS}(m,n)$ is defined via the presentation 
    \[
    \mathrm{BS}(m,n) := \langle a,b \mid  b a^m b^{-1} = a^n  \rangle .
\] For instance, $\mathrm{BS}(1,1) \cong  \Z^2$. The family of Baumslag-Solitar groups contain many intriguing examples of groups. For example, $\mathrm{BS}(2,3)$ is a group with only two generators and one relation that is \emph{non-Hopfian}, i.e., there exists a surjective group homomorphism $\mathrm{BS}(2,3) \to \mathrm{BS}(2,3)$ that is \emph{not} an isomorphism, namely the homomorphism given by \[
\mathrm{BS}(2,3) \mapsto  \mathrm{BS}(2,3), \quad a \mapsto a^2, \quad b\mapsto b.
\] Proving that this homomorphism is not injective requires more advanced techniques.
\end{example}
\begin{example}
    The group \[
    G:= \langle x, y \mid xyx^{-1} = y^2, yxy^{-1} = x^2 \rangle 
    \] is trivial. Write $x = xyx ^{-1} x y^{-1}$, using the relation $xyx ^{-1} = y^2$ we have $x= y^2 x y^{-1}= yyxy^{-1}$. Using the relation $yxy^{-1}=x^2$ we have $x=yx^2$, so $x =y ^{-1}$. So $y ^{-2}=x^2=yxy^{-1}=yy^{-1}y^{-1}=y^{-1}$, and $x=y=e$.
\end{example}
\begin{definition}[]
    A group $G$ is \textbf{finitely presented}  if there exists a finite generating set $S$ and a finite set $R \subseteq (S \cup S^{-1}) ^*$ of relations that $G \cong \langle S \mid R \rangle $.
\end{definition}
Clearly any finitely presented group is finitely generated. The converse is not always true.
\begin{example}
   The group \[
       \langle s,t \mid \{[t ^n  s t ^{-n}, t^m s t ^{-m}] \mid  n,m \in \Z\}  \rangle 
   \]  is finitely generated but not finitely presented.
\end{example}

\subsection{New groups of old}
\begin{definition}[]
    Let $I$ be a set and $\left( G_i  \right) _{i \in I}$ be a family of groups. The \textbf{direct product group}  $\prod _{i \in I}G_i $ of $(G_i ) _{i \in I}$ is the group whose underlying set is the cartesian product $\prod _{i \in I}$ and whose composition is componentwise: \[
        \prod _{i \in I}G_i  \times \prod _{i \in I}G_i  \to  \prod _{i \in I}G_i ,\quad \left( \left( g_i  \right) _{i \in I}, \left( h_i  \right) _{i \in I} \right) \mapsto  \left( g_i  \cdot h_i  \right) _{i \in I}.
    \] 
\end{definition}
\begin{definition}[]
    Let $Q$ and $N$ be groups. An \textbf{extension} of $Q$ by $N$ is an exact sequence \[
    I \to N \xrightarrow iG \xrightarrow{\pi} Q \to 1
\] of groups. Note that this implies $Q \cong G / \im(N)$.
\end{definition}
\begin{definition}[]
    Let $N$ and $Q$ be groups, and $\varphi  \colon Q \to \Aut(N)$ be a group homomorphism (i.e., $Q$ acts on $N$ via $\varphi $). The \textbf{semi-direct product} of $Q$ by $N$ with respect to $\varphi $ is the group $N \rtimes_{\varphi }Q$ whose underlying set is the cartesian product $N \times Q$ and whose composition is  \[
        (N \rtimes _{\varphi }Q)\times (N \rtimes _{\varphi }Q) \to (N \rtimes _{\varphi }Q), \quad \left( \left( n_1,q_1 \right)  ,\left( n_2,q_2 \right) \right) \mapsto  (n_1 \cdot \varphi (q_1)(n_2), q_1 \cdot q_2)
    \] 
\end{definition}
\begin{remark}
    A group extension $1 \to N \xrightarrow iG \xrightarrow{\pi} Q \to 1$ \textbf{splits} if there exists a group homomorphism $s \colon Q \to G$ such that $\pi \circ s = \id_Q$. If $\varphi  \colon Q \to \Aut(N)$ is a homomorphism, then \[
    1 \to N \xrightarrow iN \rtimes _{\varphi }Q \xrightarrow{\pi} Q \to 1
\] is a split extension with the split given by $Q \to N \rtimes _{\varphi }Q, \ q \mapsto (e,q)$, since $\pi$ is just projection onto $Q$. Conversely, for a split extension, the extension is a semi-direct product of the quotient by the kernel. Let $1 \to  N \xrightarrow iG \xrightarrow{\pi} Q \to 1$ be a group extension splitting by $s \colon Q \to G$. Then \[
N \rtimes _{\varphi }Q \rightleftarrows G,\quad (n,q) \mapsto  n \cdot s(q), \quad g \mapsto  (g \cdot (s \circ \pi(g))^{-1},\pi(g)) 
\] are well-defined mutually inverse group homomorphisms, where $\varphi  \colon Q \to \Aut(N), q \mapsto \left(n \mapsto  s(q) \cdot n \cdot s(q) ^{-1}\right)$. However, there are group extensions that do \emph{not} split, for example the extension \[
1 \to \Z \xrightarrow 2\Z \to \Z /2 \to 1
\] does not split because there is no non-trivial homomorphism from the torsion group $\Z /2$ to $\Z$.
\end{remark}
\begin{example}
    Some examples of semi-direct products:
    \begin{itemize}
    \setlength\itemsep{-.2em}
        \item 
            If $N,Q$ are groups and $\varphi \colon Q\to  \Aut(N)$ is the trivial homomorphism, then the identity map yields and isomorphism $N \rtimes _{\varphi }Q \cong  N xQ$.
        \item Let $n \in  \N _{\geq 3}$. Then the dihedral group $D_n  = \langle s,t \mid s^n , t^2, tst^{-1}=s ^{-1} \rangle $ is a semi-direct product \[
                D_n \leftrightarrows \Z /n \rtimes _{\varphi } \Z /2 , \quad s \mapsto ([1], 0), \quad t \mapsto (0, [1])
            \]  where $\varphi  \colon \Z /2 \to \Aut(\Z /n)$ is given by multiplication by $-1$. Similarly, the infinite dihedral group $D_{\infty}=\langle s,t \mid t^2 , tst ^{-1} = s ^{-1} \rangle \cong  \mathrm{Isom}(\Z)$ can be written as a semi-direct product of $\Z /2$ by $\Z$ with respect to multiplication by $-1$.
        \item Semi-direct products of $\Z ^n  \rtimes _{ \varphi } \Z$ lead to interesting examples of groups provided the automorphism $\varphi (1) \in \mathrm{GL}(n,\Z) \subseteq \mathrm{GL}(n, \R)$ is chosen suitably, e.g., if $\varphi (1)$ has interesting eigenvalues.
        \item Let $G$ be a group. Then the \textbf{lamplighter group over} $G$ is the semi-direct product group $\left( \prod _{\Z} G\right) \rtimes _{\varphi }\Z$, where $\Z$ acts on the product $\prod _{\Z}G$ by shifting the factors: \[
                \varphi  \colon \Z \to \Aut\left( \prod _{\Z}G \right) , \quad z \mapsto \left( (g_n ) _{n \in \Z} \mapsto (g _{n+z}) _{n \in \Z} \right) .
        \] 
    \item More generally, the \textbf{wreath product} of two groups $G$ and $H$ is the semi-direct product $\left( \prod _HG \right) \rtimes _{\varphi }H$, where $\varphi $ is the shift action of $H$ on $\prod _HG$. The wreath product of $G$ and $H$ is denoted by $G \wr H$.
    \end{itemize}
\end{example}

\subsection{Free products and free amalgamated products}
We now described a construction that ``glues'' two groups along a common subgroup, analogous to the universal property of pushouts in category theory.

\begin{definition}[]
    Let $A$ be a group, and let $\alpha_1 \colon A \to G_1$ and $\alpha _2 \colon A \to G_2$ be group homomorphisms. A group $G$ together with homomorphisms $\beta_1 \colon G_1 \to G $ and $\beta_2 \colon G_2 \to G$ satisfying $\beta_1 \circ \alpha_1= \beta_2 \circ \alpha_2$ is called an \textbf{amalgamated free product} of $G_1$ and $G_2$ over $A$ (wrt $\alpha_1$ and $\alpha_2$) if the following universal property is satisfied: for any group $H$ and any two group homomorphisms $\varphi_1 \colon G_1 \to H$ and $\varphi_2 \colon G_2 \to H$ with $\varphi_1 \circ \alpha_1 = \varphi_2 \circ \alpha_2$ there is exactly one homomorphism $\varphi  \colon G \to H$ of groups with $\varphi  \circ \beta_1 = \varphi  \circ \beta_2$.
\[
\begin{tikzcd}
                                                 & G_1 \arrow[rd, "\beta_1"] \arrow[rrrd, "\varphi_1", bend left]    &                                 &  &   \\
A \arrow[ru, "\alpha_1"] \arrow[rd, "\alpha_2"'] &                                                                   & G \arrow[rr, "\varphi", dotted] &  & H \\
                                                 & G_2 \arrow[ru, "\beta_2"'] \arrow[rrru, "\varphi_2"', bend right] &                                 &  &  
\end{tikzcd}
\] 
    Such a free product with amalgamation is denoted by $G_1 *_A G_2$. If $A$ is the trivial group, we write $G_1 * G_2 := G_1 *_A G_2$ and call $G_1 *G_2$ the \textbf{free product} of $G_1$ and $G_2$.
\end{definition}

\begin{example}
    Free groups can be viewed as free products of several copies of $\Z$; e.g., the free group of rank 2 is $\Z * \Z$.
\item $D _{\infty}\cong \mathrm{Isom}(\Z)$ is isomorphic to the free product $\Z /2 * \Z /2$, for instance, reflection at 0 and reflection at 1/2 provide generators of $D _{\infty}$ corresponding to the obvious generators of $\Z /2 * \Z /2$.
\item The matrix group $\mathrm{SL}(2, \Z)$ is isomorphic to the free amalgamated product $\Z /6 *_{\Z /2}\Z /4$.
\item Free amalgamated products occur naturally in topology: by van Kampen, $\pi_1$ of a pointed space glued together of several components is a free amalgamated product of $\pi_1$ of the components over $\pi_1$ of the intersection.
\end{example}
\begin{theorem}
    All free products with amalgmation exist and are unique up to canonical isomorphism.
\end{theorem}

\begin{definition}[HNN-extension]
    Let $G$ be a group, let $A,B \subseteq G$ be two subgroups, and let $\vartheta \colon A \to B$ be an isomorphism. Then the \textbf{HNN-extension of} $G$ \textbf{with respect to} $\vartheta$ is the group \[
        G * _{\vartheta} := \left\langle \left\{ x_g \mid g \in G \right\} \amalg \{t\} \mid  \{t ^{-1} x_a t = x _{\vartheta(a)}\mid a \in A\} \cup R_G  \right\rangle 
    \] where \[
    R_G := \{x_g x_h x_k ^{-1} \mid g,h ,k \in G \text{ with } g \cdot h =k \text{ in } G\} .
    \] Using an HNN-extension, we can force two given subgroups to be conjugate. Topologically they arise as fundamental groups of mapping tori of maps that are injective on the level of fundamental groups.
\end{definition}
