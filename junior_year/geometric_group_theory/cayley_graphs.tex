\section{Groups to geometry I: Cayley graphs}

\subsection{Review of graph notation}
Unless otherwise stated, graphs are undirected and simple.
\begin{definition}[]
    A \textbf{graph} is a pair $G = \left( V,E \right) $ of disjoint sets where $E$ is a set of subses of $V$ that contain exactly two elements, i.e., \[
        E \subseteq V ^{[2]} := \{e \mid e \subseteq V, |e| =2 \} .
    \] The elements of $V$ are the \textbf{vertices}, the elements of $E$ are the \textbf{edges} of $G$.
\end{definition}
Graphs are a different POV on relations, and graphs are used to model relations. Classical graph theory has many applications, notably in anything involving networks (computer science).

\begin{definition}[]
    Let $(V,E)$ be a graph.
    \begin{itemize}
    \setlength\itemsep{-.2em}
        \item We say two vertices $v,v' \in V$ are \textbf{neighbors} or \textbf{adjacent} if they are joined by an edge, i.e., if $\{v, v' \} \in  E$.
        \item The number of neighbours of a vertex is the \textbf{degree} of this vertex.
        \item A \textbf{complete} graph has all vertices neighbors of each other.
    \end{itemize}
\end{definition}
\begin{definition}[]
    Let $G= (V,E)$ and $G'=(V',E')$ be graphs. The graphs $G$ and $G'$ are \textbf{isomorphic} if there is a \textbf{graph isomorphism} between $G$ and $ G'$, i.e.,  a bijection $f \colon V \to V'$ such that for all $v,w \in V$ we have $\{v,w \} \in E$ if and only if $\{f(v),f(w)\} \in E'$; i.e., isomorphic graphs only differ in the labels of the vertices.
\end{definition}

\begin{definition}[]
    Let $G= (V,E)$ be a graph.
    \begin{itemize}
    \setlength\itemsep{-.2em}
        \item  Let $n \in \N \cup \{\infty\} $. A \textbf{path} in $G$ of length $n$ is a sequence $v_0, \cdots ,v_n $ of different vertices $v_0, \cdots ,v_n \in V$ with the property that $\{v_j , v_{j+1}\}  \in E$ holds for all $j \in \{0,\cdots ,n-1\} $; if $n < \infty$, then we say that this path \textbf{connects} the vertices $v_0$ and $v_n $.
        \item The graph $G$ is called \textbf{connected} if any tow of its vertices can be connected by a path in $G$.
        \item Let $n \in \N _{>2}$. A \textbf{cycle} in $G$ of length $n$ is a sequence $v_0, \cdots ,v_{n-1}$ of different vertices $v_0, \cdots ,v_{n-1} \in  V$ with $\{v_{n-1},v_0\} \in  E$ and moreover $\{v_j , v_{j+1}\} \in E$ for all $j \in \{0,\cdots ,n-2\} $.
    \end{itemize}
\end{definition}
\begin{definition}[]
    A \textbf{tree} is a connected graph containing no cycles. A graph containing no cycles is a \textbf{forest} (so a tree is a connected forest).
\end{definition}
\begin{prop}
    A graph is a tree iff for every pair of vertices there exists exactly one path connecting these vertices.
\end{prop}
\begin{proof}
    Let $G$ be a graph such that for every pair of vertices can be connected by exactly one path, so $G$ is connected. Assume that $G$ contains a cycle $v_0, \cdots ,v_{n-1}$; because $n>2$, the two paths $v_0, v_{n-1}$ and $v_0, \cdots ,v_{n-1}$ are different. So $G$ is a tree.

    Conversely, let $G$ be a tree, so $G$ is connected. Assume that there exists two vertices that can be connectd by differing paths $p, p'$. We can construct a cycle from these two paths, a contradiction. So every two vertices can be connected by exactly one path in $G$.
\end{proof}

\subsection{Cayley graphs}
\begin{definition}[]
    Let $G$ be a group and let $S \subseteq G$ be a generating set of $G$. Then the \textbf{Cayley graph} of $G$ with respect to $S$ is the graph $\mathrm{Cay}(G,S)$ whose 
    \begin{itemize}
    \setlength\itemsep{-.2em}
        \item set of vertices is $G$, and whose
        \item set of edges is \[
                \left\{\{g, g\cdot s\} \mid g \in G, s \in (S \cup S^{-1}) \setminus \{e\} \right\} .
        \] 
    \end{itemize}I.e., two vertices in a Cayley graph are adjacent iff they differ by an element of the generating set in question.
\end{definition}
\begin{example}
    Some examples:
    \begin{itemize}
    \setlength\itemsep{-.2em}
        \item Consider the Cayley graphs of $\Z$ wrt $\{1\} $ and $\{2,3\} $. From ``far away'' these have the same global structure (the real line). In more technical terms, these graphs are quasi-isometric wrt to the corresponding word metrics. 
        \item The Cayley graph of $\Z^2$ wrt $\{(1,0), (0,1)\} $ looks like the integer lattice in $\R^2$, when viewed far away it looks like the Euclidean plane.
        \item The Cayley graph of the cyclic group $\Z /6$ looks like a cycle graph.
        \item Let $\tau = (1\ 2) \in S_3, \sigma = (1\ 2\ 3) \in S_3$. The Cayley graph of $S_3$ wrt $\{\tau, \sigma\} $ looks like two triangles. The Cayley graph of $S_3$ is a complete graph on six vertices; similarly, $\mathrm{Cay}(\Z /6, \Z /6)$. In particular, we see that non-isomorphic groups may have isomorphic Ccayley graphs wrt certain generating sets. However, the isomorphism type of any Cayley graph of a finitely generated abelian group is still rigid enough to remember the size of the torsion part.
        \item The Cayley graph of a free group wrt to a free generating set is a tree.
    \end{itemize}
\end{example}
\begin{remark}
    Some elementary properties of Cayley graphs:
    \begin{enumerate}[label=(\arabic*)]
    \setlength\itemsep{-.2em}
        \item Cayley graphs are connect as each vertex $g$ can be reached from the vertex of $e$ by walking along the edges corresponding to a presenttaion of minimal length of $g$ in terms of the given generators.
        \item Cayley graphs are regular in the sense that every vertex has the same number $|(S \cup S^{-1})\setminus \{e\}| $ of neighbors.
        \item A Cayley graph is locally finite iff the generating set is finite; a graph is said to be \textbf{locally finite} if each vertex has only finitely many neighbors.
    \end{enumerate}
\end{remark}
\begin{remark}
    They are higher dimensional analogues of Cayley graphs in topology; associated with a presentation of a group, these is the \emph{Cayley complex}, a 2-dimensional object. More generally, every group admits a \emph{classifying space} whose fundamental group is the given group, and higher dimensional homotopy groups are trivial. These spaces allow us to model group theory in topology and play an important role in the study of group cohomology. Note that Cayley graphs and complexes require additional data on the group (generating sets and presentations) and, viewed as combinatorial objects, are only functorial with resepct to maps/homomorphism respecting this additional data.
\end{remark}

\subsection{Cayley graphs of free groups}
A combinatorical characterisation of free groups can be given in terms of trees:
\begin{theorem}
    Let $F$ be a free group, freely generated by $S \subseteq F$. Then the corresponding Cayley graph $\mathrm{Cay}(F,S)$ is a tree.
\end{theorem}
\begin{example}
The converse is \emph{not} true in general.
\begin{itemize}
\setlength\itemsep{-.2em}
    \item The Cayley graph $\mathrm{Cay}(\Z/2, [1])$ consists of two vertices joined by an edge; clearly this graph is a tree, but the group $\Z /2$ is not free.
    \item The Cayley graph $\mathrm{Cay}(\Z, \{-1,1\} $ coincides with $\mathrm{Cay}(\Z,\{1\} )$, which is a tree. But $\{-1,1\} $ is not a free generating set of $\Z$.
\end{itemize}    
However these are the only two things that can go wrong.
\end{example}
\begin{theorem}
    Let $G$ be a group and let $S \subseteq G$ be a generating set satisfying $s \cdot t \neq e$ for all $s,t \in S$. If the Cayley graph $\mathrm{Cay}(G,S)$ is a tree, then $S$ is a free generating set of $G$.
\end{theorem}

\subsection{Free groups and reduced words}
\begin{definition}[]
    Let $S$ be a set, and let $(S \cup \overline{S})^*$ a be the set of words over $S$ and formal inverses of elements of $S$.
    \begin{itemize}
    \setlength\itemsep{-.2em}
        \item Let $n \in \N$ and let $s_1, \cdots ,s_n  \in S \cup \overline{S}$. The word $s_1\cdots s_n $ is \textbf{reduced} if \[
        s _{j+1}\neq \overline{s_j } \quad \text{and} \quad \overline{s_{j+1}}\neq s_j 
    \] holds for all $j \in \{1,\cdots ,n-1\} $. (In particular, $\varepsilon $ is reduced).
\item We write $F_{\mathrm{red}}(S)$ for the set of all reduced words in $(S \cup \overline{S})^*$.
    \end{itemize}
\end{definition}

\begin{prop}
    Let $S$ be a set. The set $F_{\mathrm{red}}(S)$ of reduced words over $S \cup \overline{S}$ forms a group wrt the composition $F_{\mathrm{red}}(S) \times F _{\mathrm{red}}(S) \to F_{\mathrm{red}}(S)$ given by \[
        (s_1 \cdots s_n ,s_{n+1}\cdots s_m) \mapsto  (s_1 \cdots s_{n-r}s _{n+1+r}\cdots s_{n+m})
    \] where $s_1 \cdots s_n , s_{n+1}\cdots s_m \in F _{\mathrm{red}}(S)$, and \[
    r:= \max \{k \in \{0, \cdots ,\min(n,m-1)\} \mid \forall _{j \in \{0,\cdots ,k-1\} }\ s_{n-j}= \overline{s _{n+1+j}} \ \vee \  \overline{ s _{n-j}}= s _{n+1+j}\} .
\] Furthermore, the group $F _{\mathrm{red}}(S)$ is freely generated by $S$.tubhub day of or close to gametim
\end{prop}
\begin{cor}
    Let $S$ be a set. Any element of $F(S) = (S \cup \overline{S}) ^* / \sim $ can be represented by exactly one reduced word over $S \cup \overline{S}$.
\end{cor}
\begin{cor}
    The word problem in free groups wrt to free generating sets is solvable; consider and compare reduced words.
\end{cor}
\begin{remark}
    Using the same method proof, one can describe free products $G_1 *G_2$ of groups $G_1$ and $G_2$ by reduced words; in this case, one calls a word  \[
        g_1 \cdots g_n  \in (G_1 \sqcup G_2)^*
    \] with $n \in \N$ and $g_1 ,\cdots ,g_n  \in G_1 \sqcup G_2$ \textbf{reduced} if for all $j \in \{1, \cdots ,n-1\} $, either $g_j  \in G_1 \setminus \{0\} $ and $g _{j+1} \in G_2 \setminus \{e\} $, or $g_j  \in G_2 \setminus \{e\} $ and $g_{j+1} \in G_1 \setminus \{e\} $.
\end{remark}

\subsection{Free groups to trees}
\begin{proof}[Proof that Cayley graphs of free groups are trees.]
    Suppose the group $F$ is freely generated by $S$. Then $F$ is isomorphic to $F_{\mathrm{red}}(S)$ via an isomorphism that is the identity on $S$; WLOG we can assume that $F$ is $F_{\mathrm{red}}(S)$. Because $S$ generates $F$, the graph $\mathrm{Cay}(F,S)$ is connected. Assume that $\mathrm{Cay}(F,S)$ contains a cycle $g_0,\cdots ,g_{n-1}$ of length $n$ with $n\geq 3$; in particular, the elements $g_0, \cdots ,g_{n-1}$ are distinct, and \[
        s _{j+1}:= g_{j+1} \cdot  g_j ^{-1} \in S \cup S^{-1}
    \] for all $j \in \{0,\cdots ,n-2\} $, as well as  $s _n := g_0 \cdot g_{n-1}^{-1} \in S \cup S^{-1}$. Because the vertices are distinct, the word $s_0 \cdots s_{n-1}$ is reduced; on the other hand, we obtain \[
    s _n \cdots s_1 = g_0 \cdot g_{n-1}^{-1} \cdots  \cdots g_2 \cdot g_1 ^{-1} *g_1 \cdot g_0 ^{-1} = e = \varepsilon 
\] in $F= F_{\mathrm{red}}(S)$, a contradiction. Therefore $\mathrm{Cay}(F,S)$ cannot contain any cycles, and is a tree.
\end{proof}

\subsection{Trees to free groups}

