\section{Probability review} 
Do the homeworks (homework 0 is a syllabus quiz, homeworks 1 and 2 are probability reviews). We will not review sample spaces, conditional probability, etc; look at the videos on the course website.
\begin{definition}[Cumulative distribution]
    For any random variable $X$, the \textbf{cumulative distribution function} (cdf) of $X$ is a function $F_X \colon \R \to [0,1] $, where $F_X (x) = \P[X \leq x]$ for all $x \in \R$.
\end{definition}
{\color{red}todo:valid cdf figure}. The cumulative distribution function gives us complete information about the distribution of a random variable. This is a nifty way to encode things in one function/one object. How do we know that the cumulative distribution function is well defined?
\begin{namedthing}{Question} 
    What is $\lim _{x \to - \infty}F_X(x)$? This is zero because as we let $x$ approach $- \infty$, the less room our probability has to land. OTOH what is $\lim _{x \to + \infty}F_X(x)$? This is one because as we let $x$ approach $+ \infty$, the more room our probability has to land.
\end{namedthing}
\begin{note}
    The cumulative distribution function is non-decreasing (monotone increasing). Any function with these properties that is right continuous is the cumulative distribution function of some random variable.
\end{note}
\begin{namedthing}{Question} 
    What if your cdf is a step function (piecewise flat)? Then your random variable is discrete, and can take countably many values.
\end{namedthing}
Most of the things in the graph of a cdf are useless. The only useful things are where the jumps happen and how big the jumps are. When there is a jump those are the values that the probability can take, and the size is such probability. Taking this data and condensing it gives us the \textbf{probability mass function} (pmf). It is usually more convenient to express the distribution of a discrete random variable using this probability (mass) function.

In general, the \textbf{support} $\mathrm{supp}(X)$ of a random variable $X$ is (vaguely) the set of all the values it can take. In the discrete case, the support is where the jumps in the cdf happen. 
\begin{definition}[Probability mass function]
    For these points $x \in \mathrm{supp}(X)$, the pmf is defined as $p_X (x) = \P [X = x]$, which is the size of the jump, or $F_X(x) - F_X (x -)$ (left limit).
\end{definition}
What is the simplest random variable one can consider (non-deterministic)? This is the Bernoulli trial, or the coin toss. If we know the exact outcome of a trial then probability is not interesting, we say it's deterministic and the outcomes are degenerate. 
\begin{definition}[Bernoulli distribution]
    Our first non-trivial example is a Bernoulli trial. There are only two possible outcomes, more precisely, the support of an $X$ with the \textbf{Bernoulli distribution} is $\{0,1\} $. We usually interpret ``1'' as ``success'' and ``0'' as ``failure''. We denote the probability of success in a single Bernoulli trial by $p$\footnote{Representing \emph{p}roportion of successes.}. 
\end{definition}
\begin{namedthing}{Notation} 
    We write $X\sim \mathrm{Bernoulli}(p)$ for \[
   X\sim 
   \begin{cases}
       1 & \text{with probability}  \ p,\\
       0 & \text{with probability} \ 1-p.
   \end{cases}
    \]We can also say that $p_X(1) = p, p_X(0)= 1-p$. Drawing the cdf, we know for sure this is a step function since there are only two points in the support; a line at 0 from $(-\infty,0)$. Then there is a line at $1-p =q$ from $[0,1)$, and finally a line at 1 from $[1,\infty)$.
\end{namedthing}
\begin{definition}[Binomial distribution]
    This models the number of successes in a set of \emph{independent} identically distributed (set of probabilities is the same) Bernoulli trials. Denote the probability of success in a single trial as $p$, and $n$ denote the number of trials. If $Y$ is \textbf{binomial}, then we write $Y \sim \mathrm{Binomial}(n,p)$. We have $\mathrm{supp}(Y) = \{0,1, \cdots ,n\} $ (all successes leads to  $n$), and the pmf of $Y$ is given by $p_Y(k)= {n \choose k} p^k(1-p) ^{n-k}$ (success is $p^k$, failure is $(1-p)^{n-k}$, choice is ${n \choose k} $).
\end{definition}
