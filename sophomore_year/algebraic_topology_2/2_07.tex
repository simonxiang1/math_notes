\section{Classifying spaces for topological categories and groups}
Last time we got classifying spaces for small categories, then groups. Today we do it for topological categories, and get a special class of spaces called ``Eilenberg-Maclane'' spaces, which arise from cohomology functors. We explain them in terms of principle bundles, which may be slightly sketchy.

Consider a small category $\mathcal{C} $. We have a pair of sets $\mathrm{ob}(\mathcal{C} ), \mathrm{mor}(\mathcal{C} )$, which come with structural maps $s,t \colon \mathrm{mor}(\mathcal{C} ) \to \mathrm{ob}(\mathcal{C} )$ (source and target maps), where $s \mathcal{C} (X,Y)=X$ and $t\mathcal{C} (X,Y)=Y$. There is an identity morphism $e \colon \mathrm{ob}(\mathcal{C} ) \to \mathrm{mor}(\mathcal{C} ), X \mapsto  e_X \in \mathcal{C} (X,X)$. Finally, there is composition $\mathrm{mor}\mathcal{C} \times_{\mathrm{ob}\mathcal{C} } \mathrm{mor}(\mathcal{C}) \to \mathrm{mor}(\mathcal{C}) $, which is the set $\{\left( f,g \right) \in \mathrm{mor}(\mathcal{C} ) \times \mathrm{mor}(\mathcal{C} ) \mid tg=sf\} $ (which is associative and unital). This is a way of structuring the small category, and a \textbf{topological} category $\mathcal{C} $ is one where $\mathrm{ob}(\mathcal{C} ),\mathrm{mor}(\mathcal{C} )$ are \emph{spaces}, and the four structural morphism are \emph{continuous}.
\begin{example}
    A topological group (or monoid) $G$ defines a 1-object topological category.
\end{example}
To a topological category $\mathcal{C} $, we can attach to it a ``classifying space'' $B\mathcal{C} $. Consider the nerve $N_{\bullet}\mathcal{C} $, which is the simplicial set with $N \mathcal{C} _n = \{ \text{functors} [n] \to \mathcal{C} \} = $ chains of $n$ composable morphisms $= \mathrm{mor}(\mathcal{C} ) \times _{\mathrm{ob}(\mathcal{C} )}\cdots \times _{\mathrm{ob}(\mathcal{C} )}\mathrm{mor}(\mathcal{C} )$ $n$ times, $N_0(\mathcal{C} )=\mathrm{ob}(\mathcal{C} )$. This is more than a simplicial set, it's a \textbf{simplicial space}, ie. a functor $\left( \mathsf{Ord}_{\leq} \right) ^{\mathrm{op}} \to \mathsf{Top} $. A simplicial space $X_{\bullet}$ still has a geometric realization $|X_{\bullet}|,$ where $|X_{\bullet}|= X_{\mathrm{pre}}/\sim$. Last time we had $X_{\mathrm{pre}}=\coprod _{n \geq 0}X_n  \times |\Delta ^n |$; this time the $X_n $ are spaces parametrizing $n$-simplices. For $\theta \colon [m] \to [n]$ increasing, we identify $(\theta^* \sigma,x)\sim(\sigma,\theta_*x)$.

From here, define the classifying space of our category to be the geometric realization of the nerve, or $B\mathcal{C} := |N_{\bullet}\mathcal{C} |$. For $X_{\bullet}$ a simplicial space, $|X_{\bullet}|$ is Hausdorff, but not a CW complex in general. The natural map $|(X \times Y)_{\bullet}| \to |X_{\bullet}| \times |Y_{\bullet}|$ is still a continuous bijection, and a homeomorphism when $|X_{\bullet}|$ is \emph{compact}. In general, $|X \times Y_{\bullet}| \to |X_{\bullet}| \times |Y_{\bullet}| \xrightarrow{\id} K(|X_{\bullet}| \times |Y_{\bullet}|)$. Here $K$ is the compactly generated topology, where a subset of closed if its intersection with a compact subspace is is closed in the original topology. We tacitly apply ``$K$'' to products as needed.

What do natural transformations induce? Suppose we have \emph{continuous} functors $F_0,F_0 \colon \mathcal{C}  \to \mathcal{DD} $ between topological categories, $\theta \colon \mathrm{ob}(\mathcal{C} ) \to \mathrm{mor}(\mathcal{D} )$, and a \emph{continuous} natural transformation $F_0 \implies F_1$. This still induces a homotopy between $BF_0 \simeq  BF_1 \colon B\mathcal{C}  \to B\mathcal{D} $. The same argument applies: $F_0,F_1, \theta$ lead to a continuous function $[1] \times \mathcal{C} \to \mathcal{D} $, which implies $B([1] \times \mathcal{C} ) \to B\mathcal{D} ,B[1]=I, [1]=I \times B\mathcal{C} $. 

\subsection{Classifying spaces of topological groups}
We apply this to topological groups. For now, let $G$ be a topological monoid, which leads to the spaces $N_{\bullet}G,BG$. Like in the discrete discussion, we can also form the translation category $tG$, where $\mathrm{ob}(tG)=G, tG(g_0,g_1)= \{h \in G \mid hg_1=g_2\} $, again a topological category under the subspace topology. We get a functor $tG\xrightarrow P G$ which induces $p=BP$, $B(tG) \to BG$. It is tradition to set $EG=B(tG)$, with projection $EG \xrightarrow pBG$. Arguing as before, $EG= B(tG) \simeq  B[1] = \text{point} $. So we find that $EG$ is contractible when $G$ is a topological \emph{group}.

\subsection{Eilenberg-Maclane spaces}
For $G,H$ topological monoids, we have $B(G \times H) \xrightarrow{\cong} BG \times BH, t(G\times H) =tG \times tH$. So $E(G \times H) \xrightarrow{\cong} EG \times EG$. Now say that $G$ is s commutative. Then the multiplication $ m \colon G \times G \to G$ is actually a homomorphism. So we get a map $Bm \colon BG \times BG \simeq  B(G \times G) \to BG$. $BG$ now has this binary operation $Bm$ which is associative, because $m$ is associative and $B$ is functorial. The unit $\{1\} \xrightarrow eG$ induces $Be \colon B \{1\} =\{\mathrm{pt}\}  \to BG$, i.e. $BG$ comes with a basepoint (0-simplex) $Be$. This is a unit for $Bm$. So $BG$ is a commutative topological monoid. We can iterate and define $B^n G=B(B^{n-1}G)$. This is essentially the definition of an Eilenberg-Maclane space.
\begin{definition}[E-M spaces]
    For $\pi$ a group made discrete, define a space $K(\pi,1)=B\pi$. For $\pi$ an abelian group, define $K(\pi,n)=B^n \pi$ for $n \geq 1$.
\end{definition}
For a commutative topological \emph{group} $G$, inversion $i \colon G \to G, g \mapsto  g^{-1}$ is a continuous homomorphism. This induces $B_i  \colon BG \to BG$ showing that $BG$ is a topological abelian group. This applies to E-M spaces $K(\pi,n)$ where $\pi$ is abelian. 
We save the principal bundle story for next time. Next time, we will outline (but not give a full proof) that if $\pi$ is a discrete group, $K(\pi,1)=B \pi$ can be characterized up to homotopy equivalence as any \emph{aspherical space} $X$ with $\pi_1(X) \simeq  \pi$. For instance, $K(\Z,1) \simeq S^1 $ since $\pi_1(S^1 )\simeq  \Z$.
Key points about $K(\pi,n)$; 
\begin{itemize}
\setlength\itemsep{-.2em}
    \item The homotopy groups $\pi_i K( \pi,n)$ turn out to be trivial if $i \neq n$, and  $\pi$ if $i=n$. From the perspective of homotopy groups these are a basic gadget.
    \item  We will discuss singular cohomology a lot going forward. The point is that $H^n (X; \Z) $ is identified with homotopy classes of maps $[X,K(\pi,n)]$. So Eilenberg-Maclane spaces are called representing spaces for cohomology functors.
\end{itemize}
