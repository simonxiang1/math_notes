\section{Filtrations and homology} 
Discussion section on Monday, 3-4 PM in room 12.166. The topology seminar has booked the room until half past three but it is unlikely they'll stay in there that long.

We previously asserted that the following chain maps are quasi-isomorphisms:
\begin{itemize}
\setlength\itemsep{-.2em}
    \item For $A_{\bullet}$ a simplicial abelian group, we have the quotient map $A_* \to NA_*=A_* / DA_*$ of the simplicial chains mod the degenerate chains a quasi-isomorphism.
    \item For $X_{\bullet }$ a simplicial set, the map from simplicial chains $C_*(X_{\bullet})=\Z X_* \to S_*(|X_{\bullet}|)$. 
\end{itemize}We lay out a homological framework for proving such assertions.
\subsection{Mapping cones}
There is a homological and topological version of this concept, and it is really the homological version we want to use. But we start with the topological version for context. Let $f \colon X \to Y$ be a map up spaces, then define the \textbf{mapping cone} by \[
    Cf:= (I \times X) \amalg Y / \sim, \quad (0,x)\sim f(x), (1,x)\sim(1,x')
\] for all $x,x' \in X$. In other words, this is the mapping cylinder with the top collapsed to a point, hence the cone. Then there is a pushout diagram \[
\begin{tikzcd}
X \arrow[r, "{x\mapsto (0,x)}", hook] \arrow[d, "f"'] & I\times X / \{1\} \times X \arrow[d] \arrow[rdd, bend left] &   \\
Y \arrow[r] \arrow[rrd, bend right]                   & Cf \arrow[rd, "\exists!", dotted]                           &   \\
                                                      &                                                             & Z
\end{tikzcd}
\]  A map $Cf\to Z$ corresponds to $g \colon Y \to Z, h \colon I \times X \to Z$, where $h$ is a nullhomotopy of $g \circ f$. Now we get to the homological version, where there is a close analogy to the topological situation. We have $f \colon C_* \to D_*$ a map of chain complexes. This leads to a new chain complex $\mathrm{cone}(f)$ over some ring $R$, where \[
(\mathrm{cone}f)_n = C_{n-1}\oplus D_n ,\quad \delta \colon (\mathrm{cone}f)_n  \to (\mathrm{cone}f)_{n-1}, \quad \delta(x,y)=(-\partial _C x, fx+\partial _D y).
\] From here we can see $\delta^2=0$. There is a notion of a shift of a chain complex; for $m \in \Z$,  $C_*[x]=C_{n+m}$. The boundary operator is $(-1)^m=(-1)^m \partial _C$. By construction, there is a short exact sequence of complexes \[
\begin{tikzcd}
0 \arrow[r] & D_* \arrow[r] & \mathrm{cone}f \arrow[r] & {C_*[-1]} \arrow[r] & 0
\end{tikzcd}
\] which leads to a long exact sequence in homology, given by \[
\begin{tikzcd}
\cdots \arrow[r] & H_qD \arrow[r] & H_q(\mathrm{cone}f) \arrow[r] & H_{q-1}C \arrow[r, "\text{connecting}"] \arrow[r, "\text{map}\ \beta "'] & H_{q-1}D \arrow[r] & \cdots
\end{tikzcd}
\] The connecting map $\beta $ actually turns out to be the induced map on homology $Hf$. What is $\beta [x]$? We have $\partial _Cx=9$, then we left $x$ to $(x,0) \in \mathrm{cone}f$. Apply $\delta$, then we get $(-\partial _Cx,fx)$. Project to $D_x$ to get $[fx]$. This is all obtained by following the general formula to find the connecting map of a long exact sequence.

\begin{lemma}
    $f$ is a quasi-isomorphism iff $H_*(\mathrm{cone}f)=0$.
\end{lemma}
\begin{proof}
    Use the LES. Recall that it's given by \[
    H_*D \to H_* \mathrm{cone f}\to H_{*-1}C \xrightarrow{Hf} H_{*-1}D \to \cdots 
    \] Subsitute zero in the appropiate places and apply exactness.
\end{proof}
So mapping cones convert quasi-isomorphisms into acyclic complexes. We haven't actually made the connection to the topological version; a chain map has the form $\mathrm{cone}f \xrightarrow{h+g}  E_*$, where $g \colon D_* \to E_*$, $ h \colon C_{*-1} \to E_*$. This is a chain map iff $g$ is a chain map and $h$ is a chain homotopy from 0 to $g \circ f$.
\begin{remark}
    Suppose we take a map of spaces $f \colon X \to Y$. Then there is an induced map $f_* \colon S_*X \to S_*Y$, so we can take $\mathrm{cone}(f_*)$. On the other hand, we could also take the singular chains of the topological cone $S_*(Cf)$. We claim there is a chain map from $\mathrm{cone}(f_* \colon S_*X \to S_*Y) \to S_*(Cf)$. We have $Y\lhook\joinrel\xrightarrow{i}Cf $, which implies there ia a map $i_* \colon S_*Y \to S_*(Cf)$. So there is a chain nullhomotopy of $i_* f_*$, which leads to $h \colon S_*X \to S_{*+1}(Cf)$, a nullhomotopy of $i_*f_*$. Moreover, this map is actually a quasi-isomorphism (we will need excision to prove this part).
\end{remark}

\subsection{Filtered complexes}
Let $C_*$ be a chain complex, and take an increasing filtration on it. This means we have a sequence of subcomplexes \[
\cdots \subset F_p \subset C_* \subset F_{p+1}C_* \subset \cdots \subset C_*
\] Why are filtrations worth talking about? They come up all over the place.
\begin{example}
    Say $X$ is a space with nested subspaces $X_p \subset X_{p+1}\subset \cdots \subset X$. Then the singular chains $S_*X$ are filtered by $F_p(S_*X)=\im (S_*X_p \to S_*X)$. For example, CW complexes are filtered by their skeleta, and preimages of the skeleta of the base of fiber bundles give an interesting filtration (leading to the Serre spectral sequence).
\end{example}

From $\{F_pC_*\} $ a filtered chain complex, we get an associated graded chain complex \[
    \mathrm{gr}(C_*)=\bigoplus_{p \in \Z}\mathrm{gr}_p(C_*)=F_p C_* / F_{p-1}C_*.
\] Even though $\mathrm{gr}C$ forgets a lot of information, it still knows a lot. Say we have a finite filtration $\cdots \subset 0 \subset\cdots \subset 0 \subset F_p C_* \subset\cdots  \subset F_{p'} C_*=C_* \subset C_* \subset\cdots $
\begin{lemma}
    If $\{F_p C_*\} $ is a finite filtration and $H_*(\mathrm{gr}C)=0$, then $H_*(C)=0$.
\end{lemma}
\begin{proof}
    The number of steps $p'-p$ is finite. The idea is to induct on the number of steps. Use the LES to get our result.
\end{proof}
\begin{theorem}
    If $f \colon C_* \to D_*$ is a chain map that preserving filtrations $\{F_pC_*\} $ and $\{F_pD_*\} $, and if the induced map $\mathrm{gr}C_* \to \mathrm{gr}D_*$ is a quasi-isomorphism, then $f$ is a quasi-isomorphism.
\end{theorem}
\begin{proof}
    $\mathrm{cone}f_*$ inherits a filtration from whose associated graded is acyclic by assumption. By the lemma,  $\mathrm{cone}f$ is acyclic, which exactly means that $f$ is a quasi-isomorphism. If you wanted you could also prove this directly with the five-lemma.
\end{proof}
