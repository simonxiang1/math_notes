\section{Cap and slant products, plus Poincar\'e duality}  
Previously: \[
    \boxed{H^* \ \text{is naturally a graded (unital) ring} } 
\] The big idea of today is the following: \[
\boxed{H_{- *}\left( X \right) \ \text{is naturally a graded (unital) module over} \ H^* X} 
\] Recall the cup product $\smile$ on  $H^*$ comes from 
\begin{itemize}
\setlength\itemsep{-.2em}
    \item dualized $E-Z$ Zilber map, where $E-Z = \lambda \colon S_*(X \times Y) \to S_*X \otimes S_*Y$. Use $\lambda^{\vee} \colon S^*X \otimes S^* Y \to S^*(X \times Y)$
    \item diagonal $X \xrightarrow{\mathrm{diag}} X \times X$
\end{itemize}
The \textbf{cap product} on the other hand comes from $H^p X \otimes H_nX \to H_{n-p}X, c\otimes x \mapsto  c \smile x$. It is built from the same ingredients. Some formal properties: it makes homology a module over cohomology, where $(b \smile c) \frown x= b \frown (c \frown x)$, $1 \frown x = x$. For $f \colon X \to Y$, for $c \in H^*Y$ take the pullback class $f^* c \frown x$ for $ x \in H_*X$. Then  \[
    f_*(f^* c \frown x) = c \frown f_*x.
\] This is also called the ``projection formula''. There is some reason for this nomenclature but Dr.\ Perutz doesn't remember why.

\subsection*{Construction of the cap product}
We start with $S^* X \otimes S_{-*}X \xrightarrow{\id \otimes \mathrm{diag}_*} S^*X \otimes S_{-*}(X \times X)$. We send this second factor to a tensor product $S _{-*}X \otimes S_{-*}X$ by the Eilenberg-Zilber quasi-isomorphism. Then evaluate cochains on chains by definition to get $\Z \otimes S_{-*}=S _{-*}X$. This defines $\frown \colon S^*X \otimes S_{-*}X \to S _{-*}X$. In summary, \[
\frown \colon S^* X \otimes S_{-*}X \xrightarrow{\id \otimes \mathrm{diag}_*} S^*X \otimes S_{-*}(X \times X) \xrightarrow{\id\otimes \lambda} S ^*X\otimes S _{-*}X \otimes S_{-*}X\xrightarrow{\mathrm{ev}\otimes \id}  \Z\otimes S _{-*}X = S _{-*X}
\] For $c = [\phi] \in  H^p X, x=[z] \in H_n X$, $c \frown x = [\phi]\frown [z]=[\phi \frown z]$. Here $S_{-*}X$ means a cochain complex of degree $n$, $S _{-n}X$, cohomology equals $\partial $.


\subsection*{Construction of the slant product}
This is a map $H_p(X) \otimes H^n (X \times Y) \to H ^{n-p}(Y), x\otimes c \mapsto  c /x$. This is not quite division (it is bilinear), but it resembles division in the following sense; for $a \in H^* X, b \in H^* Y$, this leads to $a \times b \in  H^*(X\times )$ (where $a \times b =  \mathrm{pr}^*_x a \smile \mathrm{pr}^*_y b$). Then $(a \times b) / x = a(x) b$ for $x \in H_pX$, and this is how it resembles division.

Let's construct this. We start with $S _{-*}X \otimes S^*(X \times Y) \xrightarrow{\id\otimes \kappa^{\vee}}  S _{-*}X\otimes S^* X\otimes S^* Y$ where $\kappa \colon S_*X \otimes S_* Y \to S_*(X \times Y)$ is the Eilenberg (something) map. Then evaluate like in the cap product. In short, \[
- / - \colon S _{-*}X \otimes S^*(X \times Y) \xrightarrow{\id\otimes \kappa^{\vee}} S _{-*}X\otimes S^* X\otimes S^* Y \xrightarrow{\mathrm{ev}\otimes \id} S^*Y
\] What is the point of the slant product? Let the algebraic variety $V$ be the module of vector bundles on $V$, called $\mathcal{M} $. Then $V \times \mathcal{M} $ carries a natural $H^*$ chain $c$, and $H_*V \to H^*M, x \mapsto c / x$. This is used a lot in the theory of algebraic vector bundles, gauge theory, etc. We will soon use it to construct Poincar\'e duality.

About the evaluation map. We have been slightly sloppy with degrees, denoting everything with a $*$, but this should clear things up. Precisely, $\mathrm{ev }\colon  S_m X \otimes  S^n  X \to \Z$, where 
\[
\mathrm{ev}(z, \phi)=
\begin{cases}
    \phi(z),& \text{if} \ m=n\\
    0, & \text{otherwise.} 
\end{cases}
\] 
\subsection*{Manifolds}
Let's talk about topological $n$-manifolds $M$. Some useful notation: for $K \subseteq M$, set  $H_p(M \mid K) = H_p(M,M \setminus K)$. These groups are \textbf{contravariant} with respect to $K$, that is, $K \lhook\joinrel\xrightarrow iL \subseteq M $ implies $i_* \colon H_*(M \mid L) \to H_p(M \mid K) $. The important case is that for $x \in M$, we have groups $\mathrm{or}_x=H_n (M \mid x)=H_n (M, M \setminus \{x\} )$. This is local homology, and it turns out they control orientation of $x$. 

Say $x \in U \subseteq M$ homeomorphic to $\R^n $, our chart. Then there is a map $H_n (M \mid x) \xleftarrow{ \text{excision} } H_n (U \mid x)$ by inclusion, and excision tells us this is an isomorphism. So the rest of the manifold is irrelevant to local homology, and we only need to pay attention to a neighborhood. On the other hand, $\phi \colon (U,x) \xrightarrow{\cong} (\R^n ,0) ,$ which is a copy of $\Z$ in the $n$th degree. What is the generator of this group $\Z$? It is just a single simplex $\sigma \colon |\Delta ^n | \hookrightarrow \R^n  $ with 0 in its interior. 

Note also that a choice of (ordered) basis for $\R^n $ determines a generator. Only the orientation of this basis matters. For example, permuting basis vectors changes the generator of $H_n (\R^n  \mid 0)$ via sign of permutation. This motivates the following definition.
\begin{definition}[]
    An \textbf{orientation} for $M^n $ is a coherent family of isomorphisms $\omega_x \colon H_n (M \mid x) \xrightarrow{\cong} \Z  $ for $x \in \Z$.
\end{definition}
What does ``coherent'' mean here? And how does this topological definition match the top form definition for smooth manifolds?

