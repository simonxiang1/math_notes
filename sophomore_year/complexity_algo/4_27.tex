\section{Reductions} 
Dr.\ Price forgot his notes at home today, so the lecture will be slightly scatterbrained. Recall:  P consists of problems solvable in polynomial, NP consists of verifiable problems in polynomial time, NP-hard problems once solved can solve all NP problems. Today we show how to show problems are NP-hard, hence implying they are probably not in P.
\begin{namedthm}{Cook's Theorem} 
   Circuit SAT is $NP$-complete, i.e., CSAT is both in $NP$ and $NP$-hard.
\end{namedthm}
A description of CSAT is as follows: ``Does there exists an input such that a given circuit outputs 1?''\footnote{Here circuit refers to set of logic gates, a Boolean circuit.} We can see that CSAT is in NP because we just plug it in. Furthermore, CSAT reduces to SAT, which consists of satisfiable boolean \emph{formulas}. It is easy to formulate a SAT formula as a SAT circuit, but the other way is not as easy. The idea is that you make a new variable for each wire. The formula is ``3-CNF'' (3-conjunction normal form), the intersection of unions of three clauses at a time. So CSAT, SAT, and 3SAT are all NP-complete.

Consider the problem max independent set, where given a graph $G$, an \emph{independent set} $I$ is a subset of $V$ such that no edge $e \in E$ has $|e \cap S|=2$. The max independent set asks if there exists an independent set $S$ of size $|S| \geq k$. The claim is that max independent set is NP complete, or that it in NP and is NP-hard.We do this by reducing from 3SAT. Recall that reducing from $X$ (3SAT) to $Y$ (max independent set) means that we can solve $X$ (3SAT) from $Y$ (max independent set), or $Y$ (max independent set) is no harder than $X$ (3SAT, hard). How do we do this?
\begin{enumerate}[label=(\arabic*)]
\setlength\itemsep{-.2em}
    \item Construct a transformation from an \emph{arbitrary} $X$ instance $x$ to a \emph{special} $Y$ instance $y$.
    \item If $x$ is a YES instance, then $y$ is a YES instance.
    \item If $y$ is a YES instance, then $x$ is a YES instance.
\end{enumerate}
Given a 3SAT instance $(a \cup b \cup c) \cap (b \cup  \overline{c}\cup  \overline{a})\cap (a \cup  b \cup  d) \cap (a \cup  \overline{c}\cup d)$, this is YES because set $a$ and $b$ to be true. Make a graph with vertices as above, grouped into threes. Edges just form some  strongly connected components (formally called a ``clause gadget''). Now we need some variable gadgets, expressing the fact that each triple relates to each other. We do this by drawing an edge from each instance of $a$ to each instance of $\overline{a}$, $b$ to $\overline{b}$, etc.

The claim is that if 3SAT is satisfiable, then the corresponding graph has an independent set of size greather than or equal to the number of clauses. 3SAT being satisfiable means that there exists an assignment  of $x_i  $ to $\{0,1\} $ such that each clause has greater than or equal to 1 satisfied variable. For our specific instance, our assignment is $a=1,b=1,d=1,c=0$ (one for each clause). Take $S$ to be one of those satisfied variable vertices per clause gadget, where $|S|=\# $ of clauses, then $S$ is independent. So if 3SAT is satisfiable, there exists a large independent set.

Now we need to show the other direction, that is, if our independent set instance is a YES, then so is the 3SAT instance. First observe than in each clause, $S$ has exactly one vertex for each clause gadget. Furthermore, $S$ never has both $x_i $and $\overline{x}_i $ labeled vertices. Set \[
x_i =
\begin{cases}
    1 & \text{if any} \ x_i  \ \text{vertex} \in S\\
    0 & \text{if any} \ \overline{x}_i  \ \text{vertex} \in S\\
    0 & \text{otherwise} 
\end{cases}
\]This assignment is consistent and satisfies all clauses. So this is a complete NP hardness proof. 

\begin{example}[Mario]
    Now for a ``fun'' reduction example, Super Mario Bros the original. This is NP hard. We can construct a giant level in Mario Maker, and we detail why solving a general Mario level is NP hard. We do this by reducing from 3SAT. Now our clause gadgets and variable gadgets, rather than being graphs, are pieces of a level. Mario often needs to make a choice if some thing $x_1$ is true or false (variable gadgets), some choice $x_2$, etc. Then he runs back through clause gadgets to make it to the final. Zelda is also NP hard (opening doors with swords).
    
    ???? Stars/mushrooms? So clauses are like stages. So someone could give you a password, to PSAT, to SAT, to 3SAT, to a certain Mario level. Zelda is even harder, it is PSPACE-hard. In order be in NP, we need to be able to check an answer; it takes exponential time to prove (play levels, due to backtracking).
\end{example}
Karp's 21 problems
