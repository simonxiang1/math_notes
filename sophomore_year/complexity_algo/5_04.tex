\section{Broader complexity classes} 
Recall we have P (polynomial time) and NP (nondeterministic polynomial time), which means there exists a proof of \textsc{yes} that can be verified in P. We also have ``coNP'', where we take an NP problem and just negate it; does there exist a proof of \textsc{no} that can be verified in P? For example, using Hamiltonian cycle, we can ask if there doesn't exist a cycle visiting each vertex once. Then if this were false, we would have a cycle.

An interesting problem in between P and NP is factoring, and this is both in NP and coNP. The decision version of factoring asks the following: is there a prime factor of $x$ less than $k$? Then find the smallest one, use binary search and keep factoring. This is clearly in NP; why is this in coNP? We can give a complete prime factorization as a proof.

Let's go bigger. Another natural class to think about is \textbf{exponential time} EXP, basically polynomial time but exponential (precisely, $2 ^{n ^{O(1)}}$). The second class is PSPACE, which is \textbf{polynomial space}. We claim that the following inclusions hold: \[
\mathrm{P}\subseteq \mathrm{NP}\subseteq \mathrm{PSPACE}\subseteq \mathrm{EXP}
\] Clearly  $\mathrm{P}\subseteq \mathrm{NP}$. Why is $\mathrm{NP}\subseteq \mathrm{PSPACE}$? We need to show how to solve SAT in polynomial space. The naive answer tries every answer, which is exponential time but polynomial space. Therefore $\mathrm{NP}\subseteq \mathrm{PSPACE}$. We could also do it directly; given a verifier of proof in polynomial time, we just try all proofs taking polynomial space. Why is $\mathrm{PSPACE}\subseteq \mathrm{EXP}$? If we have polynomial space, then our configurations can only be in $2 ^{n ^{O(1)}}$ configurations before we loop. If we take more than $2 ^{n ^{O(1)}}$ time, we must have gotten into a loop from which we cannot leave. So any PSPACE algorithm can only take exponential time.

These are pretty easy containments. One class you may consider is NPSPACE, which is \textbf{non-deterministic polynomial space}; the question is ``does there exists a proof of \textsc{yes} that can be verified in PSPACE''? This is kind of silly because it turns out PSPACE=NPSPACE. Clearly $\mathrm{PSPACE}\subseteq \mathrm{NPSPACE}$ because you can ignore the verifier and solve the proof yourself. It turns out to be no more powerful because with a polynomial space verifier, try all polynomial proofs, which ends up being PSPACE but exponential time.
We do get power from gaining exponential time, so consider NEXT, \textbf{non-deterministic exponential time}. We have $\mathrm{EXP}\subseteq \mathrm{NEXP}\subseteq \mathrm{EXPSPACE}\subseteq \mathrm{EEXP}$ (double exponential time, $2 ^{2 ^{n ^{O(1)}}}$ time), and we could continue this chain for quite a while. For now we stick with this picture of P, NP, PSPACE, and EXP. 

For more intuition on what these classes are, consider a circuit (or function) $f$. A question in P asks ``what is $f(x)$?'' NP says ``does there exists an $x$ such that $f(x)=1$?'', which is SAT, a puzzle. PSPACE is a two player game; ``does there exist an $x_1$ such that for all $x_2$, does there exists an $x_3$ such that for all $x_4$, all the way to $x_n $, such that $f(x)=1$?'' This is a PSPACE-complete problem called ``True Quantified Boolean Formula'' (TQBF), think of it like chess (each quantifier represents a move). So this is no longer a single player game, this is a two player game. Any given game of chess is only polynomially long, but we can't easily say who wins. Some other examples include Mario (NP hard, but not NP since we don't know how long a game is). It turns out we can write TQBF in Zelda using switches, so the path is exponentially long. So Zelda is PSPACE-complete. In Go (chinese rules), we know that it is EXP-complete, since it could take exponentially many moves to finish.

Another way you might think about it is like this: we are trying to solve a chess column in the newspaper, with each line giving a path. The column only explains a couple paths, but not the full tree. A chess grandmaster could explain a lot more at each step and be more convincing. This suggests another complexity class, \textbf{interactive proofs}, called IP. The prover sends a message to the verifier given an input, and the verifier checks it. They send $n ^{O(1)}$ messages, and the verifier has to be able to know what's going on. We only require \textsc{yes/no} with probability $P=3/4$. It is clear that $\mathrm{NP}\subseteq \mathrm{IP}$, since we just take the first proof, and it is not hard to see that $\mathrm{IP}\subseteq \mathrm{PSPACE}$ since polynomially many messages that are polynomial in size. PSPACE asks ``does white win this position in chess?'' and IP asks ``could a chess god convince me that white wins this position''? There seems to be a gap, since Magnus Carlsen could probably fake a result for black. However, it turns out that IP=PSPACE\footnote{This is the most recent result covered in this class, from 1990.}; the basic idea is that there is a tree of how the game works. We can make the prover say something global about the tree, and ask questions to make him fill in the details in a way that we can catch if the answer is false.
\subsection*{Randomness}
What about a randomized algorithm, like QuickSort? How do we formalize a randomized algorithm? The standard class is BPP, \textbf{bounded error probabilistic polynomial time}. Our input is $A(x,r)$ for $r \in  \{0,1\} ^{n ^{O(1)}}$, where $r$ is uniform. We want to say that it is correct with $P \geq 2/3$ over $r$.\footnote{We use 2/3 because it is greater than 1/2 and less than 1, this threshold is by convention. Any value strictly greater than 1/2 is OK, and  $1-2 ^{-n}$ is OK.} We also have the class PP, \textbf{probabilistic polynomial time correct up to} $\mathbf 1/\mathbf 2$. This is a funny class because it turns out  $\mathrm{PP}\supseteq \mathrm{NP}$. To solve SAT, try a random $x_1, \cdots , x_n $, and see if $f(x)=1$. If \textsc{yes}, output 1, and if \textsc{no}, output \textsc{yes} with probability $ \frac{1}{2}- \frac{1}{2 ^{n+1}}$. If the truth is \textsc{no}, this is correct with probability $\frac{1}{2}+ \frac{1}{2 ^{n+1}},$ and if the truth is \textsc{yes}, this is correct with probability greater than or equal to \[
    \left( \frac{1}{2^n } \right) + \left( 1 - \frac{1}{2^n } \right) \left( \frac{1}{2}- \frac{1}{2 ^{n+1}} \right) = \frac{1}{2}+ \frac{1}{2 ^{2n+1}} > \frac{1}{2}.
\] So NP is reducible to PP. This is kind of silly, but it works. Something happened and we got BQP, \textbf{bounded error quantum polynomial} space. So we have a new containment chart as follows: \[
\mathrm{P}\subseteq \mathrm{BPP} \subseteq \mathrm{BQP} \subseteq \mathrm{PSPACE}
\] Notice the tiny equal sign under each. Are these inequalities equal or not? We don't have unconditional bounds. Among all these things, we know that $\mathrm{P} \neq \mathrm{EXP}$, and $\mathrm{PSPACE}\neq \mathrm{EXPSPACE}$. The standard conjecture for what is going on this: $\mathrm{P}=\mathrm{BPP}$, and all the other ones are different. Then $\mathrm{BPP}\subset\mathrm{NP}$, and $\mathrm{BQP}$ and $\mathrm{NP}$ are incompatible. There are further classes like SETH, the \textbf{strong exponential time hypothesis}, which says 3SAT takes at least $2 ^{ 0.99 n}$ time. A recent line of work says ``suppose this is true'', then we can get a lower bound that says finding triangles in a graph takes $n ^{2.97}$ time.

