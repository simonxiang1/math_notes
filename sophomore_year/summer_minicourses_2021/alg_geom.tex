\section{Introduction to Varieties and Schemes (Desmond Coles and Saad Slaoui)} 
\epigraph{\emph{Even more so than what we call ``key theorems'' in mathematics, it is the fertile viewpoints wihch, in our art, constitute the most powerful tools of discovery --- or rather, they are not tools, but they are the very eyes of the researcher who passionately strives to understand the nature of mathematical things.}}{\emph{—\,Alexander Grothendieck, R\'ecoltes et Semailles}}

\subsection{The basic framework, and problems the algebro-geometrically minded think about (June 28)}

Welcome to ``Introduction to varieties and schemes'', or ``Grothendieck's paradise'' (based off Hilbert's quote ``No one shall be able to expel us from the paradise that Cantor created for us.'') Today we cover the basic framework, some problems algebraic geometers think about, and varieties vs schemes. The standard references include the introduction section of \cite{ega}\footnote{I want this to be EGA but it seems difficult to do with with BibTeX...}, Springer edition, and some lecture notes by another French mathematician.

The goal is to study spaces of solutions to systems of polynomial equations with coefficients in a commutative ring $k$ via geometric methods. Classically, $k$ was a field, say $\R$. Descartes, Fermat, and Euler worked with polynomial equations in two or three variables, which leads to things like curves or surfaces in $\R^2$ or $\R^3$. The next big step was letting $k=\C$, and work from Abel and Riemann laid the foundation for complex algebraic geometry. This gives an interplay between analysis, complex geometry, and algebra.

One of the successes of scheme theory is that it allows for $k$ to be a general commutative ring, like the integers. When we involve integers, we are actually doing number theory, in this case Diophantine geometry. At first glance, looking for rational solutions seems to be a discrete problem that doesn't involve geometry at all (there is no real sense in where the integers form a space). And yet, we will see by the end of this lecture how scheme theory gives a place for these sorts of questions in a geometric setting.

\orbreak 
Recall the goal is to study spaces of solutions to systems of polynomial equations with coefficients in a commutative ring $k$ via geometric methods. Let $S=\{f_j \}_j  $ be a family of polynomial equations over some indexing set, so we can view $\{f_j \} _j  \subseteq k[(t_i ) _{i \in I}]$. Most often $I=\{1,\cdots ,n\} $ and we are studying $k[t_1,\cdots ,t_n ]$. We are interested in the subset $V(S):= \{a \in k^I \mid f_j (a)=0\} \subseteq k^I$ for all $j$ in our family. We often denote $k^I $ by $\A^{|I|}_k$, called \textbf{affine} $\mathbf k$\textbf{-space}. Then $V(S)$ is an \textbf{affine algebraic set} associated to the system $S$.

\begin{note}
    As long as $I$ is finite and $k$ is a Noetherian ring (thought of as a finiteness condition on the ring), we get that $(S)$, the ideal generated by $S$, is equivalent to the ideal generated by finitely many polynomials, denoted $(g_1,\cdots ,g_r)$. In other words, $(S)$ is finitely generated. The point is that for $I$ finite and $k$ Noetherian, affine algebraic sets are always cut out by finitely many equations. Then $V(-)$ assigns to any subset $S$ of $k[(t_i )_{i \in I}]$ a subset of affine space $\A_k^{|I|}$ by $S \mapsto  V(S)$.

    \begin{ex}
        You can check that the choice of $S$ doesn't sense the passing to ideals. In other words, $V(S)=V(I)$ where  $I=(S) \triangleleft k[(t_i )_{i \in I}]$.\footnote{Each $f_j $ is written in terms of one $t_i $ right? Are coefficients polynomials (elements of $k[(t_i )_{i \in I}]$) or elements of $k$?}
    \end{ex}
    %\begin{proof}
        %Recall that $(S):=(\{f_j \} _{j \in J})= \{a_j f_j \} $ for $a_j  \in  k[(t_i ) _{i \in I}]$. The easy direction is $V(S) \subseteq V(I)$, since any solution
    %\end{proof}
\end{note}
The equation $f(a)=0$ for $f$ a polynomial (in one variable, $f=\sum a_i t ^i , \ a_i \in k$) in $k[(t_i )_{i \in I}]$ makes sense not only for $a$ a tuple in $k^I$, but also for \emph{any} field extension $L$ of $k$ (if $k$ is a field). There is more to this: consider $f=\sum a_i t ^i $ in one variable, for $a_i  \in k$. In fact, $a$ can live in $A^I$ for any $k$-algebra $A \in \mathsf{CAlg} _k$. For a given system $S$, we don't just get one vanishing set associated to $k$, we get infinitely many vanishing sets associated to any possible $k$-algebra $A$. So we have an assignment $V_s(-) \colon \mathsf{CAlg} _k \to \mathsf{Set} $ by $A \mapsto V_s(A):= \{a \in A^I \mid  f_j (a)=0 \ \text{for all} \ j\} $. 

\begin{ex}
    The suggestive notation from earlier was chosen for a purpose. Show that any  $k$-algebra homomorphism $A \to B$ induces a map between the solution sets $V_s(A) \to V_s(B)$. This implies that $S$ gives rise to a set valued functor out of the category of $k$-algebras.
\end{ex}
\begin{ex}
    Let $R:= k[(t_i ) _{i \in I}] / (S) \in \mathsf{CAlg} _k$. Check that we have isomorphisms $\Hom _{\mathsf{CAlg} _k}(R,A) \xrightarrow{\simeq } V_s(A)$ for all $A \in \mathsf{CAlg} _k$.
\end{ex}
The principle is that a given system $S=\{f_1, \cdots , f_r\} $ should be thought of as a recipe for producing infinitely many solution sets as the $k$-algebra of admissible solutions varies. The goal is to take the categorically defined object (the functor $\mathsf{CAlg} _k \to \mathsf{Set} $) and build a ``geometric avatar'' for this large amount of data. In essence, what we're after is a certain geometric space that captures the geometric data that can be extracted from all of these solution spaces. That is the point of scheme theory.

\begin{namedthing}{Goal} 
    Given a system $S= \{f_j \} _j  \subseteq k[(t_i )_{i \in I}]$, we want to associate with it a \textbf{scheme} $X$, encoding the geometric data contained in $V_s(A)$ for $A \in \mathsf{CAlg} _k$.
\end{namedthing}
\begin{definition}[]
    A \textbf{scheme} (in the sense of Grothendieck) is a figure offering a simplified and functional (faithful) representation of an object (of algebraic origin) or equivalently, of an (instantiation) procedure.
\end{definition}
This is not really a precise definition. We'll work toward it. The idea is that $S \leadsto X$ (a geometric object) $\leadsto V_s(k)=X(k)$, the ``classical'' solutions or $V_s(A)=X(A)$, the ``$A$-valued'' solutions of the system.

In the Diophantine case where  $k=\Z$, we start with a system of polynomial equations with integral coefficients, and to it we asssociate a scheme $X$. We are really after $X(\Z)$, the integral solutions to the system, but we can also look at reducing the solutions mod $p$ and looking for solutions in the field $\Z /p \Z$. So to the scheme we associate $X(\F_p)$ the solutions mod $p$, or we could go the other road and look for complex solutions $X(\C)$. Because the complex plane has an analytic topology, we are hoping this is a more geometric set. Both of these come from the same scheme. The \textbf{Weil conjectures} say we can draw a bridge between $X(\F_p)$ and $X(\C)$.
\begin{note}
    So far, we've only described the ``affine case'', where $S $ is associated to some geometric object $\mathrm{Spec}(R),$ where $ R=k[(t_i )_i ] / (S)$, the \textbf{affine scheme} associated to $R$. In parallel to smooth/complex manifold theory, we would like to use these as building blocks for general schemes, in the same way a smooth manifold is made up of an atlas of charts locally resembling Euclidian spaces glued together in a sensible way. 

    Why do we care? The central object of study in complex algebraic geometry ($k=\C$) are called \textbf{smooth projective varieties} $X$, which come with an embedding $X \hookrightarrow \C \mathrm{P}^r$. To formulate this we need to make sense of complex projective space, which is a non-affine scheme.
\end{note}
\orbreak
Let's talk about some problems in algebraic geometry. 
\begin{itemize}
\setlength\itemsep{-.2em}
    \item \textbf{Counting problems}: ``Is there a solution?'' to a system $S$. For example with $k=\F_q$, this is a matter of checking finitely many values. The question is, what is $\# V_s(\F_{q=p^n })$? This is closely related to the Weil conjectures. This is an extremely hard question for $k=\Z$, and a lot of number theory studies this question studies this question for integral or rational solutions. One of the simplest and hardest examples is $x^n +y^n-z^n =0$. For $n \geq 3$, $V_s(\Z)= \{(0,0,0)\} $, and this is Fermat's last theorem.

        The other counting problem is intersecting various algebro-geometric objects and asking how many points are in the intersection, aptly named \emph{intersection theory}.
    \item \textbf{Relationship to ``other geometries''}: If $k$ is a topological field (eg $\R,\C,\Q_p$)\footnote{Here $\Q_p$ is the ring of $p$-adic numbers.}, we can ask whether we can exploit the geometry of $X(k)$ equipped with the corresponding ``analytic'' topology as a subspace of a topological field.
        \begin{itemize}
            \item \textbf{Serre's GAGA principle}: Here $k=\C$, $S=f_1,\cdots ,f_r \in  \C[t_0, \cdots ,t_n ]$ where the $f_i $ are homogeneous polynomials. Then we can associate a scheme $X$ which is naturally embedded in $\P^n $, since homogeneous polynomials do not care if you scale the input. Similarly, $X(\C) \hookrightarrow \P^n _{\C}$, and if we denote $X(\C)$ by $X^{\mathrm{an}}$ this is the \textbf{analytification of} $\mathbf X$, and has the structure of a compact complex K\"ahler manifold. Serre's GAGA principle establishes a correspondence between structures on this complex manifold and the algebraic variety $X$, which Desmond will be talking about on day 8.
            \item If $k \hookrightarrow \C$ is a subfield, eg $\Q$, and $X$ is the scheme associated to the system $S$, then $X(\C)=X^{\mathrm{an}}$ has a topological fundamental group $\pi_1(X^{\mathrm{an}})$. On the other hand, there exists a theory of \textbf{\'etale fundamental groups} in algebraic geometry denoted $\pi_1 ^{\text{\'et}}(X)$ which make sense for schemes. A theorem says that there is an ``inclusion'' $\widehat{ \pi_1^{\text{top} }(X^{\mathrm{an}})}\to \pi_1^{\text{\'et}}(X)$ where the big hat denotes the profinite completion of the standard fundamental group. Then we have a sequence \[
                    1 \to  \underset{\text{geometric} }{\underbrace{ \widehat{ \pi_1^{\text{top} }(X^{\mathrm{an}})}}} \to \pi_1^{\text{\'et}}(X) \to \underset{\text{arithmetic} }{\underbrace{ \mathrm{Gal}( \overline{k}/k) }} \to 1
                \] where the absolute Galois group of the seperable closure of $k$ over the ground field $k$ measures in a sense the failure of the map to be an isomorphism. So this gives a link between the geometric and arithmetic. 

                The principle is that doing algebraic geometry over a field $k$ is approximately doing geometry over the algebraic closure $\overline{k}$, and amalgamate this with the arithmetic $k$, often in the form $\mathrm{Gal}(\overline{k}/k)$. For example in $\Q_p$ the topology is really nasty, it's totally disconnected and each ball has every point as the center. There isn't an obvious way to do analysis, but there is a theory called ({\color{red}todo:brokenage?}) geometry, which has a GAGA theorem and other good stuff. Tropical geometry falls out of this.
            \item We can consider $H_{\mathrm{sing}}^*(X^{\mathrm{an}};\Q)$, our classic singular cohomology with $\Q$ coefficients from algebraic topology. We can ask ``which cohomology classes can be obtained algebraically?'' By algebraically, we consider our scheme $X$ as a geometric space and analyze the cycles $Z \hookrightarrow X$, subobjects of a certain codimension. If we have an algebraic object, then Poincar\'e duality can extend this to the Poincar\'e dual class $\alpha _Z$. The \textbf{Hodge conjecture} pretty much asks if given an arbitrary cohomology class of $H_{\mathrm{sing}}^*(X^{\mathrm{an}};\Q)$, can we express it as some linear combination of Poincar\'e dual classes of algebraic cycles on $X$? 
        \end{itemize}
    \item \textbf{Classification problems}: When are two varieties or schemes non isomorphic? We start by building invariants, including:
        \begin{itemize}
            \item \emph{Numerical}: What's the dimension? Genus? Degree of the polynomial?
            \item \emph{Analogues of algebro-topological invariants}: $\pi_1$ becomes $\pi_1^{\text{\'et}}$, which detects a lot more than the classical fundamental group. We can also look for analogues of the singular cohomology for $X$ defined over (with coefficients in) finite field $\F_q$, which exists and is called \textbf{\'etale cohomology} $H^*_{\text{\'et}}(X;\Q_t)$. 

                $K$-theory is the study of vector bundles over a space, and step one is to make sense of algebraic bundles. Then the question is ``can we look at equivalence classes of vector bundles?'', then we do some group completion. This is called the \textbf{algebraic} $\mathbf K$\textbf{-theory of schemes}, which makes sense.
            \item \emph{Birational geometry}: The basic idea is that open sets in the Zariski topology is huge. The question is then ``do two varieties or schemes have dense open subsets that are isomorphic?'' The minimal model program asks about the geometry of these dense open sets, and the resolution of singularities uses birational transformations to make varieties smooth.
        \end{itemize}
    \item \textbf{Classification of structures}: Can we classify certain types of structures in algebraic geometry over a fixed scheme $X$? For example, vector bundles over a fixed spaces or elliptic curves over $k$.
        \begin{itemize}
            \item \emph{Moduli spaces}: Can we geometrically study families of objects of a certain type? This results again in a functor, and we again ask if we can study that geometrically. The answer is ``yes'', and this is the theory of moduli spaces. The functor $\mathcal{M} _{\mathcal{B} ,X}\colon \mathsf{CAlg} _k \to \mathsf{Set} $ is defined by \[
                    \mathcal{M} _{\mathcal{B} ,X}(A):=
                    \left\{  \text{families of} \ \mathcal{B}\text{-objects over} \ X \ \text{parametrized by} \ \operatorname{Spec}A\right\} .
            \] The best case scenario is if this functor is \textbf{representable}, i.e. there exists some \textbf{classifying space} $Z$ such that for any $A \in \mathsf{CAlg} _k$, we have an isomorphism \[
            \mathcal{M} _{\mathcal{B} ,X}(A) \xrightarrow{\simeq } \Hom_{\mathrm{Sch}_k}(\operatorname{Spec} A, Z)
            \] where $\mathrm{Sch}_k$ denotes schemes over $k$. Note that $Z$ has a canonical family of $\mathcal{B} $-objects, say $\mathcal{U} $. Then a map $\operatorname{Spec}A\to Z$ formally pulls back to $f^*\mathcal{U} $, and this is how the isomorphism is established.
            \begin{example}
                Projective space $\P^n _k$ for $k$ a field can be built this way. The intuition is that projective space ``parametrizes lines in $k^{n+1}$'', and we can turn this intuition into a functor of this type. This is an example where we have a representing object.


                Other times we don't, like $\mathcal{M} _g$, the ``absolute'' moduli functor of curves (schemes of dimension 1) of genus $g$, which is \emph{not} representable. Other examples include principal $G$ bundles on a scheme $X$ for $G$ a group, where $\mathrm{Bun}_G(X)$ is also not representable. 

                The problem with these objects is that the family of $\mathcal{B} $-objects is too sophisticated for them to fit into $\mathsf{Set} $-valued functors. Automorphisms of elliptic curves make $\mathsf{Set} $-valued functors too coarse to accomodate this kind of data. The solution is to change the category of sets to the category of groupoids, and this is where \textbf{stacks} come into play.
            \end{example}
        \end{itemize}
\end{itemize}
\orbreak
\subsubsection*{Comparing varities and schemes}
Fix $S=\{f_j \}\subseteq k[(t_i ) _{i \in I}] $ throughout this section.

