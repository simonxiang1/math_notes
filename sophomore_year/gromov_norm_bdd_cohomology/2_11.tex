\section{Bounded cohomology} 
Like we said last time, we shift to the dual theory of bounded cohomology. The Gromov norm is like an $\ell^1$-norm, so there should be a dual $\ell^{\infty}$ norm. For an $\ell^{\infty}$-norm to make sense, we need bounds, hence the ``bounded'' in bounded cohomology. First we talk about ordinary group homology and cohomology, then bounded cohomology of groups, then bounded cohomology of spaces and explain why we only need them for groups. We use a topological point of view.

Let $G$ be a group. We would like to talk about the homology and cohomology groups $H_*(G;R),H^*(G;R)$ for some ring $R$ (usually $\Z$ or $\R$). From the topological POV we use the $K(G,1)$ space, or Eilenberg-Maclane space, or classifying space of $G$ as a discrete group.
 \subsection{Eilenberg-Maclance spaces}

\begin{definition}[]
    We say $X$, a connected CW complex is a $K(G,1)$ space if we have the following:
    \begin{itemize}
    \setlength\itemsep{-.2em}
\item $\pi_1(X)=G$,
\item $X$ is \textbf{aspherical}, or $\pi_n (X)=0$ for all $n>1$. Equivalently, the universal cover $\widetilde X$ is contractible.
    \end{itemize}
    There are two facts; first, they exist. The second is a universal property of sorts, which implies that they are unique up to homotopy equivalence.
\end{definition}
\begin{lemma}
    Let $X$ be a $K(G,1)$ space and $Y$ some connected CW complex. Any homomorphism $\varphi  \colon \pi_1(Y,y) \to \pi_1(X,x)$ is realized by some map $f \colon (Y,y) \to (X,x)$ (i.e. $f_*= \varphi $) and $f$ is unique up to homotopy.
\end{lemma}
This corresponds to the fact that maps between spaces induce maps on $\pi_1$;  here maps between $\pi_1$'s induce a unique map on the space, given that the target is a $K(G,1)$ space.
\begin{cor}
    If $X$ and $Y$ are both $K(G,1)$, then any isomorphism $\pi_1(X,x) \to \pi_1(Y,y)$ is induced by a homotopy equivalence $f \colon X \to Y$.
\end{cor}
\begin{proof}
    Realize $\varphi $ by $f \colon (X,x) \to (Y,y)$, $\varphi ^{-1}$ by $g \colon (Y,y) \to (X,x)$. The composition $g \circ f \colon (X,x) \to (X,x)$, and $(g \circ f) _* = \varphi  ^{-1} \circ \varphi =\id _{\pi_1 X}$. Not only do we have existence we have uniqueness, which implies $g \circ f \simeq  \id_X$. Similarly, flipping $g$ and $f$, $f \circ g\simeq \id_Y$. Therefore $g$ is a homotopy inverse.
\end{proof}
\begin{definition}[]
    Given a ring $R$, define the homology and cohomology of a group $G$ as $H_*(G;R):=H_*(K(G,1);R)$ and $H^*(G;R):= H^*(K(G,1);R)$.
\end{definition}
\begin{example}
    Some examples.
    \begin{itemize}
    \setlength\itemsep{-.2em}
\item $H_0(G;R)=H_0(K(G,1);R)=R$, since $G$ is connected.
\item $H_1(G;\Z)=H_1(K(G,1);\Z)=\mathrm{Ab}(\pi_1(K(G,1)))=\mathrm{Ab}(G)$. Similarly, $H^1 (G;R)=\Hom(G,R).$
\item What is $H_*(\Z;\Z)$? Let $X=S^1 $ which is a $K(\Z,1)$ space. Then \[
        H_k(\Z;\Z)=
        \begin{cases}
            \Z & k=0,1\\
            0 & k>1.
        \end{cases}
    \] %So if we have a way to realize a  $K(G,1)$ 
    There is a notion of the geometric dimension of $G$, which is the smallest dimension of $X$ for $X$ a $K(G,1)$ space. Formally, this is defined as $\mathrm{gd}(G):=\min \{\dim X \mid  X \ \text{is} \ K(G,1)\} $. There is a dual notion $\mathrm{cd}(G)=\min \{K \mid H_n (G;R)=0 \ \text{for all} \ n>k, \ \text{all} \ R\} $. Here we need to take twisted coefficients (comes with a $G$-action), the important this is that the cohomological dimension is always less than or equal to the homological dimension, or $\mathrm{cd}(G) \leq \mathrm{gd}(G)$.
    \end{itemize}
\end{example}

\subsection{Manifolds as $K(G,1)$ spaces}
Every finitely presented group can be represented as $\pi_1$ of some closed 4-manifold. What if we want to realized a $K(G,1)$ space as a manifold? The question is, given $G$, is there a manifold that is a $K(G,1)$ space? In other words, is there an aspherical manifold with $\pi_1=G$? 
\begin{example}
    For $G=F_2$ this manifold exists, for example the inside of the 2-torus (handlebody with body removed), the punctured torus with boundary removed, or the thrice punctured sphere with boundary removed.
\end{example}
What if we look for $M$ closed? Then $H_n (M;\Z /2) \cong \Z /2  $ for $M^n $ closed. If we further require that $M^n $ is closed orientable, then $H_n (M;\Z)\cong \Z$ generated by the fundamental class.

\begin{lemma}
    \begin{enumerate}[label=(\arabic*)]
    \setlength\itemsep{-.2em}
\item If $M^n $ is closed, aspherical with $\pi_1(M)=G$, then $H_n (G; \Z /2) \simeq 2) \cong \Z /2, H_k(G;R)=0$, $k>n$.
\item If $M^n $ is orientable in addition, $H_n (G;\Z) \cong \Z,$ etc.
    \end{enumerate}
\end{lemma}

\begin{example}
    Continuing our example, for $G=F_r$, $X=S^1 \vee S^1 $, $H_k(G;\Z)=0$ for $k>1$, $\Z$ for $k=0$, and $\Z^r $ for $k=1$ ($r$ is the rank). This leads to the following corollary.
\end{example}
\begin{cor}
    We cannot realize $K(F_r,1)$ as a closed manifold when $r>1$.
\end{cor}
For $G=\Z /n$, $H_k(\Z /n;\Z)=\Z$ for $k=0$, $\Z /n$ when $k$ is odd, and $0$ when $k>0 $, even. In particular, the cohomological dimension of $G$ is infinite, so the geometrical dimension of $G$ is infinite as well, so there is no finite dimensional $K(G,1)$. In particular, it cannot be a manifold.
\begin{prop}
    Let $G$ be a finite group, $G$ acting on $\R^n $. Then this action is not free.
\end{prop}
\begin{proof}
    Suppose the action of $G$ is free. Let $H=\Z /m$ be a cyclic subgroup (take the powers of an element of $G$), then $H$ acts on $\R^n $ freely. Then the projection $\R^n  \to \R^n  /H$ to the quotient is a covering map, since $H$ acts freely and is properly discontinuous. Define $X:= \R^n  /H$. Then $X$ is a $K(\Z /m,1)$ space since $\R^n $ is contractible, a contradiction.
\end{proof}
