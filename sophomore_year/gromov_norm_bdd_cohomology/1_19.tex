\section{First day of class}
This is a topics course on Gromov's simplicial norm and bounded cohomology. There will be optional problems to discuss online or over email. Office hours are tentatively Wednesdays 3:30 to 4:30, and Friday 12:30 to 1:30 (also by appointment). There may be specific topics that we won't have time to go into, so there may be a presentation at the end of the course. Grades are based off participation. For the first few weeks we are meeting over zoom, and we will decide how to give the lectures afterward.

It's time to talk about math!

\subsection{Motivation}
There are two notions popularized by Gromov's paper {\color{red}todo:references}, one being Gromov's simplicial norm and a dual theory called bounded cohomology. We start by introducing the simplicial norm, then we will talk about bounded cohomology, and later we will use bounded cohomology to prove more intricate things about the simplical norm.

What is the general idea? Suppose we have a topological space $X$ (thought of as a closed manifold), and we have homology groups $H_n (X;\R)$ (typically $\R$-vector spaces). We want to equip these spaces with a (semi-)norm $\|\cdot \|_1$ to measure the size of a homology class $\sigma \in H_n (X; \R)$. If $X=M^n $ is an oriented connected closed manifold, we know that $H_n (M ;\R) \simeq \R$ generated by a fundamental class $[M]$. If this space is equipped with a norm, we can talk about the norm of the fundamental class $\| [M ]\|_1$ which should measure the ``volume'' of $M$. We usually think about the volume form or a Riemannian metric when talking about volume, which depends on the metric. If there is a natural way to induce a norm, this will be a topological invariant.

How could volume be a topological invariant? There is some evidence that this is true. 
\begin{itemize}
\setlength\itemsep{-.2em}
    \item One piece of evidence is \textbf{Gauss-Bonnet}; if $M=S_g$ (surface of genus $g$), then $-2 \pi \cdot \chi(S_g)=\mathrm{area}(S_g)$. In general we need to specify a metric, but the magic of Gauss-Bonnet is that $\mathrm{area}(S_g)$ is the same for any hyperbolic metric on $S_g$. 
    \item Another piece of evidence is \textbf{Mostow's rigidity}; if $M^n $ for $n \geq 3$ admits some hyperbolic metric (the quotient of the $n$-dimensional hyperbolic plane mod some cocompact lattice), it is a theorem of Mostows that the hyperbolic metric is unique, so volume is ``well-defined'' or only depends on the topology.
\end{itemize}
In both cases, we introduce hyperbolic geometry and say ``it doesn't depend on the hyperbolic metric'', which isn't purely topological. What Gromov did was introduce a purely topological definition, then showed that it agrees with the hyperbolic geometry.

A fun exercise from algebraic topology. If $S, S'$ are both conected closed surfaces, say $g(S') >g(S)$. Can you find a map $f \colon S \to S'$ with $\deg (f) \neq 0$? There are several ways to map higher genus to lower genus surfaces (collapse, double cover), but can you map the lower genus surface into the higher genus surface? Intuition says we cannot do this because area increases. The simplest solution uses Gromov's simplicial norm.

\subsection{Defining Gromov's simplicial norm}
Let $X$ be a topological space, $H_n (X;\R)$ be the singular homology of $X$. We should have a chain complex $C_n (X;\R)$, the space of all real $n$-chains with basis $S_n (X)$, where $S_n (X)$ is the set of all singular $n$-simplices. Here $C \in S_n (X)$ is a map $C \colon \Delta ^n  \to X$. With a chosen basis, we can talk about all kinds of norms; $C_n (X;\R)$ comes with an $\ell^1$-norm, and a chain  $C \in C_n (X;\R)$ is a unique expression $C=\sum_{i=1}^{k} \lambda_i C_i $ with $\lambda_i  \in \R$. Then $|C|_1=\sum_{i=1}^k |\lambda_i |$ is a norm at the chain level. To get homology, we have a sequence \[
    C_{n+1}(X; \R) \xrightarrow{\partial _{n+1}} C_n (X; \R) \xrightarrow{\partial _n } C_{n-1}(X;\R). 
\] Then we have \emph{boundaries} $B_n :=\im \partial _{n+2}$ and \emph{cycles} $Z_n :=\ker \partial _n $. The relation is $\partial ^2=0$, so $B_n \subseteq Z_n  \subseteq (C_n , |\cdot |_1)$. Recall from functional analysis that a normed quotient vector space comes with a quotient norm of its own. Here $H_n (X; \R) = Z_n  / B_n $ has an induced semi-norm $\| \cdot \|_1$. Concretly, given a homology class $\sigma$ and the origin, the quotient norm is the infimum over all cycles. 

\begin{definition}
    We define \textbf{Gromov's simplicial norm} as $\|\sigma\|_1= \underset{C \in Z_n }{\underset{[C]=\sigma}{\inf}} |C|_1$, where $|C|_1$ is the number of simplices in the cycle $C$.
\end{definition}
This can be thought of as the ``smallest'' distance to the origin {\color{red}todo:figure}. In words, the seminorm $\|\sigma\|_1$ is the infimal number of simplices to represent  $\sigma$ as a cycle. Some properties:
\begin{prop}[Functoriality]
   For $f \colon X \to Y$ (implied to be continuous), then the induced map $f_* \colon H_n (X) \to H_n \left( Y \right) $ ($\R$-coefficients are implied) is non-increasing with respect to $\|\cdot \|_1$, i.e., for any $\sigma \in H_n (X)$, $\|f_* \sigma\|_1 \leq \| \sigma\|_1$.
\end{prop}
\begin{proof}
    Let $\sigma=[c]$ for a cycle $c$, $c=\sum \lambda_i c_i $, $\lambda_i  \in \R$, $c_i  \in S_n (X)$. Then we have the push-forward $f_* c= \sum \lambda_i f_* c_i =\sum \lambda_i ( f \circ c_i )$. \[
    \begin{tikzcd}
\Delta^n \arrow[r, "c_i"] \arrow[rr, "f_*c_i"', bend right] & X \arrow[r, "f"] & Y
\end{tikzcd}
    \] So $f_* \sigma=[f_* c]$. So $\| f_* \sigma\|_1 \leq \|f_*c \|_1 \leq \sum|\lambda_i |=|c|_1$. The left hand size doesn't depend on $c$, so take the infimum over $c$ with $[c]=\sigma$. This implies that $\|f _*\sigma\|_1 \leq \| \sigma\|_1$.
\end{proof}
\begin{cor}[Invariance]
    If $f \colon X \to Y$ is a homotopy equivalence, then $f_* \colon H_n (X) \to H_n (Y)$ is an isometric (preserving $\| \cdot \|_1$) isomorphism. More generally, let $f \colon X \to Y$. If there is a  $g\colon Y \to X$ such that $g_* f_* =\id _{H_n (X)}$, then $f_*$ is an isometric embedding.
\end{cor}
\begin{proof}
    If $f$ is a homotopy equivalence, then we have a homotopy inverse $g \colon Y \to X$ with the property that $(f \circ g) \simeq \id_Y$, $(g \circ f) \simeq \id_X$. In particular these maps induce identity maps $f_*g_*=\id _{H_n (Y)}, g_* f_*=\id _{H_n (X)}$. Then the isomorphism follows from applying the second part twice.

    To show the second part, all we need to do is show that $f_*$ preserves the norm (injectivity is already shown), or $\| f_* \sigma\|_1=\|\sigma\|_1$. To prove the reverse inequality, we need to show that $\| \sigma\|_1 \leq \|f_* \sigma\|_1$ for every $\sigma \in H_n (X)$. By assumption $g_*f_*$ is the identity, so \[
        \| \sigma\|_1= \|(g_* f_*) \sigma\|_1=\| g_* (f_*\sigma)\|_1 \leq \| f_* \sigma\|_1
    \] where the last inequality follows by the functoriality of $g$.
\end{proof}
This implies that the norm $\|\cdot \|_1$ is a homotopy invariant. Next time we will continue this line and talk about simplicial volume. This gets us a way to estimate degree and be able to do some examples. As homework, how does $\|\cdot \|_1$ behave on $H_0(X;\R)$?
