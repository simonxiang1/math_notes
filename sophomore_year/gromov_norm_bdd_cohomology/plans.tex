\documentclass{article}
\usepackage{amsmath, amssymb, amsthm}
\newtheorem{theorem}{Theorem}
\newtheorem{definition}{Definition}
\newtheorem{lemma}{Lemma}
\renewcommand\qedsymbol{$\boxtimes$}

\newcommand\N{\ensuremath{\mathbb{N}}} 
\newcommand\R{\ensuremath{\mathbb{R}}} 
\newcommand\A{\ensuremath{\mathbb{A}}} %affine space
\newcommand\Z{\ensuremath{\mathbb{Z}}} 

\begin{document}
\section{Final reading project} 
\begin{itemize}
\setlength\itemsep{-.2em}
    \item quasimorphism
    \item rotation quasimorphisms (iterate $n$ times w/basepoint, average speed)
    \item poincare: this number characterize the dynamics (properties of self maps), measurable and topological setting. self map by homeomorphisms (fixed points, periodic points, dense orbit). rotation number is real, they differ by integers, so for the map downstairs there's a rotation number defined by $\R/\Z=S^1 $. whether or not this is rational tells us periodic orbits, irrational goes to dense orbits. then the theorem says that $f$ behaves like rigid rotations by rotation number. property $\mathrm{rot}(f^n )= n \mathrm{rot}(f)$. if group is amenable then homomorphism. todo:digest up to here 
    \item generalize that by Ghys theorem.
    \item general problem; let $M = \partial W$, $M$ has immersion into some large manifold  $N$. ex $S^1 =M= \partial D^2$, immerses in $\R^2$. the question is can we immerse $M$ in $N$ such that preserve $\partial $? some of these can't extend to disk, as boundary of idsk unit tangent rotates around once. if immersion; gauss map, map $S^1  \to S^1 $, degree of map called the turning number. calculate degree by counting preimages; come with signs, turning counterclockwise. then it can't be an immersion because degree has to be one.
    \item turning number similar to rotation number. extend the problem to a surface, curve on a surface (immerse, self intersect). $T_p S$ identify in a nice way; arbirary ones differ by choice of isomorphism to $\R^2$. can we make a choice to differ continuously? connections give a local identification (tangent bundle is locally trivial), works exactly when the tangent bundle is trivial (or $TM= M \times \R^n $). surfaces with boundary can be immersed into $\R^2$ (unwrapping punctured tori), use euler char (increase genus). increase boundary components by making punctures, immersion $S _{g,n}$. doing this lets you trivialize the tangent bundle, then with an immersed curve we can talk about how many time tangent vector rotates around in the tangent space.
    \item we can also put a hyperbolic metric, where boundary is a geodesic boundary. turning number is related to hyperbolic area and some sort of rotation number.
\end{itemize}
references: paper by calegari, called ``faces of scl norm ball''. immersions. bound immersed surfaces by other surfaces, look for references from there.
\end{document}
