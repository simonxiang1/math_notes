\section{November 8, 2021} 
Last time, we formulated a classification theorem for finitely generated modules over PIDs. We saw the ``torsion case'' (not defined), and for $A=\Z$ we got the classification for finite abelian groups, and for $A=k[t],k=\overline{k}$, we got the Jordan form for matrices.

\begin{definition}[]
    A \textbf{simple} (or \textbf{irreducible}) $A$-module is an $A$-module $M$ with exactly two submodules.
\end{definition}
Note that the two submodules must be 0 and $M$; in particular, $M\neq 0$.
\begin{lemma}
    Any simple $A$-module $M$ is isomorphic to $A / \mathfrak m$ where $\mathfrak m \subsetneq A$ is a maximal (left) ideal.
\end{lemma}
\begin{proof}
    Choose $m\neq 0$ in $M$. Then $A \xrightarrow fM, 1 \mapsto  m$. For $0\neq m \in \im(f) \subseteq F$ a submodule, by simplicity $\im(f)=M$ iff $M$ is surjective. So $M \simeq A \ \ker (f)$. We want to show that $\ker(f)$ is a maximal (left) ideal. Suppose $\ker(f) \subsetneq  I \subseteq A$. This implies that $I / \ker(f) \subseteq A /\ker(f) =M$. Then since $I/\ker(f) = A / \ker(f)=M,$ so $I=A$.
\end{proof}
\begin{example}Some examples:
    \begin{enumerate}[label=(\arabic*)]
    \setlength\itemsep{-.2em}
        \item If $A$ is a PID, we saw that every maximal ideal in $A$ is $(f) := \alpha  f$ for $f$ irreducible. So sijmple $A$-modules are $A /f$ for $f$ irreducible.
        \item If $A=\Z$, simple $\Z$-modules (abelian groups) is  $\Z /p$ for some prime $p$.
    \end{enumerate}
\end{example}
\begin{definition}[]
    An $A$-module $M$ is of \textbf{finite length} if $M$ has a sequence of submodules $0=M_0 \subseteq M_1 \subseteq  \cdots  \subseteq M_n =M$, such that $M _{i+1} / M_i $ is simple for all $i$.
\end{definition}
In general, finite length modules are special cases of finitely generated modules. For example, $\Z$ is not of finite length as an abelian group; finite length for abelian group is equivalent to being finite.

\begin{definition}[]
    An $A$-module $M$ is \textbf{torsion} if for every $m \in M$, there exists an $0\neq f \in A$ such that $f \cdot m=0$.
\end{definition}
\begin{example}
    For $A$ commutative, $A / f$ for $f\neq 0$ is torsion.
\end{example}
\begin{prop}\label{flfg} 
    If $A$ is a PID and $M$ is an $A$-module, then $M$ is finite length iff $M$ is finitely generated and torsion.
\end{prop}
\begin{lemma}
    For arbitrary $A $, $M$ of finite length over $A$, then $M$ is finitely generated.
\end{lemma}
\begin{proof}
    Choose $0 \subseteq M_0 \subseteq M_1 \subseteq  \cdots  \subseteq  M_n $ as in the definition of finite length. We proceed by induction on $n$. If $n=0$, we are done. Assume for $n-1$ that there exists $m_1, \cdots ,m_r \in  M _{n-1}$ generating $M _{n-1}$. Choose any element $m _{r+1}\in M$ and not in $M_{n-1}$.
    \begin{claim}
        The $m_1, \cdots , m_{r+1}$ generate $M$.
    \end{claim}
    \begin{proof}[Proof of claim]
       Let $N \subseteq M$ be the submodule generated by $m_1 ,\cdots ,m _{r+1}$. Then $M _{n-1} \subseteq N \subseteq M, \ o \neq N / M_{n-1}  \subseteq M / M_{n-1}.$ So $N / M_{n-1}= M / M_{n-1}$.
    \end{proof}
    \begin{lemma}\label{fltorsion} 
        For $A$ an integral domain but not a field, suppose  $M$ is finite length over $A$. Then $M$ is torsion.
    \end{lemma}
    \begin{proof}[Proof of \cref{fltorsion}]
        If $M \simeq  A / m$, by hypothesis on $A$, $m\neq 0$.Choose $0\neq f \in  M$, then $f$ acts by zero on $A / m$ which implies torsion. Like before, by induction we have $M _{n-1}\subseteq M$ and $0\neq f_0 \in A$ such that $f_0 M_{n-1}=0$ and $M / M _{n+1}$ is simple. We can choose $0\neq f_1 \in A$ such that $f _1 \cdot ( M / M _{n-1})=0$ (what we just showed). In other words, $f_1 \cdot  M \subseteq  M_{n-1}$.
    \end{proof}
\end{proof}
