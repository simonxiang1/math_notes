\section{October 29, 2021} 
Today we'll talk about modules, the goal being the classification of modules over a PID. The motivation is the following: no one gets excited over the definition of a module. Modules are organizational tools, and serve many purposes:
\begin{enumerate}[label=(\arabic*)]
\setlength\itemsep{-.2em}
    \item Many interesting problems can be phrased in terms of modules. Module theory leads to tools (induction, extensions, homological algebra, ...).
    \item Properties of modules in the aggregate tell you interesting things about rings.
\end{enumerate}
The setup: let $A$ be a ring, possibly non-commutative.
\begin{definition}[]
    A (left) $\mathbf A$\textbf{-module} is the data $(M, \mathrm{act})$ where $M=(M,+)$ is an abelian group, $\mathrm{act}\colon A \times M \to M, a,m \mapsto a\cdot m$ such that 
    \begin{itemize}
    \setlength\itemsep{-.2em}
\item $a\cdot (m_1+m_2)=a \cdot m_1+a\cdot m_2$ for every $a \in A, m_1,m_2 \in M$,
\item $a\cdot (b\cdot m)=(a\cdot b)\cdot m$ for every $a,b \in A, m \in M$, 
\item $(a+b)\cdot m=a\cdot m+b\cdot m$ for every $a,b \in A, m \in M$,
\item $1\cdot m =m$ for every $m \in M$.
    \end{itemize}
    A \textbf{right} $\mathbf A$\textbf{-module} is one with a map $M \times A \to M, (m,a) \mapsto  ma$ such that analogous stuff happens.
\end{definition}
Define $A ^{\mathrm{op}}:=A$ as an abelian group, with multiplication $a \overset{\mathrm{op}}{\cdot }b:=ba $. This is a perfectly good procedure, so left $A ^{\mathrm{op}}$-modules are equivalent to right $A$-modules. In particular, left and right $A$-modules are the same when $A$ is commutative.
We think of groups acting on sets (symmetries), while we think of rings acting on modules.
\begin{example}
    Some examples of modules:
    \begin{enumerate}[label=(\arabic*)]
    \setlength\itemsep{-.2em}
        \item Let $M=A$ with $\mathrm{act}(a,b):=a\cdot b$, this is an $A$-module.
        \item For $I \subseteq A$ a (left) ideal (closed under addition and multiplication on the left), then $I$ is a submodule of $A$. Conversely, a submodule of $A$ is a left ideal.
        \item A $\Z$-module is an abelian group. Specifically, for $M$ a $\Z$-module, then $M$ is an abelian group, and the action must be $(n,m) \to \underset{n \ \text{times} }{\underbrace{m + \cdots +m}  } $ or $(n,m) \to \underset{n \ \text{times} }{\underbrace{=m - \cdots -m}  } $.
        \item If $A=k$ is a field, then a $k$-module is precisely a vector space over $k$.
    \end{enumerate}
\end{example}
Recall that $G$ acting on $X$ is the same as a map $G \to \Aut(X)$, which is sometimes a convenient perspective. Analogously, for $M$ an abelian group, define $\mathrm{End}_{\mathrm{gps}}(M) := \{\varphi  \colon M \to M \mid \varphi \ \text{is a homomorphism} \} $. Then $\mathrm{End}(M)$ is naturally a ring, where for $\varphi_1,\varphi_2 \in \mathrm{End}(M), (\varphi_1+\varphi_2)(m):= \varphi_1(m)+\varphi_1(m)$. For $(\varphi_1+\varphi_2)  $ to be an endomorphism, we need $M$ to be abelian. Furthermore, $\varphi_1\cdot \varphi_2  := \varphi_1 \circ \varphi_2$, i.e., $(\varphi_1\cdot \varphi_2)(m):= \varphi_1(\varphi_2(m))    $. In this case, $1 \in \mathrm{End}(M)$ is equal to $\id_M$. Then unwinding things, an $A$-module structure on $M$ is equivalent to a ring homomorphism $A \to \mathrm{End}(M), a \mapsto  \varphi_a $, where $\varphi _a(m)=a\cdot m$ (recall we are considering group endomorphisms, which are universal like the automorphism group).

Briefly, a \textbf{homomorphism} of $A$-modules is a map $f \colon M \to N$ such that $f$ is a homomorphism of abelian groups and $f(a\cdot m)=a\cdot f(m)$ for every $a \in A, m \in M$. We use the same language as groups, etc a submodule is closed under stuff, an automorphism of modules is a bijective homomorphism of modules, etc.

    Fix $k$ a field, and let $A=k[t]$. 
    \begin{namedthm}{Question} 
        What are $k[t]$-modules?  
    \end{namedthm}
    \begin{namedthing}{Answer} 
        Let $V$ be a $k[t]$-module. Then $V$ is a $k$-module via the homomorphism $k \to k[t]$, also $ t\in k[t]$ determines a map $T := (t \cdot  -) \colon V \to V$ (the action of $t$). Note  that $T$ is a linear transformation (AKA $k$-module endomorphism). For example, $T(\lambda v):= (t \cdot \lambda\cdot v)=(\lambda t)\cdot v= \lambda \cdot (t \cdot v)=\lambda T(v) $ for every $\lambda \in k, v\in V$. Conversely, given $V /k$ a vector space and $T \colon V \to V$ a linear transformation, for $f (t)= \sum_{}^{} a_i  t ^i  \in k[t], f \cdot v := \sum a_i  T ^i (v) \in V$ defines a $k[t]$-module structure.
    \end{namedthing}
    \begin{namedthing}{Slogan} 
        $k[t]$-modules are vector spaces with an endomorphism. 
    \end{namedthing}
    Our goal is going to be classification of modules over PIDs. We have two great PIDs, the integers and $k[t]$. One corresponds to finite groups, and the other is the classification of finite dimensional vector spaces with endomorphism. This is the idea of Jordan canonical forms, giving a relationship between vector spaces with endomorphisms and ideals inside the polynomial algebra. They seem unrelated, but the connection between the two is module theory.

