\section{October 8, 2021} 
Last time we ended with a question: given a field $k$ and a polynomial $f \in k[t]$ of degree $n$, when is $f$ \emph{separable}, i.e., when does $f$ have exactly $n$ distinct roots in some extension field $K /k$?
\begin{example}
    $t(t-1)$ or $t^2+1$ are (usually separable). $t^2,(t-1)^2, (t-1)^2(t^2+1)$ are not separable.
\end{example}
\begin{definition}[]
    Given $f=a _n  t ^n  + \cdots +a_0 \in k[t]$, define the \textbf{derivative} of $f$ as $f'(t):= n a _n  t ^{n-1}+(n-1) a _{n-1}t ^{n-2}+ \cdots + a_1$.
\end{definition}
Note that $(fg)'=f'g+fg'$.
\begin{prop}
    For $f \in k[t]$ with $\deg (f)=n>0$, the following are equivalent:
    \begin{enumerate}[label=(\arabic*)]
    \setlength\itemsep{-.2em}
        \item $f$ is separable (has $n$ distinct roots in some $K /k$),
        \item For every extension field $K/ k$ and $\lambda \in K$, $(t-\lambda)^2$ does not divide $f$.
        \item The ideal generated by $f$ and $f'$ is equal to the whole ring $k[t]$, or $(f,f'):=k[t]f+k[t]f'=k[t]$. In other words, there exist $\alpha ,\beta  \in k[t]$ such that $\alpha f+\beta  f'=1$.
        \item There exists an extension field $K /k$ such that $K[t]f+K[t]f'=K[t]$. In other words, there exist $\alpha ,\beta  \in K[t]$ such that $\alpha  f + \beta  f'=1$.
    \end{enumerate}
\end{prop}
\begin{proof}
    We know (1) is equivalent to (2) because we know there exists some $K / k$ and $c \in  k ^{\times }$, $\lambda_1, \cdots , \lambda_n  \in K$ such that $f=c \cdot \prod _{i=1}^n  (t- \lambda_i ) \in  K[t]$. Assuming (2), all roots must be distinct, so (2) implies (1). Assuming (1), if $f$ is divisible by $(t-\lambda)^2$, this implies that $f=c (t-\lambda)^2 \cdot \prod _{i=1}^{n-2}(t-\lambda_i )$. The better way to say this is, by induction on degree, factorization into monic irreducible factors is unique.

    Tautologically, the two halves of (3) and (4) are equivalent. (3) implies (4) because take the field extension in (4) to be $k$. Now we want to show that (2) implies (3). Now $(f, f')=(g)$ for some $g \in k[t]$ as $k[t]$ is a PID. WLOG, $g$ is monic. If $g=1$, we are done. Otherwise, $\deg(g)>0$, so there exists some extension field $K /k$ where $g$ has a root by last time. The claim is that $(t-\lambda)^2$ divides $f$ in $K[t]$ in this case, giving a contradiction.
    We know $(t-\lambda) \mid g \mid f$, which implies that $(t-\lambda)\mid f$. Let $\widetilde f:= \frac{f}{(t-\lambda)}$, we want to see that $\lambda$ is a root of $\widetilde f$. In other words, $f=\widetilde f\cdot (t-\lambda)$. By the product rule, $f'=(\widetilde f)'(t-\lambda)+\widetilde f\cdot 1$. Then $(t-\lambda)\mid f'$, and \[
        0=f'(\lambda)=(\widetilde f)'(\lambda)\cdot (\lambda-\lambda)+\widetilde f(\lambda).
    \] This implies $\widetilde f(\lambda)=0$, and $(t-\lambda) \mid  \widetilde f$. Multiplying through, $(t-\lambda)^2\mid f$. So (2) implies (3). Now we want to show that (3) implies (2), which is the same kind of idea. Suppose $\alpha  f +\beta  f'=1$, and suppose that $K / k$ and $\lambda\in K$ such that $(t-\lambda)^2 \mid f$. Then $f=(t-\lambda) \widetilde f$ with $\widetilde f(\lambda)=0$. Then $f'=1 \widetilde f +(t-\lambda)\widetilde f$ which implies $f'(\lambda)=\widetilde f(\lambda)=0$. This implies that $\alpha (\lambda)f(\lambda)+\beta (\lambda)f'(\lambda)=1$, so $0=1$, a contradiction.

    It remains to show that (4) implies (2), here is the sketch. The idea is that given some $K$ and $\alpha f+\beta f'=1$ in $K[t], $ in (2), given $\widetilde K$, $\lambda \in \widetilde K (t-\lambda)^2 \mid f$. The claim is that there exists a non-zero commutative ring $A$ and embeddings \[
    \begin{tikzcd}
             & A                       &                         \\
K \arrow[ru] &                         & \widetilde K \arrow[lu] \\
             & k \arrow[lu] \arrow[ru] &                        
\end{tikzcd}
    \] We skip showing that $K\otimes_k \widetilde K=A$. Define $\widetilde { \widetilde K}:=A /$ a maximal ideal means that $\widetilde { \widetilde K}$ is a field, so we have embeddings
\[
\begin{tikzcd}
                        & \widetilde{\widetilde K} \arrow[rd] \arrow[ld] &                                    \\
K \arrow[ru] \arrow[rd] &                                                & \widetilde K \arrow[lu] \arrow[ld] \\
                        & k \arrow[lu] \arrow[ru]                        &                                   
\end{tikzcd}
\] 
    Run (3) implies (2) in $\widetilde {\widetilde K} $, and we are done.
\end{proof}
Given a field $k$, there is an important invariant $\mathrm{char}(k)$, the \textbf{characteristic} of $k$. Either $\mathrm{char}(k)=0$, or $\mathrm{char}(k)$ is some prime number. The construction is as follows: for every $A$ a ring, there exists a unique ring homomorphism $\Z \xrightarrow{i_A}  A$, where $0 \mapsto 0, 1 \mapsto 1$, $2 \mapsto  2 :=1+1$ ($\Z$ is the initial object for $\mathsf{Ring} $). If $k$ is a field, then $\ker(i_k)=(p) \subseteq \Z$, which is sprime because $\Z / \ker(i_k) \hookrightarrow k$. This $p$ is the characteristic of $k$.
For $\F_p := \Z /p$, this has characteristic $p$. $\Q,\R$, and $\C$ have characteristic $0$, since $\Z$ maps injectively into these fields.
