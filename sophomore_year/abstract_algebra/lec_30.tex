\section{November 5, 2021} 
Last time: let $A$ be a ring and $M$ be an $A$-module. We defined a set $\{m _i  \} _{ i \in I}\subseteq M$ to be \textbf{generators} of $M$ if $A ^{\oplus I}\to  M$ is surjective. The set $\{v _j \}  _{ j \in J}\subseteq A ^{\oplus I} \ (r_j  = \sum _{\text{finite} }a _{ij}e_i $ sometimes expressed) is a set of \textbf{relations} for the generators $\{ m_i \} $ if $r_j  \in \ker (A^{\oplus I}\to M)$ (or $\sum a_{ij}m_i =0$ for every $j$) and $\{r_j \} $ generates $\ker (A^{\oplus I}\to M)$.

\begin{example}
    $\Z /2$ has generator $1 =m$, and relations $2\cdot m_1=0$.
\end{example}
\begin{remark}
    Given generators and relations for $M$, we obtain a ``mapping out'' universal property for $M$. Namely, giving a map $M \xrightarrow{f} N$ is equivalent to elements $n_i  \in N$ for $i \in I$ such that $\sum a _{ij}n _i =0$ for every $j \in J$.
\end{remark}
\begin{proof}
    $M= \coker(A ^{\oplus J}\to A ^{\oplus I})$.
\end{proof}
\begin{definition}
    An $A$-module is \textbf{finitely generated} if there exists if there exists a finite set of generators for $M$. This is equivalent to the existence of an $A ^{\oplus n}:= A ^{\oplus \{1, \cdots , n\} }\to  M$ surjective.
\end{definition}
\begin{example}
    Let $A=\Z$.
    \begin{enumerate}[label=(\arabic*)]
    \setlength\itemsep{-.2em}
\item $\Z /n$ is finitely generated by 1 (or every element).
\item $\Z $ is generated by 1.
\item The direct sum $\bigoplus_A \Z$ for $A$ countably infinite is not finitely generated.
\item $\Q$ is not finitely generated, since any finite subset of $\Q$ is contained in $ \frac{1}{n}\Z$ for some $n$.
    \end{enumerate}
\end{example}
Some takeaways:
    \begin{enumerate}[label=(\arabic*)]
    \setlength\itemsep{-.2em}
\item Any quotient of a finitely generated module is finitely generated.
\item If we have submodules $M_0 \subsetneq M_1 \subsetneq M_2 \subsetneq  \cdots  \subsetneq M $ with $M = \bigcup_{i=0} ^{\infty}M_i $, then $M$ is not finitely generated.
    \end{enumerate}
    The idea behind (2) is why (3) and (4) is not finitely generated. Inside $\bigoplus \Z$, consider $\Z \subsetneq \Z \oplus \Z \subsetneq \cdots $ and in $\Q$ we can do a bunch of things. The reason why (2) is true is because each generator lies in one of the $M_i $'s, and the maximal one won't generate $M_{i+1}$.

    \begin{theorem}
        Let $A$ be a PID. Then every finitely generated $A$-module $M$ is isomorphic to $\bigoplus _{i=1}^n A / f_i$ for some $f _i \in A$.
    \end{theorem}
    \begin{theorem}
        If $M$ is finitely generated over some PID  $A$, then there exist a unique $n \geq 0$, $m_1, \cdots , m_r >0$, $\pi_1, \cdots ,\pi_r \in A$ irreducible such that $M \simeq  \bigoplus _{i=1}^r A / \pi_i ^{m_i }\oplus A^{\oplus n}$.
    \end{theorem}
    \begin{cor}
        Let $A=\Z$. Then every finitely generated abelian group is isomorphic to $\bigoplus _{i=1}^r \Z / p_i  ^{m_i }\oplus \Z ^{\oplus n}$ for some primes $p_1, \cdots ,p_r$ and $n\geq 0, m_1,\cdots ,m_r>0$. In particular, every finite abelian group is isomorphic to $\bigoplus _{i=1}^r \Z / p_i  ^{m_i }$.
    \end{cor}
    Let $k$ be an algebraically closed field (every $\deg >0$ polynomial $f \in k[t]$ has a root in $k$), e.g $k=\C$ (fundamental theorem of algebra). Then take $A=k[t]$; every irreducible element in $A$ is of the form $(\text{non-zero})\cdot (t-\lambda)$ for some $\lambda \in k$. Let $V$ be a $k[t]$-module that is finite dimensional as a  $k$-vector space (``torsion case''), $T \colon V \to V$. By the theorem, there exist $\lambda_i \in K, m _i  >0$ for $i=1,\cdots ,r$ such that $V \simeq  k[t] / (t-\lambda) ^{m_1\oplus \cdots \oplus k[t]} / (t-\lambda_r)^{m_r}$ as a  $k[t]$-module.
    \begin{example}
        Let $\lambda=0$, then $k[t] / t^m$ has a basis  $1, t , \cdots , ^{m=1},\ 1 \mapsto t, t \mapsto t^2, \cdots , t ^{m-2 }\mapsto  t ^{m-1}, t ^{m-1}\mapsto 0$. In matrix form, \[
        \begin{pmatrix}
            0 & 0 & \cdots  & 0 &0\\
            1 & 0 & \cdots  & 0&0\\
            0 & 1 & \cdots  & 0&0\\
            0 & 0 & \ddots  & 0&0\\
            0 & 0 & \cdots  & 1&0\\
        \end{pmatrix}
    \] For a general $\lambda$, take the basis $k[t] / (t- \lambda)^m, 1, (t-\lambda), \cdots ,(t-\lambda)^{m-1}, T(1)=t=\lambda\cdot t\cdot (t-\lambda)$. Then the matrix looks like this:\[
        \begin{pmatrix}
            \lambda & 0 & \cdots  & 0 &0\\
            1 & \lambda & \cdots  & 0&0\\
            0 & 1 & \ddots  & 0&0\\
            0 & 0 & \ddots  & \lambda&0\\
            0 & 0 & \cdots  & 1&\lambda\\
        \end{pmatrix}
    \]This is called the \textbf{Jordan block}, and a corollary of this is the \textbf{Jordan canonical form}.
    \end{example}
