\section{September 8, 2021} 
Observe that all fibers of the map $\pi \colon G \to G /H$ are in \emph{non-canonical} bijection.

\begin{definition}[]
    A \textbf{fiber} of a map $f \colon X \to Y$ is a subset of $X$ of the form $f^{-1} (y)$ for some $y \in Y$.
\end{definition}
We know that $\pi$ is surjective. Therefore, there exists a section $\tau \colon G / H \to G$ such that \[
\begin{tikzcd}
G/H \arrow[r, "\tau"] \arrow[rr, "\id_{G/H}"', bend right] & G \arrow[r, "\pi"] & G/H
\end{tikzcd}
\] This is called ``choosing coset representatives'', since the fibers of $\pi$ are right $H$-cosets. Then given $x \in G /H$, $\pi ^{-1}(x) = \sigma(x) \cdot H$. This is in bijection with $H$ via the map multiplication on the left with $\sigma(x)$.

\begin{namedthm}{Lagrange's Theorem} 
    If $G$ and $H$ are finite, then $|G|=|G /H| \cdot |H|$. 
\end{namedthm}
\begin{proof}
    If we have $\pi \colon G \to G /H$, all fibers have order $|H|$.
\end{proof}

\begin{definition}[]
    A \textbf{map/morphism/homomorphism} of sets with $G$-action is a map $f \colon X \to Y$ such that $f(gx)=gf(x)$ for all $x \in X$, $g \in G$. A \textbf{isomorphism} of $G$-sets is a morphism $f$ with an inverse $g$ ($fg=\id_Y,\ gf=\id_X$). This is equivalent to $f$ being bijective.
\end{definition}

\begin{definition}[]
    For $X$ a $G$-set and $x \in X$, the \textbf{stabilizer} of $x$ is the subgroup $\mathrm{stab}(x)=\mathrm{stab}_G(x)= \{g \in G\mid g \cdot x=x\} $.
\end{definition} The implicit claim is that $\mathrm{stab}(x)$ is a subgroup. To check this, $1 \in \mathrm{stab}(x)$ as $1 \cdot x=x$. For $g,h \in \mathrm{stab}(x)$, then $(gh)x=g(hx)=gx=x$, so $gh \in \mathrm{stab}(x)$. Finally, if $g \in \mathrm{stab}(x)$, then $x=1\cdot x=g^{-1}\cdot g\cdot x=g^{-1}x$, so $g^{-1} \in \mathrm{stab}(x)$.

\begin{example}
    Let $X=G/H$, $x = \pi(1) \in G /H$. Then $\mathrm{stab}(x)=H$, since $g \in \mathrm{stab}(x) $ iff $gH=H$ iff $g \in H$.
\end{example}
\begin{example}
    Let $G=S_n $ act on $\{1,\cdots ,n\} =X$. Then $\mathrm{stab}(n)=S _{n-1}$, since we fix $n$ and shuffle $1,\cdots ,n-1$.
\end{example}
\begin{lemma}
    Suppose we are given a $G$-set $X$ and a point $x \in X$ such that 
    \begin{enumerate}[label=(\alph*)]
    \setlength\itemsep{-.2em}
\item $| X / G| =1$ (there exists a unique orbit),
\item $\mathrm{stab}(x)=H \subseteq G$.
    \end{enumerate}
    Then there exists a unique isomorphism of $G$ between $G / H \xrightarrow{\simeq } X$ such that $\pi(1) \mapsto x$.
\end{lemma}
\begin{proof}
     To figure out which way maps should go, recall the universal property of $G /H$.\[
     \begin{tikzcd}
G \arrow[d, "\pi"'] \arrow[rd, "\widetilde f"] &   \\
G/H \arrow[r, "f", dotted]                     & X
\end{tikzcd}
\] Define $\widetilde f(g)= gx$. Then by the universal property of quotients, we see that $f$ exists iff $\widetilde f(gh)= \widetilde f(g)$ for all $g \in G, h\in H$.

{\color{red}todo:not sure what happened here} 
\end{proof}

\begin{lemma}
    Giving a morphism of all $G$-sets is equivalent to a point $x \in X$, $f(\pi(1))=x$, such that $\mathrm{stab}(x) \supseteq$.
\end{lemma}

Suppose $G$ acts on $X$. Then $X \simeq  \prod _{ i \in I}G / H$ as a $G$-set for $I$ some (?). The construction is as follows: let $I= X /G$. Choose some representatives of each orbit, i.e. a section of the map $X \to  X / G$. For each $i \in I=X /G$, choose $x _i $ in that orbit. For $i \in I$, take $H_i = \mathrm{stab}(x)$. Now apply our previosu result to obtain the deomposition of $X$.
