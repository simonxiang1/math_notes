\section{November 11, 2021} 
Last time, we talked about \cref{flfg}. Recall that $M$ is \emph{torsion} if for every $m \in M$, there exists an $f \neq 0$ in $A$ such that $fm=0$. We want to show that if $M$ is finitely generated and torsion, then $M$ is of finite length. 

A special case is where $M= A /I$ for some ideal $I \subseteq A$, $I=(f)$ for $f \in A$. Note that $f\neq 0$ because $M$ is torsion. Write $f=g_1 \cdots g_r$ where $g _i  \in A$ is irreducible. Proceed by induction on $r$. If $r=1$, then $f$ is irreducible and $(f)$ is maximal. Last time, we showed this is equivalent to $A /f$ being simple, so it is of finite length. Otherwise, suppose $r>1$. Then we can construct a short exact sequence \[
    0 \longrightarrow A / g_1 \cdots g_{r-1} \longrightarrow  A /f = A / g_1 \cdots g_r \xrightarrow{\text{projection}} A / g_r \longrightarrow 0
\] where $1 \mapsto g_r, (f)=(g_1 \cdots g_r) \subseteq (g_r)$. Explaining this short exact sequence, giving a map of $A$-modules $A / Ah \to M$ is equivalent to $m \in M$, $h\cdot m=0$ ($m$ is defined as the image of 1). In our case, $M = A /f, h=g_1 \cdots g_{r-1},$ and $m \in A /f$ is $g-r$, which implies the first map exists.

Now to back up the claim that this is indeed a short exact sequence. 
\begin{enumerate}[label=(\alph*)]
\setlength\itemsep{-.2em}
    \item $\beta $ \emph{is surjective}: True.
    \item $\alpha $ \emph{is injective}: Given $x \in A$ such that $x \cdot g_r=0\pmod f$, this implies $g_1 \cdots  g_{r-1}$ divides $x$, because $g_r$ is irreducible.
        \begin{proof}
            $x g_r=f \cdot y=g_1 \cdots  g_r y \implies  \frac{x}{g_1 \cdots g_{r-1}}=y$.
        \end{proof}
    \item $\ker \beta =\im \alpha $: $\ker (\beta )= (g_r) / (g_1 \cdots g_r)$ is generated by $g_r$, which is exactly the image of $\alpha $.
\end{enumerate}
In the general case, here is a useful general trick for dealing with finitely generated modules. We proceed by induction on the generators; suppose $M$ is generated by $m_1 ,\cdots , m_n  \in M$. Take $m_0 \in M$ to be the submodule generated by $m_1 , \cdots ,m_{n-1}$. Then we have the short exact sequence \[
0 \to  M_0 \to  M \to  M / M_0 \to 0
\] and since $M / M_0$ is generated by $m_n $, we have $M / M_0 \simeq  A / I$ for some $I$.

For us, we have $M$ is finitely generated and torsion over a PID. Then \[
0 \to  M_0 \to M \to  A / I \to 0
\] as just described, where $M_0$ is generated by $n-1$ elements and torsion, by induction we have $M_0$ of finite length. Similarly, $A /I$ is also torsion. By last time, we have $A / f$ simple, which implies it is of finite length.

Our present goal is to classify finitely generated torsion modules over a PID.
\begin{example}
    Some examples:
    \begin{enumerate}[label=(\arabic*)]
    \setlength\itemsep{-.2em}
\item For $A =\Z$, finitely generated and torsion just means finite abelian groups.
\item For $A = \overline{k}[t]$, we have $V $ finite dimensional over $k$ with $T \colon V \to V$.
    \end{enumerate}
\end{example}
\begin{definition}[]
    An $A$-module $M$ is \textbf{indecomposable} if $M\neq 0$ and for any pair of $A$-modules $M_1,M_2$ with an isomorphism $M_1 \oplus M_2 \simeq M$, either $M_1=0$ or $M_2=0$.
\end{definition}
Last time we talked about simple modules, which have no interesting submodules (the module incarnation of a prime number). Indecomposable modules is a different notion which somewhat resembles the notion of a prime number.
\begin{example}
    Let $A= \Z$.
    \begin{enumerate}[label=(\arabic*)]
    \setlength\itemsep{-.2em}
        \item $\Z /p$ is indecomposable. In general, simple implies indecomposable.
        \item $\Z / 6 \simeq  \Z /2 \oplus \Z /3$ (remainder theorem), so $\Z /6$ is not indecomposable.
        \item $\Z /4$ is indecomposable but not simple. In general, $\Z / p^n $ is indecomposable, and finitely generated indecomposable abelian groups will have this form.
    \end{enumerate}
\end{example}
\begin{lemma}
    Let $M$ be a finite length $A$-module. Then $M$ is isomorphic to a direct sum $M \simeq M_1 \oplus \cdots \oplus M_r$, where $M_i $ is indecomposable.
\end{lemma}
\begin{proof}
    If $M$ is indecomposable, then we are done. Otherwise, $M \simeq M' \oplus M''$ for $M',M''$ non-trivial submodules. Keep going, we have to stop eventually since $M $ is of finite decreasing length.
\end{proof}
The summary is that now we need to classify indecomposable torsion modules.
