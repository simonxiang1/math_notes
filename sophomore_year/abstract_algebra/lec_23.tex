\section{October 20, 2021} 
Today we'll talk about the recognition criterion for semidirect products.

\begin{definition}[]
    Given groups $G,H$, an \textbf{extension} of $G$ by $H$ is the data $(E,\pi,i)$ where $E$ is a group, $\pi \colon E \twoheadrightarrow G , i \colon H \hookrightarrow E $, and we have that $\pi i (y)=1$ for every $h \in H$. Then the map $H \xrightarrow i \ker (\pi) \subseteq E$ is an isomorphism, and $\pi$ is surjective iff $E /\ker (\pi) \xrightarrow{\simeq } G $ iff $E /H \xrightarrow{\simeq } G$. In short, $E$ is a group, $H \subseteq E$ is normal, and $E /H \simeq G$.
\end{definition}
\begin{note}
    A note on notation. We write $1 \to H \xrightarrow i E \xrightarrow{\pi} G \to 1$, with an exception by replacing the 1s with 0s. The group $H$ (resp $G$) is uniquely written positively (resp $G$). A sequence like this is a \textbf{short exact sequence} (or SES for short) of groups.
\end{note}
\begin{example}
    Some examples of short exact sequences:
    \begin{enumerate}[label=(\arabic*)]
    \setlength\itemsep{-.2em}
        \item $0 \to  \Z /2 \xrightarrow{1 \mapsto 2} \Z/4 \xrightarrow{1 \mapsto 1} \Z /2 \to 0$ is a short exact sequence.
        \item $0 \to  \Z /2 \xrightarrow{x \mapsto (x,0)} \Z /2 \times \Z /2 \xrightarrow{(x,y)\mapsto y} \Z /2 \to $ is a short exact sequence.
        \item For every $n,m \geq 1$, $0 \to \Z/n \xrightarrow{1 \mapsto  m} \Z / nm \xrightarrow{ 1 \mapsto  1} \Z / m \to 0$ is a short exact sequence.
        \item For every $G,H,$ we have $1 \to  H \xrightarrow{h \mapsto (1,h)} G \times H \xrightarrow{(g,h) \mapsto g} G \to 1$ is a short exact sequence.
        \item Suppose $G$ acts on $H$ by group automorphisms. Then by last time, $1 \to  H \xrightarrow{h \mapsto (1,h)} G \ltimes H \xrightarrow{(g,h) \mapsto g} G \to 1$ is a short exact sequence.
    \end{enumerate}
\end{example}
\begin{namedthing}{Principle} 
   $G\ltimes H$ is the simplest extension of $G$ by $H$. 
\end{namedthing}
\begin{definition}[]
    A \textbf{map of extensions} of $G$ by $H$ is a commutative diagram \[
    \begin{tikzcd}
        1\arrow[r] & H \arrow[d,"\id"]\arrow[r,"i_1"] & E_1 \arrow[r,"\pi_1"]\arrow[d,"f"] & G \arrow[r]\arrow[d,"\id"] & 1\\
        1\arrow[r] & H \arrow[r,"i_2"] & E_2 \arrow[r,"\pi_2"] & G\arrow[r] & 1
    \end{tikzcd}
    \]for $f$ a homomorphism. 
\end{definition}
\begin{lemma}
    Any map $f \colon E_1 \to E_2$ is a group isomorphism, and $f ^{-1}$ is a map of extensions.  
\end{lemma}
\begin{proof}
    Some claims:
    \begin{enumerate}[label=(\arabic*)]
    \setlength\itemsep{-.2em}
\item Suppose $g \in E$, with $f(g)=1 \in E_2$, then 
    \begin{align*}
        1= \pi_2(1)=\pi_2f(g)=\pi_1(g)& \implies \pi_1(g)=1\\
                                      &\implies g \in \ker(\pi_1)\\
                                      &\implies  g \in H \subseteq E\\
                                      &\implies g= i_1(h).
    \end{align*}So $1=f(g)=f(i_1(h))=i_2(h)$, which implies $i_2(h)=1$, so $i$ is injective implying that $h=1$, which subsequently implies $g=1$.
\item $f$ is surjective, $g \in E_2$. Then 
    \begin{align*}
        \pi_2(g) \in G = E_1 /H & \implies \exists \gamma  \in E_2 \ \text{s.t.} \ \pi_1(\gamma ) =\pi_2(f(\gamma )) =\pi_2(g)\\
                                &\implies \pi_2(f(\gamma )\cdot g ^{-1})=1 \\
                                &\implies \exists h \in H \ \text{s.t.} \ i_2(h)=f(\gamma )\cdot g ^{-1}\\
                                &\implies g=i_2(h ^{-1} ) \cdot f(\gamma ).
    \end{align*}Set $\widetilde  \gamma  = i_1(h ^{-1}) \cdot \gamma  \in E_1$, then $f(\widetilde \gamma )-f i_1(h ^{-1})\cdot f(\gamma )=g $ which implies $g \in \im(f)$.\qedhere
    \end{enumerate}
\end{proof}
\begin{definition}[]
    A \textbf{splitting} of an extension $1 \to H\to E\to G\to 1$ is a map of groups $\sigma \colon G \to E$ such that $\pi \sigma=\id _G$. \[
    \begin{tikzcd}
            &                  & G \arrow[d, "\sigma"'] \arrow[rd, "\id"] &             &   \\
1 \arrow[r] & H \arrow[r, "i"] & E \arrow[r, "\pi"]                       & G \arrow[r] & 1
\end{tikzcd}
    \] An extension is said to be \textbf{split} if there exists a splitting.
\end{definition}
\begin{example}
    $0 \to  \Z /2 \to  \Z /4 \to \Z /2 \to 0$ is a non-split extension.
\end{example}
Given a split extension, this leads to an action of $G$ on $H$: define ${}^gh :=\sigma(g)h(\sigma(g) ^{-1})$. Then $\pi(\text{RHS} ) = \pi \sigma(g) \pi(h) \pi(\sigma(g)) ^{-1}=g \cdot 1\cdot g ^{-1}=1$, which implies the RHS is in $H$.
\begin{lemma}
    Given a splitting $\sigma$ for an extension $1 \to  H \to  E \to  G \to  1$, there exists a unique isomorphism of split extensions $ E \simeq  G \ltimes H$ by this action of $G$ on $H$.
\end{lemma}
\begin{proof}
    An exercise is to show the map $G\ltimes H \xrightarrow{(g,h) \mapsto  i(h)\sigma(g)} E$ is a homomorphism. Then this map is an isomorphism of split extensions. We'll continue this next time.
\end{proof}
