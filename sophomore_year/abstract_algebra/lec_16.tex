\section{October 4, 2021} 
Today is our first in person class! Today we'll do more on polynomials, discussing irreducibility and roots of polynomials.
Let $k$ be a field.
\begin{definition}[]
    A polynomial $f \in k[t]$ is \textbf{irreducible} if the following conditions hold:
    \begin{enumerate}[label=(\arabic*)]
    \setlength\itemsep{-.2em}
\item $f\neq 0$,
\item $f$ is not a unit ($\deg f > 0$),
        \item whenever $f=f_1 \cdot f_2$, $f_1$ is a unit or $f_2$ is a unit.
    \end{enumerate}
\end{definition}
A reminder that being a unit in this polynomial ring means a polynomial has degree zero.
\begin{example}Some examples:
    \begin{enumerate}[label=(\arabic*)]
    \setlength\itemsep{-.2em}
\item  $t \in k[t]$ is irreducible.
\item $\deg(f)=1$ implies that $f$ is irreducible, since if $f=f_1\cdot f_2$ then $\deg(f_1)+\deg(f_2)=1$, therefore one of the numbers has to be zero.
\item For $k=\Q,\R$, and $f(t)=t^2 +1$, then $f$ is irreducible. If $f=f_1\cdot f_2$, then $\deg(f_i)=0 \implies \deg(f_i)=1$. We can assume that the $f_i $'s are monic, which implies $f_i =t-\lambda _i $ for $\lambda _i  \in k$. So $t^2+1=(t-\lambda_1)(t-\lambda_2)$ for $\lambda_1,\lambda_2 \in \Q,\R$. This is impossible, because for $t=\lambda_1$, $\lambda_1^2+1=0,$ a contradiction.
    \end{enumerate}
\end{example}
\begin{remark}
    Sometimes it's convenient to think of ``plug in $\lambda$ for $t$'' as considering the (unique) homomorphism $k[t] \xrightarrow{t \mapsto \lambda}  k$, which is the identity on $k \subseteq k[t]$.
\end{remark}
\begin{note}
    A note on notation: for $A$ a commutative ring, $f \in A$, then we denote $(f):=A f \subseteq A$, the ideal generated by $f$ (repeated multiples of $f$). More generally, for $f_1,\cdots ,f_n  \in A$, we have $(f_1, \cdots , f_n ):= Af_1 +\cdots + Af_n $ the ideal generated by the $f_i $.
\end{note}
\begin{lemma}
    A polynomial $f \in k[t]$ is irreducible if and only if $(f)$ is maximal.
\end{lemma}
\begin{proof}
Suppose that $f$ is irreducible. We want to show that $(f)$ is maximal. Suppose $(f) \subseteq I \subsetneq k[t] $ where $I$ is some ideal. We want to show that $f=I$. We previously showed that $k[t]$ is a PID, so $I(g) $ for some non-unit polynomial $g$. Now \[
    (f) \subseteq (g) \implies  f \in (g) \implies f=g \cdot h
\] for some $h \in k[t]$. Since $g $ is not a unit, and $f$ is irreducible, then $h$ is a unit. Furthermore, \[
h ^{-1} \cdot  f =g \implies  g \in (f) \implies  (g) \subseteq  f \implies (f) =(g).
\] Conversely, suppose $f$ is maximal. We want to show that $f$ is irreducible. If $f=g\cdot h$ with $g$ not a unit, we want to show that $h$ is a unit. Suppose $(f ) \subseteq (g) \subsetneq k[t]$, by maximality we have $(f)=(g)$. This implies there exists an $\eta \in k[t]$, or $\eta \cdot  f=g$. Multiplying by (non-zero) $h$, we have \[
h \cdot \eta \cdot f=hg =f \underset{\text{domain} }{  \implies } h \cdot \eta=1.
\] So $h$ is a unit with inverse $\eta$.
\end{proof}
\begin{remark}
    All of this is valid in any PID. In general, it's difficult to stare at a polynomial and tell whether it's irreducible or not. In degree zero and one this is trivial, and for degree two plug it into the quadratic formula and find the discriminant.
\end{remark}
\begin{lemma}
    Let $f \in k[t]$ be non-zero. Then there exist $f_1, \cdots ,f_n $ irreducible such that $f=\prod _{i=1}^n f _i $.
\end{lemma}
\begin{proof}[Proof 1]
    This proof is for $k[t]$. If $f$ is not irreducible, then $f=g\cdot h$ for $g, h $ not units. If $\deg f=0$ then we are done. If $\deg f=1$ we have $f$ irreducible so we are done. For $g,h$ not units, $\deg(g) , \deg(h) < \deg(f)$, which implies $g=g_1, \cdots ,g_r$, $h=h_1, \cdots ,h_s$ with $g_i, h_j $ irreducible. So $f$ is a product of irreducibles and we are done. 
\end{proof}
\begin{proof}[Proof 2]
    This proof is valid for any PID. Given $f$ non-zero, by last time there exists some maximal ideal $\mathfrak m_1 \supseteq (f)$. We have $\mathfrak m_1=(f_1)$ for $f_1$ irreducible, which implies $f_1 \mid f$. If $\frac{f}{f_1}$ is a unit, then we are done. Otherwise, we can take the ideal $\left( \frac{f}{f_1} \right) \subseteq \mathfrak m_2=(f_2)$ for $f_2$ irreducible. We get that $f_2 \mid \frac{f}{f_1}$- if $\frac{f}{f_1f_2}$ is a unit we are done, otherwise we repeat. If the process $f=f_1 \cdots  f_n \cdot  \frac{f}{f_1 \cdots f_n }$ doesn't terminate (for all $n>0,f_i $ irreducible), $f \in (f_1), f\in (f_1f_2) \subseteq (f_1)$. Eventually, \[
        (f_1) \supsetneq (f_1f_2) \supsetneq \cdots 
    \] which is a strict inclusion since $f_2$ is not a unit, and so on. For $I= \bigcap_{n>0} (f_1 \cdot  \cdots  \cdot  f_n )$, we want to show that $I=(0)$. This gives a contradiction since $f \in I$ with $f$ non-zero. Since we are in a PID, this implies $I=(g) \subseteq (f_1 \cdots  f_n )$ for all $n >0$. We can then form $\frac{g}{f_1 \cdot  \cdots \cdot f_n },$ where \[
    (g) \subsetneq \left( \frac{g}{f_1} \right) \subsetneq \left( \frac{g}{f_1f_2} \right) \subsetneq \cdots 
\]Decreasing sequences of ideals are more subtle than increasing sequences of ideals, which we have transformed our increasing sequence into by the fact that we live in a PID. Form $J = \bigcup_{n>0} \left( \frac{g}{f_1 \cdots f_n } \right) ,$ and $J=(h)$. But this is a union, so $h \in \left( \frac{g}{f_1 \cdots f_n } \right) $ for some $n >0$. Moving some symbols around,
\begin{align*}
    \frac{g}{f_1 \cdots  f_{n+1}}& \in J=(h),\\
    \frac{g}{f_1 \cdots f _{n+1}} & \in  \left( \frac{g}{f_1 \cdots f_n } \right), \\
    \frac{g}{f_1 \cdots f_{n+1}}&=\eta \cdot \frac{g}{f_1 \cdots f_n }.
\end{align*}
Since we live in a domain, $1=f_{n+1}\cdot \eta$, so $f_{n+1}$ is a unit, a contradiction.
\end{proof}

That was a lot of manipulation, how do we set up the argument? There is some sort of built in ``finiteness'', the fact that we can't be divisible by an infinite amount of irreducible polys. So for ideals, the union eventually stabilizes, and the increasing sequence eventually stops.
