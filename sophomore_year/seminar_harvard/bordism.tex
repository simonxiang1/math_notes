\section{Quantum Mechanics} 
Today's speaker is Justin Kulp from the Perimeter Institute for Theoretical Physics. We will talk about ``gapped phases of quantum matter''. There are different camps interested in this: the condensed matter camp/quantum information (QI), the math camp, and the high energy physics (HEP) camp. They may say things like this:
\begin{itemize}
\setlength\itemsep{-.2em}
    \item Cond-MAT/QI: Gapped system, microscopic Hamiltonians, phases, SPt, anyons
    \item MATH: Formal TFTs, $\pi_0$, homotopy, group cohomology, cobordism, MTC
    \item HEP: Gauge theory, TQFTs, field, Dijkgraaf-Witten.
\end{itemize}
We will not talk about quantum mechanics.\footnote{Resources for QM: Ryan Hall- QM. Freed. Mackey's book on QM. Varadarajan. Kapustin 1303.6917?} Regardless of choice of axioms, we have three objects everyone agrees on.
\begin{itemize}
\setlength\itemsep{-.2em}
    \item $\mathcal{H} $ the \textbf{state space}, a complex separable Hilbert space
    \item $\mathrm{End}(\mathcal{H} )$, some algebra of operators on $\mathcal{H} $, which will contain something called \textbf{observables} 
    \item $H$ the \textbf{Hamiltonian}, a non-negative self-adjoint operator.
    \item \textbf{Unitary evolution of states}, a one parameter group acting on $\mathcal{H} $ generated by $H$. In other words, a map $\R \mapsto U(\mathcal{H} ), t \mapsto U_t= e^{-it H \mid  \hbar}$ ?? called the time-evolution operator.
\end{itemize}Since $H$ is non-negative, $z \mapsto  U_z = e ^{i \tau H \mid  \hbar}$, where $\tau$ is \textbf{Euclidian time}, $\tau \mapsto U_{\tau}=e^{-\tau H \mid \hbar}$, $\tau>0$. Why? This turns oscillatory things into exponentially decaying things. This makes QFT analogous to Statistical Field Theory.

\begin{example}
    A nice system is a particle on a ring. Consider the classical Lagrangian $L=\frac{1}{2}\dot x^2$, then after identifying $x\sim x+2\pi$ we can view $x$ as a particle on a ring. After Hamiltonification we get $\hat{H}=- \frac{1}{2}\partial _x^2$, and $\mathcal{H} =L^2(S^1 ;\C), \tau \mapsto  U _{\tau}=e^{-\tau \partial ^2_x}$. Our eigenfunction is $\psi _n (x)= \frac{e^{inx}}{\sqrt{2 \pi} }$, and evaluation is $E_n = \frac{n^2}{2}$. Then $L= \frac{1}{2}\dot x^2+ \frac{1}{2 \pi}\theta \dot x$. Formally, $\hat{H}= \frac{1}{2}\left( -i \partial _x- \frac{1}{2\pi}\theta \right) ^2,\mathcal{H} =L^2(S^1 ;\mathcal{L} _{e^{i\theta}})$. Okay this isn't worth it I will spend my time watching the previous lectures instead.
\end{example}

