\section{August 26, 2021}
Algebraic geometry is about trying to think about algebraic objects in terms of geometry, for example visualizing $x^2+y^2=1$ as the circle. For example, we can discuss things like symmetries, groups, parametrizations, calculus, length ($2\pi$), the topology, etc. Grothendieck says that we should be able to do this to \emph{all} algebraic objects.

There is a natural map from Geometry  $\to $ Algebra by looking at functions. Let $k$ be a field, like $\R$ or $\C$. Let $X$ be a set or space that we do geometry. Then consider the map\[
    X\mapsto \mathrm{Fun}_k(X):= \{f \colon X \to k\} ,
\] where $\mathrm{Fun}_k(X)$ forms a commutative ring. Addition is given by $f(x)+g(x)=(f +g)(x)$ and multiplication is $f(x) \cdot g(x)=(fg)(x)$. For example, if $X =S^1  \subseteq \R^2$, consider polynomial functions on $X$, the restriction $\{ \text{polynomials on} \ \R^2\} =k[x,y]$. This ring of polynomial functions $\mathrm{Poly}(S^1 )\simeq k[x,y] / (x^2+y^2-1) $ naturally recovers our polynomials valued on the circle.

The first topic we define is the notion of a \textbf{scheme}, a class of geometric spaces or setting to do geometry. Inside of schemes we have $\operatorname{Spec}R$, where $R$ is a commutative ring.

\subsection{Preamble to algebraic geometry}
Let $R= \Z[a,b,c] / (a^n +b^n -c^n )$, which results in a geometric object called $\Spec R$ (analogously with $\Spec \Z$). ``Spectrum'' refers to a rainbow, like the spectrum of an operator, a subset of $\R$ given by eigenvalues on a matrix. If $A \in \mathrm{Mat}_{n \times n}(\R)$, we look at $\R[x] / ( \text{characterisitic polynomials of} \ A)$. Algebraic topologists also like the work spectrum, but these ideas are unrelated.

Let $X$ be a space, and $x \in X$. How do points relate to functions? Given a function, we can evaluate it at that point. Let $R= \mathrm{Fun}_k(X)$, then we get a ring homomorphism $\mathrm{eval}_x \colon R \to k$, where $f \mapsto f(x)$. Then $\ker( \mathrm{eval}_x)=m_x=\{f / f(x)=0\} \subseteq R$, which is a \emph{maximal ideal}. Modding out by maximal ideals gives a field, and the kernel is maximal since if $m_x \subseteq I \subseteq R$, $0 \subseteq I /m \subseteq R /m=k$. This results in an ideal between 0 and a field, therefore $m_x$ is maximal.

The basic algorithm is that points $=R \leadsto$ maximal ideals of $R$ ($R /m\simeq k)$, which we want to realize as  $R=\mathrm{Fun}_k(X)$. Gelfand duality says that if $X$ is a compact Hausdorff topological space, then $X \mapsto C(X)$ where $C(X)$ is a commutative $C^*$-algebra. So this idea is not unique to algebraic geometry. We want to take this idea and modify it to work for arbitrary commutative rings. In particular, we allow
\begin{itemize}
\setlength\itemsep{-.2em}
    \item Rings with nilpotent elements, eg $\varepsilon  ^2=0$. This was outlawed in classical AG. In the world of schemes, this corresponds to analysis, in particular derivatives.
    \item We are not going to assumed that our fields are algebraically closed, ($k= \overline{k}$), or even a field at all. This corresponds to arithemtic, since one of the things we want to do is study the integers by AG.
    \item The third generalization of classical AG is gluing. Just one $\Spec R$ is not all the spaces we want, we glue them together like manifolds which corresponds to topology.
\end{itemize}
These are the main differences between pre-Grothendieck AG and modern AG.
\subsection{Introduction to category theory}
Category theory is an indispensible tool in many areas of math. There isn't an obvious home for it in the curriculum, so this is the home.

\begin{definition}[]
    A category $\mathcal{C} $ consists of two collections[sets]\footnote{We can't say sets because of some logic issue with Russell's paradox, but think of them as sets.}:
    \begin{itemize}
    \setlength\itemsep{-.2em}
        \item $\operatorname{Ob}\mathcal{C} $, the set of objects,
        \item $\operatorname{Mor}\mathcal{C} \to \operatorname{Ob}\mathcal{C} \times \operatorname{Ob}\mathcal{C} $, the set of morphisms.
    \end{itemize}
    For $X,Y \in \operatorname{Ob}\mathcal{C} $, this maps to $\Hom_{ \mathcal{C} }$ {\color{red}todo:go back and finish}. Finally, we have a composition: for all $X,Y,Z \in \mathcal{C} $, \[
        \overset{f}{\overbrace{\Hom _{\mathcal{C} }(X,Y)} } \times  \overset{g}{\overbrace{\Hom _{\mathcal{C} }(X,Y)} } \to \overset{g \circ f}{\overbrace{\Hom _{\mathcal{C} }(X,Y)} }
    \] {\color{red}todo:finish def} 
\end{definition}
\begin{example}
    A category $\mathcal{C} $ with one object $X$ has the data $\Hom _{\mathcal{C} }(X,X):=A$. Then $A$ is a monoid, since $1_X \in A$ is the unit and multiplication is composition. If we require the morphism to be invertible, then $A$ is a group.
\end{example}

\begin{definition}[]
    We say $f \in \Hom _{\mathcal{C} }(X,Y)$ is an \textbf{isomorphism} if there exists a  $g \in \Hom _{\mathcal{C} }(Y,X)$ such that $f \circ g=1_Y$, $g \circ f=1_X$. A category $\mathcal{C} $ is a \textbf{groupoid} if all morphisms are isomorphisms.
\end{definition}
\begin{example}
    A groupoid with only one object is a group. The most famous groupoid is the fundamental groupoid (or Poincar\'e groupoid) $\pi_{\leq 1}(X)$, where $\operatorname{Ob}:= \{x \in X\} $ points, and morphisms are $\Hom(x,y)=$ paths from $x$ to $y$ up to homotopy. In other words, this is $\pi_1(X)$ without reliance on the basepoint.
\end{example}
\begin{example}[Posets as categories]
   Let $S$ be a set with a partial order $\leq$, that is for all $x,y,z \in X$,
   \begin{itemize}
   \setlength\itemsep{-.2em}
       \item $x \leq x$,
       \item $x\leq y \leq z \implies x \leq z$,
        \item $x\leq y$ and $y \leq x \implies x=y$.
   \end{itemize}
   This gives rise to a category $\mathcal{C} , $ where $\operatorname{Ob}\mathcal{C} =S$, and morphisms are $\Hom _{\mathcal{C} }(X,Y)= \{x \leq y\} $ (there is an arrow from $x$ to $y$ if $x\leq y$). This is a \textbf{small} category.
\end{example}
\begin{example}
    Here are examples of ``big'' categories. Before, we could list the objects. 
    \begin{itemize}
    \setlength\itemsep{-.2em}
        \item $\mathsf{Set} $ is the category where objects are sets and morphisms are maps between sets.
        \item $\mathsf{Group} $ is the category where objects are groups and morphisms are groups homomorphisms.
        \item $\mathsf{Top} $ is the category of topological spaces with continuous maps as morphisms.
        \item $\mathsf{Ring} $ is the category of rings with ring homomorphisms. 
        \item $R$-$\mathsf{Mod} $ is the category of modules over a ring $R$, where $k$-modules are just vector spaces in $\mathsf{Vect} _k$ and $\Z$-modules are just abelian groups in $\mathsf{Ab} $.
    \end{itemize}
    Objects are irrelevant. The whole point of category theory is we should never think about categories without thinking about the morphisms. This leads into the next idea: we have categories as objects, but how do two categories talk to each other?
\end{example}

\begin{definition}[]
    Let $\mathcal{C} ,\mathcal{D} $ be categories. Then a \textbf{functor} $F \colon \mathcal{C}  \to \mathcal{D}  $ is 
    \begin{itemize}
    \setlength\itemsep{-.2em}
        \item 
    a function $\operatorname{Ob}\mathcal{C} \xrightarrow{F}  \operatorname{Ob}\mathcal{D} $, where $X \in \mathcal{C} \mapsto  F(X) \in \mathcal{D} $,
\item given $X,Y \in \mathcal{C} $, \[
        \Hom _{\mathcal{C} }(X,Y) \xrightarrow{F} \Hom _{\mathcal{D} }(F(X),F(Y))
\] which preserves identities and composition: \[
\begin{tikzcd}
{\Hom_{\mathcal C}(X , Y)\times \Hom_{\mathcal C}(Y,Z)} \arrow[r] \arrow[d, "F\times F"] & {\Hom_{\mathcal C}(X,Z)} \arrow[d, "F"] \\
{\Hom_{\mathcal D}(F(X) , F(Y))\times \Hom_{\mathcal D}(F(Y),F(Z))} \arrow[r]            & {\Hom_{\mathcal D}(F(X),F(Z))}         
\end{tikzcd}
\] This diagram \emph{commutes}, eg it doesn't matter which way you compose the arrows.
    \end{itemize}
    This defines what we call a \textbf{covariant functor}. A \textbf{contravariant functor} reverses the direction of arrows, eg $\Hom _{\mathcal{C} }(X,Y) \xrightarrow{F} \Hom _{\mathcal{C} }(F(Y),F(X))$. 
\end{definition} 
In terms of monoids, we can define the opposite monoid as $a \cdot b=b\cdot a$, and we can do the same thing in terms of categories: morphisms go the other way. Therefore we can think of contravariant functors as standard covariant functors in $\mathcal{C} ^{\mathrm{op}}$.
\begin{example}[Functions]
    This is the example our class is built on. Let $\mathrm{Fun}_k \colon \mathsf{Set}  \to \mathsf{Ring} $\footnote{From now on in this class, all rings are commutative.} be given by $X \mapsto \mathrm{Fun}_k(X)$. What make this a functor is that it's compatible with morphisms: given a map $X \to Y$, how do we relate functions on $Y$ to functions on $X$? Functions pull back: 
\[
\begin{tikzcd}
X \arrow[rr, "\pi"] \arrow[rd, "\pi^*f"'] &   & Y \arrow[ld, "f"] \\
                                          & k &                  
\end{tikzcd}
\] 
This tells us that $(\pi^*f)(x)=f(\pi(x))$. So $\mathrm{Fun}_k(Y) \xrightarrow{\pi^*} \mathrm{Fun} _k(X)$ is a contravariant functor, which we call $\Spec$. This gives rise to the slogan for this class: Algebra=(Geometry)$^{\mathrm{op}}$.
\end{example}


