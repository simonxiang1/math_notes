\section{September 7, 2021} 
Today we'll talk more about the theme of optimization problems in categories by universal properties. A more useful way of thinking about initial and final objects are limits and colimits.

\subsection{Limits and colimits}
Consider $\mathcal{C} $ a category, and we consider a functor $A \colon I \to \mathcal{C} $. We say $A$ is an $I$\textbf{-shaped diagram}, where $I$ is some small (simple) category. This functor takes the shape of $I$ and makes a picture  in $\mathcal{C} $. \[
\begin{tikzcd}
i \arrow[rr] \arrow[rd] &              & j \\
                        & k \arrow[ru] &  
\end{tikzcd} \qquad \implies  
\qquad 
\begin{tikzcd}
A_i \arrow[rr] \arrow[rd] &                & A_j \\
                          & A_k \arrow[ru] &    
\end{tikzcd}
\] A \textbf{cocone} of $A \colon I \to \mathcal{C} $ is an object $B \in \mathcal{C} $ and maps $f _i  \colon A_i  \to B$ for all $i \in I$ commuting with all morphisms in $I$. \[
\begin{tikzcd}
                                      & A_j \arrow[dd] \arrow[rrd] &  &   \\
A_i \arrow[ru] \arrow[rd] \arrow[rrr, crossing over] &                            &  & B \\
                                      & A_k \arrow[rru]            &  &  
\end{tikzcd}
\] A \textbf{cone} is defined analogously, where the maps are from $B \to A_i $. \[
\begin{tikzcd}
                                       &  &                           & A_j \arrow[dd] \\
B \arrow[rrru] \arrow[rr] \arrow[rrrd] &  & A_i \arrow[ru] \arrow[rd] &                \\
                                       &  &                           & A_k           
\end{tikzcd}
\] 
\begin{definition}[]
    A \textbf{colimit} of a diagram $A \colon I \to C$ is an initial object of cocones $\varinjlim A$. A  \textbf{limit} of $A \colon I \to C$, $\varprojlim A$, is a final cone.
\end{definition}
So colimits are ``filling in'' the diagram on the right, while limits are ``filling in'' the diagram on the left. This is one of the most important ideas in category theory, along with adjoint functors which essentially say the same thing.

\begin{example}
    Let $I$ be the category with two objects $\{1,2\} $ and no maps between them. Then a functor $A \colon I \to \mathcal{C} $ gives two objects $A_1,A_2 \in \mathcal{C} $. To figure out the limit $\varprojlim A$, we need something mapping  \emph{to} the $A_i $. That is, a cone such that anything mapping into the $A_i $ factors through $\varprojlim$. This turns out to be the  \textbf{product} $A_1\times A_2$. \[
    \begin{tikzcd}
                                               &  &                                                    & A_1 \\
B \arrow[rrrd] \arrow[rrru] \arrow[rr, dashed] &  & \varprojlim :=A_1 \times A_2 \arrow[ru] \arrow[rd] &     \\
                                               &  &                                                    & A_2
\end{tikzcd}
    \] Similarly, the \textbf{coproduct} is defined analogously. \[
    \begin{tikzcd}
A_1 \arrow[rd] \arrow[rrrd] &                                                                        &  &   \\
                            & \varinjlim:=A_1\coprod A_2 \arrow[rr, "\exists !" description, dashed] &  & B \\
A_2 \arrow[ru] \arrow[rrru] &                                                                        &  &  
\end{tikzcd}
    \]In $\mathsf{Set} $, let $X,Y$ be sets. Then for a set $C$ with maps $C\xrightarrow{f} X$ and $C\xrightarrow{g} Y$, the maps factor through the unique map $(f,g)$ by projection. In the same way, for maps $X\xrightarrow{h} B,Y\xrightarrow{k} B$, they factor through the disjoint union by inclusion. \[
    \begin{tikzcd}
  &  &                                                                                                              & X \arrow[rd, "\iota_X"'] \arrow[rrrd, "h"] \arrow[llld, "f"'] &                                                                    &  &   \\
C &  & \varprojlim:=X \times Y \arrow[ru, "\pi_X"'] \arrow[rd, "\pi_Y"] \arrow[ll, "\exists !" description, dotted] &                                                               & \varinjlim:=X\coprod Y \arrow[rr, "\exists !" description, dashed] &  & B \\
  &  &                                                                                                              & Y \arrow[ru, "\iota_Y"] \arrow[rrru, "k"'] \arrow[lllu, "g"]  &                                                                    &  &  
\end{tikzcd}
    \] 
\end{example}
\begin{example}
    In $\mathsf{Grp} $, the product is just $G \times H$, but the coproduct is actually the free product. This illustrates the idea that in general colimits are more complicated for limits.
\end{example}
Consider the forgetful functor $\mathrm{For}\colon \mathsf{Grp} \to \mathsf{Set} $, where $G\mapsto $ the underlying set and $\varphi \colon G \to H$ maps to the underlying set map. The functor $\mathrm{For}$ takes products to products, but \emph{doesn't} preserve coproducts. The general principle is that forgetful functors preserve limits. A mneumonic is that limits $\leftrightarrow$ subs and colimits  $\leftrightarrow$ quotients, and quotients are more complicated. Limits map \emph{in}, and colimits map \emph{from}.

\begin{example}
    Consider the categories $\mathsf{Vect} _k$ or $R$-$\mathsf{Mod}$. It turns out that products and coproducts are the same, the direct product $V \oplus W$. The category $\mathsf{Ab} $ is different from the category of groups, because the direct some $M\oplus N$ is a biproduct while for groups the product $M\times N$ and the free product $M * N$ are very different. We defined the (co)product by the (co)limit of two points, and similarly the repeated direct sum $\bigoplus _i V_i $ is the (co)limit of a bigger diagram.

    In $\mathsf{Vect} $, we have infinite products and infinite coproducts. But they are different. The product of vector spaces $\prod V_i =V_1 \times V_2 \times  \cdots  x \cdots $ is the set theoretic $\{(v_1,v_2, \cdots )\} $, while the infinite direct sum is $\bigoplus V_i = \{(v_1,v_2,\cdots ) \mid  \text{all but finitely many} \ v_i =0\} $..
\end{example}

\subsection{Fiber products/pullbacks}
Consider our index category $\bullet_1 \to \bullet_2$ with a functor to $\mathcal{C} $. This gets send to $A_1 \to A_2$. A colimit would be the best thing that $A_1,A_2$ map to in a compatible way. $A_1\to A_2,$ and $A_2 \to A_2$ by $\id$, so $\varinjlim\{A_i \} =A_2$, similarly $\varprojlim \{A_i \} =A_1$. This category has a final and initial project, then the colimit $\varinjlim A=A(\text{final object} )$, similarly $\varprojlim A= A( \text{initial object} )$.


Now consider the diagram $ \bullet_1 \rightarrow \bullet_3 \leftarrow \bullet_2$. This diagram has a final object but no initial object. We ask for the limit (???) {\color{red}todo:figure out limit or colimit, initial or terminal, diagram} 

The \textbf{pushout} is defined in a similar way, as the \emph{colimit} of the diagram $\bullet_1 \leftarrow \bullet_3 \rightarrow \bullet_2$. {\color{red}todo:diagram} 

set:disjoint union mod equiv relation, in grp its frere product with amalgamation.

Our next class of examples is (co)equalizers. 

localization and complection as limits and coliimits?
