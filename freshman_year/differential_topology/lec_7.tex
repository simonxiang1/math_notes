\section{February 9, 2021}
\subsection{Embeddings and Submanifolds}
Recall that $f \colon M \to N$ is an \textbf{embedding} if it's injective, an immersion, and a homeomorphism onto its image. We say $Q \subseteq N$ is a \textbf{submanifold} if about every $q \in Q$ there exists a submanifold chart. {\color{burntorange} TODO: read the notes about a submanifold chart}.

\begin{theorem}
    Let $f \colon M \to N$ be an embedding. Then $Q:= f(M) \subseteq N$ is a submanifold.
\end{theorem}
\begin{proof}
    Fix $q \in Q$. There exists a unique $p \in M$ such that $f(p)=q$ since $f$ is onto. By the normal form, choose charts $(U,x)$ about $p$ and $(V,q)$ about
\end{proof}
{\color{burntorange} TODO: something happened w/short exact sequences}
