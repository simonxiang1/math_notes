\section{February 9, 2021}
We have three basic ways to associate a ``shape'' to a function $f \colon M \to N$: the \emph{image} $f(M)\subseteq N$, the \emph{preimage} $f^{-1}(q)\subseteq M$, and the \emph{graph} $\Gamma(f) \subseteq M\times N$ of $f$. 
\begin{itemize}
    \item For $M,N$ smooth manifolds and $f$ a smooth function, the graph $\Gamma (f)$ is always a submanifold of $M \times N$, and is diffeomorphic to the domain $M$.
    \item If $f$ is an embedding, then $f(M)$ is diffeomorphic to $M$.
    \item If $q$ is a regular value, then $f ^{-1}(q)$ is a submanifold of the domain $M$.
    \item We will soon study \emph{transversality}, the condition for the inverse image $f ^{-1}(Q) \subseteq M$ of a submanifold $Q \subseteq N$ to be a submanifold.
\end{itemize}

\subsection{Embeddings and submanifolds}
\begin{theorem}
    Let $f \colon M \to N$ be an embedding. Then $f(M)\subseteq N$ is a submanifold.
\end{theorem}
\begin{proof}
    Let $Q$ denote $f(M)$. Fix $q \in Q$: we must construct a submanifold chart about $q$. Let $p \in M$ be the unique point so that $f(p)=q$. Since $f$ is immersive, we have charts $(U,x)$ about $p$ and $(V,y)$ about $q$ such that $y \circ f\circ x^{-1}(x^1,\cdots ,x^m)=(x^1,\cdots ,x^m,0,\cdots ,0)$. We claim there exists an open subset $V' \subseteq V$ such that the restricted chart $(V',y)$ is a submanifold chart. 

    If the condition for being a submanifold chart fails, then we have a sequence $\{p_k\} _{k=1}^{\infty} \subseteq M \setminus  U$ such that $\lim _{k \to \infty}y^j (f(q_k))=0$ for $j=m+1,\cdots ,n$. So the sequence $\{f(q_k)\} \subseteq V$ converges to a point of $f(U)$, and since $f$ is a homeomorphism onto its image we conclude that $\{p_k\} \subseteq M \setminus U$ converges to a point of $U$, which is a contraction\footnote{This is supposed to be ``contradiction'', but I find my typo funnier.} since $M \setminus U$ is closed in $M$.
\end{proof}

\subsection{Regular values and submanifolds}
Let us talk about the algebra.
\begin{definition}[]
    A sequence \[
    V \overset{T}{\longrightarrow} W \overset{S}{\longrightarrow} X
    \] of linear maps of vector spsaces is \textbf{exact} if $S \circ T=0$ and $\ker S= \im V$ as subspaces of $W$. A \textbf{long exact sequence} \[
    \cdots \longrightarrow V^i \longrightarrow V^{i+1}\longrightarrow V^{i+2}\longrightarrow\cdots 
    \] is a sequence of linear maps in which every two consecutive maps forms an exact sequence. A \textbf{short exact sequence} is a LES of the form \[
    0 \longrightarrow V' \overset{T}{\longrightarrow} V \overset{S}{\longrightarrow} V''\longrightarrow 0.
\] In the above equation, the linear map $T \colon V' \to V$ is injective with cokernel ($V / \im T)$ isomorphic to $V''$, and the linear map $S \colon V \to V''$ is surjective with kernel isomorphic to $V'$. Furthermore, if $V',V,V''$ are finite dimensional, then $\dim V=\dim V'+\dim V''$.
\end{definition}
\begin{definition}
Let $P \subseteq M$ be a submanifold and $p \in P$. 
\begin{enumerate}[label=(\arabic*)]
    \item The \textbf{codimension} of $P$ in $M$ at $p$ is defined by $\operatorname{codim}_p(P \subseteq M)=\dim_p M-\dim_p Q=\dim (T_pM /T_pP)$.
    \item The quotient space $T_p M /T_p P$ is the \textbf{normal (space)} to $P$ at $p$.
\end{enumerate}
Sometimes we use the notation $v_p=T_p M /T_p P$ to denote the normal space at $p$. Observe that there is a short exact sequence \[
0 \longrightarrow T_p P \longrightarrow T_p M \longrightarrow v_p \longrightarrow 0.
\] 
\end{definition}
\begin{theorem}
    Let $f \colon M \to N$ be a smooth map of smooth manifolds and $q \in N$ a regular value. Then $P:= f ^{-1}(q) \subseteq M$ is a submanifold of codimension equal to $\dim_q N$. Furthermore, if $p \in P$, \[
        T_p P=\ker (df_p \colon T_p M \to T_p N).
    \] \end{theorem}
We can express this with the short exact sequence \[
    0 \longrightarrow T_p P \longrightarrow T_p M \overset{df_p}{\longrightarrow} T_q N\longrightarrow 0,
    \] illustrated in {\color{red}todo:figure.} In general, the codimension is a locally constant function $\operatorname{codim}\colon P \to \Z^{\geq 0}$. Our theorem asserts that if $P$ is cut out by a single function, then $\operatorname{codim}$ is constant.

    \begin{proof}
        {\color{red}todo:} 
    \end{proof}
\begin{example}
    The $2$-sphere $S^2 \subseteq \A^3_{x,y,z}$ is cut out by the single function $f \colon \A^3 \to \R$, $(x,y,z) \mapsto x^2+y^2+z^2$. Namely, $S^2=f^{-1}(1)$. The differential $df=2x \,dx+2y \,dy+2z\, dz$ does not vanish at any point of $f ^{-1}(-1)$. Note that $0$ is a critical point, but $f ^{-1}(0) \subseteq \A^3$ is a submanifold (not of the expected codimension of $\dim \R$, though).
\end{example}
\begin{example}
    {\color{red}todo:$O^n $ orthogonal group, lie groups, a proposition,}
\end{example}

\subsection{A counting invariant; the fundamental theorem of algebra}
\begin{theorem}
    Let $M$ be a compact smooth manifold, $N$ a smooth manifold with $\dim M=\dim N$, and  $f \colon M \to N$ a smooth function. Set $N _{\text{reg} }\subseteq N$ the subset of regular values. Then the function \[
        \# \colon N_{\text{reg} } \to \Z^{\geq 0}, \quad q \mapsto \# f ^{-1}(q)
    \] 
\end{theorem}
The conclusion is that for any regular value $q \in N$ the subset $f^{-1}(q)\subseteq M$ is finite and its cardinality is a locally constant function of the regular value.
\begin{proof}
    {\color{red}todo:this. also todo: fundamental thm of algebra} 
\end{proof}
