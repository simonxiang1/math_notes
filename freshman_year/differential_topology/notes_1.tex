\part{Guillemin and Pollack}
\section{Chapter 1: Manifolds and Smooth Maps}
\begin{center}
    \sc{Introduction} 
\end{center}
These are supplementary notes, following the classic text on differential topology Guillemin and Pollack. Here are some things we should know before starting:

A cover $\{V_{\beta }\} $ is a \textbf{refinement} of another cover $\{U_{\alpha }\} $ if every set $V_{\beta }$ is contained in at least one $U_{\alpha }$. Since $\R^n $ is second-countable, every open cover $\{U_{\alpha }\} $ in $\R^n $ has a countable refinement. For a quick proof, take the collection of all open balls contained in some $U_{\alpha }$ with rational radii, centered at points with rational coordinates.

If $X\subseteq \R^n $, then $V\subseteq X$ is \textbf{(relatively) open} in $X$ if it can be written as the intersection of $X$ with an open subset of $\R^n$, or $V= \widetilde V \cap X$, where $\widetilde V$ is open in $\R^n $. If $Z\subseteq X$, we can also speak of open covers of $Z$ in $X$, meaning coverings of $Z$ by relatively open subsets of $X$. Every such cover of $Z$ may be written as the intersection of $X$ with a covering of $Z$ by open subsets of $\R^n $. Since $\R^n $ is second countable, every open cover of $Z$ relative to $X$ has a countable refinement. To see this, given $\{U_{\alpha }\} $ relatively open in $X$, write $U_{\alpha }= \widetilde U_{\alpha }\cap  X$. Then let $\widetilde V _{\beta }$ be a countable refinement of $\{\widetilde U_{\alpha }\} $ in $\R^n $, and define $V_{\beta }=\widetilde V_{\beta }\cap X$.
\orbreak
A mapping $f \colon U \to \R^m $ of an open $U\subseteq \R^n $ is called \textbf{smooth} if $f$ has continuous partial derivatives of all orders.
