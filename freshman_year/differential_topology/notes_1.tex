\part{Guillemin and Pollack}
\section{Chapter 1: Manifolds and Smooth Maps}
\begin{center}
    \sc{Introduction} 
\end{center}
These are supplementary notes, following the classic text on differential topology Guillemin and Pollack. Here are some things we should know before starting:

A cover $\{V_{\beta }\} $ is a \textbf{refinement} of another cover $\{U_{\alpha }\} $ if every set $V_{\beta }$ is contained in at least one $U_{\alpha }$. Since $\R^n $ is second-countable, every open cover $\{U_{\alpha }\} $ in $\R^n $ has a countable refinement. For a quick proof, take the collection of all open balls contained in some $U_{\alpha }$ with rational radii, centered at points with rational coordinates.

If $X\subseteq \R^n $, then $V\subseteq X$ is \textbf{(relatively) open} in $X$ if it can be written as the intersection of $X$ with an open subset of $\R^n$, or $V= \widetilde V \cap X$, where $\widetilde V$ is open in $\R^n $. If $Z\subseteq X$, we can also speak of open covers of $Z$ in $X$, meaning coverings of $Z$ by relatively open subsets of $X$. Every such cover of $Z$ may be written as the intersection of $X$ with a covering of $Z$ by open subsets of $\R^n $. Since $\R^n $ is second countable, every open cover of $Z$ relative to $X$ has a countable refinement. To see this, given $\{U_{\alpha }\} $ relatively open in $X$, write $U_{\alpha }= \widetilde U_{\alpha }\cap  X$. Then let $\widetilde V _{\beta }$ be a countable refinement of $\{\widetilde U_{\alpha }\} $ in $\R^n $, and define $V_{\beta }=\widetilde V_{\beta }\cap X$.
\orbreak
A mapping $f \colon U \to \R^m $ of an open $U\subseteq \R^n $ is called \emph{smooth} if $f$ has continuous partial derivatives of all orders. If the domain of $f$ is not open, we usually cannot speak of partial derivatives (for the concept to work, we need to be able to find a neighborhood around each point). So we generalize this definition a little. A map $f \colon X \to \R^m$ defined on an arbitrary $X$ in $\R^n $ is \textbf{smooth} if it can be locally extended to a smooth map on open sets, that is, if around each $x \in X$ there is an open set $U \subseteq \R^n $ and a smooth map $F \colon U \to \R^m$ such that $F$ equals $f$ on $U \cap X$. 

A smooth map $f \colon X \to Y$ of two subsets of Euclidian spaces is a \textbf{diffeomorphism} if it is a bijection, and the inverse map $f ^{-1} \colon Y \to X$ is also smooth. In this course, diffeomorphic sets are intrinsically equivalent.

\orbreak
\subsection{Tangent Space and the Differential}
These are actually supplementary notes handed out by Dr.\ Freed, not from G\&P.
\begin{definition}
    Let $\{V_{\alpha }\} _{\alpha \in A}$ be a collection of vector spaces indexed by a set $A$. Then the \textbf{direct product} $\prod _{\alpha \in A}V_{\alpha }$ is the Cartesian product of the sets $V_{\alpha }$ with componentwise addition and scalar multiplication. It is a vector space, possibly infinite dimensional. An element of the direct product is denoted $\xi = \{\xi _{\alpha }\} \in  \prod _{\alpha \in A}V_{\alpha }$; the $\alpha $-component of $\xi $ is $\xi _{\alpha }$. The sum is defined by $(\xi + \eta)_{\alpha }=\xi _{\alpha }+\eta _{\alpha }$, or $\{\xi _{\alpha }\} +\{\eta _{\alpha }\} = \{\xi_{\alpha }+\eta _{\alpha }\} $.
\end{definition}
Let $X$ be a smooth manifold with atlas $\mathcal{A} = \{(U_{\alpha },x_{\alpha })\} _{\alpha \in A} $. For $p \in X$, let $A_p\subseteq A$ be the set of indices $\alpha \in A$ such that $p \in U_{\alpha }$ and set $\mathcal{A} _p= \{( U_{\alpha },x _{\alpha })\} _{\alpha \in A}$. Suppose the dimension of $X$ at $p$ is $n$.
\begin{definition}
    The \textbf{tangent space} $T_p X$ is the subspace of the direct product $\prod _{\alpha \in  A_p}\R^n  _{\alpha }$ consisting of vectors $\xi = \{\xi _{\alpha }\} $ such that \[
        \xi _{\beta }= d (x _{\beta }\circ x _{\alpha }^{-1}) _{x _{\alpha }(p)}(\xi _{\alpha })
    \] for all $\alpha ,\beta \in  A_p$. Here $\R^n  _{\alpha }$ denotes the vector space $\R^n $ thought of as displacements in the codomain of the coordinate map $x _{\alpha }\colon  U_{\alpha } \to \A^n $.
\end{definition}

\subsection{The Inverse Function Theorem and Immersions}
If $X$ and $Y$ are smooth manifolds, then a smooth map $f \colon X \to Y$ is a \textbf{local diffeomorphism} if it diffeomorphically maps a neighborhood of point $x$ onto its image (a nbd of $y=f(x)$). Note that for local diffeomorphisms, the mapping $df_x \colon T_x(X) \to T_y(Y)$ is an isomorphism.

\begin{namedthm}{The Inverse Function Theorem}
  Suppose that $f \colon X \to Y$ is a smooth map whose derivative $df_x$ at the point $x$ is an isomorphism of tangent spaces. Then $f$ is a local diffeomorphism at $x$.
\end{namedthm}

\subsection{Sard's Theorem and Morse Functions}
What is measure zero? Transversality is a generalization of regularity, which is useful. We say some $A \subseteq \R^{\ell}$ has \textbf{measure zero} if it can be covered by a countable number of rectangular solids with arbitrary small total volume. A rectangular solid in $\R^{\ell}$ is a product of $I$ intervals in $\R$ with volume the product of the lengths. So $A$ has measure zero if for every $\varepsilon >0$, there exists a countable collection $\{S_1 ,S_2,\cdots \} $ of solids in $\R^{\ell}$ such that $A$ is contained in the union of the $S_i $, and \[
    \sum_{i=1}^{\infty} \operatorname{vol}(S_i )<\varepsilon .
\] 


\subsection{The Tangent Bundle}
The tangent spaces to $X$ at points are vector subspaces of $\R^n $ that generally overlap. The \textbf{tangent bundle} $T(X)$ is an artifice used to ``pull them apart''. Specifically, we have $T(X) \subseteq X\times \R^n $ defined by \[
    T(X) = \{(x,v) \in X \times \R^n \mid v \in T_x(X)\} .
\] $T(X)$ contains a copy $X_0$ of $X$, consisting of the points $(x,0)$. 

\subsection{Integration on Manifolds}
Why do we integrate over forms? 
\begin{namedthm}{Change of variables in $\R^k$} 
   Suppose $f \colon V \to U$ is a diffeomorphism of open sets in $\R^k$ and $a$ is an integrable function on $U$. Then \[
       \int_{U} a\, dx_1 \cdots  dx_k = \int _V (a \circ f) \,|\det (df)|\, dy_1 \cdots dy_k.
   \] 
\end{namedthm}
Changing variables by $f$ sends $a$ to the obvious pullback $a \circ f$, while scaling by the volume form $|\det (df)|$. The forms counteract this volume change. Consider the integrand to be a $k$-form $\omega= a\,dx_1\wedge \cdots \wedge dx_k$. Then define \[
\int_U \omega= \int_U a\,dx_1 \cdots dx_k.
\] So $\omega$ pulls back to the form $f^*(\omega)= (a \circ f) \det (df) dy_1 \wedge  \cdots \wedge  dy_k$. If $f$ preserves orientation, then $\det(df)>0$, so $f^* \omega$ is the integrand on the right in the Change of Variables theorem. We say $\omega$ is integrable if $a$ is.
\begin{namedthm}{Change of Variables in $\R^k$} 
   Assume $f \colon V \to U$ is an orientation-preserving diffeomorphism of open sets in $\R^k$ or $\H^k$, and let $\omega$ be an integrable $k$-form on $U$. Then \[
   \int _U \omega= \int_V f^* \omega.  
   \] If $f$ is orientation-reversing, then $\int_U \omega=-\int _V f^* \omega$.
\end{namedthm}
Recall the \textbf{support} of $\omega$ is the closure of the set of points where $\omega(x)\neq 0$; assume that this closure is compact, or $\omega$ is \textbf{compactly supported}. Say $\supp \omega \subseteq W \subseteq X$ where $W$ is open and parametrizable. If $h \colon U \to W$ is an orientation-preserving diffeomorphism of $W$ with $U \subseteq H^k$ open, $h^* \omega$ is a compactly supported smooth $k$-form on $U$. So $h^* \omega$ is integrable, and $\int_X\omega= \int_U h^* \omega$. If $g \colon V \to W$ is another parametrization of $W$, then $f = h ^{-1} \circ g$ is an orientation preserving diffeomorphism $V \to U$, and \[
\int _U  h^* \omega  = \int _V f^* h^* \omega=\int _V g^* \omega.
\] So $\int_X \omega$ is independent of parametrization. {\color{red}todo:something partitions of unity}. You can check that \[
\int_X (\omega_1+\omega_2)= \int _X \omega_1+ \int_X \omega_2 \qquad \text{and} \qquad \int_X c\omega=c \int_X \omega
\] for $c \in \R$.
\begin{theorem}
    If $f \colon Y \to X$ is an orientation-preserving diffeomorphism, then \[
    \int_X \omega=\int_Y f^* \omega
\] for every compactly supported, smooth $k$-form on $X$ (where $k=\dim X=\dim Y$).
\end{theorem}
We can only integrate $k$-forms over $X$ a $k$-manifold, but we can integrate lower dimensional forms over submanifolds. If $Z$ is an oriented submanifold of $X$ and $\omega$ is a form on $X$, our abstract operations give us a natural way of ``restricting'' $\omega$ to $Z$. Let $i\colon Z \hookrightarrow X$ denote the inclusion, and define the \textbf{restriction} of $\omega$ to $Z$ as the form $i^* \omega$. If $\omega$ is a 0-form, then  $i^*\omega$ is the usual restriction of $\omega$ to $Z$. If $\dim Z= \ell$ and $\omega$ is an $\ell$-form whose support intersects $Z$ in a compact set, define \[
\int_Z \omega=\int_Z i^* \omega.
\] 
\begin{example}
    Suppose $\omega=f_1dx_1+f_2dx_2+f_3dx_3$ is a smooth 1-form on $\R^3$, and $\gamma \colon I \to \R^3$ is a simple curve (diffeomorphism of $I$ onto $C=\gamma (I)$ a compact 1-manifold with boundary). Then \[
    \int_C \omega=\int_I \gamma ^* \omega.
\] For $\gamma (t)=(\gamma_1(t),\gamma_2(t),\gamma_3(t))  $ we have $\gamma ^* dx_i =d\gamma _i = \frac{d\gamma _i }{dt}dt$, and \[
\int _C \omega= \sum_{i=1}^{3} \int_{0}^{1} f_i [\gamma (t)] \frac{d\gamma _i }{dt}(t) \, dt.
\] If $\mathbf F$ is the vector field $(f_1,f_2,f_3)$ in $\R^3$, then this is the \textbf{line integral} of $\mathbf F$ over $C$ denoted $\oint \mathbf F\,d \gamma $.
\end{example}
\begin{example}
    Consider the compact supported 2-form on $\R^3$ given by $\omega=f_1 dx_2\wedge dx_3+f_2 dx_3\wedge dx_1+ f_3 dx_1\wedge dx_2$. {\color{red}todo:area form} 
\end{example}

\subsection{Summary of facts about the exterior derivative}
\begin{enumerate}[label=(\arabic*)]
\setlength\itemsep{-.2em}
    \item Linearity: $d(\omega_1+\omega_2)=d\omega_1+d\omega_2$ 
    \item Multiplication law: $d(\omega \wedge \theta)=(d\wedge ) \wedge  \theta+(1)^p \omega \wedge d\theta$
    \item $d(d\omega)=0$.
    \item $d$ is unique
    \item For $g \colon Y \to X$ a smooth map of manifolds with boundary, for every form $\omega$ on $X$, $d(g^* \omega)=g^* (d \omega)$.
    \item $d \circ g^* = g^* \circ d$ (5) but concise
    \item $df= \frac{\partial f}{\partial x_i } dx^i $
\end{enumerate}

\subsection{Cohomology}
All gradient vector fields have curl zero, but the converse depends on the domain of definition. Two closed $p$-forms are \textbf{cohomologous}, denoted $\omega \sim \omega'$, if $\omega-\omega'=d \theta$ (their difference is exact). Then $H^p(X)$ is the set of equivalent classes, and forms a real vector sapce. The 0 cohomology class is the collection of exact forms, since $\omega=\omega'=d \theta=0$.

If $f \colon X \to Y$ is a smooth map that pulls $p$-forms on $Y$ back to $p$-forms on $X$, since $f^*$ commutes with the derivative, $f^*$ pulls back closed forms to closed forms and exact forms to exact forms. In fact, if $\omega\sim \omega'$, then $f^* \omega\sim f^* \omega'$. So $f^*$ induces a pullback on cohomology classes, or a mapping $f^{\#} \colon H^p(Y) \to H^p(X)$. Note that $f^{\#}$ \emph{pulls back}, that is, if $f \colon X \to Y$, then $f^{\#}\colon H^p (Y) \to H^p(X)$.

You can show that
\begin{enumerate}
    \item If $X \overset{f}{\longrightarrow} Y \overset{g}{\longrightarrow} Z$, then $(g \circ f)^{\#}=f ^{\#}\circ g^{\#}$.
    \item $H^p(X)=0$ for all $p>\dim X$.
    \item $\dim H^0(X)$ is precisely the number of connected components in $X$.
\end{enumerate}

\subsection{Stokes Theorem}
We have a remarkable relationship between the operators $\int$ and $d$ on forms and the operation $\partial $ which to each manifold with boundary associates its boundary.

\begin{namedthm}{The Generalized Stokes Theorem} 
    Suppose $X$ is any compact-oriented $k$-dimensional manifold with boundary (which implies $\partial X$ is a $(k-1)$-manifold with the boundary orientation). Then if $\omega$ is any smooth $(k-1)$-form on $X$, \[
    \int _{\partial X}\omega= \int_X d\omega.
    \] 
\end{namedthm}
\begin{proof}
    Suppose $\omega$ has compact support contained in the image of a local parametrization $h \colon U \to X$, where $U \subseteq \R^k$ or $\mathbb H^k$\footnote{We omit the part of the proof where $U \subseteq \mathbb H^k$ for time constraints.} is open. If $U \subseteq \R^k$ is open, then $h(U)$ does not intersect the boundary. So \[
        \int _{\partial  X}\omega=0\quad \text{and} \quad \int_X d\omega= \int_U h^*(d \omega) = \int _U dv,
    \] where $v= h^*\omega$. Since $v$ is a $(k-1)$-form, write $v= \sum_{i=1}^{k} (-1)^{i-1}f_i  dx_1 \wedge  \cdots \wedge  \widehat{dx_i } \wedge \cdots \wedge dx_k$. Then $dv= \left( \sum _i  \frac{\partial f_i }{\partial x_i }dx_1 \wedge \cdots \wedge dx_k \right) $, so \[
    \int _{\R^k}=\sum _i  \int _{\R^k}\frac{\partial f_i }{\partial x_i }dx_1\cdots dx_k.
\] This is our usual notion of a multivariable integral, so you can change the order of the terms (Fubini's theorem). Integrating the $i$th term with respect to $x_i $, we get \[
\int _{\R^{k-1}}\left( \int_{-\infty}^{\infty} \frac{\partial f_i }{\partial x_i }dx_i  \right) dx_1 \cdots \widehat{dx_i } \cdots  dx_k.
\] Since $\int_{-\infty}^{\infty} \frac{\partial f_i }{\partial x_i }dx_i $ is the function of $x_1,\cdots , \hat{x}_i ,\cdots ,x_k$ that assigns to any $(k-1)$-tuble $(b_1,\cdots ,\hat{b}_i ,\cdots ,b_k)$ the number $ \int_{-\infty}^{\infty} g'(t) \, dt$, where $g(t)=f_i (b_1,\cdots ,t,\cdots ,b_k)$. Since $v$ has compact support, $g$ vanishes outside a sufficiently large interval $(-a,a)$ in $\R^1$. So by the FTC, \[
\int_{-\infty}^{\infty} g'(t) \, dt= \int_{-a}^{a} g'(t) \, dt=g(a)-g(-a)=0-0=0.
\] Thus $\int_X d\omega=0$.
\end{proof}
