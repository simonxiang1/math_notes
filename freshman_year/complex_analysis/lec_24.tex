\section{November 17, 2020}
Wasn't feeling too well today, no notes.
\section{November 19, 2020}
See above rip
\section{November 24, 2020}
Test today, some review.
\orbreak
\Large{\S 54 }\normalsize 
Maximum modulus
\begin{theorem}[Maximum modulus]
If $f$ is analytic and nonconstant in a domain $D$, then $|f(z)|$ has no maximum value in $D$, that is, there exists no $z_0 \in D$ st $|f(z)|\leq |f(z_0)|$ for all $z$ in $D$.
\end{theorem}
\begin{cor}
Suppose that $f$ is continuous on a closed bounded region $R$, and that $f$ is analytic and nonconstant in the interior of $R$. Then the maximum value of $|f(z)| \in R$ (which is always attained) occurs somewhere in the boundary and never in the interior.
\end{cor}
liouvilles them: every bounded entire function is constant.\\
\Large{\S 56 }\normalsize
\begin{itemize}
    \item Limits of sequences, easy
    \item basic facts abt series, easy
\end{itemize}
\Large{\S 59 }\normalsize
Some useful series expansions
\begin{itemize}
    \item $e^{z}=\sum_{n=0}^{\infty} \frac{z^n}{n!},\quad |z|<\infty$
    \item $\sin z=\sum_{n=0}^{\infty} (-1)^n \frac{z^{2n+1}}{(2n+1)!},\quad |z|<\infty$
    \item $\cos z= \sum_{n=0}^{\infty} (-1)^{n}\frac{z^{2n}}{(2n)!}, \quad |z|<\infty$
    \item $\sinh z =\sum_{n=0}^{\infty} \frac{z^{2n+1}}{(2n+1)!},\quad |z|<\infty$ 
    \item $\cosh z= \sum_{n=0}^{\infty} \frac{z^{2n}}{(2n)!},\quad |z|<\infty$ 
    \item $\frac{1}{1-z}=\sum_{n=0}^{\infty} z^n ,\quad |z|<1$
\end{itemize}
\Large{\S 62}\normalsize \ 
Laurent series
\begin{theorem}
    Suppose $f$ analytic in an annulus $R_1<|z-z_0|<R_2$ centered at $z_0$ and let $C$ denote a positively oriented closed countour in the annulus. Then for all $z$ in the annulus, $f(z)$ has the following series representation:
    \[
        f(z)=\sum_{n=0}^{\infty} a_n (z-z_0)^n +\sum_{n=1}^{\infty} \frac{b_n }{(z-z_0)^n },\quad R_1<|z-z_0|<R_2
    \] where \[
    a_n =\frac{1}{2\pi i}\int_{C}^{} \frac{f(z)}{(z-z_0)^{n+1}} \, dz \ (n\in \N),\quad b_n =\int_{C}^{} \frac{f(z)}{(z-z_0)^{-n+1}} \, dz \ (n\in \N\setminus \{0\} ).
    \] 
\end{theorem}
\begin{example}
\[
e^{\frac{1}{z}}=\sum_{n=0}^{\infty} \frac{1}{n!z^n }=1+\frac{1}{1!z}+\frac{1}{2!z^2}+\frac{1}{3!z^3}+\cdots, \quad 0<|z|<\infty
\] 
\end{example}
\Large{\S 66}\normalsize \
Integration and differentiation of power series
\begin{theorem}
    Let $C$ be a contour in the circle of convergence of a power series of the form $S(z)=\sum_{n=0}^{\infty} a_n (z-z_0)^n ,$ and $g(z)$ be some function continuous on $C$. Then 
    \[
        \int_{C}^{} g(z)S(z) \, dz=\sum_{n=0}^{\infty} a_n \int_{C}^{} g(z)(z-z_0)^n  \, dz
    \] 
\end{theorem}
\begin{cor}
    The sum $S(z)$ of power series is analytic at each point $z$ interior to the circle of convergence of that series.
\end{cor}
\Large{\S 67}\normalsize \
Multiplication and division of power series
\[
    f(z)g(z)=\sum_{n=0}^{\infty} c_n (z-z_0)^n ,\quad |z-z_0|<R
\] 
\[
    [f(z)g(z)]^{(n)}=\sum_{k=0}^{n} {n \choose k}f^{(k)}(z)g^{(n-k)}(z),\quad n,k\in \N,\quad {n\choose k}=\frac{n!}{k!(n-k)!}
\] 
\[
    c_n =\sum_{k=0}^{n} \frac{f^{(k)}(z_0)}{k!}\cdot \frac{g^{(n-k)}(z_0)}{(n-k)!}=\sum_{k=0}^{n} a_kb_{n-k}
\] 
\[
    \therefore f(z)g(z)=\sum_{k=0}^{n} a_kb_{n-k}(z-z_0)^n ,\quad |z-z_0|<R
\] 
\Large{\S 71}\normalsize \
Residues and stuff
\begin{theorem}[Cauchy's residue theorem]
   Let $C$ be a simple closed contour, then if $f$ is analytic inside and on $C$ save for a few singularities $z_k\in \mathring C$ we have \[
       \int_{C}^{} f(z) \, dz=2\pi i \sum_{k=1}^{n} \, \underset{z=z_k}{\operatorname{R es}} \, f(z).
   \] 
\end{theorem}
Some useful equations
\[
    b_n =\frac{1}{2\pi i}\int_{C}^{} \frac{f(z) \, dz}{(z-z_0)^{-n+1}},\, n=1 \implies \int_{C}^{} f(z) \, dz=2\pi i\,\underset{z=z_0}{\operatorname{R es} }\,f(z)
\] 
\[
    \int_{C_0}^{} f(z) \, dz=2\pi i\, \underset{z=\infty}{\operatorname{R es}} f(z).
\] 
\[
    \int_{C_0}^{} f(z) \, dz=-2\pi i \,\underset{z=0}{\operatorname{R es}} \left[ \frac{1}{z^2}f\left( \frac{1}{z} \right)  \right] \implies \underset{z=\infty}{\operatorname{R es}f(z)} =- \, \underset{z=0}{\operatorname{R es}} \left[ \frac{1}{z^2}f\left( \frac{1}{z} \right)  \right] .
\] 
\begin{theorem}
    If $f$ is analytic everywhere in the finite plane except a finite number of singular points interior to a positively oriented simply closed contour $C$ then     
\[
    \int_{C}^{} f(z) \, dz=2\pi i\, \underset{z=0}{\operatorname{R es}}\, \left[ \frac{1}{z^2}f\left( \frac{1}{z} \right)  \right]  .
\] 
\end{theorem}
\noindent\Large{\S 72}\normalsize \ 
The three types of isolated singularities
\begin{itemize}
    \item If principal part is infinite, essential singularity
    \item Finite $\implies $ pole of order $n$
    \item If principal part zero, then removeable singularity
\end{itemize}
\noindent\Large{\S 74}\normalsize \ 
Residues at poles
\begin{theorem}
    An isolated singularity $z_0$ of $f$ is a pole of order $m$ iff $f(z)$ can be written in the form \[
        f(z)=\frac{\phi(z)}{(z-z_0)^m}
    \] where $\phi(z)$ is analytic and nonzero. Moreover, \[
    \underset{z=z_0}{\operatorname{R es}}  \, f(z)=\phi(z_0) \ \text{if} \ m=1,\quad \underset{z=z_0}{\operatorname{R es}} \,f(z)= \frac{\phi^{(m-1)}(z_0)}{(m-1)!}\ \text{if} \,\geq 2.
    \] 
\end{theorem}
\noindent\Large{\S 76}\normalsize \ 
Zeroes and poles
\begin{theorem}
    Suppose that two functions $p$ and $q$ are analytic at a point $z_0$ and $p(z_0)\neq 0$ and $q$ has a zero of order $m$ at $z_{0}$. Then the quotient $P(z)/q(z)$ has a pole of order $m$ at $z_0$.
\end{theorem}
\begin{theorem}
    Let two functions $p$ and $q$ be analytic at a point $z_0$. If $p(z_0)\neq 0$, $q(z_0)=0$, and $q'(z_0)\neq 0$, then $z_0$ is a simple pole of the quotient $p(z)/q(z)$ and \[
        \underset{z=z_0}{\operatorname{R es}} \, \frac{p(z)}{q(z)}=\frac{p(z_0)}{q'(z_0)}.
    \] 
\end{theorem}
\noindent\Large{\S 79}\normalsize \ 
Real valued integration
