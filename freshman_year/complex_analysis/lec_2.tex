\section{September 1, 2020}
\subsection{Units and Zero Divisors in the Complex Numbers}
Recall from last time: A complex number can be defined as $(x,y)=x+iy,$ where $x, y \in \R$. Addition is easy: $(x_1+iy_1)+(x_2+iy_2)=(x_1+y_1)+i(y_1+y_2)$. In particular, $(0,0)=0+i\cdot 0=0$. For multiplication, assume $i^2=-1.$ Then 
\begin{align*}
    (x_1+iy_1)(x_2+iy_2)&=(x_1x_2+iy_1x_2+iy_2x_1+i^2y_1y_2)\\
                        &=x_1x_2-y_1y_2+i(y_1x_2+y_2x_1).
\end{align*}
On division: what does it mean to divide complex numbers? We say the multiplicative unit of a complex number (wrt the ring $\C$ ) as the unique $\frac{1}{z}=z^{-1}$ s.t. $z\cdot z^{-1}=z^{-1}\cdot z=(1,0) \in \C$ (the unity of $\C$). Assume $(x,y)(x,y)^{-1}=(1,0)$.  Then do $u$ and $v$ exist such that the system of equations
\begin{equation*}
    \begin{cases}
    xu-yv=1 \\
    xv+yu=0
    \end{cases}
\end{equation*}
holds? Yes, iff the determinant $\left| \begin{smallmatrix} x & -y \\ y & x\end{smallmatrix} \right| =x^2+y^2$ is non zero.

\begin{definition}[Complex Conjugate]
    We have $(x, -y)$ the complex conjugate of the complex number $z=(x,y)$, denoted $\bar{z}$.
\end{definition}

We show that  $\C$ has no zero divisors and is therefore an integral domain. WLOG, assume there exists $z_1,z_2$ such that $z_1 \neq 0, \, z_1z_2=0$: then we have $z_1^{-1}$ exists. So $z_1^{-1}z_1z_2=1z_2=0$, therefore $z_2=0.$ For example: the group $\operatorname{GL}_n(\R)$ is not an integral domain, since we have zero divisors (two matrices that when multipled equal zero).

\subsection{Polar Coordinate Notation}
\begin{definition}[Polar Coordinates]
Think of $(x,y)$ as rectangular coordinates in the $xy$-plane, and consider the \emph{polar coordinate} notation $z=[r,\theta]$, where $r=\sqrt{x^2+y^2} = |z|$ (modulus of $z$), and $\theta = \arctan (\frac{y}{x}).$ So $[r,\theta]=(r \cos \theta, r \sin \theta)$.
\end{definition}

\begin{example}[Multiplication with Polar Coordinates]
   We have \[
       [r_1,\theta_1][r_2,\theta_2]=(r_1\cos\theta_1,r_1\sin\theta_1)(r_2\cos\theta_2,r_2\sin\theta_2).
   \]
   Then 
   \begin{gather*}
       (r_1\cos\theta_1+i r_1\sin\theta_1)(r_2\cos\theta_2+i r_2\sin\theta_2) = \\ 
       r_1r_2\left[ \cos\theta_1\cos\theta_2-\sin\theta_1\sin\theta_2 \right] + ir_1r_2\left[ \sin\theta_1\cos\theta_2 + \sin\theta_2\cos\theta_1 \right]=\\
       r_1r_2\cos(\theta_1+\theta_2)+r_1r_2i\sin(\theta_1+\theta_2)=\\
       [r_1r_2,\theta_1+\theta_2].
   \end{gather*}
       %finish this later
\end{example}
\begin{example}
    Assume that a complex number $z=(x,y)$ is nonzero. Then \[
        \frac{1}{(x,y)}=\frac{1(x,-y)}{(x,y)(x,-y)}=\frac{(x,-y)}{x^2+y^2}.
    \]
\end{example}
\subsection{On the Norm (Modulus) of a Complex Number}
\begin{example}
Some properties of the modulus (norm) $|z|$: 
\begin{enumerate}
    \item $|z_1z_2|=|z_1| |z_2|$,
    \item $\left| \frac{z_1}{z_2} \right| = \left| z_1\cdot \frac{1}{z_2} \right| = \left| z_1\cdot \frac{\bar{z_2}}{|z_2|^2} \right| = |z_1|\frac{\left| z_2 \right| }{\left| z_2 \right| ^2}= \frac{\left| z_1 \right| }{\left| z_2 \right| }$ (clearly $|\bar{z_2}|=|z_2|$),
    \item $|z_1+z_2| \leq |z_1|+|z_2|$ ($\C$ is a metric space, so the triangle inequality holds),
    \item $|z_1+z_2| \geq \big| |z_1|-|z_2| \big| $ (reverse triangle inequality).
\end{enumerate}
\end{example}
We prove the Reverse Triangle Inequality.
\begin{proof}
    We have $|z_1|=|z_1+z_2-z_2| \leq |z_1+z_2|+|z_2|$, so $|z_1+z_2|\geq|z_1|-|z_2|$. A similar argument holds for $z_2$.
\end{proof}
Think of the polar angle as only well defined for multiples of $2\pi$. Define the argument (angle) as $\operatorname{Arg} = -\pi < \theta \leq \pi$ (what??). So $\operatorname{Arg}(1,1)=\frac{\pi}{4}, \, \operatorname{Arg}(-1,0)=\pi$. OTOH, we would have $\operatorname{arg}(1,1) = \frac{\pi}{4}+2\pi n$.
\subsection{Euler's Formula}
\begin{theorem}[Euler's Formula]
    We claim \[
    e^{i\theta}=\cos\theta+i\sin\theta.
    \]
\end{theorem}
\begin{proof}
    Try using Maclaurin series.
\end{proof}
This suggests $e^{i\theta_1}e^{i\theta_2}=e^{i(\theta_1+\theta_2)}$. We proved this when we showed $[r_1,\theta_1][r_2,\theta_2]=[r_1r_2,\theta_1+\theta_2]$.

The reason why Dr. Radin says to "forget about Euler" is because he's trying to make a semi-rigorous (or self-contained) construction of the complex numbers. I think it's fine to rely on intuition from other courses, this isn't Real Analysis (nowhere near as rigorous). If we truly were to construct the field $\C$, we would have to cover polynomial rings and the fields generated by PID's quotient irreducible polynomials, then show that $\C \simeq \R[x]/\langle x^2+1 \rangle $ (and show that this new field is algebraically closed too!). Of course this isn't feasible. So let's just think of this as Euler's Formula, and not some weird definition!

Back to math: using our newfound formula, we can simply say $\operatorname{arg} z = \theta$ such that $z=re^{i\theta}$ for any $z\in \C$. Similarly, $\operatorname{Arg}z$ is just $\theta$ restricted to the interval  $(-\pi,\pi]$.
\begin{example}
    If $z=re^{i\theta}$ nonzero, then what is the polar form of $\frac{1}{z}$? It must be \[
    \frac{1}{r}e^{-i\theta}.
    \]
\end{example}
\begin{example}
 We've seen that $e^{i\theta_1}e^{i\theta_2}=e^{i(\theta_1+\theta_2)}$. Then \[
    e^{i\theta_1}\left( e^{i\theta_2}e^{i\theta_3} \right) =e^{i\theta_1}e^{i(\theta_2+\theta_3)}= e^{i(\theta_1+\theta_2+\theta_3)}.
\]
So $\left( \cos\theta+i\sin\theta \right)^{m}=\cos(m\theta)+i\sin(m\theta)$. This is known as \emph{de Moivre's formula} .   
\end{example}
