\section{September 15, 2020}
I could be studying fundamental groups right now, but instead I'm sitting here verifying limits and derivatives by hand. Why?? OK so Gradescope is a meme. Anything new?

Everything so far has been awfully boring. But now it gets interesting. Finally, I've been waiting for this.

\subsection{Cauchy-Riemann Equations}
Suppose we have a function $f \colon \C \to \C$, write it as $f(z)=u(x,y)=iv(x,y)$. Assume $f'(z_0)$ exists, and is equal to \[
    \lim_{(x,y)\to (x_0,y_0)}\frac{\left( u(x,y)+iv(x,y) \right) -\left( u(x_0,y_0)+iv(x_0,y_0) \right) }{(x+iy)-(x_0+iy_0)}.
\] Then we rewrite this to get \[
f'(z_0)=\lim_{(x,y)\to (x_0,y_0)}\frac{u(x,y)-u(x_0,y_0)+i\left[ v(x,y)-v(x_0,y_0) \right] }{(x-x_0)+i(y-y_0)}.
\] Consider two special ways $(x,y)$ can be ``near'' $(x_0,y_0)$. First, let $x=x_0$ but $y\neq y_0$. Then the quotient becomes 
\begin{gather*}
\frac{u(x_0,y)-u(x_0,y_0)+i[v(x_0,y)-v(x_0,y_0)]}{i(y-y_0)}=\\
\frac{u(x_0,y)-u(x_0,y_0)}{i(y-y_0)}+\frac{v(x_0,y)-v(x_0,y_0)}{y-y_0}.
\end{gather*} Then the limit is equal to \[
\frac{1}{i}\frac{\partial u}{\partial y}(x_0,y_0)+\frac{\partial v}{\partial y}(x_0,y_0).
\] Now let $y=y_0$ but $x\neq x_0$. Then the quotient becomes \[
\frac{u(x,y_0)-u(x_0,y_0)+i\left[ v(x,y_0)-v(x_0,y_0) \right] }{x-x_0}=\frac{u(x,y_0)-u(x_0,y_0)}{x-x_0}+\frac{i\left[ v(x,y_0)-v(x_0,y_0) \right]}{x-x_0} ,
\] so the limit $f'(z_0)$ is equal to \[
\frac{\partial u(x_0,y_0)}{\partial x}+i \frac{\partial v}{\partial x}(x_0,y_0).
\]  Why are we doing this? It's because if the limit exists, it should be the same whichever direction you approach it from, so you can derive some cool equalities.

The two equations must agree, so \[
    \frac{1}{i} \frac{\partial u}{\partial y}(x_0,y_0)+ \frac{\partial v}{\partial y}(x_0,y_0)=\frac{\partial u}{\partial x}(x_0,y_0)+i \frac{\partial v}{\partial x}(x_0,y_0).
\] Examine the real and imaginary parts, so we have 
\begin{equation}
\frac{\partial v}{\partial y}(x_0,y_0)=\frac{\partial u}{\partial x}(x_0,y_0) \quad \text{and} \quad \frac{\partial v}{\partial x}(x_0,y_0)=- \frac{\partial u}{\partial y}(x_0,y_0).
\end{equation}
 These are known as the \emph{Cauchy-Riemann Equations}. Furthermore, \[
f'(z_0)=\frac{\partial u}{\partial x}(x_0,y_0)+i \frac{\partial v}{\partial x}(x_0,y_0)=\frac{\partial v}{\partial y}(x_0,y_0)-i \frac{\partial u}{\partial y}(x_0,y_0).
\] What does this tell us? If your function is differentiable at a point, then we have a way to compute the derivative at that point. The converse does not hold, that is, if the Cauchy-Riemann equations hold this doesn't necessarily guarantee the existence of a derivative at that point.

\begin{example}
    Recall that the function $f(z)=|z|$ is nowhere differentiable. However, consider $g(z)=|z|^2=x^2+y^2=u(x,y)+iv(x,y)$, where $v(x,y)=0$ and $u(x,y)=x^2+y^2$. Let's check to see if this function satisfies the Cauchy-Riemann equations. $\frac{\partial u}{\partial x}=2x,\, \frac{\partial u}{\partial y}=2y, \, \frac{\partial v}{\partial x}=0,\, \frac{\partial v}{\partial y}=0$. Does $\frac{\partial u}{\partial x}=\frac{\partial v}{\partial y}?$ Only if $2x=0 \implies x=0$. Does $\frac{\partial u}{\partial y}=-\frac{\partial v}{\partial x}?$ Only if $2y=0\implies y=0$. So this function only satisfies the Cauchy-Riemann equations at the origin, which implies that the function is nowhere differentiable at any other point ($g'(z_0)$ does not exists if $z_0\neq 0$). Now it does satisfy the CR equations at $0$, but we don't have the existence of the derivative guaranteed.

    Let's check the case if $z_0=0$. \[
        \frac{g(z)-g(0)}{z-0}=\frac{|z|^2-0}{z-0}=\frac{\overline{z}z}{z}=\overline{z}.
    \] Does $\lim_{z\to 0}\overline{z}$ exist? Yes, and it's equal to $0$. So $g'(0)$ exists and is equal to $0$.
\end{example}
Let's talk about the opposite of the CR equations.
\begin{theorem}[Weak Converse of Cauchy-Riemann Equations]
    Let $f=u+iv$ be defined on a neighborhood of $z_0=x_0+iy_0$. Suppose the partial derivatives of $u$ and $v$ exist in that neighborhood, and are continuous at $z_0$. Furthermore, suppose the functions $u$ and $v$ satisfy the CR-equations at $z_0$. Then $f'(z_0)$ exists.
\end{theorem}
\begin{note}
    We claim the hypotheses hold for $|z|^2: u(x,y)=u^2+y^2,\, v(x,y)=0$. Oops, I went to the restroom here. I don't think I missed anything interesting though.
\end{note}
Now the next topic is very important. 
\begin{example}
    Let $f(x,y)=e^{x}(\cos(y)+i\sin(y))=e^{x}\cos(y)+ie^{x}\sin(v)$. Note: $u$ and $v$ are \emph{nice}. Let's compute the CR equations: $\frac{\partial u}{\partial x}=e^{x}\cos(y)=\frac{\partial v}{\partial y}=e^{x}\cos(y)$. We also have $\frac{\partial u}{\partial y}=-e^{x}\sin(y)=-\frac{\partial v}{\partial x}=-e^{x}\sin(y)$. Then $f$ is differentiable everywhere, furthermore, 
    \[
    f'=\frac{\partial u}{\partial x}+i \frac{\partial v}{\partial x}=e^{x}\cos(y)+ie^{x}\sin(y)=e^{x}(\cos(y)+i\sin(y))=f.
\] So $f$ is equal to its derivative everywhere. This is probably the single most important function in the entire course. 
\end{example}
We will eventually denote this function as $\exp(z)$. Notee: if we use Euler's formula $e^{i\theta}=\cos\theta+i\sin \theta$, then \[
    \exp(z)=e^{x}e^{iy}=e^{x+iy}=e^{z}.
\] But we have to make sure we can add the exponents first. Before discussing this further, consider polar coordinates for $z$. For any function $g$, write $f(z)=u(r,\theta)+iv(r,\theta)$. Then after the change of coordinates we have \[
\frac{\partial u}{\partial r}=\frac{\partial u}{\partial x}\frac{\partial x}{\partial r}+\frac{\partial u}{\partial y}\frac{\partial y}{\partial r}
\] and \[
\frac{\partial u}{\partial \theta}=\frac{\partial u}{\partial x}\frac{\partial x}{\partial \theta}+\frac{\partial u}{\partial y}\frac{\partial y}{\partial \theta}.
\] where $x=r\cos\theta$, $y=r\sin \theta$. We have $\frac{\partial x}{\partial r}=\cos\theta,\, \frac{\partial x}{\partial r}=\sin \theta,\, \frac{\partial x}{\partial \theta}=-r\sin \theta,\, \frac{\partial y}{\partial \theta}=r\cos\theta$. Use these to get \[
\frac{\partial u}{\partial r}=\frac{\partial u}{\partial x}\cos\theta+\frac{\partial u}{\partial y}\sin\theta
\] and \[
\frac{\partial u}{\partial \theta}=-\frac{\partial u}{\partial x}r\sin\theta + \frac{\partial u}{\partial y}r\cos\theta.
\] If $f$ is differentiable, $\frac{\partial u}{\partial x}=\frac{\partial v}{\partial y}$, $\frac{\partial u}{\partial y}=-\frac{\partial v}{\partial x}$
Plugging these into the CR equations, we get \[
\frac{\partial u}{\partial r}=\frac{1}{r}\frac{\partial v}{\partial \theta},\, ?? stop moving the page
\] 
Next time: no. We'll do it next time. We have a test in 2 weeks BTW.
