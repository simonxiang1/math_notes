\section{Cauchy's Theorem and Its Applications}
These will follow Stein and Shakarchi \S 2.
\orbreak
Last time we talked about some good stuff: open sets, holomorphic functions, integration along curves. Our first cool theorem relates the three, which is (you guessed it) Cauchy's theorem. It loosely states that if $f$ is holomorphic on an open set $\Omega$ and $\gamma\subseteq \Omega$ is a closed curve whose interior is also contained in $\Gamma$, then \[
    \int_{\gamma}^{} f(z) \, dz=0.
\] Cool stuff happens, including the calculus of residues. Right now we restrict ourselves to easy curves (toy contours) for simplicity, but we'll deal with the general case soon. We'll also get to Cauchy's integral formula, which says that for $f$ holomorphic in an open set containing a circle $C$ and its interior, then for all $z\in C$, \[
f(z)=\frac{1}{2\pi i}\oint_{C}^{} \frac{f(\zeta)}{\zeta-z} \, d\zeta.
\] From here, by differentiating we'll see that as long as the first derivative exists, all of them do! From there, we'll get
\begin{itemize}
    \item The property of ``analytic continuation'' (finally), namely that a holomorphic function is determined by its restriction to any open subset of its domain, which follows from the fact that holomorphic functions are analytic.
    \item Liouville's theorem, which can prove the fundamental theorem of algebra.
    \item Morera's theorem, which gives a simple integral characterization of holomorphic functions, preserved under uniform limits.
\end{itemize}
\subsection{Goursat's theorem}
We know that if $f$ has a primitive in $\Omega$ open, then \[
    \int_{\gamma}^{} f(z) \, dz=0
\] for any closed curve $\gamma \subseteq \Omega$. We also can show that if the relation above holds, then a primitive will exist.
\begin{theorem}[Goursat's theorem]
If $\Omega$ is an open set in $\C$, and $T \subset \Omega$ is a triangle whose interior is also contained in $\Omega$, then \[
    \int_{T}^{} f(z) \, dz=0,
\] given $f$ is holomorphic in $\Omega$.
\end{theorem}
\begin{proof}
    Given a triangle $T^{(0)}$ with a fixed positive orientation, consider its barycentric subdivision yielding the similar triangles $T_1^{(1)},T_2^{(1)},T_3^{(1)},T_4^{(1)}$, defined to be the same orientation as $T^{(0)}$. So 
    \begin{equation}\label{4}
        \int_{T^{(0)}}^{} f(z) \, dz=\int_{T_1^{(1)}}^{} f(z) \, dz + \int_{T_2^{(1)}}^{} f(z) \, dz + \int_{T_3^{(1)}}^{} f(z) \, dz + \int_{T_4^{(1)}}^{} f(z) \, dz.
    \end{equation}
     Then there exists a $j$ such that \[
    \left| \int_{T^{(0)}}^{} f(z) \, dz \right| \leq 4 \left| \int_{T_j ^{(1)}}^{} f(z) \, dz \right| ,
\] because if that weren't the case, it would contradict \cref{4}. Denote such $T_j ^{(1)}$ by $T^{(1)}$: note that $d^{(1)}=\frac{1}{2}d^{(0)}$ and $p^{(1)}=\frac{1}{2}p^{(0)}$, where $d^{(i)}$ and $p^{(i)}$ are the diameters and perimeters of the $T^{(i)}$'th triangle, respectively. Continuing on, we have a sequences of triangle $T^{(0)},T^{(1)},\cdots ,T^{(n)},\cdots $ such that \[
\left| \int_{T^{(0)}}^{} f(z) \, dz \right| \leq 4^n \left| \int_{T^{(n)}}^{} f(z) \, dz \right|,\quad d^{(n)}=2^{-n}d^{0}, \ p^{(n)}=2^{-n}p^{(0)}.
\] Denote the \emph{solid} closed triangle with boundary by $\mathcal{T} ^{(n)}$, and note that our construction yields a sequence of nested compact sets \[
\mathcal{T} ^{(0)}\supset \mathcal{T} ^{(1)}\supset \cdots \supset \mathcal{T} ^{(n)}\supset \cdots 
\] whose diameter approaches zero. By the NIP, there exists a unique $z_0\in \bigcup_{i=1}\mathcal{T} ^{(i)}$. Now $f$ is holomorphic at $z_0$, so we write $f(z)=f(z_0)+f'(z_0)(z-z_0)+\psi(z)(z-z_0)$, where $\psi(z)\to 0$ as $z\to z_0$. Since $f(z_0)$ and the linear function $f'(z_0)(z-z_0)$ have primitives, then we integrate and get \[
\int_{T^{(n)}}^{} f(z) \, dz = \int_{T^{(n)}}^{} \psi(z)(z-z_0) \, dz.
\] Now $z_0\in \overline{\mathcal{T} ^{(n)}}$ and $z\in \partial \mathcal{T} ^{(n)}\implies |z-z_0|\leq d^{(n)}$. Recall that $\left| \int_{\gamma}^{} f(z) \, dz \right| \leq \sup_{z\in \gamma}|f(z)|\cdot \operatorname{length}(\gamma)$. So we have the estimate \[
\left| \int_{T^{(n)}}^{} f(z) \, dz \right| \leq \varepsilon _n d^{(n)}p^{(n)},
\] where $\varepsilon _n =\sup_{z\in T^{(n)}}|\psi(z)|\to 0$ as $n\to \infty$. Therefore we have \[
\left| \int_{T^{(n)}}^{} f(z) \, dz \right| \leq\varepsilon _n 4^{-n}d^{(0)}p^{(0)}
\implies \left| \int_{T^{(0)}}^{} f(z) \, dz \right| \leq 4^n  \left| \int_{T^{(n)}}^{} f(z) \, dz \right| \leq \varepsilon _n d^{(0)}p^{(0)}.
\] Letting $n\to \infty$ concludes the proof, since $\varepsilon _n \to 0$.
\end{proof}
Attempt to sum up the proof in easy to read words: take a triangle and subdivide it naturally into four more triangles. Choose one of them such that it follows a nice inequality, and continue subdiving until infinity. Since $\C$ is complete, we can find a point in all their closures: rewriting the formula for a derivative, we can apply the inequalities and get our result.
\begin{cor}
    If $f$ is holomorphic in an open set $\Omega$ that contains a rectangle $R$ and its interior, then \[
        \int_{R}^{} f(z) \, dz.
    \] 
\end{cor}
\begin{proof}
    Immediate by constructing the $n$-gon with $n-2$ simplices.
\end{proof}

\subsection{Local existence of primitives and Cauchy's theorem in a disk}
Ah, so I was complaining about the jank-connectedness definition by polygonal paths earlier in my notes. I see why they do it that way now.
\begin{theorem}
    A holomorphic function in an open disc has a primitive in that disk.
\end{theorem}
\begin{proof}
    WLOG, let $D$ be a disk centered at the origin, $z\in D$ be a point. Consider a curve from $0$ to $z$ made by joining straight line segments, starting by joining $0$ to $\operatorname{Re}z$ then from $\operatorname{Re}z$ to $z$, with the orientation you'd expect. We'll denote this polygonal path by $\gamma_z$. Define the unique function $F(z)=\int_{\gamma_z}^{} f(w) \, dw$: we claim $F$ is holomorphic in $D$ and $F'(z)=f(z)$. Let $z,h\in \C$ and choose $h>0$ such that $z+h\in D$. Consider \[
        F(z+h)-F(z)=\int_{\gamma_{z+h}}^{} f(w) \, dw-\int_{\gamma_z}^{} f(w) \, dw.
    \] 
\end{proof}





