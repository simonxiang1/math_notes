\section{October 13, 2020}
\subsection{More on integration}
Soon we'll get to Cauchy's theorem, the most important theorem in this course (integral on a closed curve is equal to zero). Thank goodness I read the other book, it covered in two pages what we cover in two weeks (sans calculations).
\begin{theorem}\label{int}
   Suppose $f$ is continuous on a domain $D$. Then TFAE:
   \begin{enumerate}
       \item $f$ has a primitive $F$ on $D$.
       \item $\int_{\Gamma }^{} f(z) \, dz$ along paths $\Gamma \subseteq D$ only depend on the endpoints of $\Gamma $.
       \item $\oint_{\Gamma }^{0} f(z) \, dz=0$ for all $\Gamma$ a closed path.
   \end{enumerate}
\end{theorem}
\begin{proof}
    ($1\implies 2) $ Assume $f=\frac{df}{dz}$ in $D$. Then
    \begin{align*}
        \int_{\Gamma }^{} f \, dz&=\int_{a}^{b} f[(w(t)]w'(t) \, dt\\
                                 &=\int_{a}^{b} \frac{dF}{dz}[w(t)]w'(t) \, dt\\
                                 &=\int_{a}^{b} \frac{d}{dt}F[w(t)] \, dt\\
                                 &=F[w(b)]-F[w(a)]
    \end{align*} by the FTC, finishing the first implication.

    ($2\implies 3)$ Assume $\Gamma $ is closed loop, choose a basepoint $\gamma$, then $\oint_{\Gamma }f(z)\,dz=F[w(\gamma)]-F[w(\gamma)]=0$. Wait, is my proof wrong? Dr.\ Radin is splitting the curve in two, then noting that they have opposite orientation, implying that the left and right derivatives will cancel.

    ($3\implies 1$) Assume that $\oint_{\Gamma }f(z)\,dz=0$ for $\Gamma $ a closed path. Define $F(w)=\int_{\Gamma }^{} f(z) \, dz$ where $w$ is an endpoint of $\Gamma $. From here, it's not hard to show that $\frac{dF}{dz}=f$, finising the proof.
\end{proof}
RIP for Dr.\ Radin's internet, we lost a good one.

\begin{example}
    Let $\Gamma \colon w(t)=e^{it}$, where $-\frac{\pi}{2}\leq t \leq \frac{\pi}{2}$. Let $f(z)=z^4$. Then \[
        I=\int_{\Gamma }^{} f(z) \, dz=\int_{-\frac{\pi}{2}}^{\frac{\pi}{2}}e^{4it}ie^{it}  \, dt=i \int_{-\frac{\pi}{2}}^{\frac{\pi}{2}} e^{5it} \, dt= \left.i \frac{e^{5it}}{5i}  \right|_{-\frac{\pi}{2}}^{\frac{\pi}{2}}=\frac{1}{5[2i]}=\frac{2i}{5}.
    \] Alternatively, use the theorem above.
\end{example}
\begin{example}
    Let $D$ be the annulus and $\Gamma $ be a path in the annulus, then $\oint_{\Gamma }f(z)dz=0$.
\end{example}
\begin{example}
    Let $w(t)=e^{it}$ for $0\leq t \leq 2\pi$, set $f(z)=\frac{1}{z}$. Now $f$ has an antiderivative, say $F=\operatorname{Log}(z)$, but $F$ is not defined on any $D$ containing $\Gamma $, so we can't directly apply the theorem. Trick: define $I_{\varepsilon }=\int_{\Gamma _{\varepsilon }}^{} \frac{1}{z} \, dz$, where $\Gamma _{\varepsilon }$ is the open interval $I\setminus B(z_0,\varepsilon )$ for some basepoint $z_0$, $\varepsilon >0$. Then this integral is equal to $\operatorname{Log}(-1+\varepsilon i)-\operatorname{Log}(-1-\varepsilon i)=\ln |-1+\varepsilon i|+i\operatorname{Arg}(-1+i\varepsilon ) - \ln|-1-\varepsilon i|-i \operatorname{Arg}(-1-i\varepsilon )$. After some estimates, the $\ln$'s reduce to approximately zero, and we get $\sim \pi i -(\sim -\pi i)\cong 2\pi i $.
\end{example}
\begin{definition}[Simple curve]
    A parametrized path is simple if $w(t)\neq w(t')$ for $t\neq t'$. If $\Gamma $ is closed, we can make an exception for the endpoints $w(a)=w(b)$.
\end{definition}
\subsection{Cauchy's Theorem}
Here we are.
\begin{theorem}[Cauchy's Theorem]\label{ct}
   If $f$ is analytic on $\Gamma ^{\circ }$ for $\Gamma $ a simple closed curve, then \[
       \oint_{\Gamma }^{} f(z) \, dz=0. 
   \] 
\end{theorem}
Basically, when we talk about interiors we mean the connected open bounded region made from a simple closed path, which exists by the Jordan curve theorem. This is really powerful, since we don't require the existence of an antiderivative, we just need $f$ to be analytic. The earlier theorem is quite elementary, but Cauchy's theorem is much more advanced. Once we prove this, it will follow that analytic functions $f$ have antiderivatives, and we can go on to say that $f$ has second, third, and so on derivatives. This is one of the first ``cool'' theorems of complex analysis in that it demonstrates how much nicer analytic functions are compared to real valued functions. No proof, unfortunately.

We want to get some consequences from this theorem.
\begin{definition}[Simply connected]
    A domain $D$ is simply connected if every loop $\gamma$ is nullhomotopic, that is, the fundamental group $\pi_1(D)$ is trivial. How it's formulated in complex analysis: the interior of every loop is contained in the domain.
\end{definition}
\begin{cor}
    If $f$ is analytic in a simply-connected domain $D$, then \[
        \oint_{\Gamma }^{} f(z) \, dz=0
    \] for any closed $\Gamma \subseteq D$.
\end{cor}
\begin{proof}
    Since the interio(unfinished)
\end{proof}
\begin{cor}
    If $f$ is analytic in a simply-connected domain, then $f$ has an antiderivative there.
\end{cor}
