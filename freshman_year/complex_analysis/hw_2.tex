\section{August 28, 2020: Homework 2}
\textbf{Section 3:} Problems 2,4,7.

\subsection{Question 2}
\begin{prob}
    Show that \[
    \frac{1}{1/z}=z,
    \]
    where $z \neq 0.$
\end{prob}
\begin{solution}
    We know that $z^{-1}=1/z$ exists and is equal to 
    \[
        \left( \frac{x}{x^2+y^2},\frac{-y}{x^2+y^2} \right) 
    \]
    since $z$ is non-zero. Continuing on, we have $(1/z)^{-1}=\frac{1}{1/z}$ exists ($z \neq 0)$, and with a simple application of the previous formula is equal to
    \[
        \left( \left( \frac{\left( \frac{x}{x^2+y^2} \right) }{ \left( \frac{x}{x^2+y^2} \right)^2 + \left( \frac{-y}{x^2+y^2} \right)^2 } \right), \left( \frac{-\left( \frac{-y}{x^2+y^2} \right) }{\left( \frac{x}{x^2+y^2} \right)^2 + \left( \frac{-y}{x^2+y^2} \right) ^2} \right) \right).
    \]  
   This may look intimidating, but we can easily reduce this to 
   \[
       \left( \frac{\frac{x}{x^2+y^2}}{\left( \frac{x^2+(-y)^2}{\left( x^2+y^2 \right) ^2} \right) }, \frac{\frac{-(-y)}{x^2+y^2}}{\left( \frac{x^2+(-y)^2}{\left( x^2+y^2 \right) ^2} \right) } \right),
   \]
   which once again simplifies to
    \[
        \left( \frac{\frac{x}{x^2+y^2}}{\frac{x^2+y^2}{\left( x^2+y^2 \right) ^2}}, \frac{\frac{y}{x^2+y^2}}{\frac{x^2+y^2}{\left( x^2+y^2 \right) ^2}}\right) = \left( \frac{\frac{x}{x^2+y^2}}{\frac{1}{x^2+y^2}}, \frac{\frac{y}{x^2+y^2}}{\frac{1}{x^2+y^2}} \right) = (x,y) = z.
    \]
\end{solution}
    
\subsection{Question 4}
\begin{prob}
    Prove that if $z_1z_2z_3=0,$ then at least one of the three factors is equal to zero.
\end{prob}
\begin{proof}
    Let $z_1z_2z_3=(z_1z_2)z_3=0.$ Then either $(z_1z_2)$ or $z_3$ is zero (proof from the book): WLOG, assume that  $(z_1z_2)z_3=0$ and $(z_1z_2) \neq 0$. Since the complex numbers form a field, we have $(z_1z_2) \in \C$ so $(z_1z_2)^{-1} \in \C$, and $z\cdot0 =0$ for all $z \in \C$. So
        \begin{align*}
            z_3&=z_3\cdot1\\
               &=z_3\left( (z_1z_2)(z_1z_2)^{-1} \right) \\
               &= \left( (z_1z_2)^{-1}(z_1z_2)z_3 \right) \\
               &= (z_1z_2)^{-1}\left( (z_1z_2)z_3 \right) \\
               &= (z_1z_2)^{-1}\cdot0\\
               &=0.
        \end{align*}
        If $z_3$ is zero, then we are done. If $(z_1z_2)$ is zero, then we apply the same logic again to conclude that either $z_1$ or $z_2$ is zero. So either way, one of the factors $z_1,z_2,$ or $z_3$ must be zero, and we are done (note that you can prove that this holds for any number of factors by induction).
\end{proof}

\subsection{Question 7}
\begin{prob}
    Use the associative law for addition and the distributive law to show that 
    \[
        z(z_1+z_2+z_3)=zz_1+zz_2+zz_3.
    \]
\end{prob}
\begin{proof}
    We have
    \begin{align*}
        z(z_1+z_2+z_3)&=z((z_1+z_2)+z_3) \ \small{\text{by Associativity of Addition}} \\
                &=z(z_1+z_2)+zz_3 \ \small{\text{by the Distributive Law}} \\
                &=(zz_1+zz_2)+zz_3 \ \small{\text{by the Distributive Law}} \\
                &=zz_1+zz_2+zz_3 \ \small{\text{ by Associativity of Addition, }} \\
    \end{align*}
    completing the proof.
\end{proof}

