\section{Residues, Meromorphic Functions, and Poles}
These will follow Stein and Shakarchi \S 3.
\orbreak
In order of increasing severity, we have the following types of singularities:
\begin{itemize}
    \item Removable singularities
    \item Poles
    \item Essential singularities
\end{itemize}
The first type are harmless since you can extend a function to be holomorphic at its removable singularities. At essential singularities, functions tend to oscillate and grow faster than powers, which makes them hard to understand. Poles are deeply connected with the calculus of residues, which we will see soon. We know for $f$ holomorphic in an open set containing a curve $\Gamma $ and its interior, we have $\int_{\Gamma }^{} f(z) \, dz=0$. What if $f$ has a pole in the middle? Recall that for $f(z)=\frac{1}{z}$ and $C$ a circle centered at $0$, then \[
\int_{C}^{} \frac{dz}{z}=2\pi i.
\] This turns out to be the key ingredient.
