\section{February 2, 2021}
\begin{definition}[]
    A metric space $X$ is \textbf{homogeneous} if all points look the same, that is, there exists an isometry that takes $p$ to $q$ for all $p,q\in X$.
\end{definition}
\begin{example}
    Some examples of homogeneous spaces:
    \begin{enumerate}
        \item Sphere
        \item Plane
        \item Torus
    \end{enumerate}
\end{example}
Let a group $G$ act transitively on a space $X$, that is, for all $x,y \in X$ there exists a $g$ such that $gx=y$. Define $H_x= \{h\in G \mid h\cdot x=x\} $. 
\begin{enumerate}
    \item Show that $H_x$ and $H_y$ are conjugate.
    \item $X\simeq G /H$.
\end{enumerate}
If $G$ is the set of isometries of $X$, then $H_x= \{\text{isometries that sent} \ x \ \text{to itself} \ \} =$ isotropy group at $x$ (which we proved is the same for all $x$), sometimes called the ``little group''.
