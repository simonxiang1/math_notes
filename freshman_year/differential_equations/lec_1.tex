\section{First Order Linear Differential Equations (8/27/20)}
\begin{definition}[Order]
    We have the \emph{order} of a differential equation the highest derivative of a function $y$ that appears in the equation. For example, the order of the differential equation \[
        \frac{dy}{dt}=3y^2\sin(t+y)
    \]
    is $1$, while the order of \[
    \frac{d^3y}{dt^3}=e^{-y}+t+\frac{d^2y}{dt^2}
    \]
    is $3$. We would call the first example a \emph{first-order differential equation} and the second a \emph{third order differential equation}.
\end{definition}
\begin{definition}[Solution]
    The \emph{solution} of a differential equation is a continuous function $y(t)$ that together with its derivatives satisfies the given relationship.
\end{definition}
\begin{example}
    The function \[
        y(t)=2\sin t - \frac{1}{3}\cos 2t
    \]
    is the solution of the second-order differential equation \[
        \frac{d^2y}{dt^2}+y= \cos 2t
    \]
    since 
    \begin{align*}
        \frac{d^2}{dt^2}\left( 2\sin t - \frac{1}{3}\cos 2t \right) + \left( 2\sin t - \frac{1}{3}\cos 2t \right) \\
        = \left( -2\sin t + \frac{4}{3} \cos 2t \right) + 2 \sin t - \frac{1}{3}\cos 2t = \cos 2t.
    \end{align*}
\end{example}

Goal: Given a differential equation of the form \[
    \frac{dy}{dt}=f(t,y)
\]
and the function $f(t,y)$, find all functions $y(t)$ that satisfy the equation above.

What we have: As of now, all we can solve is a differential equation of the form \[
    \frac{dy}{dt}=g(t)
\]
given $g(t)$ is integrable. Very sad!
\begin{definition}[Linear ODE]
    The general first-order linear differential equation is of the form \[
        \frac{dy}{dt}+a(t)y=b(t),
    \]
    where $a(t)$ and $b(t)$ are continuous (assumed to be functions of time).
\end{definition}
\begin{definition}[Homogeneous Linear ODE]
    The equation \[
        \frac{dy}{dt}+a(t)y=0
    \]
    is called the \emph{homogeneous} first-order linear differential equation, and the previous definition is called the \emph{nonhomogeneous} first-order linear differential equation for $b(t)$ not necessarily zero.
\end{definition}
\begin{example}
    Let us solve the homogeneous first-order linear differential equation. Rewrite it in the form \[
        \frac{\frac{dy}{dt}}{y}=-a(t).
    \]
    Second, note that \[
        \frac{\frac{dy}{dt}}{y}\equiv \frac{d}{dt}\ln |y(t)|.
    \] Then we can write the differential equation in the form \[
    \frac{d}{dt}\ln |y(t)| = -a(t),
    \] so we have \[
    \ln|y(t)|= - \int a(t) \, dt + c_1.
    \] Continuing on, \[
    |y(t)|=\exp\left( -\int a(t) \, dt + c_1 \right) = c \exp \left( - \int a(t) \, dt \right) 
    \]
    or \[
        \left| \, y(t)\exp\left( \int a(t) \, dt \right)  \right| = c.
    \]
    Now $y(t)\exp \left( \int a(t) \, dt \right) $ is continuous and we know its absolute value is constant which implies that the function itself is constant (which follows from the IVT, assuming $g(t_1)=c$ and $g(t_2)=-c$ for $g$ a function, $c$ a constant. So we have $y(t)\exp\left( \int a(t) \, dt \right) = c$, or 
    \begin{equation}\label{eq 1}
        y(t) = c \exp \left( - \int a(t) \, dt \right) .
    \end{equation}
    Equation \eqref{eq 1} is the \emph{general solution} of the homogeneous equation. Note that there exist infinitely many solutions since for all $c$ we have a distinct $y(t)$.
\end{example}
\begin{example}
    To solve the Linear ODE \[
    \frac{dy}{dt}+2ty=0,
    \]
    simply apply Equation \eqref{eq 1} to yield \[
        y(t)=c\exp\left( -\int 2t \, dt \right) = c\exp\left(-t^2 \right) .
    \] (This is taking too long! I'll type notes with less rigor next time).
\end{example}
Usually scientists are not interested in the general solution given by Equation \eqref{eq 1}, rather we look for solutions to a specific $y(t)$ which at some time $t_0$ has the value $y_0$, or we want to determine a $y(t)$ such that \[
    \frac{dy}{dt}+a(t)y=0, \quad y(t_0)=y_0.
\]
Please accept the derivation that a general solution to this type of problem is 
\begin{equation}\label{eq_2}
    y(t)=y_0\exp\left( -\int_{t_0}^{t} a(s) \, ds \right)  
\end{equation}
without proof (there is nothing of interest about the derivation process).
\begin{example}
    To solve \[
        \frac{dy}{dt}+(\sin t)y=0, \quad y(0)=\frac{3}{2},
    \]
    let  $a(t)=\sin t$, $t_0=0$, $y_0=\frac{3}{2}$. Then \[
        y(t)=\frac{3}{2}\exp\left( -\int_{0}^{t} \sin s \, ds \right)  =\frac{3}{2} \exp \left( \cos t - 1 \right).
    \]
\end{example}
\begin{example}
    To solve the initial value problem \[
        \frac{dy}{dt}+ \exp(t^2)y=0, \quad y(1)=2,
    \]
    simply PLUG IT IN (reee) to get \[
        y(t)=2\exp\left( -\int_{1}^{t} e^{s^2} \, ds \right).
    \]
    Recall that this is the \emph{Gaussian Integral} and can be solved by a change to double integration by polar coordinates (yielding $\int_{-\infty}^{\infty} e^{-x^2} \, dx = \sqrt{\pi}$), but in general has no closed form solution.
\end{example}



    
