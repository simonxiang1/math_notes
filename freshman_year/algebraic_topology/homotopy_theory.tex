\section{Homotopy theory} 
Here comes a long block of Hatcher exposition, read if interested, skip if not.
\orbreak
We have met the first homotopy group already, the fundamental group $\pi_1(X)$. The higher dimensional analogues $\pi_n (X)$ are the \emph{homotopy groups}, which have some similarities to the homology groups: $\pi_n (X)$ is abelian for $n\geq 2$, and there are relative homotopy groups fitting into a LES similar to homology. However, neither Seifert-van Kampen's nor excision holds, making the homotopy groups much harder to compute. 

However, these groups are still important: one reason is \emph{Whitehead's theorem}, which states that a map between CW complexes inducing isomorphisms on the homotopy groups is a homotopy equivalence. However the stronger statement that if two complexes have isomorphic homotopy groups then they're homotopy equivalent is false usually, aside from the case where we only have one nontrivial homotopy group— these spaces are called \emph{Eilenberg-MacLane spaces}.

Another more direct connection between homology and homotopy is the \emph{Hurewicz theorem}, which says that the first nonzero homotopy group $\pi_n (X)$ of a simply-connected space $X$ is isomorphic to the first nonzero homology group $\widetilde H_n (X)$. Though excision doesn't always hold, in some important special cases it does for a range of dimensions. This leads to the idea of \emph{stable homotopy groups}, the beginning of stable homotopy theory. If you figure out how to compute the stable homotopy groups of spheres, you can pick up your Fields medal at the door.

We'll also talk a little about fiber bundles which somewhat generalize the idea of covering spaces for higher homotopy groups, purely to lead toward fibrations. These allow us to describe how the homotopy type of a CW complex is inductively built up from its homotopy groups by forming `twisted products' of Eilenberg-MacLane spaces, which is the notion of a \emph{Postnikov tower}.
\orbreak

\subsection{Whitehead's theorem and relative homotopy groups}
Let $I^n $ be the $n$-cube, and the boundary $\partial I^n $ be the subspace of points with at least one coordinate equal to $0$ or $1$. 
\begin{definition}[Higher homotopy groups]
    For a pointed space $X,x_0$, define the $\mathbf n$\textbf{-th homotopy group} $\pi_n (X,x_0)$ as the set of homotopy classes $f \colon (I^n ,\partial I^n ) \to (X,x_0)$ where homotopies $f_t$ are required to satisfy $f_t(\partial I^n )=x_0$ for all $t$. If $f$ and $g$ are two maps from the $n$-cube into $(X,x_0)$, then we define the composition as \[
        (f+g)(s_1,s_2,\cdots ,s_n )=
        \begin{cases}
            f(2s_1,s_2,\cdots ,s_n ),\quad & s_1\in [0,\sfrac{1}{2}]\\
            g(2s_1-1,s_2,\cdots ,s_n )&s_1\in [\sfrac{1}{2}, 1].
        \end{cases}
    \] 
    If we use $[f]$ to denote the homotopy classes of $f$ (rel $\partial $), then $[f]+[g]=[f+g]$. We have $(f+g)\in \pi_n (X,x_0)$ since $\partial I^n \to 0$, and thus we have given $\pi_n (X,x_0)$ a group structure. Visualizing composition in terms of spheres, we crush the equatorial $S^{n-1}$ to a point yielding a wedge of two $S^n $'s, and from here the picture is the same as with cubes. Associativity is most naturally proven by the cubical picture.
\end{definition}
\begin{theorem}
    $\pi_{n\geq 2} $ is commutative.
\end{theorem}
\begin{proof}
    Consider $f+g$ as the composition of two maps of spheres, with the equator glued to the basepoint. We will show this is the same as $g+f$. The homotopy takes place in the domain $S^n $ (slogan ``work in the domain''), which is just a rotation of the sphere until the equator returns to itself, exchanging the top and bottom hemispheres. This defines a homotopy $S^n \times I\to S^n $, so following the rotation, we end up with $g+f$. The simplest way to state this for higher dimensions is by suspending the spheres, but the key point is that this \emph{doesn't} work for $n=1$, since you can't homotope a circle with a basepoint to exchange the top and bottom.
\end{proof}
This extends to $\pi_0$ by letting $I^0$ be a point and $\partial I^0$ be empty, so $\pi_0(X,x_0)$ is just the set of path-components of $X$. It turns out that $\pi_1$ has a lot of complications since every type of group is possible. The complications for \textbf{higher homotopy groups} (for $n\geq 2$) are totally different, in particular, $\pi_{n\geq 2}(X,x_0)$ is always abelian. Although we formalize things with cubes, most visualizations use balls and spheres, with the correspondence $(I^n ,\partial I^n )\cong (D^n ,S^{n-1})$.
\begin{example}
    The basic example is that $\pi_1(X,x_0)$ is the set of maps from the interval into $(X,x_0)$ up to homotopy rel $\partial I$, which is just the fundamental group. For $\pi_2(X,x_0)$, this is defined as the set of maps from the square to $(X,x_0)$, where the edges of the square $\mapsto x_0$. This can be visualizes as a small square near $x_0$ with a $2$-cell behind it, the reason why it's drawn offset is to distinguish between squares and spheres (more precisely, to indicate that the domain is $I^2$).
\end{example}
Often we think of $\pi_n $ as the set of homotopy classes of maps $(S^n ,\text{point} )\to (X,x_0)$ or classes of maps $(D^n ,\partial D^n )\to (X,x_0)$. These are the same sets, but the formal definition by cubes is often better.
\begin{namedthm}{Whitehead's Theorem}
    If $f\colon X \to Y$ for $X,Y$ connected cell complexes induces isomorphisms on each $\pi_n $, then $f$ is a homotopy equivalence.
\end{namedthm}
\begin{note}
    This does \emph{not} say that if $\pi_i (X)\simeq \pi_i (Y)$ for all $i$, then $X$ and $Y$ are homotopy equivalent! This might happen in a way such that there is no specific function inducing isomorphisms on the homotopy groups. However Whitehead's theorem is still good, it says that any topological map that is an algebraic isomorphism is also a ``topological isomorphism'' (homotopy equivalence).
\end{note}
If $f\colon X \to Y$ sends $x_0\in Y$ to $y_0\in Y$, then $f_* \colon \pi_n (X,x_0) \to \pi_n (Y,y_0)$ is defined as follows: for all $\alpha \colon (I^n ,\partial I^n )\to (X,x_0) $, $f_*(x)$ is the composition $(I^n ,\partial I^n )\overset{\alpha }{\longrightarrow} (X,x_0)\overset{f}{\longrightarrow} (Y,y_0)$. This formulates things as the level of maps of spaces. We must check that this preserves the equivalence relation of homotopy rel $\partial I$, which isn't hard. This is exactly the same as for $\pi_1$. The omission of basepoint is also similar as for $\pi_1$, the point being that if we have a space $X$ with a path $\gamma$ from $x_0\to x_1$, then we have an induced isomorphism $\pi_n (X,x_0)\simeq \pi_n (X,x_1)$. In particular, the isomorphism type of $\pi_n $ is independent of basepoint, so whether $f\colon (X,x_0) \to (Y,y_0)$ is an isomorphism on $pi_n $ is independent of choices of $x_0,y_0$, assuming that $X,Y$ are path connected. How do we define this? You can view it as stretching out the $n$-sphere from one point to another via the map $\gamma$.
\begin{note}
    The isomorphism between $\pi_n (X,x_0)$ and $\pi_n (X,x_1)$ depends on the choice of $\gamma$, just like $\pi_1.$ For example, take a cube minus a vertical pillar in the middle. We can take $\gamma$ to loop around the pillar from the left or from the right to connect to points on opposite ends. These two elements of $\pi_2(X,x_0)$ are different, since moving one loop to the other requires moving the basepoint which isn't allowed.
\end{note}
\begin{definition}[Relative homotopy groups]
    We will define for $x_0\in A\subseteq X$, the \textbf{relative homotopy groups} $\pi_n (X,A,x_0)$ (note $\pi_1$ is not actually a group, just a set with a distinguished point) as the homotopy classes of maps $I^n \to X$ such that 
    \begin{enumerate}[label=(\roman*)]
        \item  $ \overset{I^{n-1}\times \{0\} \subseteq I^{n-1}\times I}{\left( I^{n-1}\subseteq I^n \right) } \to A$ ,
        \item all other facets of $I^n $ map to $x_0$.
    \end{enumerate}
    If we replace (i) by the stronger condition $I^{n-1}\to x_0$, then we get $\pi_n (X,x_0)$. So it's like $\pi_n (X,x_0)$ except that one facet of $I^n $ can be moved around inside $A$.
\end{definition}
    These satisfy a long exact sequence as such:\[
        \cdots \to \pi_{n+1(X,A)}\longrightarrow \pi_n (A)\overset{i_*}{\longrightarrow}  \pi_n (X) \longrightarrow \pi_{n}(X,A) \longrightarrow \pi_{n-1}(A)\overset{i_*}{\longrightarrow} \pi_{n-1}(X)\cdots \to \pi_1(X,A).
    \] Since the $i_*$ are isomorphisms by Whitehead's thm (where $i_*$ is induced by the inclusion $A\hookrightarrow X$), then $\pi_n (X,A)=0$ for all $n$. We also have a geometric fact (the compression criterion): an element of $\pi_n (X,A)$ is 0, say as a map $(D^n ,S^{n-1}, s_0)\overset{f}{\to } (X,A,x_0)\iff f$ is homotopic rel $S^{n-1}$ to a map $D^n \to A$. The image is a cube $X=I^3$, and $A$ the bottom face. Then a ``bulge'' (element of $\pi_2(X,A)$) can be flattened to the disk in $A$, so it's zero in $\pi_2(X,A)$. Repeating this process is like building a deformation retraction of $X$ to $A$.

        The relative homotopy groups are groups for $n\geq 2$, it's the same construction as for ordinary $\pi_n $ except that one coordinate on $I^n $ is reserveed for the definition of relateive $\pi_n $ (the last coordinate). What goes wrong for $\pi_1$? One facet being able to move around in $\pi_1$ means you get paths you can't combine.
        \begin{theorem}[Long exact sequence]
            If $x_0\in A\subseteq X$, then the following sequence is exact (suppressing basepoints): \[
                \pi_{n+1}(X,A)\overset{\partial }{\longrightarrow}  \pi_n (A) \overset{i_*}{\longrightarrow} \pi_n (X)\overset{j}{\longrightarrow}  \pi_n (X,A) \overset{\partial }{\longrightarrow}  \pi_{n-1}(A) \overset{i_*}{\longrightarrow} \pi_{n-1}(X)\to \cdots \to \pi_1(X,A)
            \] where the $i_*$ are induced by inclusion, and $j\colon \pi_n (X) \to \pi_n (X,A)$ is the obvious map. That is, a map $(I^n ,\partial I^n )\to (X,x_0)$ where $(I^n ,I^{n-1})\to (X,A)$, and $(I^n , \, \text{rest of} \ \partial I^n )\to x_0$. The boundary maps $\partial \colon \pi_n (X,A) \to \pi_{n-1}(A)$ are also ``obvious'': if $f\colon I^n  \to X$ represents an element of $\pi_n (X,A)$, then $\left. f \right| _{I^{n-1}}\to A$ sends $\partial I^{n-1}\to x_0$, ie it represents an element of $\pi_{n-1}(A)$, which is precisely $\partial f$.
\end{theorem}
\begin{proof}
    Is it time for a diagram chase? We'll break this up into part (B), (C), (D), there is no part (A). (B) will be about $\beta \in \pi_n (A)$, (C) about $\gamma \in \pi_n (X)$, and (D) about $\delta \in \pi_n (X,A)$.
\[
                \pi_{n+1}(X,A)\overset{\partial }{\longrightarrow}  \underset{\beta \in }{\pi_n (A)}  \overset{i_*}{\longrightarrow} \underset{\gamma\in }{\pi_n (X)} \overset{j}{\longrightarrow}  \underset{\delta\in }{\pi_n (X,A)}  \overset{\partial }{\longrightarrow}  \pi_{n-1}(A) \overset{i_*}{\longrightarrow} \pi_{n-1}(X)\to \cdots \to \pi_1(X,A)
            \]
    \noindent\textbf{Part B:} Suppose $\beta \in \pi_n (A)$. We must show that
    \begin{enumerate}[label=(\roman*)]
        \item $ji_*\beta =0$. Applying $j$ means one facet of $I^n $ is allowed to move within $A$. We must show that if it allows our facet of $I^n $ to move within $A$, then we can shrink $\beta $ to a constant map. Shrink $I^n $ onto $\partial I^n \setminus I^{n-1}$ (Hatcher calls this space $J^{n-1}$) rel $\partial I^n \setminus I^{n-1}$. So $\beta_t $ sends the endpoints of the bottom edge of the cube to itself, $t=0\implies \beta_0=\beta _t$, $t$ small $\implies \beta _t\approx$ the pinched cube, when $t$ is large then $\beta _t\approx$ a shrunken pinched cube, and $t=1\implies \beta_1 $ is constant to $x_0$. Slogan: $\beta $ itself is a nullhomotopy of $\partial \beta $.
        \item If $\beta \in \pi_n (A)$ and $i_*\beta =0$, then $\beta =\partial \alpha $ for $\alpha \in \pi_{n+1}(X,A)$. In this case we have to find an $\alpha $ st $\partial \alpha =\beta $, the image of $\alpha $ leaving $A$ describes a map $I^{n+1}\to  I^{n+1}/J^n =B^n \to X$ restricting to $\beta $ on $I^n $. The fact that you can draw something that bounds $\beta $ is precisely finding an element that bounds $\beta $.
    \end{enumerate}
\end{proof}
