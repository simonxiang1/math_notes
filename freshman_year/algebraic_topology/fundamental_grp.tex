\section{The Fundamental Group}
OK guys, let's decompose big spaces into smaller ones and compute their fundamental groups. These notes follow Hatcher \S 1.2, Lee \S 10, and May \S 1,\S 2.7.
\subsection{Defining the fundamental group}
Let $X$ be a space, we say two paths $f,g \colon I \to X$ from $x$ to $y$ are \textbf{equivalent up to homotopy} if there exists a homotopy $h\colon I\times I \to X     $ such that 
\[
    h(s,0)=f(s),\ h(s,1)=g(s),\ h(0,t)=x,\ h(1,t)=y
\] for all $s,t\in I$. In other words, starting at $0$ for the second interval ensures you begin at $f$, and evaluating at the end takes you to $g$. Since the starting position of $f$ and $g$ are the same, for the first interval starting at $0$ must give $x$, and similarly for $y$. Denote the homotopy equivalence class of $f$ as $[f]$: we say that $f$ is a \textbf{loop} if $f(0)=f(1)$.
\begin{definition}[Fundamental group]
    The \textbf{fundamental group} of a space $X$ with a basepoint $x$ denoted $\pi_1(X,x)$ is the set of loops up to homotopy starting and ending at $x$. For paths $f,g$ we have \[
        (g\circ f)(t)=
        \begin{cases}
            f(2t)\quad & \text{if} \ 0\leq t \leq \sfrac{1}{2},\\
            g(2t-1) &\text{if} \ \sfrac{1}{2} \leq t \leq 1.
        \end{cases}
    \] Similarly, we define $f^{-1}(t)=f(1-s)$, $f$ transversed the other way around, and the identity is the constant loop, $c_x(t)=x$. Multiplication of equivalence classes is given by $[g][f]=[g\circ f]$, which is well-defined, under this operation $\pi_1(X.x)$ becomes associative and unital.
\end{definition}
We can define a change-of-basepoint homomorphism between pointed groups $\gamma[a] \colon \pi_1(X,x)\to \pi_1(X,y)$ for $x,y\in X$ by $\gamma[a][f]=[a\cdot f\cdot a^{-1}],$ where $a$ is a path from $x$ to $y$ (note that $X$ has to be path-connected!). To see that $\gamma[a]$ is a homomorphism, consider $\gamma[a][fg]=[afga^{-1}]=[afa^{-1}aga^{-1}]=[afa^{-1}][aga^{-1}]=\gamma[a][f]\circ \gamma[a][g]$. Note that $\gamma[b\cdot a]=\gamma[b]\circ \gamma[a]$, then $\gamma[a]$ becomes an isomorphism by $\gamma[a]\circ \gamma[a^{-1}]=\gamma[a^{-1}]\circ \gamma[a]=\operatorname{id}_{\pi_1(X)} .$

We can also define the induced map on the fundamental group of a map between spaces $p \colon X \to Y$ as $p_* \colon \pi_1(X,x) \to \pi_1(Y,p(x))$ by $p_*[f]=[p\circ f]$, where $f$ is a loop $I\to X$. This is a homomorphism, and $\operatorname{id}\colon X \to X$ induces the identity homomorphism. Note that for $q \colon Y \to Z$, $q_* \circ p_* = (q\circ p)_*$. We would like to say that for $p \simeq q$ homotopic by a homotopy $h$ that $p_*=q_*$, but this isn't quite true because homotopies don't respect basepoints. But $h$ determines a path $a \colon p(x) \to q(x)$ by $a(t)=h(x,t)$, which helps. Then, we can draw a diagram and show the fundamental group is preserved under homotopy type.
\subsection{Fundamental group of the circle}
If this is a first introduction to fundamental groups, then our first fundamental group of real interest is $\pi_1(S^{1} )=\Z$. Before we do this, let's do a quick calculation to show $\pi_1(\R)=0$. Take the origin as a convenient basepoint. Define $k \colon \R\times I \to \R$ by $k(s,t)=(1-t)s$. Then $k$ is a homotopy from the identity to the constant map at $0$. For a loop $f \colon I \to \R$ at $0$, define a homotopy from this map to a point (via $k$) by $h(s,t)=k(f(s),t).$ Then $f$ is equivalent to a constant $c_0$ by the homotopy $h$.

Now let's talk about circles: we can view $S^{1} $ as the circle group (let's denote it $U^1$), that is, $U^1 = \{z\in \C\mid |z|=1\}.$ Multiplication is continuous, so this is a topological group. Take the identity $1$ as a convenient basepoint for  $S^{1} $.
\begin{theorem}
    We have the fundamental group of a circle isomorphic to the integers, that is, \[
        \pi_1(S^{1} ,1)\cong\Z.
    \] 
\end{theorem}
\begin{proof}
    Define (the covering map) $p \colon \R \to S^1 $ by $p(s)=p(s)=e^{2\pi is}$, wrapping each interval $[n,n+1]$ around the circle. Now $f_n =p\circ \widetilde f_n $, where $\widetilde f_n(s)=sn $ is the uniquely lifted path in $\R$. This is true because exponentiation is a homomorphism between multiplication and addition, and the path lifting holds true for arbitrary paths. That is, for any path $f \colon I \to S^1 $ with $f(0)=I$, we have a uniquely lifted path $\widetilde f\colon I \to \R$ such that $\widetilde f(0)=0$ and $p\circ \widetilde f=f$. Observe that the preimage of any connected neighborhood in $S^1 $ in $\R$ is a disjoint union of a copy of such nbd in the intervals $(r,r+1),(r+1,r+2),\cdots $ for some $r\in \R$. Since $I$ is compact, we can subdivide $I$ into a finite amount of closed subintervals, so $f$ takes each subinterval into such nbd. Working backward, we thus determine the unique lifting of $f$ by using the fact that each lifting of the subintervals is determined inquely by the lifting of its initial point, then repeat.

    Consider $j \colon \pi_1(S^1 ,1) \to \Z$ by $j[f]=\widetilde f(1)$, where then endpoint lifts to. Since $p( \widetilde f(1))=1$, this number is an integer. To show such integer is independent of the choice of $f$ in the homotopy class $[f]$, say we have a homotopy $h \colon f \simeq g$ through loops at $1$, then this lifts uniquely to a homotopy $\widetilde h \colon I\times I \to \R$ such that $\widetilde h(0,0)=0$ and $p\circ \widetilde h=h$. This is the homotopy lifting property. Since paths lift uniquely, the paths $c(t)=\widetilde h(0,t)$ and $d(t)=\widetilde h(1,t)$ determine constant paths since $h(0,t)=h(1,t)=1$ for all $t$. Now $c$ is constant at $0$, so by the unique lifting property we have \[
        \widetilde f(s)=\widetilde h(s,0) \quad \text{and} \quad \widetilde g(s)=\widetilde h(s,1).
    \] Then our second constant path starts at $\widetilde f(1)$ and ends at $\widetilde g(1)$, thus $j(\widetilde f(1))$ is independent of the choice of path in $[f]$.

    Since $j[f_n ]=n$, $j\circ i \colon \Z \to \Z$ is the identity, since $i$ takes integers to the path with such winding number, and $j$ sends a path with such winding number back down to the integers. All we have left to check is that $j$ is injective, since the composition of injections results in an injection, which the identity must be (it is a bijection)\footnote{Actually, apparently the converse of this is an example of something I missed in discrete math called the \textbf{Cantor-Bernstein-Schr\"oeder Theorem}, which states that if we have injections $A\to B$ and $B\to A$, then $A$ and $B$ must be in bijection.}. Suppose $j[f]=j[g]$, then $\widetilde f(1)=\widetilde g(1)$. Then $\widetilde g^{-1}\cdot \widetilde f$ is a loop at $0$ in $\R$. Since $\R$ is contractible, $[\widetilde g^{-1}\cdot \widetilde f]=[c_0]$, thus projecting down yields \[
        [g^{-1}][f]=[g^{-1}\cdot f]=[c_1].
    \] Then $[f]=[g]$, finishing the proof.
\end{proof}
To summarize: we defined a function from the integers into the fundamental group $\pi_1(S^1 )$ by considering the class of loops with winding number $n$, and defined a complementary function by considering the winding number of any class of loops in $S^1 $. How we showed this function was an injection is by considering the upstairs covering space $\R$, in which loops lift uniquely and preserve homotopy type. The fact that $\R$ covers $S^1 $ was also used in the definition of the complementary function assigning winding numbers, precisely as the endpoint of the lifted loop. We carefully made sure that homotopy type mattered not when assigning winding numbers to functions. Then considering that the endpoint of two lifted  loops, such paths being equal meant that they ended in the same place upstairs, and composing one with the others inverse results in a loop in $\R$ centered at zero. But since $\R$ is contractible, this loop becomes zero, and projecting downstairs also makes the composition zero, implying that the two loops are equal. After some function mumbo jumbo, we can conclude that the first map is a bijection, finishing the proof.

\orbreak
We can start with our first real-world application, that is, Brouwers fixed point theorem.
\begin{prop}
    There is no continuous map $r \colon D^2 \to S^1 $ such that $r \circ i=\operatorname{id}_{S^1 }$, where $i$ denotes the inclusion $S^1 \hookrightarrow D^2$.
\end{prop}
\begin{proof}
    If such an $r$ existed, then \[
        \pi_1(S^1 ,1) \overset{i_*}{\lhook\joinrel\longrightarrow} \pi_1(D^2,1) \overset{r_*}{\longrightarrow} \pi_1(S^1 ,1)
    \] would compose to the identity, since the identity induces the identity homomorphism on the fundamental groups. Since the identity homomorphism of $\Z$ doesn't factor through the zero group, this cannot happen.
\end{proof}
\begin{namedthm}{Brouwer's Fixed-point Theorem}
    Any continuous map $f\colon D^2 \to D^2$ has a fixed point.
\end{namedthm}
\begin{proof}
    Suppose $f$ has no fixed point, that is, $f(x)\neq x$ for all $ x$. Define $r(x)\in S^1 $ to be the intersection of the ray starting at $f(x)$ and ending at $x$ with the boundary $S^1 $. This is well defined since $f(x)\neq x$ for all $x.$ Note that $r(x)=x$ if $x\in S^1 $. Since $r$ is continuous, this contradicts our proposition, since this is a retraction of the disk onto the boundary that, when composed with inclusion, becomes the identity on the circle, since $r(x)=x$ for $x\in S^1 $. This finishes the proof.
\end{proof}

\subsection{The van Kampen Theorem (Hatcher)}
Let's take a space $X$ and say it's the union of path-connected open subsets $A_{\alpha}$, each of which contains the basepoint $x_0\in X$. Then the homomorphisms $j_{\alpha} \colon \pi_1(A_{\alpha}) \to \pi_1(X)$ induced by the inclusions $A_{\alpha}\hookrightarrow X$ extend to a homomorphism $\Phi \colon *_{\alpha}\pi_1(A_{\alpha}) \to \pi_1(X)$. The van Kampen theorem will say that $\Phi$ is often onto but in general, we can expect $\Phi$ to have a nontrivial kernel. 

For if $i_{\alpha\beta} \colon  \pi_1(A_{\alpha}\cap A_{\beta}) \to \pi_1(A_{\alpha})$ is the homomorphism induced by the inclusion $A_{\alpha}\cap A_{\beta}\hookrightarrow A_{\alpha}$ then $j_{\alpha}i_{\alpha\beta}=j_{\beta}i_{\beta\alpha}$, both of these compositions being induced by the inclusion $A_{\alpha}\cap A_{\beta}\hookrightarrow X$, so the kernel of $\Phi$ contains all the elements of the form $i_{\alpha\beta}(\omega)i_{\beta\alpha}(\omega)^{-1}$ for $\omega \in \pi_1(A_{\alpha}\cap A_{\beta})$. 

Van Kampen says under fairly broad hypotheses that this determines all of $\Phi$.

\begin{theorem}
    If $X$ is the union of path-connected open sets $A_{\alpha}$ each containing the basepoint $x_0\in X$ and if each intersection $A_{\alpha}\cap A_{\beta}$ is path-connected, then the homomorphism \[
        \Phi \colon *_{\alpha}(A_{\alpha}) \to \pi_1(X)
    \] is onto. Furthermore, if each intersection $A_{\alpha}\cap A_{\beta}\cap A_{\gamma}$ is path-connected, then the kernel of $\Phi$ is the normal subgroup $N$ generated by all elements of the form $i_{\alpha\beta}(\omega)i_{\beta\alpha}(\omega)^{-1}$ for $\omega \in \pi_1(A_{\alpha}\cap A_{\beta})$, and hence $\Phi$ induces an isomorphism \[
    \pi_1(X)=*_{\alpha}\pi_1(A_{\alpha}) /N.
    \] 
\end{theorem}
\begin{example}[Wedge Sums]
    I like the visual of the wedge sum but the terminology of the smash product. If only we could keep the \texttt{\textbackslash vee} symbol ($\vee$) and say we ``smash the spaces together'' at a point. 

    We define the wedge sum $\bigvee_{\alpha}X_{\alpha}$ with basepoints $x_{\alpha}\in X_{\alpha}$ as the disjoint union $\amalg_{\alpha}X_{\alpha}$ with all the basepoints $x_{\alpha}$ identified to a single point. If each $x_{\alpha}$is a deformation retract of an open neighborhood $U_{\alpha }$ in $X_{\alpha }$, then $X_{\alpha }$ is a deformation retract of its open neighborhood $A_{\alpha }=X_{\alpha }\bigvee_{\beta\neq\alpha }U_{\beta}$. The intersection of two or more distinct $A_{\alpha }$'s is $\bigvee_{\alpha }U_{\alpha }$, which deformation retracts to a point. Then by van Kampens theorem, \[
        \Phi \colon *_{\alpha }\pi_1(X_{\alpha }) \to \pi_1(\bigvee_{\alpha }X_{\alpha })
    \] is an isomorphism, provided each $X_{\alpha }$ is path-connected, hence also each $A_{\alpha }$. Therefore for a wedge sum of circles, $\pi_1(\bigvee_{\alpha }S_{\alpha }^{1})$ is a free group, the free product of copies of $\Z$.
\end{example}
\orbreak
I know it always helps to see something done somewhere else. For me, the above definition fails to make any sense at all whatsoever. So, let's revisit van Kampens from two more lens: one from the words of Lee (\emph{Introduction to Topological Manifolds}) and another from the categorical perspective.
\subsection{The van Kampen Theorem (Lee)}
Let's say we have a space $X$ that's made up of the union of two open subsets $U,V\subseteq X$. We know how to compute the fundamental groups of $U,V$, and $U\cap V$ (each of which is path-connected). Every loop can be written as a product of loops in $U$ or $V$ (visualized as the free product $\pi_1(U)*\pi_1(V)$), but any loop in $U\cap V$ only represents a single element of $\pi_1(X)$, even though it represents two distinct elements of the free product (one in $\pi_1(U)$ and one in $\pi_1(V)$). So $\pi_1(X)$ can be though of as the quotient of this free product modulo some relations from $\pi_1(U\cap V)$ that express this redundancy.

Let's do some setup so we can state a precise version of van Kampens. Let $X$ be a topological space and $U,V\subseteq X$ such that $U \cup V=X$ and $U\cap V\neq \O$. Let $q\in U\cap V$. Then the four inclusion maps shown below,
            \begin{figure}[H]
                \centering
                \begin{tikzcd}
                                         & U \arrow[rd, "k"]  &   \\
U\cap V \arrow[ru, "i"] \arrow[rd, "j"'] &                    & X \\
                                         & V \arrow[ru, "l"'] &  
\end{tikzcd}
            \end{figure}
            induce fundamental group homomorphisms as shown below.
                        \begin{figure}[H]
                \centering
                \begin{tikzcd}
                                                        & {\pi_1(U,q)} \arrow[rd, "k_*"] &              \\
{\pi_1(U\cap V,q)} \arrow[ru, "i_*"] \arrow[rd, "j_*"'] &                                & {\pi_1(X,q)} \\
                                                        & {\pi1(V,q)} \arrow[ru, "l_*"'] &             
\end{tikzcd}
            \end{figure}
            Now insert the free product $\pi_1(U,q)*\pi_1(V,q)$ in the middle of the diagram and let $\iota_U \colon \pi_1(U,q) \hookrightarrow \pi_1(U,q)*\pi_1(V,q)$ and $\iota_V \colon \pi_1(V,q) \hookrightarrow \pi_1(U,q)*\pi_1(V,q)$ be the canonical injections. By the characteristic property (unique induced homomorphisms) of the free product, $k_*$ and $l_*$ induce a homomorphism $\Phi \colon \pi_1(U,q)*\pi_1(V,q) \to \pi_1(X,q)$ such that the right half of the following diagram commutes:

                        \begin{figure}[H]
                \centering
                \begin{tikzcd}
                                                                                & {\pi_1(U,q)} \arrow[rd, "k_*"] \arrow[d, "\iota_U" description]  &              \\
{\pi_1(U\cap V, q)} \arrow[ru, "i_*"] \arrow[rd, "j_*"'] \arrow[r, "F", dotted] & {\pi_1(U,q)*\pi_1(V,q)} \arrow[r, "\Phi"]                        & {\pi_1(X,q)} \\
                                                                                & {\pi_1(V,q)} \arrow[ru, "l_*"'] \arrow[u, "\iota_V" description] &             
\end{tikzcd}
            \end{figure}
            Finally, define a map $F \colon \pi_1(U\cap V,q) \to \pi_1(U,q)*\pi_1(V,q)$ by setting $F(\gamma)=(i_*\gamma)^{-1}(j_*\gamma)$\footnote{$F$ is not a homomorphism.}. Let $\overline{F(\pi_1(U\cap V,q))}$ denote the \emph{normal closure}\footnote{the \emph{normal closure} of a set means the smallest normal subgroup that contains such set.} of the image of $F$ in $\pi_1(U,q)*\pi_1(V,q)$.

\begin{theorem}[Seifert-Van Kampen]
    Let $X$ be a topological space. Suppose $U,V\subseteq X$ are open, $U\cup V=X$, and $U,V$, and $U\cap V$ are path-connected. Then for any $q\in U\cap V$, the homomorphism $\Phi$ is surjective, and its kernel is $\overline{F(\pi_1(U\cap V,q))}$. Therefore we have \[
        \pi_1(X,q)\cong \pi_1(U,q)*\pi_1(V,q) \big/ \overline{F(\pi_1(U\cap V,q))}.
    \] 
\end{theorem}
When the fundamental groups in question are finitely presented, the theorem has a useful reformulation in terms of generators and relations.
\begin{cor}
    In addition to the hypothesis of van Kampen, assume that the fundamental groups of $U,V$, and $U\cap V$ have the following finite presentations:
    \begin{gather*}
\pi_1(U,q)\cong \langle \alpha_1,\cdots,\alpha_m \mid \rho_1,\cdots,\rho_r \rangle;\\
\pi_1(V,q)\cong\langle \beta_1,\cdots,\beta_n \mid \sigma_1,\cdots,\sigma_s \rangle; \\
\pi_1(U\cap V,q)\cong \langle \gamma_1,\cdots,\gamma_p \mid \tau_1,\cdots,\tau_t \rangle .
    \end{gather*}
    Then $\pi_1(X,q)$ has the presentation \[
        \pi_1(X,q)\cong \langle \alpha_1,\cdots,\alpha_m,\,\beta_1,\cdots,\beta_n \mid \rho_1,\cdots,\rho_r,\,\sigma_1,\cdots,\sigma_s,\,u_1=v_1,\cdots,u_p=v_p \rangle 
    \] where for each $a=1,\cdots,p$, $u_a$ is an expression for $i_*\gamma_a\in \pi_1(U,q)$ in terms of the generators $\{\alpha_1,\cdots,\alpha_m\} $, and $v_a$ similarly expresses $j_*\gamma_a\in \pi_1(V,q)$ in terms of $\{\beta_1,\cdots,\beta_n\} $.
\end{cor}
\subsection{The fundamental groupoid}
We backtrack a little to talk about categorical nonsense. This doesn't fit too well with the section on category theory, so it's here. These will follow May \S 2.5.
\orbreak
We often talk of pointed spaces, but it would to nice to talk about spaces without making such a choice. We define the fundamental groupoid $\prod_{}^{} (X) $ of a space $X$ to be the category whose objects are the points of $X$ and whose morphism $x \to y$ are the equivalence classes of paths from $x$ to $y$\footnote{Recall Example 1.4 of a group being realized as a category with all its arrows isomorphisms.}. Then the set of endomorphisms of the object $x$ is the fundamental group $\pi_1(X,x)$.

We say ``groupoid'' because a group is simply a groupoid with only one object (the class of morphisms or symmetries on an object). However, the category of groupoids has several objects. We also defined groupoids as categories whose morphisms are all isomorphisms. If morphisms are functors, then we have the category $\mathsf{Grpd} $ of groupoids. So we can see $\prod $ as a functor $\mathsf{Top_*} \to \mathsf{Grpd} $.

Let's talk about skeletons. We have the skeleton of a category $\mathcal{C} $ denoted by $\operatorname{sk}(\mathcal{C}) $. This is a ``full'' subcategory with one object from each isomorphism class of objects of $\mathcal{C} $, ``full'' meaning that the morphisms between two objects of $\operatorname{sk}(\mathcal{C} )$ are all of the morphisms between these objects in $\mathcal{C} $. The inclusion functor $J \colon \operatorname{sk}(\mathcal{C} ) \to \mathcal{C} $ is an equivalence of categories. We can find an inverse functor $F \colon \mathcal{C}  \to \operatorname{sk}(\mathcal{C} )$ by letting $F(A)$ be the unique object in $\operatorname{sk}(\mathcal{C} )$ that is isomorphic to $A$, choosing an isomorphism $\alpha _A \colon A \to F(A)$, and defining $F(f)=\alpha _B \circ f \circ \alpha_A^{-1}$ for a morphism $f \colon  A \to B$ in $\mathcal{C} $. Choose $\alpha $ to be the identity morphisms if $A\in \operatorname{sk}(\mathcal{C} )$, then $FJ=\operatorname{id}_{\operatorname{sk}(\mathcal{C} )}$; the $\alpha _A$ specify a natural isomorphism $\alpha \colon \operatorname{id} \to JF $.

A category is connected if any two objects can be connected by a sequence of morphisms. Then a groupoid is connected iff any two of its objects are isomorphic. The group of endomorphisms of any object $C$ is then a skeleton of $\mathcal{C} $, so we can generalize our results about skeletons to give the relationship between a fundamental group and a fundamental groupoid of a path connected space $X$.
\begin{prop}
    Let $X$ be a path connected space. Then for each $x\in X$, the inclusion $\pi_1(X,x) \to \prod (X)$ is an equivalence of categories.
\end{prop}
\begin{proof}
    View $\pi_1(X,x)$ as a groupoid with on object $x$: then $\pi_1(X,x)$ is a skeleton of $\prod (X)$ and we are done.
\end{proof}
May's presentation and proofs are very concise and elegant. I like this.



