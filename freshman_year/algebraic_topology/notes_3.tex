\section{Van Kampen's Theorem}
OK guys, let's decompose big spaces into smaller ones and compute their fundamental groups. These notes follow Hatcher \S 1.2.
\subsection{The van Kampen Theorem}
Let's take a space $X$ and say it's the union of path-connected open subsets $A_{\alpha}$, each of which contains the basepoint $x_0\in X$. Then the homomorphisms $j_{\alpha} \colon \pi_1(A_{\alpha}) \to \pi_1(X)$ induced by the inclusions $A_{\alpha}\hookrightarrow X$ extend to a homomorphism $\Phi \colon *_{\alpha}\pi_1(A_{\alpha}) \to \pi_1(X)$. The van Kampen theorem will say that $\Phi$ is often onto but in general, we can expect $\Phi$ to have a nontrivial kernel. 

For if $i_{\alpha\beta} \colon  \pi_1(A_{\alpha}\cap A_{\beta}) \to \pi_1(A_{\alpha})$ is the homomorphism induced by the inclusion $A_{\alpha}\cap A_{\beta}\hookrightarrow A_{\alpha}$ then $j_{\alpha}i_{\alpha\beta}=j_{\beta}i_{\beta\alpha}$, both of these compositions being induced by the inclusion $A_{\alpha}\cap A_{\beta}\hookrightarrow X$, so the kernel of $\Phi$ contains all the elements of the form $i_{\alpha\beta}(\omega)i_{\beta\alpha}(\omega)^{-1}$ for $\omega \in \pi_1(A_{\alpha}\cap A_{\beta})$. 

Van Kampen says under fairly broad hypotheses that this determines all of $\Phi$.

\begin{theorem}
    If $X$ is the union of path-connected open sets $A_{\alpha}$ each containing the basepoint $x_0\in X$ and if each intersection $A_{\alpha}\cap A_{\beta}$ is path-connected, then the homomorphism \[
        \Phi \colon *_{\alpha}(A_{\alpha}) \to \pi_1(X)
    \] is onto. Furthermore, if each intersection $A_{\alpha}\cap A_{\beta}\cap A_{\gamma}$ is path-connected, then the kernel of $\Phi$ is the normal subgroup $N$ generated by all elements of the form $i_{\alpha\beta}(\omega)i_{\beta\alpha}(\omega)^{-1}$ for $\omega \in \pi_1(A_{\alpha}\cap A_{\beta})$, and hence $\Phi$ induces an isomorphism \[
    \pi_1(X)=*_{\alpha}\pi_1(A_{\alpha}) /N.
    \] 
\end{theorem}
\begin{example}[Wedge Sums]
    I like the visual of the wedge sum but the terminology of the smash product. If only we could keep the \texttt{\textbackslash vee} symbol ($\vee$) and say we ``smash the spaces together'' at a point. 

    We define the wedge sum $\bigvee_{\alpha}X_{\alpha}$ with basepoints $x_{\alpha}\in X_{\alpha}$ as the disjoint union $\amalg_{\alpha}X_{\alpha}$ with all the basepoints $x_{\alpha}$ identified to a single point. If each $x_{\alpha}$is a deformation retract of an open neighborhood $U_{\alpha }$ in $X_{\alpha }$, then $X_{\alpha }$ is a deformation retract of its open neighborhood $A_{\alpha }=X_{\alpha }\bigvee_{\beta\neq\alpha }U_{\beta}$. The intersection of two or more distinct $A_{\alpha }$'s is $\bigvee_{\alpha }U_{\alpha }$, which deformation retracts to a point. Then by van Kampens theorem, \[
        \Phi \colon *_{\alpha }\pi_1(X_{\alpha }) \to \pi_1(\bigvee_{\alpha }X_{\alpha })
    \] is an isomorphism, provided each $X_{\alpha }$ is path-connected, hence also each $A_{\alpha }$. Therefore for a wedge sum of circles, $\pi_1(\bigvee_{\alpha }S_{\alpha }^{1})$ is a free group, the free product of copies of $\Z$.
\end{example}


