\section{Homology} 
The big boy has arrived. These notes will follow Hatcher \S 2.1.
\begin{remark}
    This is something I heard even before I enrolled in this course. The homotopy groups are easy to define, but impossible to compute and work with. The homology groups take a lot of work to define, but the resulting groups are much nicer and easier to work with.
\end{remark}
\orbreak
The fundamental group is a cool tool when dealing with low-dimensional spaces (the pride and joy of UT Austin), but it doesn't do well with higher dimensional spaces, for example, it can't distinguish between the $n$-spheres $S^n$ for $n\geq 2$. We can get rid of this limitation by considering the higher homotopy groups $\pi_n(X)$, which are defined in terms of maps from the $n$-dimensional cube $I^n$ and homotopies $I^n \times  I \to X$ of such maps. Cool things about higher homotopy groups: for $X$ a CW complex, $\pi_n(X)$ only depends on the $(n+1)$-skeleton, and $\pi_i(S^n)=0$ for $i<n$ and $\Z$ for $i=n$, as expected. However, the drawback is that they're extremely difficult to compute in general— take the ``simple'' task of computing $\pi_i(S^n)$ for $i>n$.

Enter the homology groups $H_n(X)$. Similar to $\pi_n(X)$, $H_n(X)$ for $X$ a CW complex depends only on the $(n+1)$-skeleton, and for the spheres $H_i(S^n)\simeq \pi_i(S^n)$ for $1\leq i\leq n$, but the homology groups have the advantage in that $H_i(S^n)=0$ for $i>n$. However, everything has a price. How exactly do we define these so called homology groups? We start by motivating, then doing simplicial homology, before moving onto singular homology. Most efficient method for computing homology groups is called cellular homology. We'll also talk about Mayer-Vietoris sequences, the analogue of the van Kampens for the fundamental group.

Something interesting about homology: most of the time we only use the basic properties of homology, not the definition itself. So we could almost invoke an axiomatic approach, which will happen soon. We could also skip the algebra and talk about geometry, but then Dr.\ Brand would be unhappy (and so would I), so we'll approach it with a mix of the two (talk about intuition first then state the axioms later).
\subsection{The big idea of homology}
Issues with homotopy groups: things get really wacky because $S^{2} $ has no cells of dimension greater than $2$, but some (infinitely many) of the higher homotopy groups $\pi_n(S^{2} )$ are nontrivial. $\langle god\, shattering\, star\, noises\rangle $ However, homology groups are (directly) related to cell structures, in that you can regard them as an algebraization of how cells of dimension $n$ attach to cells of dimension $n-1$.

Imagine a circle with two antipodal points $x$ and $y$, with four arrows $a,b,c,d$ drawn in the direction from $x$ to $y$, which we'll denote by $X_1$.
\begin{figure}[H]
    \centering
    \incfig[0.3]{homology_basic}
    \caption{The graph $X_1$, consisting of two vertices and four edges.}
    \label{x1}
\end{figure}
Usually loops are nonabelian, so suppose we abelianize the loops. That is, the loops $ab^{-1}$ and $b^{-1}a$ are ``the same circle'' (but with a different starting point), so we'll just say they're equal. Formally (not really), rechooisng the basepoint just permutes the letters cyclically, so by abelianizing we can cast off our silly worries about the basepoint. So we make the transition from loops (chosen basepoint) $\longrightarrow$ cycles (no chosen basepoint).

Now we abelian, and all the cool abelian groups use additive notation. So a cycle looks something like $a-b+c-d$ now, a linear combination of edges with integer coefficients. We'll call these linear combinations \textbf{chains} of edges. We can decompose these into cycles by several ways, eg $(a-c)+(b-d)=(a-d)+(b-c)$, so it's better just to say cycles are any LC of edges st at least one decomposition makes geometric sense. When is a chain a cycle? Cycles are distinguished by the fact that they enter and exit a vertex the same amount of times. So for an arbitrary chain $ka+lb+mc+nd$, it enters $y$ about $k+l+m+n$ times (one for each thing) and enters $x$ (or leaves it) $-k-l-m-n$ times. So if we want $ka+lb+mc+nd$ to be a cycle, we just need to require $k+l+m+n=0$.

To generalize this, let $C_1$ be the free abelian group with a basis set $\{a,b,c,d\} $ (edges), and $C_0$ be the free abelian group with basis $\{x,y\} $ (vertices). Elements of $C_1$ are chains of edges, and elements of  $C_0$ are linear combinations of vertices. Define a homomorphism $\partial \colon C_1 \to C_0$ by sending each basis element to $y-x$, then $\partial (ka+lb+mc+nd)=(k+l+m+n)y-(k+l+m+n)x$, so cycles are precisely $\operatorname{ker}\partial $. It can be seen that $a-b,\,b-c,$ and $c-d$ form a basis for $\operatorname{ker}\partial $, so every cycle in $X_1$ is a unique linear combination of these three elts. Basically, $X_1$ has three ``holes'', the three gaps in between the four edges.

Now let's attach a $2$-cell to $X_1$ to get $X_2$, as seen below.
\begin{figure}[H]
    \centering
    \incfig[0.3]{x2}
    \caption{$X_1$ with a $2$-cell attached, denoted $X_2$. Have you ever seen a $2$-cell that looks like cloth?}
    \label{x2}
\end{figure}
The $2$-cell is attached along the cycle $a-b$, forming the $2$-skeleton $X_2$. Now the cycle is trivial (homotopically), which suggest we form a quotient by factoring out the subgroup generated by $a-b$. For example, $a-c$ and $b-c$ are now equivalent, since they're homotopic in $X_2$. Algebraically, we define a pair of homomorphisms $C_2 \overset{\partial_2}{\longrightarrow}C_1\overset{\partial_1}{\longrightarrow}C_0  $, where $C_2$ is the infinite cyclic group generated by $A$, and $\partial_2(A)=a-b$. $\partial_1 $ is the boundary homomorphism, defined earlier. We are interested in $\operatorname{ker}\partial_1 / \operatorname{im}\partial_2  $, that is, the $1$-dimensional cycles modulo the boundaries (multiples of $a-b$). Remember, factor groups collapse everything we don't like to the identity. This quotient group is the \textbf{homology group} $H_1(X_2)$. If we were to talk about $X_1,$ since it has no $2$-cells $C_2$ is simply zero, so $H_1(X_1)=\ker \partial_1 / \operatorname{im}\partial_2=\ker \partial_1  $, which is free abelian on three generators. $H_1(X_2)$ is free abelian on two generators ($b-c$ and $c-d$), which expresses the geometric observation that there are two holes remaining after filling one of them in with the $2$-cell $A$.

Let's go farther. Add another $2$-cell to the pre-existing $2$-cell $A$, to get the $3$-complex $X_3$.
\begin{figure}[H]
    \centering
    \incfig[0.3]{x3}
    \caption{The $3$-complex  $X_3$, formed by attaching a $2$-cell to $X_2$.}
    \label{x3}
\end{figure}
This gives a $2$-dimensional chain group $C_2$ consisting of linear combinations of $A$ and $ B$, and the boundary homomorphism $\partial_2 \colon C_2 \to C_1 $ sends $A,B$ to $a-b$. $H_1(X_3)=\ker \partial_1 / \operatorname{im}\partial_2  =H_1(X_2)$, but now $\partial_2 $ has a nontrivial kernel (the infinite cyclic group generated by $A-B$). We view $A-B$ as a $2$d cycle generating $H_2(X_3)=\ker \partial_2\simeq\Z $. The second homology detects the $2$d ``hole'' in $X_3$. 

Unfortunately the diagrams will have to stop now, but let's go even farther and make the complex $X_4$ from $X_3$ by attaching a $3$-cell $C$ along the $2$-sphere by $A$ and $B$, creating a chain group $C_3$ generated by $C$. The boundary homomorphism $\partial_3 \colon C_3 \to C_2 $ that sends $C$ to $A-B$ should be seen as the boundary of $C$, similar to how $a-b$ is the boundary of $A$. Now we have a sequence of boundary homomorphisms $C_3 \overset{\partial_3}{\longrightarrow}C_2\overset{\partial_2}{\longrightarrow} C_1 \overset{\partial_1}{\longrightarrow}C_0$, and $H_2(X_4)=\ker \partial_2 /\operatorname{im}\partial_3  $ is now trivial. $H_3(X_4)=\ker \partial_3=0, $ note that $H_1(X_4)=H_1(X_3)\simeq\Z\times \Z$, so this is the only homology group of $X_4$ that isn't trivial.
\orbreak
You can pretty much see where this is going. For a cell complex $X$, we have chain groups $C_n(X)$ free abelian with basis the $n$-cells of $X$, with boundary homomorphisms $\partial_n \colon C_n(X) \to C_{n-1} (X)$, by which we define the homology group $H_n(X) = \ker \partial_n / \operatorname{im}\partial _{n+1} $. So what's the problem? It's how to define $\partial_n $ in general— for $n=1$ this is easy, it's the vertex head minus the one at the tail. For $n=2,$ it still isn't hard per say, if the cell is attached on a loop of edges, just take the cycle of edges, keeping in mind orientation. This is much tricker for higher dimension cells, even with restrictions to polyhedral cells and nice attaching maps we still have to worry about orientation and stuff.

So what do we do? Use triangles, of course. We can subdivide arbitary polyhedra into certain special types of polyhedra called simplices (what we talked about in class day 1), so there isn't any loss of generality (but there is a loss of efficiency). This gives rise to our more basic \textbf{simplicial homology}, which deals with cell complexes from simplices. However, we are still quite limited in what we can do.

So, what do we really do this time? Make things less simple, and make your life difficult by considering the collection of all possible continuous maps of simplices into a space $X$ (wow). The chain groups $C_n(X)$ are tremendously large, but the quotients $H_n(X)=\ker \partial_n / \operatorname{im}\partial _{n+1} $, the \textbf{singular homology groups}, are much smaller and easier to work with\footnote{For reasonably ``nice'' spaces $X$, of course.}. For example, in the examples above the singular homology groups coincide with the ones computed from cellular chains. Furthermore (as we will see later), singular homology lets us define these nice cellular homology groups for \textit{all} cell complexes, which solves the issue of how to define boundary maps for cellular chains.

\subsection{Simplicial homology and $\Delta$-complexes}
I have a feeling we're gonna be typing a lot of \texttt{\textbackslash Delta}'s. So basically, the only thing cool kids talk about is singular homology, but it's kinda complicated so we gotta talk about the inferior version for those who have the brain capacity of a literal ape\footnote{The book simply says ``primitive'' version, so I used my imagination a little bit.}, simplicial homology, first. We talk about simplicial homology in the domain of $\Delta $-complexes. Take the standard fundamental polygons with orientation for $\mathbb{T}^2,\,\R P^2,\,$ and the Klein bottle $K$. Cut the squares in half with a diagonal to get two triangles, from here we can get the original shape by identifying in pairs. We can do this with any $n$-gon, decomposing it into $n-2$ base triangles. So we can make any closed surface from triangles, furthermore, we could also make a larger class of spaces that aren't surfaces by allowing more than two edges to be glued together at the same time.

The idea of a $\Delta $-complex is to generalize these constructions to $n$-dimensions. The $n$-dimensional triangle is the $n$-simplex, the smallest convex set in $\R^m$ containing $n+1$ points $v_0,\cdots,v_n$ that don't lie in a hyperplane of dimension less than $n$, where by ``hyperplane'' we mean the set of solutions to a system of linear equations. We could also just say that the difference vectors $v_1-v_0,\cdots,v_n-v_0$ are LI. The $v_i$ are \textbf{vertices} of the simplex, and the simplex itself is $[v_0,\cdots,v_n]$.
\begin{figure}[H]
    \centering
    \incfig[0.5]{delta_simplex}
    \caption{The $0$-simplex to the $3$-simplex, respectively (with ordered vertices and oriented edges).}
    \label{simp}
\end{figure}
For example, we have the standard $n$-simplex given by \[
    \Delta ^n =\{(t_0,\cdots,t_n )\in \R^{n+1} \mid \sum_{i}^{} t_i=1 \ \text{and} \ t_i \geq 0 \ \text{for all} \ i\},
\] whose vertices are the unit vectors along the coordinate axes. Think of this as taking the unit vectors, and drawing a triangle from each of their endpoints. For homology, orientation of vertices is really important, so $n$-simplex really means $n$-simplex with an ordering on its vertices. Ordering the vertices will determine an orientation on its subscripts, as can be seen in \cref{simp}. This also determines a canonical linear homeomorphism from the standard $n$-simplex $\Delta ^n $ onto any other simplex $[v_0,\cdots,v_n]$ that preserves the order of the vertices, given by 
\[
(t_0,\cdots,t_n )\mapsto \sum_{i}^{} t_iv_i.
\] We say the coefficients $t_i$ are the \textbf{barycentric coordinates} of the point $\sum_{i}^{} t_i v_i \in [v_0,\cdots,v_n ].$ Deleting a vertex of a $n$-simplex yields something that spans an $(n-1)$-simplex, called a \textbf{face} of $[v_0,\cdots,v_n ]$. We'll adopt the following convention:
\textit{The vertices of a face, or of any subsimplex spanned by a subset of the vertices,  will always be ordered according to their order in the larger simplex.}
That sounds reasonable enough. We say the union of all faces of $\Delta ^n $ is the \textbf{boundary} of $\Delta ^n $, written $\partial \Delta ^n $. The \textbf{open simplex} $\mathring{\Delta }^n $ is equal to $\Delta ^n \setminus \partial \Delta ^n $, the interior of $\Delta ^n $. 

A $\mathbf{\Delta }$\textbf{-complex} structure on a space $X$ is a collection of maps $\sigma_{\alpha } \colon \Delta ^n  \to X$, with $n$ depending on the index $\alpha $, such that:
\begin{enumerate}
    \item The restriction $\sigma_{\alpha }|\mathring{\Delta }^n $ is onto, and each point of $X$ is in the image of exactly one restriction $\sigma_{\alpha }|\mathring{\Delta }^n $.
    \item Each restriction of $\sigma_{\alpha }$ to a face of $\Delta ^n $ is one of the maps $\sigma_{\beta } \colon \Delta ^{n-1} \to X$. Here we are identifying the fact of $\Delta ^n $ with $\Delta ^{n-1}$ by the canonical linear order-preserving homeomorphism.
    \item A set $A\subseteq X$ is open if and only if $\sigma_{\alpha }^{-1}(A)$ is open in $\Delta ^n $ for each $\sigma_{\alpha }$.
\end{enumerate}
