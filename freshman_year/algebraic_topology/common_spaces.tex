\section{Common Topological Structures}
We'll take this section to digress a little bit and explore some examples of our favorite spaces that we work with a lot in topology.
\subsection{Manifolds (todo)}
\subsection{Cell complexes}
The big idea is this: we can build a lot of our favorite topological spaces by starting with ``removable parts'' from each dimension, then glueing them together via something called a ``boundary map''. Naturally, we start from dimension zero (points), add cells from the $1$st dimension (arcs), then $2$-cells (surfaces), and so on. 

For example, we can construct the torus $\mathbb{T}=S^1 \times S^1 $ by identifying opposite edges of a square. In general, an orientable (worry about what this means later) surface $M_g$ of genus $g$ can be constructed from a polygon with $4g$ sides by identifying pairs of edges. The $4g $ edges of the polygon becomes a union of $2g$ circles intersecting at a point. The interior of the polygon can be thought of as a $\mathbf 2$\textbf{-cell}, and the union of circles being obtained by attaching $2g$ open arcs, or $\mathbf 1$\textbf{-cells}. 
\begin{definition}[$n$-cell]
An $\mathbf{n}$\textbf{-cell} $e^n $ is defined as a space homeomorphic to the open disk $D^n \setminus \partial D^n,$ where $\partial D^n =S^{n-1}$. Note that $D^0$ and $e^0$ are simply points since $\R^0=\{0\} $.
\end{definition}
\begin{definition}[CW complexes]
We call the spaces defined by attaching $n$-cells by the name \textbf{cell complex} or \textbf{CW complex}. When it is clear that we're referring to cell or CW complexes, sometimes we will drop the beginning word and just refer to these spaces as ``complexes''. Here is a natural way to generalize the construction of such complexes:
\begin{enumerate}[label=(\arabic*)]
    \item Start with a discrete set of points $X^0$, or $0$-cells.
    \item Inductively, form the $\mathbf n$\textbf{-skeleton} $X^n $ from $X^{n-1}$ by attaching $n$-cells $e_{\alpha }^n $ via maps $\varphi _{\alpha }\colon S^{n-1} \to X^{n-1}$. Note! These maps are supposed to be hard to define! So $X^n $ is the quotient space of the disjoint union $X^{n-1}\amalg _{\alpha }D_{\alpha }^n $ of $X^{n-1}$ with a collection of $n$-disks $D^n $ under the identifications $x\sim \varphi _{\alpha }(x)$ for $x\in \partial D_{\alpha }^n $. Thus as a set, $X^n =X^{n-1}\amalg_{\alpha }e_{\alpha }^n $ where each $e_{\alpha }^n $ is an open $n$-disk.
    \item We can either stop this process at a finite step with $X=X^n $ for some $n<\infty$, or continue on forever, with $X=\bigcup_{n} X^n $. In the infinite-dimensional case, $X$ is given the weak topology: A set $A\subseteq X$ is open (or closed) iff $A\cap X^n $ is open (or closed) in $X^n $ for each $n$.
\end{enumerate}
If $X=X^n $ for some $n$, then $X$ is said to be finite-dimensional, and the smallest such $n$ is the \textbf{dimension} of $X$, or the maximum number of cells of $X$.
    
\end{definition}
\begin{example}
    To better understand what is happening in step (2) of building a CW complex, let us examine the first few cases.
    \begin{enumerate}[label=(\arabic*)]
        \item Consider $n=1$. We start with a $0$-skeleton consisting of just discrete points. Then we attach $1$-cells homeomorphic to the open interval $(0,1)$ via maps $\varphi _{\alpha }\colon S^0 \to X^0$, which are maps from the two endpoints of an interval into a discrete set of points. So $X^1$ is the quotient space of the disjoint union $X^0 \amalg_{\alpha }D^1_{\alpha }$ of $X^0$ with a collection of $1$-disks $D_{\alpha }^1$ (just points in $X^0$ plus $\alpha $ number of intervals that don't touch the points) under the identifications $x\sim \varphi _{\alpha }(x)$ for $x\in \partial D_{\alpha }^1 =S^{0} \simeq \{0,1\}$. In this case, the identifications are just ways of glueing endpoints of the interval $[0,1]$. So as a set, $X^1=X^0\amalg _{\alpha }e_{\alpha }^1$ where $e_{\alpha }^1$ is homeomorphic to the interval $(0,1)$. 

            All this is saying is start out with points, and add some closed intervals (but make sure they're disjoint). Then, identify endpoints of these intervals together with maps $\varphi _{\alpha }$ by considering $x\sim \varphi _{\alpha }(x)$. Post-identification, we have the $1$-skeleton equal to a bunch of points with disjoint open intervals. The reason they're disjoint is so points from $X^0$ can't go in the middle of the interval $(0,1)$, but \emph{can} be attached to the endpoints $0,1$ still preserving disjointedness. Basically, $[0,1]\cup \{0,1\} $ isn't a disjoint union, but $(0,1)\amalg \{0,1\} $ is, and they look the same. Note that the $1$-skeleton $X^1$ is just a \textbf{graph} as we call it in algebraic topology, with vertices the $0$-cells and edges the $1$-cells.
        \item Now consider $n=2$. We start with the $1$-skeleton $X^1$ which looks like a graph with vertices and edges. Then we attach $2$-cells homeomorphic to the open ball $B(0,1)\subseteq D^2$, which is a disk in $\R^2$ minus its boundary, $S^1 $. These look like sheets of paper cut in a circle, with an open boundary. We attach these sheets via maps $\varphi _{\alpha }\colon S^1  \to X^1$, which map from a boundary circle into a graph. Then $X^2$ is the quotient space of the disjoint union $X^1 \amalg_{\alpha }D_{\alpha }^2$ with the identification $x\sim \varphi _{\alpha }(x)$ for $x\in \partial D_{\alpha }^2=S^1 $. This time, we consider a graph with a bunch of closed sheets, then attach the sheets based off where the boundary circle gets glued onto the graph (realized by $\varphi _{\alpha }$). So as a set, $X^2=X^1 \amalg _{\alpha }e_{\alpha }^2         $, which means that the $2$-skeleton $X^2$ looks like a graph plus open sheets, glued together at the boundaries. You could also think of attaching $2$-cells as ``filling in'' regions/faces of the graph enclosed by edges.
        \item For $n=3$, we start with $X^2$, a graph with sheets glued in circles along the edges. Then we form the $3$-skeleton $X^3$ by attaching $3$-cells (imagine these as ``filling'' cells, like how a cat will conform to the shape its container, sans the container)\footnote{For a less exciting but more pedagogically effective metaphor, replace all instances of ``cat'' with the appropiate form of the word ``liquid''.}. These cats are attached via maps $\varphi _{\alpha }\colon S^2 \to X^2$, which take their containers $S^2$ (a real-life ball minus its interior, looks like a shell) into the $2$-skeleton. Then consider the $2$-skeleton together with the space-filling cats and their containers, denoted by $X^2\amalg_{\alpha }D_{\alpha }^3$, and identify $x\sim \varphi _{\alpha }(x)$ with $x\in \partial D_{\alpha }^3=S^2$, which is just saying where the containers (or empty shells) get glued onto the $2$-skeleton. Then, the cat will conform to fit its space, and thus we have attached our $3$-cells, and we express this by saying $X^3=X^2\amalg_{\alpha }e_{\alpha }^3$.
    \end{enumerate}
At this point, you should have a good idea of where this is going, although it does get more difficult to visualize in higher dimensions, which is why we have the handy topological construction we stated before moving on to cases. 
\end{example}
\begin{definition}[Euler characteristic]
    The \textbf{Euler characteristic} $\chi (X)$ of a finite cell complex is the number of even-dimensional cells minus the odd-dimensional cells. Formally, we define 
    \[
    \chi(X)=\sum_{n=0}^{k} (-1)^n c_n,
    \] 
    where $c_n $ is the number of $n$-cells of $X$ and $k$ is the dimension of $X$. For $k=2$, we have $\chi (X)=0$-cells $-1$-cells $+2$-cells, which if you think of $ 0$-cells as vertices, $1$-cells as edges, and $2$-cells as faces, is our familiar formula $V-E+F=2$ for convex polyhedron, so it can be seen that the Euler characteristic is a generalization of Euler's formula. We will later use homology to show that $\chi(X)$ depends only on the homotopy type of $X$, which is why $V-E+F=2$ holds for all convex polyhedra.
\end{definition}
\begin{example}
    The sphere $S^n $ has a cell complex structure consisting of one $0$-cell and one $n$-cell, with the $n$-cell attached by the constant map $S^{n-1}\to e^0$. You can think of this as wrapping up the ``boundary'' of the $n$-plane down to a single point, and closing it up (the best concrete example is $\R^2$ plus a point). This is formally expressed by the quotient space $D^n / \partial D^n $.
\end{example}
\begin{definition}[]
    Each cell $e_{\alpha }^n $ in a cell complex $X$ has a \textbf{characteristic map} $\phi_{\alpha }\colon D_{\alpha }^n  \to X$ extending the attaching map $\varphi _{\alpha }$, and is a homeomorphism from the interior of $D_{\alpha }^n $ onto $e_{\alpha}^n  $. We define $\varphi _{\alpha }$ as the composition $D_{\alpha }^n \hookrightarrow X^{n-1}\amalg_{\alpha }D_{\alpha }^n \to X^n \hookrightarrow X$, where the map $X^{n-1}\amalg_{\alpha }D^n _{\alpha }$ is the quotient map that builds $X^n $ from $X^{n-1}$.
\end{definition}
\begin{example}
    For examples of the characteristic map, considering the canonical CW structure on $S^n $, a characteristic map for the lone $n$-cell is the quotient map $D^n \to S^n $ collapsing $D^n $ to a point. For $\R\mathrm P^n $ (see next section), a characteristic map for $e^i $ is the quotient map $D^i \to \R\mathrm P^i \subseteq \R\mathrm P^n $ identifying antipodal points of $\partial D^i =S^{i-1}$, and similarly for $\C\mathrm P^n $.
\end{example}
\begin{definition}[Subcomplexes and CW pairs]
    A \textbf{subcomplex} of a cell complex $X$ is a closed subspace $A\subseteq X$ consisting of a union of cells of $X$. Since $A$ is closed, the characteristic map of each cell in $A$ has image contained in $A$, and thus the image of each attaching map is also in $A$, so $A$ is also a cell complex. A pair $(X,A)$ consisting of a cell complex $X$ and a subcomplex $A$ is called a \textbf{CW pair}.
\end{definition}
\begin{example}
    Each $n$-skeleton $X^n $ of a cell complex $X$ is a subcomplex.
\end{example}
\begin{example}
    With the canonical structure on $S^n $, we have natural inclusions $S^0\subseteq S^1\subseteq \cdots \subseteq S^n $, but these aren't subcomplexes since $S^n $ only has two cells. However, if we put a different CW structure on $S^n $ by attaching two hemispheres represented by $k$-cells onto the equatorial $S^{k-1}$ to obtain $S^k$, then each subspace $S^k$ of $S^n $ is a subcomplex. The $k$-cells are precisely the two remaining components of $S^k \setminus S^{k-1}$. Then the infinite sphere $S^{\infty}=\bigcup_{n=0} ^{\infty}S^n $ also becomes a cell complex. 
\end{example}
In the examples of cell complexes we've given so far, the closure of each cell is a subcomplex, and the same goes for any collection of cells. However, this doesn't have to hold in general: consider the canonical structure of $S^1 $, then if we attach a $2$-cell by a map $\varphi \colon  S^1 \to S^1 $ where $\operatorname{im}\varphi $ is a nontrivial arc in $S^1 $, then the closure of the $2$-cell is not a subcomplex since it only contains part of the $1$-cell.
\subsection{Operations on CW complexes}
We can do a lot of things to cell complexes. Here are some of them.
\begin{namedthing}{Products}
    If $X,Y$ are complexes, then $X\times Y$ has a CW structure with cells $e_{\alpha }^m\times e_{\beta }^n $ where $e_{\alpha }^m$ ranges over the cells of $X$ and $e_{\beta }^n $ the cells of $Y$. An example is the torus $\mathbb{T}=S^1 \times S^1 $ with the standard cell structure on $S^1 $. For general complexes $X$ and $Y$ there is a minor point-set topological issue, that is, the topology of $X\times Y$ as a complex is sometimes finer than the product topology, however the topologies coincide if either $X$ or $Y$ has finitely many cells or both $X$ and $Y$ have countably many cells. This kind of triviality rarely causes real issues though.
\end{namedthing}

\begin{namedthing}{Quotients}
    If $(X,A)$ is a CW pair, then the quotient space $X /A$ takes on a natural CW structure from $X$, consisting of the cells of $X \setminus A$ plus a $0$-cell representing the image of $A$, or the point it gets identified to. For a cell $e_{\alpha }^n $ of $X\setminus A$ attached by $\varphi _{\alpha }\colon S^{n-1} \to X^{n-1}$, the attaching map for such cell in the quotient $X /A$ is the composition $S^{n-1}\to X^{n-1}\to X^{n-1} / A^{n-1}$. For example, given any structure on $S^{n-1}$ we can build $D^n $ by attaching an $n$-cell, so the quotient $D^n /S^{n-1}$ is $S^n $ with the usual structure, since the old structure all gets identified to a point. 
\end{namedthing}
\begin{namedthing}{Suspension}
    Given a space $X$, the \textbf{suspension} $SX$ is the quotient of $X\times I$ with the identifications of collapsion $X\times \{0\} $ to a point and $X\times \{1\} $ to another point. The example that makes the suspension somewhat clear is $X=S^1 $: $S^1 \times I$ is the hollow cylinder, and identifying $S^1 \times \{0\} $ and $S^1 \times \{1\} $ to points ``closes up'' the ends to make a two-sided hollow top. We can regard $SX$ as the union of two copies of the \textbf{cone} $CX$, defined as $CX:=(X\times I) /(X\times \{0\} )$. If $X$ is a CW complex, then so are $C X$ and $SX$ as quotients of the product with $I$, where $I$ has the standard structure of two $0$-cells and a $1$-cell. Don't tune out just yet, suspension becomes important later in algebraic topology. It's hard to give relevant things as of now, but suspension is a functor that sends maps $f\colon X \to Y$ to $Sf \colon SX \to SY$ where $Sf$ is the quotient map of $f\times \mathbb 1 \colon X\times I \to Y\times I$, and it also induces isomorphisms with the $(n+1)$th homology group, that is, $\widetilde H_{n+1} (X)\approx \widetilde H_n (SX)$ for all $n$.
\end{namedthing}
\begin{namedthing}{Join}
    The cone $CX$ consists of all line segments joining points of $X$ to a vertex, and the suspension does the same with two vertices. In general, for spaces $X,Y$ we can define define the space of line segments joining $X$ to $Y$, called the \textbf{join} $X\star Y$. Formally, the join is the quotient space  $X\times Y\times I$ with the identifications $(x,y_1,0)\sim (x,y_2,0)$ for all $x\in X,y_1,y_2\in Y$ and $(x_1,y,1)\sim (x_2,y,1)$ for all $x_1,x_2\in X,y_2\in Y$. This collapses the subspace $X\times Y\times \{0\} $ to $X$ and $X\times Y\times \{1\} $ to $Y$. The join can be particularly confusing to wrap your head around, so it helps to start by visualizing the join of some basic spaces and working up from there until a general understanding is achieved. This is a lot of text, so feel free to skip this and move onto the wedge sum if you're not interested in how joins work.
    \begin{itemize}
        \item Let's begin by considering the join of the singleton spaces $X=\{x_0\} ,Y=\{y_0\} $. Then $X\times Y\times I$ is just an interval of the form $(x_0,y_0,i)$ for $i\in I$, so the join identifications just identify the endpoints with themselves, which does nothing. So the join of two points is an interval.
        \item Now let's consider the join of a point and an interval, or $X\star Y$ where $X=\{x_0\} $ and $Y=[0,1]$. In this case, $X\times Y\times I$ is a cube $I^2$ of the form $(x_0,i_1,i_2)$ for $i_1,i_2\in I$. The first identification joins all $i_1\in Y$ to $x_0$, collapsing the left side of the square $I^2$ to a point. But the second identification does nothing since identifying since $X$ is just a point (yielding $(x_0,y,1)\sim (x_0,y,1)$) so the join of a point and an interval is a triangle. 
        \item In this case, let $X=\{x_0\} $ and $Y=S^1 $. If $X\times Y\times I$ is the cylinder, the identification $(x_0,y_1,0)\sim(x_0,y_2,0)$ scrunches up the circle to $x_0$, while $(x_0,y,1)\sim(x_1,y,1)$ does nothing, resulting in a hollow top, which is precisely the cone $CX$ of $S^1 $. Furthermore, considering the join of $S^1 $ with two points denoted $S^1 \star S^0$, we get the suspension top as stated earlier. In general, the join of $X$ with a one point space is the cone $CX$, while the join of $X$ with the two point space $S^0$ is the suspension $SX$.
        \item Consider $X,Y$ closed intervals. Then $X\star Y$ becomes a tetrahedron: to see this, note that $X\times Y\times I$ looks like the cube $I^3$. If we take $(x,y_1,0)\sim(x,y_2,0)$, then if $x$ varies along the interval, any $y_1,y_2$ will collapse into $x$, therefore this collapses the leftmost face of the cube into a line that looks like $X$. The same argument holds for $Y$ (with the directions reversed since these are ordered pairs). 
        \item This time, let $X=S^1 $ and $Y$ be a closed interval. $X\times Y\times I$ looks like a thick washer (precisely, a washer with thickness and length 1, with a hole of diameter 2). The identification $(x,y_1,0)\sim(x,y_2,0)$ will collapse the interval onto the circle, ``thinning out'' the left side of the washer until the radius is zero. The identification $(x_1,y,1)\sim(x_2,y,1)$ is a little more confusing but still understandable: it collapses every circle onto a point of the interval. So if you think of the washer as consisting of rings, start with the inner ring, smoosh it to a point, and move all the way up until all that's left is a line. The remaining space is the join $S^1 \star I$.
        \item Finally, consider the join of two circles $S^1 \star S^1 $. This doesn't embed in $\R^3$, but it is still possible to understand what it looks like. $S^1 \times S^1 \times I$ is a torus times an interval: consider on one end of the interval the torus being smashed to one circle, and on the other end the other circle being smashed. Thus the result is a circle continuously deforming ``sideways'' into another circle. From here, you should be getting a pretty good idea of what the join of two spaces is like.
    \end{itemize}
    In general, the join $X\star Y$ contains a copy of $X$ at one end and a copy of $Y$ at the other, and every other point $(x,y,t)\in X\star Y$ is a unique line segment joining a point $x\in X$ to another point $y\in Y$, where the segment is obtained by letting $t$ vary. We can write points of $X\star Y$ as formal linear combinations $t_1x+t_2y$ with $0\leq t_i  \leq 1$ and $t_1+t_2=1$, where $0x+1y=y$ and $1x+0y=x$ corresponding to the identifications defining $X\star Y$. Think of $t_1$ as how far away $x$ is from $X\times \{0\} $, and similarly for $t_2$. They must add to 1 since an interval is of length 1\footnote{I don't have too good intuition on why they have to add up to 1 for anything greater than the $2$-simplex, though...}.

    In the same way, we can construct an interated join $X_1\star \cdots \star X_n $ of formal linear combinations $t_1x_1+\cdots +t_n x_n $ with $0\leq t_i \leq 1$ and $t_1+\cdots +t_n =1$, omitting the terms $0x_i $. A very special case is when each $X_i $ is just a point: as we saw earlier, the join of two points is a line, the join of three (join of line and point) is a triangle, and four a tetrahedron. In general, the join of $n$ points is a convex polyhedron of dimension $n-1$ called a \textbf{simplex}. Formally, given that the $n$ points form a standard basis for $\R^n $ we have their join the $(n-1)$-simplex \[
        \Delta ^{n-1}:=\{(t_1,\cdots ,t_n )\in \R^n  \mid t_1+\cdots +t_n =1 \ \text{and} \ t_i \geq 0\} .
    \] If $X$ and $Y$ are CW complexes, then we have a natural CW structure on $X\star Y$ with $X,Y$ subcomplexes, the remaining cells being the product cells of $X\times Y\times (0,1)$.
\end{namedthing}
\begin{namedthing}{Wedge Sum}
    This is one of the more important operations on spaces. Given spaces $X$ and $Y$ with points $x_0\in X$ and $y_0\in Y$, the \textbf{wedge sum} $X\vee Y$ is the quotient of the disjoint union $X\amalg Y$ with the relation identifying $x_0\sim y_0$ to a single point. For example, $S^1 \vee S^1 $ is the figure eight. Note that $S^1 \vee S^1 \vee S^1 $ is \emph{not} the chain of three cirles, but rather the fidget spinner! In general, we can form the wedge sum $\bigvee_{\alpha }X_{\alpha }$ of spaces $X_{\alpha }$ by identifying the points $x_{\alpha }\in X_{\alpha }$ to a point in $\amalg_{\alpha }X_{\alpha }$, which is why $S^1 \vee S^1 \vee S^1 $ takes on a fidget spinner structure, all the points have to get identified to one. If the $X_{\alpha }$ are cell complex and the points are $0$-cells, then $\bigvee_{\alpha }X_{\alpha }$ is a cell complex obtained from the complex $\amalg_{\alpha }X_{\alpha }$ by quotienting the subcomplex of $0$-cells to a point. 

    For any cell complex $X$, the quotient $X^n  /X^{n-1}$ is a wedge sum of $n$-spheres $\bigvee _{\alpha }S_{\alpha }^n $ with one sphere for each $n$-cell of $X$. This makes sense because turning the $(n-1)$-skeleton into a point leaves the remaining $n$-cells (which look like $n$-spheres) all connected at a single point. This will come in handy later.
\end{namedthing}
\begin{namedthing}{Smash Product}
    To be honest, I haven't ever used this but apparently it comes in handy later. In a product space $X\times Y$ there are copies of $X$ and $Y$ given by $X\times \{y_0\} $ and $\{x_0\} \times Y$ for points $x_0\in X,y_0\in Y$. The copies intersect at the point $(x_0,y_0)$, so we can view the spaces at this intersection as the wedge sum $X\vee Y$ with the intersection point $(x_0,y_0)$. Then the \textbf{smash product} $X\wedge Y$ is the quotient $(X\times Y) / (X\vee Y)$. A way to think about $X\wedge Y$ is a reduced version of the product $X\times Y$, obtained by collapsing the ``overlapping'' parts that ``aren't genuinely products'', the pointed spaces $X$ and $Y$. The example that makes this click is the the smash product $S^1 \wedge  S^1 $. Since $S^1 \times S^1 =\mathbb{T}$ the torus, a wedge $S^1 \vee S^1 $ is just two rings, one on the inner circle and one on the outer. So collapsing the inner circle removes the hole, and the outer brings the extra structure back to the surface, hence $S^1 \wedge S^1 $ is homeomorphic to the $2$-sphere. Another example: for a space $X$, the smash product $X\wedge S^0$ is just $X$, since it collapses the second copy of $X$ to a point in the first copy.

    The smash product $X\wedge Y$ is a complex if $X$ and $Y$ are complexes with $x_0,y_0$ $0$-cells, assuming that $X\times Y$ has the cell-complex topology as opposed to the product topology (in cases where the two are different). To generalize the previous example, $S^m\wedge S^n $ is the $(m+n)$-sphere $S^{m+n}$. To see why this is true, note that $S^m \times S^n $ has four cells, a $0$-cell, an $m$-cell, an $n$-cell, and an $(m+n)$-cell. The wedge sum consists of identifying the $m$-cell and the $n$-cell to the $0$-cell, and collapsing these three cells into the $0$-cell yields a complex with one $0$-cell and one $(m+n)$-cell, which is precisely the $(m+n)$-sphere $S^{m+n}$.
\end{namedthing}
\subsection{Preserving homotopy type of complexes (todo)}
Here are two ways to construct homotopy equivalences. The first is by collapsing spaces. 
\begin{prop}
    If $(X,A)$ is a CW pair consisting of a CW complex $X$ and a contractible subcomplex $A$, then the quotient map $X\to X /A$ is a homotopy equivalence.
\end{prop}
The proof may or may not come. Here are some examples of how this is applied.
\begin{example}[Graphs]
    Suppose $X$ is a finite graph. If two endpoints of any edge of $X$ are distinct, we can collapse this to a point, producing a homotopy equivalent graph with one less edge. We can repeat until all edges of $X$ become loops, then each component of $X$ is either an isolated vertex or a wedge sum of circles. 

    A natural question to ask is whether two graphs, having only one vertex in each component (post contractions) can be homotopy equivalent if they aren't isomorphic graphs? By an exercise, we can reduce this down to connected graph, so this boils down to asking whether $\bigvee_m S^1 $ is homotopy equivalent to $\bigvee_n S^1 $ for $n\neq m$, which isn't true since their fundamental groups are different. We could have also done this by the Euler characteristic, but that only works for graphs, this better version demonstrates the power of the fundamental group.
\end{example}
\begin{example}
    Consider $X$ obtained from $S^2$ by attaching an arc $A$ to two points, say the north and south poles. Let $B$ be the arc on the surface of $S^2$ connecting the poles. Contracting $A$ yields a crescent shape, while contracting $B$ yields $S^2\vee S^1 $. So $S^2 /S^0$ and $S^1 \vee S^2$ are homotopy equivalent.
\end{example}
TODO: torus with meridional disks necklace, attaching and mapping cones.


\subsection{The real projective space $\R \mathrm P^n$}
Credit to Cameron Krulewski at UChicago, who wrote up a paper on $\R \mathrm P^n$ for a Math 132 project, whose notes I am following today.
\orbreak
\begin{definition}[Real projective space]
    We define \textbf{real projective space} (or real projective $n$-space) $\R\mathrm P^n$ by the set of lines that pass through the origin in $\R^{n+1}$. Each line is determined by a nonzero vector in $\R^{n+1}$ unique up to scalar multiplication, and $\R\mathrm P^n $ is topologized as the quotient space of $\R^{n+1}\setminus \{0\} $ under the equivalence relation $v\sim \lambda v$ for scalars $\lambda\neq 0$. We'll make sense of this topologization soon. For $n=1$ this is called the real projective line, and the real projective plane for $n=2$.
\end{definition}
Manifolds are often talked about as subsets of $\R^n$, for example, we often discuss $k$-manifolds embedded in at most $\R^{2k+1}$. For $\R \mathrm P^2$ (the real projective plane), this doesn't embed in $\R^3$, but it does immerse. This won't make sense the higher we go up. A better way to think of abstract manifolds like $\R \mathrm P^n$ is as a \textbf{quotient space} by identifying points of another manifold.
\begin{claim}
    The real projective $n$-space is homeomorphic to an $n$-sphere with antipodal points identified, that is, $\R \mathrm P^n \cong S^n/ (v \sim -v)$.
\end{claim}
Why is this true? Since $\R\mathrm P^n $ is topologized as the quotient of nonzero vectors with $v\sim \lambda v$, consider vectors of length one which gives $S^n .$ Then $v\sim \lambda v$ becomes $v\sim -v$ due to lack of choice. To make sense of this, let's look at the cases. 
\begin{enumerate}[label=(\roman*)]
    \item In the trivial case, let $n=0$. Then $\R \mathrm P^0$, the set of lines through the origin in $\R$, consists of just one line $\{\R\} $, so it's homeomorphic to a singleton. What is $S^0$? It's two singletons, so if you identify them you get your expected result.
    \item Now let's look at $n=1$: we want to show that $\R \mathrm P^1$ is homeomorphic to the circle $S^{1} $. Let's parametrize the lines by their slopes, that is, the angle $\tan \left( \frac{y}{x} \right) $ for any positive pair $(x,y)$ on any given line. We choose $(x,y)$ positive since the lines extend in both directions and looking at both would mean a redundancy. Then these lines hit every angle from $0$ to $\pi$, and the $x$-axis given by $\R\times \{0\} $ has an angle of both $0$ and $\pi$ (identifying the two together). So we get that $\R \mathrm P^1 \cong S^{1} $. How is this homeomorphic to $S^{1} / (v \sim -v)$, as we claimed? Identifying antipodal points gets a semicircle, but the endpoints of the semicircle are also antipodal and get identified, so $S^{1} \cong S^{1} /(v\sim -v)$.
    \item For $n=2$, this immerses in $\R^3$ as a weird surface called \emph{Boy's surface}. To make sense of our claim for $n=2$, we can use the same argument as last time, parametrizing the lines through the origin in $\R^3$. Then these hit the entire sphere, with each line hitting both the northern and southern hemispheres, so we can get rid of one of the hemispheres. Flattening out the remaining hemisphere, we get that this looks like $B(0,1)$, a closed ball in the plane! Then we have to deal with lines through the origin in $\R^2$, which is just $\R\mathrm P^1$. This means that we can build $\R\mathrm P^2$ as $B(0,1)\cup \R\mathrm P^1$, which hints at a CW structure we'll see soon. I've thought about how to visualize this, but there's no explanation I can fit in a simple sentence or two, so I recommend going to Wikipedia and starting from there.
    \item In general, we can see this inductive building process holds. The sphere $S^n $ with antipodal points identified by $v\sim -v$ is the same as a hemisphere $D^n $ with antipodal points of $\partial D^n=S^{n-1} $ identifed, which is just $\R\mathrm P^{n-1}$. So $\R\mathrm P^n $ is obtained from $\R\mathrm P^{n-1}$ by attaching an $n$-cell, with the quotient projection $S^{n-1}\to \R\mathrm P^{n-1}$ as the attaching map. It follows by induction that $\R\mathrm P^n $ has a CW structure $e^0 \cup e^1\cup \cdots \cup e^n $ with one cell $e^i $ in each dimension $i\leq n$. For example, in the previous case with $\R\mathrm P^2=B(0,1)\cup \R\mathrm P^1$, this is the same as attaching a $2$-cell to $\R\mathrm P^1$, which was built by attaching a $1$-cell to a point, which is precisely $\R\mathrm P^0$.
\end{enumerate}
\begin{example}
    We can talk about the infinite union $\R\mathrm P^{\infty}$, since we can build $\R\mathrm P^n $ from $\R\mathrm P^{n-1}$. So $\R\mathrm P^{\infty}=\bigcup_{n=0}^{\infty} \R\mathrm P^n $ becomes a cell complex with one cell in each dimension. This can be viewed as the space of lines through the origin in $\R^{\infty}=\bigcup_{n=0} ^{\infty}\R^n $.
\end{example}
\begin{example}
    We can define \textbf{complex projective} $\mathbf{n}$\textbf{-space} $\C\mathrm P^n $ as the space of complex lines through the origin in $\C^{n+1}$. Once again, $\C\mathrm P^n $ is topologized as the quotient of $\C^{n+1}\setminus \{0\} $ by $v\sim \lambda v$ for $\lambda\neq 0$. Equivalently, this is quotient of $S^{2n+1}\subseteq \C^{n+1}$ with $v\sim \lambda v$ for $|\lambda|=1$. Similarly, we can obtain $\C\mathrm P^n $ as a quotient of $D^{2n}$ under $v\sim \lambda v$ for $v\in \partial D^{2n}=S^{2n-1}$, but I don't think it helps illuminate the example very much so you can read Hatcher if you want to know precisely how this identification works. Just trust me when I say that $\C\mathrm P^n $ is obtained from $\C\mathrm P^{n-1}$ by attaching a $2n$-cell via the quotient map $S^{n_2n-1}\to \C \mathrm P^{n-1}$. So $\C\mathrm P^n $ has a cell complex structure $\C\mathrm P^n=e^0\cup e^2\cup \cdots \cup e^{2n} $ with one cell for each even dimension, and similarly so does $\C\mathrm P^{\infty}$.
\end{example}
\begin{example}[Cellular properties of projective space]
    As stated in the last section, in $\R\mathrm P^n $ a characteristic map for $e^i $ is the quotient map $D^i \to \R\mathrm P^i \subseteq \R\mathrm P^n $ identifying antipodal points of $\partial D^i =S^{i-1}$, and is similarly defined for $\C\mathrm P^n $. We have the subcomplexes $\R\mathrm P^k\subseteq \R\mathrm P^n $ and $\C\mathrm P^k\subseteq \C\mathrm P^n$ for $k\leq n$. These turn out to be the only subcomplexes of $\R\mathrm P^n$ and $\C\mathrm P^n$.
\end{example}
