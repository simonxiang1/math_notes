\section{Covering Spaces}
Today we talk about covering spaces, another central topic in algebraic topology. The notes will follow various texts, including Hatcher, Lee, and May.
\orbreak
\subsection{Some preliminary definitions}
Sometimes we need to know what words mean so we can talk about big concepts. These notes will follow May \S 3. We can talk about the theory of covering spaces on \emph{locally contractible} spaces that are path-connected, that is, spaces with a base of contractible spaces, that is, open sets that are contractible when viewed as a space under the subspace topology. However, to get the full picture, we must talk about \emph{locally path-connected} spaces. 
\begin{definition}[Locally path-connected]
    A space $X$ is \emph{locally path-connected} if for any $x\in X$ and any neighborhood $U$ of $x$, there exists a smaller neighborhood $V$ of $x$, with each of whose points can be connected to $x$ by a path in $U$. We could also say $X$ has a base consisting of open sets that are path-connected (under the subspace topology).
\end{definition}
Note that if $X$ is connected and locally path-connected, then it is path-connected. From now on\footnote{By this, we mean any sections following May.}, we assume that spaces are connected and locally path-connected. Let's look at how May defines covering spaces.
\begin{definition}[Covering space]
    A map $p \colon E \to B$ is a covering (or cover, covering space) if it is onto and if each point $b\in B$ has an open neighborhood $V$ such that each component of $p^{-1}(V)$ is open in $E$ and is mapped homeomorphically onto $V$ by $p$. We say that a path connected open subset $V$ with this property is a fundamental neighborhood of $B$. We call $E$ the total space, $B$ the base space, and $F_b = p ^{-1}(b)$ a fiber of the covering $p$.
\end{definition}
Some notes: in other texts, we have
\begin{itemize}
    \item covering $\rightarrow $ covering map,
    \item $U$ is a fundamental neighborhood $\rightarrow U$ is evenly covered,
    \item total space $\rightarrow $ covering space,
    \item base space $\rightarrow $ ??,
    \item $F_b=p^{-1}(b)$ is a fiber of $p \,\longrightarrow\, F_b$ is the preimage of $b$ (points) in the union of sheets of $\widetilde X$ over $U_b$.
\end{itemize}   
Another definition that will come in handy when classifying covering spaces is the notion of something being semilocally simply-connected, that is, given a ``hole'' (of genus one), we can always find a neighborhood contained in that hole such that the fundamental group induced by the inclusion map is trivial in $\pi_1$ of the entire space.
\begin{definition}[Semilocally simply-connected]
    A space $X$ is \emph{semilocally simply-connected}  if for all $x\in X$, there exists a neighborhood $U_x$ containing $x$ such that the inclusion map $U \hookrightarrow X$ induces the trivial map, that is, $\pi_1(U,x)\to \pi_1(X,x)$ is trivial.
\end{definition}
We'll define this again when we need it, and talk a little more about what it means for a space to be semilocally simply-connected.
\orbreak
This is kind of out of place, but now we'll state Lebesgue's number lemma. It's useful when dealing with compact metric spaces.
\begin{lemma}[Lebesgue's number lemma]
    If a metric space $(X,d)$ is compact and we have an open cover of $X$, then there exists a $\delta>0$ such that every subset of $X$ having a diameter less than $\delta$ is contained in some member of the cover. We say $\delta$ is the \emph{Lebesgue number} of such cover.
\end{lemma}
\begin{proof}
    If the subcover is trivial then any $\delta>0$ will suffice. Otherwise, if $\bigcup_{i \in  I} A_i$ is a finite subcover, then for $i\in I$, define $C_i :=X\setminus A_i$ (note that $C_i$ is nonempty since the subcover is nontrivial). Define a function \[
        f \colon X \to R,\quad x \mapsto \frac{1}{n} \sum_{i=1}^{n} d(x,C_i).
    \] Since $f$ is continuous on a compact set, it obtains a minimum $\delta$. The key thing to note is that every $x$ is in some $A_i$, so by the extreme value theorem $\delta >0$. To show that this $\delta$ is indeed the Lebesgue number of the cover, let $x_0\in Y$, where $\operatorname{diam}(Y)<\delta$, such that $Y\subseteq B(x_0,\delta)$. Since $f(x_0)\geq \delta$, there exists at least one $i$ such that $d(x_0,C_i)\geq \delta$. But then $B(x_0,\delta)\subseteq A_i$, and so $Y\subseteq A_i$.
\end{proof}


\subsection{Covering spaces}
These notes will follow Hatcher \S 1.3.
\vspace{0.25cm}
\newline
We've already seen these briefly when we calculated $\pi_1(S^{1} )$, using the projection $\R\to S^{1} $ of a helix onto a circle. Covering spaces can be used to calculated fundamental groups of other spaces as well, but the connection runs much deeper than this. We can talk about algebraic aspects of the fundamental group through the geometric language of covering spaces, exemplified in one of the main results in this section: a one to one correspondence between connected covering spaces of a space $X$ and subgroups of $\pi_1(X)$ (spoilers, smh). This is really really similar to Galois theory, where we looked at the towers of field extensions and related them to the subgroup lattice of the Galois group of automorphisms\footnote{I actually know this! Thank goodness for an entire semester of algebra to understand an example.}.
\begin{definition}[Covering space]
A \emph{covering space} of a space $X$ is a space $\widetilde X$ together with a map $p \colon \widetilde X \to X$ (we say $p$ is a \emph{covering map}) satisfying the following condition: Each point $x\in X$ has an open neighborhood $U$ in $X$ such that $p^{-1}(U)$ is a union of disjoint open sets in $\widetilde X$, each of which is mapped homeomorphically onto $U$ by $p$. Then we say $U$ is \emph{evenly covered} and the disjoint open sets in $\widetilde X$ that project homeomorphically to $U$ by $p$ are called \emph{sheets} of $\widetilde X$ over $U$.
\end{definition}
If $U$ is connected these sheets are the connected components of $p^{-1}(U)$ so they're uniquely determined by $U$. If $U$ is not connected, however, the decomposition of $U$ into sheets may not be unique. $p ^{-1}(U)$ is allowed to be empty, so $p$ doesn't have to be onto. The number of sheets over $U$ can be given by the cardinality of $p ^{-1}(x)$, given $x\in U$. This number is a constant if $X$ is connected.
\begin{example}
A prototypical example (or way to wrap your head around) this section is the helix embedded in $\R^3$: if you think of it projecting on a circle, then $p^{-1}(U)$ is just $\amalg_{\alpha }U_{\alpha }$, where each $U_{\alpha }$ corresponds to the $U$ of a coil or wind of the helix.
\end{example}
\begin{example}
    Another example is the helicoid surface $S\subseteq \R^3$ given by $(s \cos 2\pi t, s \sin 2\pi t, t)$ for $(s,t)\in (0,\infty)\times \R$. This projects onto $\R^2\setminus \{0\} $ via the map $(x,y,z) \mapsto (x,y)$, and defines a covering space $p \colon S \to \R^2\setminus \{0\} $ since each point of $\R^2\setminus \{0\} $ is contained in an open disk $U$ in $\R^2\setminus \{0\} $ with $p ^{-1}(U)$ consisting of countably many disjoint open disks in $S$ projecting homeomorphically onto $U$. (I can't really see this example...)
\end{example}
\begin{example}
    We also have the map $p \colon S^{1}  \to S^{1} $, $p(z)=z^n$ where we view $z$ as a complex number with $|z|=1$ and $n$ any positive integer\footnote{Something about the order of $z$ I realized when thinking about this example: $z^n$ means $z$ coils around in $S^{1} $ $n$ times. So if $z$ was a fifth root of unity, the covering space would be a circle with five coils projecting onto $S^{1} $. Now what if $z$ has infinite order. Can $z$ even have infinite order? I'm not entirely sure...\\Edit: elements that are irrational multiples of $2\pi$ have infinite order. So does that mean it never winds back to itself? How is this isomorphic to $S^{1} $?}. This projection is as described in the footnote, but intersects itself in $n-1$ points (that one can't really imagine as intersections). To see this without the defect, embed $S^{1} $ in the boundary torus of a solid torus $S^{1} \times D^2$ such that it winds $n$ times monotonically around the $S^{1}$ factor without self-intersections, then restrict the projection $S^{1} \times D^2 \to S^{1} \times \{0\} $ to this embedded circle. What?
\end{example}

We usually restrict our attention to connected covering spaces, as these contain all the interesting examples.

\subsection{The covering spaces of $S^{1} \vee S^{1} $ (todo figures)}
Covering spaces of $S^{1} \vee S^{1} $ form a rich family that demonstrate the general theory very concretely. For convenience, let $X=S^{1} \vee S^{1} $. View it as a graph with one vertex and two edges, with the edges labeled $a$ and $b$. 
\begin{figure}[H]
    \centering
    \incfig[0.3]{s1xs1}
    \caption{The graph of $S^{1} \times S^{1} $.}
    \label{s1xs1}
\end{figure}
Let $\widetilde X$ be any other graph with four edges connected to each point, like $X$ at its singular vertex, and that each edge has been assigned an orientation like the ones assigned to each edge of $X$. That is, for each vertex there are two $a$-edges and $b$-edges oriented toward and away from the vertex. Help I can't include figures that are the proper size! Let's call $\widetilde X$ a $2$-oriented graph.

Given a $2$-oriented graph $\widetilde X$ we can construct a map $p \colon \widetilde X \to X$ that sends all vertices of $\widetilde X$ to the vertex of $X$, and all edges of $\widetilde X$ to the edge of $X$ with the same label. Say $p$ is a homeomorphism on the regions bounded by the edges, and preserves the orientation of the edges. Then $p$ is a covering map. Conversely, every covering space of $X$ is a graph that inherits a $2$-orientation from $X$. It can be shown that every graph with four edges at each vertex can be $2$-oriented: the proof follows from graph theory. We could also generalize this to $n$-oriented graphs, which are covering spaces of the wedge sum of $n$ circles.

How would we generate a simply-connected covering space of $X$? Start with the open intervals $(-1,1)$ in $\R^2$ (one per coordinate axis). Then for a fixed $\lambda,\,0<\lambda < \sfrac{1}{2}$, say $\lambda =\sfrac{1}{3}$, adjoin four open segments of length $2\lambda$ to the ends of the previous segments, and shift each back by a distance of $\lambda$. These new adjoined segments are perpendicular and bisected by the old ones: continue with four more new segments of distance $2\lambda^2$ at a distance $\lambda^2$ to the (now 12) end segments, and so on. Then at the $n$-th iteration, we would be adding open segments of length $\lambda^{2n-1}$ at a distance $\lambda^{n-1}$ from the previous endpoints. Then the union of the segments is a graph (the Cayley graph of the fundamental group of $S^{1} $!), with vertices the intersections, labeling horizontal edges $a$ and orienting them to the right, and vertical edges $b$, orienting them upward.

This covering space is called the \emph{universal cover} of $X$, because it covers every connected covering space of $X$.
\orbreak
\subsection{More on covering spaces}
These notes will follow Lee \S 11.
\vspace{0.25cm}
\newline
The definition of a covering space is the same as Hatcher except: the covering space $\widetilde X$ must be connected. Once again, the only interesting covering spaces are connected ones, and so we eliminate the need to frit fret around about details when introducing new theorems and just make sure covering spaces are connected in the definition.
\begin{example}
    The exponential quotient map $\varepsilon \colon \R \to S^{1} $ given by $x \mapsto e^{2\pi ix}$ is a covering map. Another example: define $E \colon \R^n \to \mathbb{T}^n$ by \[
        E(x_1,\cdots,x_n)=(\varepsilon(x_1),\cdots,\varepsilon(x_n)).
    \] We will show in an exercise that a product of covering maps is a covering map. So $E$ is a covering map.
\end{example}
\begin{example}
    Define a map $\pi \colon S^n \to \R P^n$ (where $n\geq 1$) by sending each point $x$ in the sphere to the line through the origin and $x$, thought of as a point in $\R P^n$. Then $\pi$ is a covering map, and the fiber of each point in $\R P^n$ is a pair of antipodal points $\{x,-x\} $.
\end{example}
\begin{lemma}[Elementary properties of covering maps]
   Every covering map is a local homeomorphism, an open map, and a quotient map. An injective covering map is a homeomorphism.
\end{lemma}
\begin{proof}
    Left as an exercise to the reader.
\end{proof}
\begin{prop}
    For any covering map $p \colon \widetilde X \to X$, the cardinality of each fiber $p ^{-1}(q)$ is the same for any fiber.
\end{prop}
\begin{proof}
    If $U$ is any evenly covered open set in $X$, each sheet in $p ^{-1}(U)$ contains exactly one point of each fiber. Then for any $q,q'\in U$, there are one-to-one correspondences \[
        p ^{-1}(q) \longleftrightarrow \{\text{sheets of}\,p ^{-1}(U)\} \longleftrightarrow p ^{-1}(q'),
    \] which shows that the number of sheets is constant on $U$. It follows that the set of points $q'\in X$ such that $p ^{-1}(q')$ has the same cardinality as $p ^{-1}(q)$ is open. Now let $q\in X$, and let $A$ be the set of points in $X$ whose fibers have the same cardinality as $p ^{-1}(q)$. Then $A$ is open, and $X\setminus A$ is open since it's a union of open sets (one open set for each cardinality not equal to $p ^{-1}(q)$). Since $X$ is connected and nonempty, we have $A=X$.
\end{proof}
If $p \colon \widetilde X \to X$ is a covering map, then the cardinality of any fiber is the \emph{number of sheets} of the covering. For example, the $n$-th power map ($S^{1} \to S^{1} $) is an $n$-sheeted covering, $\pi \colon S^n \to \R P^n$ is a two sheeted covering, and $\varepsilon \colon \R \to S^{1} $ is a countably sheeted covering.

\subsection{Lifting properties}
Here we'll talk about some important lifting properties, that we discussed when we proved that $\pi_1(S^{1} )$ is isomorphic to $\Z$. Recall: if $p \colon \widetilde X \to X$ is a covering map and $\varphi \colon B \to X$ is any continuous map, a \emph{lift} of $\varphi $ is a continuous map $\widetilde\varphi \colon B \to \widetilde X$ such that $p \circ \widetilde\varphi =\varphi $. See the commutative diagram below for reference.
            \begin{figure}[H]
                \centering
\begin{tikzcd}
                                                         & \widetilde X \arrow[d, "p"] \\
B \arrow[r, "\varphi"'] \arrow[ru, "\widetilde \varphi"] & X                          
\end{tikzcd}
            \end{figure}
            \begin{prop}[Unique lifting property]
  Let $p \colon \widetilde X \to X$ be a covering map. Suppose $B$ is connected, $\varphi \colon B \to X$ is continuous, and $\widetilde\varphi_1 \widetilde\varphi_2 \colon B \to \widetilde X$ are lifts of $\varphi $ that agree at some point of $B$. Then $\widetilde \varphi_1\equiv \widetilde \varphi_2$, that is, lifts are unique.
\end{prop}
\begin{proof}
    We show that the set \[
        \mathcal{S}=\{b\in B \mid \widetilde \varphi_1 (b)=\widetilde\varphi_2(b)  \} 
    \] is both open and closed in $B$, contradicting the connectedness of $B$ if $\mathcal{S} $ is a proper nontrivial subset of $B$. We conclude that $\mathcal{S} $ must be all of $B$ since $\widetilde \varphi_1 $ and $\widetilde\varphi_2 $ agree at a point (so $\mathcal{S} $ is nontrivial) and therefore $\widetilde\varphi_1 $ and $\widetilde\varphi_2 $ are unique.

    Let $b\in \mathcal{S} $ and $U\subset X$ be an evenly covered neighborhood of $\varphi (b)$, and let $U_{\alpha }$ be the component of $p ^{-1}$ containing $\widetilde \varphi_1(b)=\widetilde \varphi_2(b)  $. On the neighborhood $V=\widetilde \varphi_1^{-1}(U_{\alpha })\cap \widetilde \varphi_2^{-1}(U_{\alpha }) $ of $b$, we have $\varphi =p\circ \widetilde \varphi_1= p \circ \widetilde \varphi_2  $. Since $p$ is 1-1 on $U_{\alpha }$, this implies $\widetilde \varphi_1=\widetilde \varphi_2  $ on $V$, so $\mathcal{S} $ is open. 

    OTOH, for $b\notin \mathcal{S} $, if $U$ is an evenly covered neighborhood of $\varphi (b)$, there are disjoint components $U_1,U_2$ of $p ^{-1}(U)$ containing $\widetilde \varphi_1(b),\,\widetilde \varphi_2(b)  $ such that $p$ is a homeomorphism from each $U_i$ to $U$. Letting $V=\widetilde \varphi_1^{-1}(U_1)\cap \widetilde \varphi_2^{-1}(U_2)  $, we conclude that $\widetilde \varphi_1\neq \widetilde \varphi_2  $ on $V$, so $\mathcal{S} $ is closed. This proof is much easier to follow if you trace everything out with all the inverse relations on the commutative diagram above.
\end{proof}
\begin{prop}[Path lifting property]
    Let $p \colon \widetilde X \to X$ be a covering map. Suppose $f \colon I \to X$ is any path, and $\widetilde q_0\in \widetilde X$ is any point in the fiber of $p$ over $f(0)$. Then there exists a unique lift $\widetilde f \colon I \to \widetilde X$ of $f$ such that $\widetilde f(0)=\widetilde q_0$. 
\end{prop}
\begin{proof}
    By the Lebesgue number lemma, $n$ can be chosed large enough that $p$ maps each subinterval $[k/n,(k+1)/n]$ into an evenly covered open subset of $X$. Starting with $\widetilde f(0)=\widetilde q_0, \widetilde f$ is defined inductively by choosing an evenly covered neighborhood $U_k$ containing $f[k /n,(k+1)/n]$, a local section\footnote{A \emph{local section} of a continuous map is a continuous right inverse defined on some open subset. This exists here by Lee's Lemma 11.7, which shows the existence of local sections of covering maps.} $\sigma_k \colon U_k \to \widetilde X$ such that $\sigma_k(f(k /n))=\widetilde f(k /n)$, and setting $f=\sigma_k\circ f$ on $[k /n,(k+1)/n]$. Because $p\circ \widetilde f=(p\circ \sigma_k)\circ f=f$, this is indeed a lift, and it is unique by the unique lifting property.
\end{proof}
\begin{prop}[Homotopy lifting property]
    Let $p \colon \widetilde X \to X$ be a covering map. Suppose $f_0,f_1 \colon I \to X$ are path homotopic, and $\widetilde f_0,\widetilde f_1 \colon I \to \widetilde X$ are lifts of $f_0$ and $f_1$ such that $\widetilde f_0(0)=\widetilde f_1(0)$. Then $\widetilde f_0 \sim \widetilde f_1$. 
\end{prop}
\begin{proof}
    If $H \colon f_0 \sim f_1$ is a path homotopy, by the Lebesgue number lemma we can choose $n$ large enough that $H$ maps each square of side $\frac{1}{n}$ into an evenly covered open set. Labeling the squares $S_{ij}=[i /n,(i+1)/n]\times [j /n,(j+1)/n]$, we define a lift $\widetilde H$ of $H$ square by square along the bottom row, then the next row, and so on by induction. On each square $S_{ij}$, set $\widetilde H=\sigma \circ H$, for an appropiate local section $\sigma$ chosen such that the new definition of $\widetilde H$ matches the previous one at the corner point $(i /n,j /n)$. Then since two such definitions agree on a line segment (by restricting $H$ to it), they are equal by the unique lifting property.

    On the left-hand and right-hand edges of $I\times I$, where $s=0$ or $s=1$, $\widetilde H$ is a lift of the constant loop and therefore constant. The restriction $\widetilde H_0$ to the bottom edge where $t=0$ is a lift of $f_0$ starting at $\widetilde f_0(0)$, and therefore is equal to $\widetilde f_0$, similarly $\widetilde H_1=\widetilde f_1$. Therefore $\widetilde H$ is the required path homotopy between $\widetilde f_0$ and $\widetilde f_1$.
\end{proof}

\subsection{Connections to the fundamental group}
Back to Hatcher \S 1.3.
\vspace{0.25cm}
\newline
Here are some applications of the lifting properties with respect to the fundamental group.
\begin{prop}
    The map $p_* \colon  \pi_1(\widetilde X,\widetilde x_0) \to \pi_1(X,x_0)$ induced by a covering space $p \colon (\widetilde X,\widetilde x_0) \to (X,x_0)$ is injective. The image subgroup $p_*(\pi_1(\widetilde X,\widetilde x_0))$ in $\pi_1(X,x_0)$ consists of the homotopy classes of loops in $X$ based at $x_0$ whose lifts to $\widetilde X$ starting at $\widetilde x_0$ are loops.
\end{prop}
\begin{proof}
    An element of the kernel of $p_*$ is represented by a loop $\widetilde f_0 \colon I \to \widetilde X$ with a homotopy $f_t \colon I \to X$ of $f_0=p \widetilde f_0$ to the trivial loop $f_1$. By the homotopy lifting property, there is a lifted homotopy of loops $\widetilde f_t$ starting with $\widetilde f_0$ and ending with a constant loop. Basically, since elements of the kernel start with the same point, and there exist unique lifts to them that are nullhomotopic, we conclude the kernel is trivial and $p_*$ is 1-1.

    For the second part of the proposition, loops at $x_0$ lifting to loops at $\widetilde x_0$ represent elements of the image of $p_* \colon \pi_1(\widetilde X,\widetilde x_0) \to \pi_1(X,x_0)$. Conversely, a loop representing an element of the image $p_*$ is homotopic to a loop having such a lift, and by the homotopy lifting property, this loop must also have such a lift.
\end{proof}
\begin{prop}
    The number of sheets (cardinality of a fiber) of a covering space $p \colon (\widetilde X,\widetilde x_0) \to (X,x_0)$ with $X $ and $\widetilde X$ path-connected equals the index of $p_*(\pi_1(\widetilde X,\widetilde x_0))$ in $\pi_1(X,x_0)$.    
\end{prop}
\begin{proof}
    For a loop $g$ in $X$ based at $x_0$, let $\widetilde g$ be its lift to $\widetilde X$ starting at $\widetilde x_0$. A product $h\cdot g$ with $[h]\in H=p_*(\pi_1(\widetilde X,\widetilde x_0))$ has the lift $\widetilde h \cdot \widetilde g$ ending at the same point as $\widetilde g$ since $\widetilde h$ is a loop ($\widetilde h$ denotes the same lift as $\widetilde g$, just of $h$ instead). All this is saying is that you can lift a product of loops by a product of loops, and we're choosing one loop to the in the image subgroup of $p_*$. Then we can define a function $\Phi$ from cosets $H[g]$ to $p ^{-1}(x_0)$ by sending $H[g]$ to $\widetilde g(1)$. $H[g]$ denotes $h\cdot g$, where $h\in H$, the coset of $g$. If you think about it, these are cosets since we just vary $g$: and so the number of cosets is the index of the subgroup $p_*(\pi_1(\widetilde X,\widetilde x_0))$ in $\pi_1(X,x_0)$. Now we just have to show $\Phi$ is a bijection to complete the proof.

    $\Phi$ is onto by the path-connectedness of $\widetilde X$, since $\widetilde x_0$ can be joined to any point in $p ^{-1}(x_0)$ by a path $\widetilde g$ projecting to a loop $g$ at $x_0$. To show $\Phi$ is 1-1, note that $\Phi(H[g_1])=\Phi(H[g_2])$ implies that $g_1\cdot \overline{g_2}$ lifts to a loop in $\widetilde X$ based at $\widetilde x_0$, so $[g_1][g_2]^{-1}\in H$ and hence $H[g_1]=H[g_2]$.
\end{proof}
\orbreak
Question: for a continuous map $\varphi \colon Y \to X$, does $\varphi $ admit a lift $\widetilde \varphi $ to a covering space $\widetilde X$ of $X$? The lifting criterion can help us out.
\begin{theorem}[Lifting criterion]
    Suppose we are given a covering space $p \colon (\widetilde X,\widetilde x_0) \to (X,x_0)$ and a map $f \colon (Y,y_0) \to (X,x_0)$ with $Y$ path-connected and locally path-connected. Then a lift $\widetilde f \colon (Y,y_0) \to (\widetilde X,\widetilde x_0) $ of $f$ exists if and only if $f_*(\pi_1(Y,y_0))\subseteq p_*(\pi_1(\widetilde X,\widetilde x_0))$.
\end{theorem}
\begin{proof}
If a lift $\widetilde f$ exists, then $p\widetilde f=f$, so $f_*=p_* \widetilde f_*$
\end{proof}

\subsection{Classification of covering spaces (todo split it up)}
How can we catch all the covering spaces? This whole topic deals closely with its analogue in algebra, Galois theory, with a 1-1 correspondence between connected covering spaces of $X$ (towers of field extensions) and subgroups of $\pi_1(X)$ (subgroups of $\operatorname{Gal}(\mathbb{E} /\F)$). This comes from the function that assigns each covering space $p \colon (\widetilde X,\widetilde x_0) \to (X,x_0)$ to the subgroup $p_* (\pi_1(\widetilde X,\widetilde x_0))$ of $\pi_1(X,x_0)$.

\begin{definition}[Semilocally simply-connected]
    A space $X$ is semilocally simply-connected if for all $x\in X$, there exists a neighborhood $U_x$ containing $x$ such that the inclusion map $U \hookrightarrow X$ induces the trivial map, that is, $\pi_1(U,x)\to \pi_1(X,x)$ is trivial.
\end{definition}
Basically, the fundamental group of $U$ is trivial \emph{inside} the fundamental group of $X$, that is, loops in $\pi_1(U,x)$ are nullhomotopic in $X$ (not necessarily $U$, if that were the case, $U$ would be locally simply connected). Intuitively, there are lower bounds on the size of holes (genus-wise): if theres a hole, you can find a neighborhood smaller than it so that loops are still trivial. For example, take the Hawaiian earring: loops here are very very small, and at the base every neighborhood will contain a hole, so it's not semilocally simply-connected (a ``bad'' space)\footnote{Does anyone reading this know of a space that's path-connected but not locally path-connected? I know of many counterexamples for the converse, but without a counterexample to the implication I don't see why local path-connectedness is a necessary condition on top of path-connectedness.}.

\begin{prop}\label{main}
    If $X$ is a path-connected, locally path-connected, and semilocally simply-connected space, then for every subgroup $H$ of $\pi_1(X,x_0)$, there is a covering space $(\widetilde X,\widetilde x_0)\overset{p}{\to }(X,x_0)$ such that $p_*(\pi_1(\widetilde X,\widetilde x_0))=H$.
\end{prop}
We'll prove \cref{main} much later, let's talk about it first. If $H=1$, then $\widetilde X$ is simply-connected, so $\widetilde X$ is the universal cover of $X$. Now given an evenly covered open set $U$, any loop in $U$ will lift to a sheet in $\widetilde X$, which implies it's nullhomotopic in $\widetilde X$, and therefore nullhomotopic in $X$ (we don't know if it's nullhomotopic in $U$), we can see this just by projecting the loop with $p$. This implies that if $U \overset{\iota}{\hookrightarrow }X$ denotes the inclusion of $U$ in $X$, then $\iota_*(\pi_1(U))=1$ in $\pi_1(X)$. So $X$ must be semilocally simply-connected. The following claim shows why we need these claims for $X$ to be a ``nice'' space.
\begin{claim}
    If $X$ is path-connected, locally path-connected, and semilocally simply-connected, then there exists a universal cover of $X.$
\end{claim}
\begin{proof}
    We prove this by directly constructing a universal cover of $X$ through the fundamental groupoid. First assume that $X$ has a universal cover $\widetilde X\overset{p}{\to }X$. Let $\widetilde x_0\in \widetilde X$. Then for some other $\widetilde x\in \widetilde X$, there is a unique path homotopy class of paths from $\widetilde x_0$ to $\widetilde x$. So points in $\widetilde X$ are in a 1-1 correspondence of path homotopy classes of paths starting at $\widetilde x_0$. But by the path lifting property, these are all homotopic.

    Let's turn this around and define the universal cover of $X$ by its path homotopy classes, that is, let \[
        \widetilde X := \{[f]\in \Pi_1(X) \mid f(0)=x_0\},
    \] where $\Pi_1(X)$ denotes the fundamental groupoid of $X$. The covering is given by $p \colon \widetilde X \to X,\, [f]\mapsto f(1)$. We want to define a topology on $\widetilde X$ that makes $p$ continuous and a covering map. To do this, we define a basis $\mathscr{B}$ and check to see if the inverse image of open sets in the \emph{basis} are continuous. Albin, 24 min lecutre 8 unfinished
\end{proof}
\orbreak
Now that we've proved that for every subgroup we have a covering space, the next question is how many covering spaces per subgroup? We have two covering spaces $p_1 \colon (\widetilde X_1,\widetilde x_1)\to (X,x_0) $ and $p_2 \colon (\widetilde X_2,\widetilde x_2) \to (X,x_0)$ are \emph{equivalent} if there is a homeomorphism $f \colon (\widetilde X_1,\widetilde x_1) \to (\widetilde X_2,\widetilde x_2)$ such that $p_1=p_2\circ f$, or such that the following diagram commutes:
            \begin{figure}[H]
                \centering
                \begin{tikzcd}
{(\widetilde X_1,\widetilde x_1)} \arrow[rr, "f"] \arrow[rdd, "p_1"'] &           & {(\widetilde X_2, \widetilde x_2)} \arrow[ldd, "p_2"] \\
                                                                      &           &                                                       \\
                                                                      & {(X,x_0)} &                                                      
\end{tikzcd}
            \end{figure}

If so, it's easy to see that this is an equivalence relation.
\begin{theorem}
    The covering spaces $(\widetilde X_i,\widetilde x_i) \overset{p_i}{\to } (X,x_0)$, where $i \in \{1,2\} $ and $X$ is path-connected, locally path-connected are equivalent if and only if $p_{1_*}(\pi_1(\widetilde X_1,\widetilde x_1))=p_{2_*}(\pi_1(\widetilde X_2,\widetilde x_2))$.
\end{theorem}
So it turns out the answer to the question above is just one.
\begin{proof}
    One direction is easy: look at the diagram of induced fundamental groups, and notice that the homeomorphism induces an isomorphism on the subgroups of $\pi_1(X)$. The other direction is more interesting. Let $H_1=p_{1_*}(\pi_1(\widetilde X_1,\widetilde x_1))$ and $H_2=p_{2_*}(\pi_1(\widetilde X_2,\widetilde x_2))$. Since $H_1 \subseteq H_2$ and $p_2$ is a covering map and $X$ is path-connected and locally path-connected, there exists a lift of $\widetilde p_1$ to a map $f\colon(\widetilde X_1,\widetilde x_1) \to (\widetilde X_2,\widetilde x_2)$ by the lifting criterion, making the diagram commute. Similarly, $H_2\subseteq H_1$, so there's a lift of $p_2$ to a map $g \colon (\widetilde X_2,\widetilde x_2) \to (\widetilde X_1,\widetilde x_1)$ making the appropiate diagram commute. In particular, we have 
            \begin{figure}[H]
                \centering
                \begin{tikzcd}
{(\widetilde X_1,\widetilde x_1)} \arrow[rr, "g\circ f"] \arrow[rdd, "p_1"'] &           & {(\widetilde X_1, \widetilde x_1)} \arrow[ldd, "p_1"] \\
                                                                      &           &                                                       \\
                                                                      & {(X,x_0)} &                                                      
\end{tikzcd}
            \end{figure}
            Since the identity is also a lift of $p_1$ to $(\widetilde X_1,\widetilde x_1)$, by uniqueness of lifts we have $g\circ f$ equal to the identity, that is, $g\circ f= \operatorname{id}_{\widetilde X_1}$. Similarly, we have $f\circ g = \operatorname{id}_{\widetilde X_2}$. So $f$ is a homeomorphism. \footnote{I don't understand where $f$ came from: how can we guarantee its existence?}
\end{proof}
Now for the theorem we all came here for.
\begin{theorem}\label{galois}
    Let $X$ be a path-connected, locally path-connected, and semilocally simply-connected space. Then there is a bijection between the coverings $(\widetilde X,\widetilde x_0) \overset{p}{\to }(X,x_0)$ up to equivalence and the subgroups of $\pi_1(X,x_0)$. This bijection is given by $p \mapsto p_*(\pi_1(\widetilde X,\widetilde x_0))$. Furthermore, we also have a 1-1 correspondence between the non pointed covering spaces $\widetilde X \overset{p}{\to }X$ and the conjugacy classes of subgroups, given by the same map  $p\mapsto [p_*(\pi_1(\widetilde X))]$.
\end{theorem}
It's important that we have a covering space and choice of basepoint: if we change the basepoint, we might not necessarily give the same group. Changing the basepoint gives a conjugacy isomorphisms between fundamental groups. This conjugacy isomorphism might give rise to different subgroups, conjugating by some element of the group possibly gives a different subgroup. Hence the second part of the theorem.

It turns out there's an equivalence of posets between covers of a space $(X,x_0)$ (for $X$ a ``nice'' space) and subgroups of $\pi_1(X,x_0)$, known as the Galois correspondence. The partial order is given by defining two elements to be comparable if one is a cover of another.
\subsection{Actions on the fibers}
If $p \colon \widetilde X \to X$ a cover, $\alpha \in \pi_1(X,x_0)$, define $L_{\alpha }\in \operatorname{Sym}(p^{-1}(x_0))$ by $L_{\alpha}\widetilde x=\widetilde \alpha (0),$ where $\widetilde \alpha $ is the lift of $x$ to a path ending at $\widetilde x$. We have $L_{\alpha \beta }=L_{\alpha }\circ L_{\beta }$, since $L_{\alpha \beta }(\widetilde x)=\widetilde{\alpha \beta }(0)=L_{\alpha }(\widetilde \beta(0))=L_{\alpha }(L_{\beta }(\widetilde x)).  $ This is why we defined $L_{\alpha }(\widetilde x)$ starting at the left endpoint $0$. Albin lecture 9, 36 minutes

