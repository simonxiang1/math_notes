\section{Common Topological Structures}
We'll take this section to digress a little bit and explore some examples of our favorite spaces that we work with a lot in topology.
\subsection{Manifolds (todo)}
\subsection{Cell complexes (todo)}
The big idea is this: we can build a lot of our favorite topological spaces by starting with ``removable parts'' from each dimension, then glueing them together via something called a ``boundary map''. Naturally, we start from dimension zero (points), add cells from the $1$st dimension (arcs), then $2$-cells (surfaces), and so on. 

For example, we can construct the torus $\mathbb{T}=S^1 \times S^1 $ by identifying opposite edges of a square. In general, an orientable (worry about what this means later) surface $M_g$ of genus $g$ can be constructed from a polygon with $4g$ sides by identifying pairs of edges. The $4g $ edges of the polygon becomes a union of $2g$ circles intersecting at a point. The interior of the polygon can be thought of as a $\mathbf 2$\textbf{-cell}, and the union of circles being obtained by attaching $2g$ open arcs, or $\mathbf 1$\textbf{-cells}. 

Here is a natural way to generalize the construction of such a space:
\begin{enumerate}[label=(\arabic*)]
    \item Start with a discrete set of points $X^0$, or $0$-cells.
    \item Inductively, form the $\mathbf n$\textbf{-skeleton} $X^n $ from $X^{n-1}$ by attaching $n$-cells $e_{\alpha }^n $ via maps $\varphi _{\alpha }\colon S^{n-1} \to X^{n-1}$. Note! These maps are supposed to be hard to define! So $X^n $ is the quotient spae of the disjoint union $X^{n-1}\amalg _{\alpha }D_{\alpha }^n $ of $X^{n-1}$ with a collection of $n$-disks $D_{^n }$ under the identifications $x\sim \varphi _{\alpha }(x)$ for $x\in \partial D_{\alpha }^n $. Thus as a set, $X^n =X^{n-1}\amalg_{\alpha }e_{\alpha }^n $ where each $e_{\alpha }^n $ is an open $n$-disk.
    \item We can either stop this process at a finite step with $X=X^n $ for some $n<\infty$, or continue on forever, with $X=\bigcup_{n} X^n $. In the infinite-dimensional case, $X$ is given the weak topology: A set $A\subseteq X$ is open (or closed) iff $A\cap X^n $ is open (or closed) in $X^n $ for each $n$.
\end{enumerate}
If $X=X^n $ for some $n$, then $X$ is said to be finite-dimensional, and the smallest such $n$ is the \textbf{dimension} of $X$, or the maximum number of cells of $X$.
\subsection{The real projective plane $\R \mathrm \mathrm P^n$ (todo)}
Credit to Cameron Krulewski at UChicago, who wrote up a paper on $\R \mathrm P^n$ for a Math 132 project, whose notes I am following today.
\orbreak
Manifolds are often talked about as subsets of $\R^n$, for example, we often discuss $k$-manifolds embedded in at most $\R^{2k+1}$. What is the real projective $n$-space $\R \mathrm P^n$ exactly? It's the space of lines through the origin in $\R^{n+1}$. For $\R \mathrm P^2$ (the real projective plane), this doesn't embed in $\R^3$, but it does immerse. This won't make sense the higher we go up. A better way to think of abstract manifolds like $\R \mathrm P^n$ is as a \textbf{quotient space} by identifying points of another manifold.
\begin{claim}
    The real projective $n$-space is homeomorphic to an $n$-sphere with antipodal points identified, that is, $\R \mathrm P^n \cong S^n/ (v \sim -v)$.
\end{claim}
Why is this true? Let's look at the cases. In the trivial case, let $n=0$. Then $\R \mathrm P^0$ consists of just one line $\{\R\} $, so it's hoemomorphic to a singleton. What is $S^0$? It's two singletons, so if you identify them you get your expected result.

Now let's look at $n=1$: we want to show that $\R \mathrm P^1$ is homeomorphic to the circle $S^{1} $. Let's parametrize the lines by their slopes, that is, the angle $\tan \left( \frac{y}{x} \right) $ for any positive pair $(x,y)$ on any given line. We choose $(x,y)$ positive since the lines extend in both directions and looking at both would mean a redundancy. Then these lines hit every angle from $0$ to $\pi$, and the $x$-axis given by $\R\times \{0\} $ has an angle of both $0$ and $\pi$ (identifying the two together). So we get that $\R \mathrm P^1 \cong S^{1} $. How is this homeomorphic to $S^{1} / (v \sim -v)$, as we claimed? Identifying antipodal points gets a semicircle, but the endpoints of the semicircle are also antipodal and get identified, so suprisingly $S^{1} \cong S^{1} /(v\sim -v)$.





