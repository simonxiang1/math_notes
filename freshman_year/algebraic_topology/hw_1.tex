\section{August 29, 2020: Homework 1}
\textbf{Hatcher Chapter 0 (p.\ 18):} 1, 3ab, 17,\\
\textbf{Hatcher Section 1.1 (p.\ 38):} 3, 6, 7, 16.
\subsection{Question 1}
\begin{problem}
    Suppose $X, Y$ are compact Hausdorff spaces and $ f \colon X \to Y$ is continuous and onto. Define $\sim$ as the equivalence relation on $X$ given by $x_1\sim x_2$ if and only if $f(x_1)=f(x_2)$.
   \begin{enumerate}
       \item[(a)] Prove the quotient space $X/ \sim$ is Hausdorff.
       \item[(b)] Use this to show that the induced map $X/\sim \,\to Y$ is a homeomorphism.
       \item[(c)] Show that identifying the ends of the interval gives $S^{1}$.
       \item[(d)] Give a cooler example.
   \end{enumerate} 
\end{problem}
\begin{solution}
    We do this by
\begin{enumerate}
    \item[(a)] Let $[a],[b]$ be elements (equivalence classes) of the quotient space $X/\sim$. We want to separate $[a]$ and $[b]$ by open sets: since t

        (note $X$ is $T_4$ and you can also separate compact sets). Canonical projection map is a quotient map: therefore it maps closed sents onto closed (and preimage of closed is also closed)
\end{enumerate}
\end{solution}


\subsection{Problem 1 Chapter 0}
\begin{prob}
   Construct an explicit deformation retraction of the torus with one point deleted onto a graph consisting of two circles intersecting in a point, namely, longitude and meridian circles of the torus. 
\end{prob}

\subsection{Problem 3a}
\begin{prob}
    Show that the composition of homotopy equivalences $X \to Y$ and $Y \to Z$ is a homotopy equivalence $X \to Z$. Deduce that homotopy equivalence is an equivalence relation.
\end{prob}

\subsection{Problem 3b}
\begin{prob}
    Show that the relation of homotopy among maps $X \to Y$ is an equivalence relation.
\end{prob}

\subsection{Problem 17a}
\begin{prob}
    Show that the mapping cylinder of every map $ f \colon S^{1} \to S^{1}$ is a CW complex.
\end{prob}

\subsection{Problem 17b}
\begin{prob}
    Construct a $2$-dimensional CW complex that contains both an annulus $S^{1}\times I$ and a M\"obius band as deformation retracts.
\end{prob}

\subsection{Problem 3 Section 1.1}
\begin{prob}
    For a path-connected space $X$, show that $\pi\left( X \right) $ is abelian if and only if all basepoint-change homeomorphisms $\beta_h$ depend only on the endpoints of the path $h$.
\end{prob}

\subsection{Problem 6}
\begin{prob}
    We can regard $\pi_1\left( X,x_0 \right) $ as the set of basepoint-preserving homotopy classes of maps $\left( S^{1},s_0 \right) \to \left( X,x_0 \right) $. Let $\left[ S^{1},X \right] $ be the set of homotopy classes of maps $S^{1}\to X$, with no conditions on basepoints. Thus there is a natural map $\Phi : \pi_1\left( X,x_0 \right) \to \left[ S^{1},X \right] $ obtained by ignoring basepoints. Show that $\Phi$ is onto if $X$ is path-connected, and that $\Phi([f]) = \Phi([g])$ if and only if $[f]$ and $[g]$ are conjugate in $\pi_1\left( X,x_0 \right) $. Hence $\Phi$ induces a one-to-one correspondence between $\left[ S^{1},X \right] $ and the set of conjugacy classes in $\pi_1\left( X \right) $, when $X$ is path-connected. 
\end{prob}

\subsection{Problem 7}
\begin{prob}
    Define $ f \colon S^{1}\times I\to S^{1}\times I$ by $f(\theta, s)=(\theta + 2\pi s,s)$, so $f$ restricts to the identity on the two boundary circles of $S^{1}\times I$. Show that $f$ is homotopic to the identity by a homotopy $f_t$ that is stationary on both boundary circles. [Consider what $f$ does to the map $s \mapsto (\theta_0,s)$ for fixed $\theta_0\in S^{1}$. 
\end{prob}

\subsection{Problem 17}
\begin{prob}
    Construct infinitely many nonhomotopic retractions $S^{1} \vee S^{1} \to S^{1} $.
\end{prob}




