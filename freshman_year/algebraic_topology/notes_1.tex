\section{Category Theory}
Today we talk about abstract nonsense! These notes will follow Evan Chen's Napkin \S 60 and May's ``A Concise Course in Algebraic Topology'' \S 2.

\subsection{Motivation}
Why do we talk about categories? Categories rise from objects (sets, groups, topologies) and maps between them (bijections, isomorphisms, homeomorphisms). Algebraic topology speaks of maps from topologies to groups, which makes maps between categories a suitable tool for us.
\begin{example}
    Here are some examples of morphisms between objects:
    \begin{itemize}
        \item A bijective homomorphism between two groups $G$ and $H$ is an isomorphism. What also works is two group homomorphisms $\phi \colon G \to H$ and $\psi \colon H \to G$ which are mutual inverses, that is $\phi \circ \psi = \operatorname{id}_H$ and $\psi \circ \phi=\operatorname{id}_G$.
        \item Metric (or topological) spaces $X$ and  $Y$ are isomorphic if there exists a continuous bijection $f \colon  X \to Y$ such that $f^{-1}$ is also continuous.
        \item Vector spaces $V$ and $W$ are isomorphic if there is a bijection $T \colon V \to W$ that's a linear map (aka, $T$ and $T^{-1}$ are linear maps).
        \item Rings $R$ and $S$ are isomorphic if there is a bijective ring homomorphism $\phi$ (or two mutually inverse ring homomorphism).
    \end{itemize}
\end{example}

\subsection{Categories}
\begin{definition}[Category]
    A \emph{category} $\mathcal{A}$ consists of 
    \begin{itemize}
        \item A class of \emph{objects} , denoted $\operatorname{obj}(\mathcal{A}).$ 
        \item For any two objects $A_1,A_2\in \operatorname{obj}(\mathcal{A})$, a class of \emph{arrows} (also called  \emph{morphisms} or \emph{maps} between them). Let's denote the set of arrows by $\operatorname{Hom}_{\mathcal{A}}(A_1,A_2)$.
        \item For any $A_1,A_2,A_3\in \operatorname{obj}(\mathcal{A})$, if $f \colon A_1 \to A_2$ is an arrow and $g \colon A_2 \to A_3$ is an arrow, we can compose the two arrows to get $h=g\circ f \colon A_1 \to A_3$ an arrow, represented in the \emph{commutative diagram} below: 
            \begin{figure}[H]
                \centering
    \begin{tikzcd}
A_1 \arrow[rd, "h", dotted] \arrow[r, "f"] & A_2 \arrow[d, "g"] \\
                                           & A_3               
\end{tikzcd}
            \end{figure}
        The composition operation can be denoted as a function \[
            \circ \colon \operatorname{Hom}_{\mathcal{A}}(A_2,A_3)\times \operatorname{Hom}_{\mathcal{A}}(A_1,A_2) \to \operatorname{Hom}_{\mathcal{A}}(A_1,A_3)
        \] for any three objects $A_1,A_2,A_3$. Composition must be associative, that is, $h\circ(g\circ f)=(h\circ g)\circ f.$ In the diagram above, we say $h$ \emph{factors} through $A_2$.
    \item Every object $A\in \operatorname{obj}_{\mathcal{A}}$ has a special \emph{identity arrow} $ \operatorname{id}_{\mathcal{A}}$. The identity arrow has the expected properties $\operatorname{id}_{\mathcal{A}}\circ f=f$ and $f\circ \operatorname{id}_{\mathcal{A}}=f$.
    \end{itemize}
\end{definition}
\begin{note}
    We can't use the word ``set'' to describe the class of objects because of some weird logic thing (there is no set of all sets). But you can think of a class as a set.
\end{note}
From now on, $A\in \mathcal{A}$ is the same as $A\in \operatorname{obj}(\mathcal{A})$. A category is \emph{small} if it has a set of objects, and \emph{locally small} if $\operatorname{Hom}_{\mathcal{A}}(A_1,A_2)$ is a set for any $A_1,A_2\in \mathcal{A}$.

\begin{example}[Basic Categories]
    Here are some basic examples of categories:
    \begin{itemize}
        \item We have the category of groups $\mathsf{Grp}.$ 
            \begin{itemize}
                \item The objects of $\mathsf{Grp}$ are groups.
                \item The arrows of $\mathsf{Grp}$ are group homomorphisms.
                \item The composition of $\mathsf{Grp}$ is function composition.
            \end{itemize}
        \item Describe the category $\mathsf{CRing}$ (of commutative rings) in a similar way.
        \item Consider the category $\mathsf{Top}$ of topological spaces, whose arrows are continuous maps between spaces.
        \item Also consider the category $\mathsf{Top}_{*}$ of topological spaces with a distinguished basepoint, that is, a pair $(X,x_0), \, x_0\in X$. Arrows are continuous maps $f \colon X \to Y$ with $f(x_0)=y_0$.
        \item Similarly, the category of (possibly infinite-dimensional) vector spaces over a field $k$ $\mathsf{Vect}_{k}$ has linear maps for arrows. There is even a category $\mathsf{FDVect}_{k}$ of finite-dimensional vector spaces.
        \item Finally, we have a category $\mathsf{Set}$ of sets, arrows denote any map between sets.
    \end{itemize}
\end{example}
\subsection{Functors}
Motivation: maps between categories, objects rising from other objects. 
\begin{example}[Basic Functors]
   Here are some basic examples of functors:
   \begin{itemize}
       \item Given an algebraic structure (group, field, vector space) we can take its underlying set $S$: this is a functor from $\mathsf{Grp}\to \mathsf{Set}$ (or whatever you want to start with).
       \item If we have a set $S$, if we consider the vector space with basis $S$ we get a functor $\mathsf{Set} \to \mathsf{Vect}$.
       \item Taking the power set of a set $S$ gives a functor $\mathsf{Set}\to \mathsf{Set}$.
       \item Given a locally small category $\mathcal{A}$, we can take a pair of objects $(A_1,A_2)$ and obtain a set $\operatorname{Hom}_{\mathcal{A}}(A_1,A_2)$. This turns out to be a functor $\mathcal{A}^{\text{op}}\times \mathcal{A}\to \mathsf{Set}$.
   \end{itemize}
   Finally, the most important example (WRT this course):
   \begin{itemize}
       \item In algebraic topology, we build groups like $H_1(X),\, \pi_1(X)$ associated to topological spaces. All these group constructions are functors $\mathsf{Top} \to \mathsf{Grp}$.
   \end{itemize}
\end{example}
\begin{definition}[Functors]
    Let $\mathcal{A}$ and $\mathcal{B}$ be categories. A \emph{functor} $F$ takes every object of $\mathcal{A}$ to an object of $\mathcal{B}$. In addition, it must take every arrow $A_1 \overset{f}{\to }A_2$ to an arrow $F(A_1)\overset{F(f)}{\longrightarrow }F(A_2).$ Refer to the commutative diagram:
            \begin{figure}[H]
                \centering
                \begin{tikzcd}
               & A_1 \arrow[dd, "f"']       &  & B_1=F(A_1) \arrow[dd, "F(f)"] &                \\
\mathcal{A}\ni & {} \arrow[rr, "F", dotted] &  & {}                            & \in\mathcal{B} \\
               & A_2                        &  & B_2=F(A_2)                    &               
\end{tikzcd}
            \end{figure}
        Functors also satisfy the following requirements:
        \begin{itemize}
            \item Identity arrows get sent to identity arrows, that is, for each identity arrow $\operatorname{id}_A$, we have $F(\operatorname{id}_A)=\operatorname{id}_{F(A)}$.
            \item Functors respect composition: if $A_1\overset{f}{\to }A_2\overset{f}{\to }A_3$ are arrows in $\mathcal{A}$, then $F(g\circ f)=F(g)\circ F(f)$.
        \end{itemize}
\end{definition}
More precisely, these are covariant functors. A contravariant functor $F$ reverses the direction of arrows, so that $F$ sends $f \colon A_1 \to A_2$ to $F(f) \colon F(A_2) \to F(A_1)$, and satisfies $F(g\circ f)=F(f)\circ F(g)$ instead. A category $\mathcal{A}$ has an opposite category $\mathcal{A}^{\text{op}}$ with the same objects and with $\mathcal{A}^{\text{op}}(A_1,A_2)=\mathcal{A}(A_2,A_1)$. A contravariant functor $F \colon \mathcal{A} \to \mathcal{B}$ is just a covariant functor $\mathcal{A}^{\text{op}}\to \mathcal{B}$. 

\begin{example}
    We have already talked about \emph{free} and \emph{forgetful} functors in Example 1.3: the forgetful functors are functors from spaces to sets (the underlying set of a group) and free functors are from sets to spaces (the basis set forming a vector space).
    \begin{itemize}
        \item Another example of a forgetful functor is a functor $\mathsf{CRing}\to \mathsf{Grp}$ by sending a ring $R$ to its abelian group $(R,+)$.
        \item Another example of a free functor is a functor $\mathsf{Set}\to \mathsf{Grp}$ by taking the free group generated by a set $S$ (who would have known this is free?)
    \end{itemize}
\end{example}
Here is a cool example: functors preserve isomorphism. If two groups are isomorphic, then they must have the same cardinality. In the language of category theory, this can be expressed as such: if $G\cong H$ in $\mathsf{Grp}$ and $U \colon \mathsf{Grp} \to \mathsf{Set}$ is the forgetful functor, then $U(G)\cong U(H)$. We can generalize this to \emph{any} functor and category!
\begin{theorem}
    If $A_1\cong A_2$ are isomorphic objects in $\mathcal{A}$ and $F \colon \mathcal{A} \to \mathcal{B}$ is a functor then \[
        F(A_1)\cong F(A_2).
    \] 
\end{theorem}
\begin{proof}
Let's go diagram chasing!
            \begin{figure}[H]
                \centering
                \begin{tikzcd}
               & A_1 \arrow[dd, "f"', shift right=2]      &  & B_1=F(A_1) \arrow[dd, "F(f)"']                &                \\
\mathcal{A}\ni & {} \arrow[rr, "F", dotted, shift left=3] &  & {}                                            & \in\mathcal{B} \\
               & A_2 \arrow[uu, "g"']                     &  & B_2=F(A_2) \arrow[uu, "F(g)"', shift right=2] &               
\end{tikzcd}
            \end{figure}
The main idea of the proof follows from the fact that functors preserve composition and the identity map. 
\end{proof}
This is very very useful for us (people who are doing algebraic topology) because functors will preserve isomorphism between spaces (we get that homotopic spaces have isomorphic fundamental groups).
\begin{note}
    As a meme (or not really, but it's still funny), we can construct the category $\mathsf{Cat}$ whose objects are categories and arrows are functors.
\end{note}
\subsection{Natural Transformations}
We talked about maps between objects which led to categories, and then maps between categories which lead to functors. Now let's talk about maps between functors, the natural transformation: this is actually not too strange (recall the homotopy, a ``deformation'' from a map to another map).

In this case, we also want to pull a map (functor) $F$ to another map $G$ by composing a bunch of arrows in the target space $\mathcal{B}$.
\newpage
\begin{definition}[Natural Transformations]
    Let $F,G \colon \mathcal{A} \to \mathcal{B}$ be two functors. A \emph{natural transformation} $\alpha \colon F \to G$ denoted
            \begin{figure}[H]
                \centering
    \begin{tikzcd}
        \mathcal{A} \arrow[rr, "F", bend left] \arrow[rr, "G"', bend right] & \,\,~ \Big\Downarrow\alpha & \mathcal{B}
\end{tikzcd}
            \end{figure}
            consists of, for each $A\in \mathcal{A}$ an arrow $\alpha_A \in \operatorname{Hom}_{\mathcal{B}}(F(A),G(A))$, which is called the component of $\alpha$ at $A$. 
            Pictorially, it looks like this: 
            \begin{figure}[H]
                \centering
                \begin{tikzcd}
                                                                   &  & F(A)\in\mathcal{B} \arrow[dd, "\alpha_A"] \\
\mathcal{A}\ni A \arrow[rru, "F", dotted] \arrow[rrd, "G", dotted] &  &                                           \\
                                                                   &  & G(A)\in\mathcal{B}                       
\end{tikzcd}
            \end{figure}
            The $\alpha_A$ are subject to the ``naturality'' requirement such that for any $A_1 \overset{f}{\to }A_2$, the following diagram commutes:
            \begin{figure}[H]
                \centering
                \begin{tikzcd}
F(A_1) \arrow[r, "F(f)"] \arrow[d, "\alpha_{A_1}"'] & F(A_2) \arrow[d, "\alpha_{A_2}"] \\
G(A_1) \arrow[r, "G(f)"']                           & G(A_2)                          
\end{tikzcd}
            \end{figure}
            The arrow $\alpha_A$ represents the path that $F(A)$ takes to get to $G(A)$ (like in a homotopy from $f$ to $g$ the point $f(t)$ gets deformed to the point $g(t)$ continuously). Think of $f$ representing the homotopy and the basepoints being $F(A_1),G(A_1)$ to $F(A_2),G(A_2)$.
\end{definition}
Natural transformations can be composed. Take two natural transformations $\alpha \colon F \to G$ and $\beta \colon G \to H$. Consider the following commutative diagram: 
            \begin{figure}[H]
                \centering
                \begin{tikzcd}
                                                                                                     & F(A) \arrow[d, "\alpha_A"] \\
\mathcal{A}\ni A \arrow[ru, "F", dotted] \arrow[r, "G" description, dotted] \arrow[rd, "H"', dotted] & G(A) \arrow[d, "\beta_A"]  \\
                                                                                                     & H(A)                      
\end{tikzcd}
            \end{figure}
    We can also construct inverses: suppose $\alpha$ is a natural transformation such that $\alpha_A$ is an isomorphism for each $A$. Then we construct an inverse arrow $\beta_A$ in the following way:
            \begin{figure}[H]
                \centering
                \begin{tikzcd}
                                                                   &  & F(A)\in\mathcal{B} \arrow[dd, "\alpha_A"', shift right] \\
\mathcal{A}\ni A \arrow[rru, "F", dotted] \arrow[rrd, "G", dotted] &  &                                                         \\
                                                                   &  & G(A)\in\mathcal{B} \arrow[uu, "\beta_A"', shift right] 
\end{tikzcd}
            \end{figure}
            We say $\alpha$ is a \emph{natural isomorphism}. Then $F(A)\cong G(A)$ \emph{naturally} in $A$ (and $\beta$ is an isomorphism too!) We write $F\cong G$ to show that the functors are naturally isomorphic.
\begin{example}
    If $F \colon \mathsf{Set} \to \mathsf{Grp}$ is the free functor that sends a set to the free group on such set and $U \colon \mathsf{Grp} \to \mathsf{Set}$ is the forgetful functor sending a free group to its generating set, then we have a natural inclusion of $S\hookrightarrow UF(S)$. The functors $F$ and $U$ are left and right adjoint to each other, in the sense that we have a natural isomorphism \[
        \mathsf{Grp}(F(S),A)\cong \mathsf{Set}(S,U(A))
    \] for a set $S$ and an abelian group $A$. This expresses the ``universal property'' of free objects: a map of sets $S\to U(A)$ extends uniquely to a homomorphism of groups $F(S)\to A$. 
\end{example}
\begin{definition}
    Two categories $\mathcal{A}$ and $\mathcal{B}$ are equivalent if there are functors $F \colon \mathcal{A} \to \mathcal{B}$ and $G \colon \mathcal{B} \to \mathcal{A}$ and natural isomorphisms $FG \to \operatorname{Id}$ and $GF \to \operatorname{Id}$, where the $\operatorname{Id}$ are the respective identity functors.
\end{definition}
\subsection{Homotopy Categories and Homotopy Equivalence}
Todo

