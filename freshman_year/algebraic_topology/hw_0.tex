\section*{Homework 0}
\begin{problem}
    Prove that the finite product of manifolds is a manifold.
\end{problem}
\begin{proof}
    We prove $M \times N$ is a manifold, where $M$ is an $m$-manifold and $N$ is an $n$-manifold, which is a sufficient condition for the finite product 
\[
    \prod_{i=1}^n M_{i}
\]
to be a manifold for $M_i$ a $m_i$-manifold, $m_i \in \N$. First, note that the product of two $T_2$ spaces is $T_2$. Take $\tau_1, \tau_2$ to be topological spaces, and let $X$ be their product. We have two distinct points $(a,b), (c,d)$ in $X$, which we can separate by open sets $X_1 \times U_2, \, X_1 \times V_2 \in X$ for $X_1 \in \tau_1, \, U_2, V_2 \in \tau_2$ if $a = c$ (which implies $b \neq d$), and  $U_1 \times X_2, \, V_1 \times X_2$ for $U_1, V_1 \in \tau_1, \, X_2 \in \tau_2$ if $a \neq c.$ 

Now let $(a,b)$ be in $X$, where $a$ is in $\tau_1$ and $b$ is in $\tau_2.$ Then there exist $U_1 \in \tau_1, \, U_2 \in \tau_2$ such that $a \in U_1, \, b\in U_2,$ and $U_1$ homeomorphic to $\mathbb{R}^m$, $U_2$ homeomorphic to $\mathbb{R}^n.$ Simply take the open set $U_1 \times U_2 \in X$ containing the point $(a,b)$, and define the homeomorphism $f: M \times N \to \mathbb{R}^m \times \mathbb{R}^n$ by  $f(x,y)=(g(x),h(y)),$ where $g$ and $h$ are the homeomorphisms of $U_1$ onto $\mathbb{R}^m$ and $U_2$ onto $\mathbb{R}^n$ respectively. Clearly $f$ is continuous since $g$ and $h$ are continuous, and by the same logic $f^{-1}$ exists and is continuous, and is given by $f^{-1}(x,y) = (g^{-1}(x),h^{-1}(y))$ (whose components are continuous since $g$ and $h$ are homeomorphisms).

Finally, we have $\mathbb{R}^m \times \mathbb{R}^n$ homeomorphic to $\mathbb{R}^{m+n}=\mathbb{R}^{n+m}$, so we conclude the product manifold $M \times N$ is indeed an $n+m$-manifold.
\end{proof}
\vspace{10mm}

\begin{problem}
    Prove that a manifold is connected if and only if it is path-connected.
\end{problem}

\begin{proof}
    First, note that every path-connected space is connected. By way of contradiction, assume that a path-connected topological space $(X, \tau)$ is not connected, that is, there exist $U, V \in \tau$ such that $U \cap V = \emptyset$ and  $U \cup V = X.$ 

    Recall that an \textit{interval} is a set $I \subset \R$ such that for all $a, b \in I$, $a < x < b$ implies $x \in I.$ Furthermore, all intervals are connected (we omit the proof for brevity). Since $\tau$ is path-connected, for all $x,y \in X$ there exists a path $f: \left[ a,b \right] \to X$ such that $f$ is continuous and $f(a)=x$ and $f(b)=y.$ Now the image of the path denoted $f( [ a,b ] ) $ is connected, since the image of a connected set under a continuous function is connected. Choose $x \in U$ and $y \in V$: then the path $f$ cannot connect $x$ and $y$ since $U \cap V = \O$, and $f([a,b])$ must either be fully contained in $U$ or $V$. Therefore path-connected spaces (and manifolds) are connected, proving the reverse implication.

    For the forward implication, let $a \in M.$ Consider $X,$ the set of points that are path-connected to $a.$ Note that $a \in X$ so $X \neq \O$ (this is important). We claim $X$ and $X^c$ are open: to see this, let $x \in X.$ Then we have an open neighborhood of $x$ homeomorphic to $\R^n$, let us denote its image under the homeomorphism $f$ as $U \subset \R^n$. We can find a convex neighborhood of $f(x)$ denoted $B(f(x), \epsilon) \subset U$ that is path-connected by definition. Since path-connectedness is preserved under a continuous map, the inverse image of the convex neighborhood containing $f(x)$ under the homeomorphism $f$ denoted $f^{-1}\left( B(f(x), \epsilon \right))$ is path-connected. Note that $x \in f^{-1}\left( B(f(x), \epsilon \right)),$ so there exists a path between every point in $f^{-1}\left( B(f(x), \epsilon \right))$ and $x$, therefore $x \in f^{-1}\left( B(f(x), \epsilon \right)) \subset X$ and is open. Since for all $x \in X$ we have $x \in X^{\circ}$, we conclude $X$ is open. A similar argument follows for the fact that $X^c$ is open: simply examine $y \in X^c$ and $B(f(y), \delta)$ instead.

    We reach the final stage of this proof. By assumption, our manifold $M$ is connected. This is equivalent to the fact that the only subsets of $M$ that are both open and closed are $M$ and $\O$: if there existed an $A \subset M$ that were both open and closed, then $A \cap A^c = \O$ and $A \cup A^c = M$, contradicting the fact that $M$ is connected. Now we have constructed a path-connected set $X$ that is both open and closed— both  $X$ and $X^c$ are open, and $X \neq \O$ as stated earlier in the proof. We conclude that $X = M$, and so the manifold $M$ is path-connected. 
\end{proof}
\vspace{10mm}

\begin{problem}
        Suppose a finite group $G$ acts on a manifold~$M$.  Suppose
    the action is \textit{free}, meaning that only the identity
    element has any fixed points.  Then the orbit space $M/G$ is
    also a manifold.  (``Lying in the same $G$-orbit'' is an
    equivalence relation on~$M$.  $M/G$ means the set of equivalence
    classes.  The topology on $M$ induces one on $M/G$, which
    is the one you must work with.)
\end{problem}
\begin{proof}
    Why helpppppp

    $G$ acts on $M$: a map $*:G \times M \to M \ni ex=x \forall x \in M, \, *(g1,g2)x=*(g1*(g2,x)) \forall g_1,g_2 \in G, \, x \in M$ or alternatively $(g_1g_2)x=g_1(g_2x) \forall x \in M, \, g_1, g_2 \in G.$ $M$ is a $G$-set. 

    Free action: $g \in G \land \exists x \in X \ni gx=x \implies g=e.$

    Orbit of $x \in X$: $Gx = \{ gx \mid g \in G \}$ for some $x \in X$. x $\sim$ y iff $\exists g \in G \ni gx=y \implies$ orbits are equivalence classes under this relation. 

    Orbit space: set of all equivalence classes (under the same orbit relation), denoted $X/G$ (also called quotient, space of coinvariants).

    G-orbit
\end{proof}

