%small margins
\usepackage[margin=1in]{geometry}

%basic math stuff
\usepackage{amsmath, amssymb, amsthm}
\usepackage{mathrsfs, mathtools}
\usepackage{hyperref}
\usepackage[capitalize, noabbrev]{cleveref}
\usepackage{xfrac}
\usepackage{tikz-cd}
\usepackage{float}
\usepackage{enumitem}
\usepackage{mleftright}

%ok
\usepackage{xcolor}
\definecolor{meangreen}{RGB}{0,133,62}
\definecolor{burntorange}{RGB}{191,87,0}

%table of contents formatting
\usepackage{tocloft}
\renewcommand{\cftsecfont}{\color{black}}
\renewcommand{\cftsubsecfont}{\color{burntorange}}
\usepackage{titlesec}

%microtype, font, color support, cl
\usepackage[activate={true,nocompatibility},final,tracking=true,kerning=true,spacing=true,factor=1100,stretch=10,shrink=10]{microtype}
\usepackage{adforn}
\usepackage[charter]{mathdesign}
%\usepackage[cmintegrals,cmbraces]{newtxmath}
%\usepackage{ebgaramond-maths}
%\usepackage[T1]{fontenc}


\usepackage{fancyhdr}
\fancyhf{}
 \fancyhead[RO,LE]{\small\thepage}
 \fancyhead[LO]{\small\itshape\nouppercase{\leftmark}}
 \fancyhead[RE]{\small\itshape Lecture Notes}
 \setlength{\headheight}{11.0pt}
\pagestyle{fancy}

%figures
\usepackage{import}
\usepackage{xifthen}
\usepackage{pdfpages}
\usepackage{transparent}
\newcommand{\incfig}[2][1]{%
    \def\svgwidth{#1\columnwidth}
    \import{./figures/}{#2.pdf_tex}
}
\pdfsuppresswarningpagegroup=1

%good stuff
\newtheorem{theorem}{Theorem}[section]
\newtheorem{lemma}{Lemma}[section]
\newtheorem{cor}{Corollary}[section]
\newtheorem{problem}{Problem}
\newtheorem{prop}{Proposition}[section]
\newtheorem*{prob}{Problem}
\theoremstyle{definition}
\newtheorem*{note}{Note}
\newtheorem*{claim}{Claim}
\newtheorem*{conclusion}{Conclusion}
\newtheorem{definition}{Definition}[section]
\newtheorem{remark}{Remark}[section]
\newtheorem{example}{Example}[section]
\newenvironment{solution}
  {\renewcommand\qedsymbol{$\blacksquare$}\begin{proof}[Solution]}
  {\end{proof}}

%named theorems!
\swapnumbers
\theoremstyle{plain}
\newcommand{\thistheoremname}{}
%numbered
\newtheorem{genericthm}[theorem]{\thistheoremname}
\newenvironment{namedthmnum}[1]
  {\renewcommand{\thistheoremname}{#1}%
         \begin{genericthm}}
        {\end{genericthm}}
%unnumbered
\newtheorem*{genericthm*}{\thistheoremname}
\newenvironment{namedthm}[1]
  {\renewcommand{\thistheoremname}{#1}%
         \begin{genericthm*}}
               {\end{genericthm*}}
%general thing (definition, example, etc)
\theoremstyle{definition}
\newcommand{\thisthingname}{}
\newtheorem{genericthing}[definition]{\thisthingname}
\newenvironment{namedthingnum}[1]
  {\renewcommand{\thisthingname}{#1}%
         \begin{genericthing}}
        {\end{genericthing}}
%unnumbered thing
\newtheorem*{genericthing*}{\thisthingname}
\newenvironment{namedthing}[1]
  {\renewcommand{\thisthingname}{#1}%
         \begin{genericthing*}}
               {\end{genericthing*}}

%self explanatory
\newcommand\N{\ensuremath{\mathbb{N}}} 
\newcommand\R{\ensuremath{\mathbb{R}}} 
\newcommand\A{\ensuremath{\mathbb{A}}} %affine space
\newcommand\Z{\ensuremath{\mathbb{Z}}} 
\renewcommand\O{\ensuremath{\emptyset}} 
\newcommand\Q{\ensuremath{\mathbb{Q}}} 
\newcommand\C{\ensuremath{\mathbb{C}}}
\newcommand\F{\ensuremath{\mathbb{F}}} %field
\newcommand\E{\ensuremath{\mathbb{E}}} %field extension
\renewcommand\P{\ensuremath{\mathbb{P}}} %projective space
\renewcommand\H{\ensuremath{\mathbb{H}}} %hyperbolic space
\newcommand\im{\ensuremath{\operatorname{im}}} %image
\newcommand\id{\ensuremath{\operatorname{id}}} %identity map
\newcommand\grad{\ensuremath{\operatorname{grad}}} %gradient
\newcommand\curl{\ensuremath{\operatorname{curl}}} %gurl
\renewcommand\div{\ensuremath{\operatorname{div}}} %divergence
\newcommand\Gr{\ensuremath{\operatorname{Gr}}} %grassmannian
\newcommand\Hom{\ensuremath{\operatorname{Hom}}} %linear mappings

\newcommand{\transv}{\mathrel{\text{\tpitchfork}}}
\makeatletter
\newcommand{\tpitchfork}{%
      \vbox{
              \baselineskip\z@skip
                  \lineskip-.52ex
                      \lineskiplimit\maxdimen
                          \m@th
                              \ialign{##\crcr\hidewidth\smash{$-$}\hidewidth\crcr$\pitchfork$\crcr}
                                }%
                            }
                            \makeatother
%transversality

%essential (thanks Dr. Urbanski)
\renewcommand\qedsymbol{$\boxtimes$}

%link colors
\hypersetup{
 linktocpage,
 colorlinks,
 linkcolor={burntorange},
 citecolor={meangreen},
 urlcolor={blue!80!black}
}

%cool section divider
\newcommand{\orbreak}{
\begin{center}
    \adforn{25}\adforn{14}\adforn{53}
 \vspace{0.2cm}
\end{center}
}

\titleformat{\section}[frame]
{\normalfont}
{\filright
\footnotesize
\enspace Lecture \thesection\enspace}
{8pt}
{\Large\bfseries\filcenter}

  \titlespacing*{\section}
{0pt}{5.5ex plus 1ex minus .2ex}{4.3ex plus .2ex}

%tis me
\author{Simon Xiang}
