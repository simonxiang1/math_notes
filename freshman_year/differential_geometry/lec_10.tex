\section{March 1, 2021}
not sure which lecture we're on...
\begin{definition}[]
    If $\sigma \colon U \to S$ is a chart, then $\mathbf N_{\sigma}= \frac{\sigma_u \times \sigma_v}{\|\sigma_u \times \sigma v\|}$ is the \textbf{standard unit normal}.
\end{definition}
{\color{red}todo:questions: how is this defined for higher manifolds? what's up with the condition for regularity? transition maps are smooth/atlas is maximal? what the heck is the cross product? tpn inn higher space? is it always two dimensions? does tpn depend of choice of charts?} 
Taking the cross product gives us a vector orthogonal to $T_p S$, so dividing by the magnitude gives us a standard unit normal vector. Some stuff happened with computing the standard unit normal to a point on $S^2$.

\begin{lemma}
    If $\sigma,\widetilde \sigma$ are charts then \[
        \widetilde \sigma _{\widetilde u}\times \widetilde \sigma _{\widetilde v}=\det(J(\Phi)) \sigma_u \times \sigma_v,
    \] where $\Phi=\sigma ^{-1} \widetilde \sigma$ and $J(\Phi)$ is the matrix corresponding to $D\Phi$.
\end{lemma}
\begin{cor}
        $\mathbf N_{\widetilde \sigma}=\operatorname{sgn}(\det(J\Phi)) \mathbf N_{\sigma}.$
\end{cor}
\begin{definition}[]
    A surface $S$ is \textbf{orientable} if there is an atlas of regular surface patches for $S$ such that each $\det (J(\Phi))>0$ for the transition map $\Phi$ between any two patches in the atlas.
\end{definition}
