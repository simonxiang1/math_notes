\section{August 31, 2020}

\subsection{The Dihedral Group}
\begin{example}[Dihedral Group]
    Consider the free group $G=\langle g, \tau \rangle $ and the normal subgroup $H_n$ of $G$ generated by \[
    g^{n},\tau^2, \tau g \tau^{-1}g.
    \]
    The dihedral group $D_{2n}=G/H_n$ (sometimes denoted $D_n$), is it automatically normal? What about conjugating by powers of $g$?
\end{example}

Observe that $\langle g \rangle \simeq \langle gH_n \rangle  \subseteq D_{2n}$. $\langle g \rangle $ has order $n$ and is normal (convince yourselves of this). $\tau$ has order 2 and so does $\langle \tau g^{i} \rangle $ for any $i$. Are these subgroups normal? (Yes sometimes, no some other times).

Consider the following: $2\Z \trianglelefteq \Z \to \Z /2\Z= \{2\Z, 1+2Z\}$, $\langle (123) \rangle \trianglelefteq S_3 = \sfrac{S_3}{\langle (123) \rangle} = \{1_{S_3},\bar{(12)}\} $, $\R^{+} \setminus \{0\} \trianglelefteq \R\setminus \{0\} \to \sfrac{\R\setminus \{0\}}{ \R^{+}\setminus \{0\}}=\{\bar{1},\bar{-1}\} $. What distinguishes these groups (they all have order two)?

\subsection{Group Homomorphisms, Isomorphisms, and Automorphisms}
\begin{definition}[Homomorphisms]
    Let $G, H$ be two groups. A map $ \phi \colon G \to H$ is a homomorphism if \[
        \phi (g_1g_2)=\phi(g_1)\cdot \phi(g_2).
    \]
    
\end{definition}
\begin{definition}[Isomorphism]
    A map $\phi$ is an isomorphism is $\phi$ if a homomorphism and a bijection. If $ \phi \colon G \to H$ is an isomorphism then we write $G \simeq H$.
\end{definition}
\begin{definition}[Automorphism]
    We have $\phi$ an automorphism if $\phi$ is an isomorphism from $G$ onto itself, that is, $G=H$.
\end{definition}

\subsection{The First Homomorphism Theorem}
\begin{remark}
    Let $ \phi \colon G \to H$ be a group homomorphism. Then
    \begin{enumerate}
        \item $\phi(1_G)=1_H$ and $\phi(g^{-1})=\phi(g)^{-1}$,
        \item $\phi(G) = \operatorname{im}\phi$ is a subgroup of $H$,
        \item $\ker \phi = \{g\in G \mid \phi(g)=1 H\} $ is a normal subgroup of $G$,
        \item $\phi$ is injective $\iff$ $\ker \phi = \{1_G\}$,
        \item If $G$ is finite then $|G|=|\ker \phi|\cdot |\operatorname{im} \phi|$.
    \end{enumerate}
\end{remark}
\begin{theorem}
    Let $ \phi \colon G \to H$ be a group homomorphism. Then $\bar{\phi} \colon G/\ker\phi \to \im \phi$ is an isomorphism.
\end{theorem}
\begin{proof}
    Left as an exercise to the reader (verify that $\bar{\phi}$ is well-defined, injective, surjective, and a homomorphism).
\end{proof}
\begin{example}
    Recall the groups $\sfrac{\Z}{2\Z} = \langle 1+2\Z \rangle, \sfrac{S_3}{\langle (123) \rangle} = \langle (12)\langle (123) \rangle  \rangle, \sfrac{\R\setminus \{0\} }{\R^{+}\setminus \{0\} } = \langle (-1)\R^{+}\setminus \{0\}  \rangle $. Then we have isomorphisms onto all of them, so they are the same.
\end{example}
\begin{remark}
    Product of groups $\iff$ quotient groups. $H, K$ be groups. 
    $G=H\times K$, $H \simeq H \times \{1_K\} \trianglelefteq G$. ???
    $H,K \trianglelefteq G$, $H \cap K = \{1_G\} , HK=G \implies G \simeq H \times K$ (prove this). $\sfrac{G}{H} \simeq K$ and $\sfrac{G}{K}=H$: relax any of the implications, and the isomorphisms will fail.
\end{remark}

