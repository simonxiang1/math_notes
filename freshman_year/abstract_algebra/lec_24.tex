\section{October 23, 2020}
Last week we had a test, which I didn't go to because I'm not in the class.
\subsection{Prime ideals}
\begin{example}
    Let $R=\Z[x]\supset \langle x \rangle =I$. Recall the ideal generated by $x$, denoted $\langle x \rangle $, is the set of polynomials with constant term zero, or the polynomials such that $0$ is a root of the polynomial. Then $R / I\simeq\Z$, since every polynomial will get crushed to its constant term. Is $I$ maximal? No, since $\Z$ isn't a field. 

    If $J=\langle x,5 \rangle $, what is $R /J$? This is equal to $\Z /5\Z$, since any integer that's a multiple of $5$ will live in $J$, and we can adjust by multiples of $5$. This is determined by the isomorphism $p(x)\mapsto p(0)\pmod 5$. So this is a field, since $5$ being prime implies that $\Z /5\Z$ is a finite integral domain, which implies that it's a field. Then $\langle x,5 \rangle $ is maximal.
\end{example}
\begin{definition}[Prime ideals]
    Let $R$ be a commutative ring with identity. Then $P$ is a prime ideal if $P\neq R$, and if $ab\in P$, it follows that either $a\in P$ or $b\in P$.
\end{definition}
\begin{prop}
    Let $R$ be a commutative ring with unity and $P$ be an ideal. Then $P$ is prime if and only if $R /P$ is an integral domain.
\end{prop}
\begin{prop}
    Let $R$ be a commutative ring with unity and $P$ be an ideal. Then $P$ is maximal implies that $P$ is prime. However, the converse does not hold.
\end{prop}
\begin{proof}
    Now $P$ is maximal if and only if $R /P$ is a field, and $P$ is prime if and only if $R / P$ is an integral domain. But every field is an integral domain, showing the implication. An example to disprove the converse: let $R=\Z[x]$ and $I=\langle x \rangle $. Now $\R /I\simeq\Z$ is an integral domain but not a field, so $I$ is prime but not maximal iff $I\subsetneq J=\langle x,5 \rangle \subsetneq \Z[x]$.
\end{proof}
\begin{remark}
    In the integers, the only prime ideal that isn't maximal is $\langle 0 \rangle $.
\end{remark}


\subsection{Rings of fractions}
Let's start with a commutative ring $R$. Pick a nonempty set $S\subseteq R$ such that $0\in S$ and $S$ is closed under multiplication. That is, for all $s_1,s_2\in S$, $s_1s_2\in S$. Consider the following equivalence relation on $R\times S$: we say \[
    (r_1,s_1)\underset{S}{\sim}(r_2,s_2) \ \text{if there exist} \ s',s''\in S \ \text{st} \
    \begin{cases}
        r_1s'=r_2s'',\\
        s_1s'=s_2s''.
    \end{cases}
\] Verify in your free time that this is an equivalence relation. Let $S^{-1}R:=$ the set of equivalence classes of $R\times S$ with respect to $\underset{S}{\sim}$. Note that $\frac{r}{s}$ is how the equivalence class of $(r,s)$ is notated.

Define the following operations on $S^{-1}R$:
\begin{itemize}
    \item $\frac{r_1}{s_1}+\frac{r_2}{s_2}=\frac{r_1s_2+r_2s_1}{s_1s_2}$,
    \item $\frac{r_1}{s_1}\cdot \frac{r_2}{s_2}=\frac{r_1r_2}{s_1s_2}$. This is why we wanted $S$ closed under multiplication.
\end{itemize}
Under these operations, $S^{-1}R$ has a commutative ring structure with identity. How does this new ring relate to $R$? 
\begin{remark}
    Consider the inclusion map $\iota \colon R\hookrightarrow S^{-1}R$, where $r\mapsto \frac{rs}{s}$, independent of the choice of $s\in S$, due to our definition of the equivalence classes. Note that if $S$ contains one of the zero divisors but not the other, then $\iota$ is not injective, since $\ker \iota=\{0\} $ is just saying that $S$ contains no zero divisors.
\end{remark}
Assume $R$ has a multiplicative identity, then by functorality (speculation) we have an induced inclusion $R^{\times }\hookrightarrow (S^{-1}R)^{\times }$. What are the units in the group $(S^{-1}R)^{\times }$? Is this inclusion surjective? Note that
\[
    (S^{-1}R)^{\times }\supseteq R^{\times }\cup \left\{\frac{s_1}{s_2} \mid s_1,s_2\in S\right\} ,
\] is the $\supseteq$ an equality?  I missed something.
\begin{definition}[Ring of fractions]
    Let $R$ be a commutative ring, $S$ a multiplicatively closed subset of $R$ such that the set of zero divisors is not contained in $S$, and neither is zero. Then the ring $S^{-1}R$ is the \textbf{ring of fractions} of $S$ with respect to $R$. If $R$ is an integral domain, then the ring $(R\setminus \{0\} )^{-1}R$ is the \textbf{field of fractions} or the \textbf{quotient field} of $R$. We denote this as $\operatorname{Frac}(R)$.
\end{definition}
