\section{October 26, 2020}
Whoops, missed this lecture. It was about more on fraction rings.
\section{October 28, 2020}
Let $R=\Z$. What are we talking about? Fix a prime $p$, consider the following:
\begin{itemize}
    \item $S_p = p\Z\setminus \{0\} $ 
    \item $S_p' = \Z /p\Z$
    \item $S_p''=\{p^n   \mid n\in \N\} $ 
    \item $S_{pq}=pq \Z \setminus 0, q\in \Z$
    \item $S_p^{-1}\Z=\{\frac{a}{pk} \mid a\in \Z,pk\in \Z\} $ 
    \item $S_{pq}\subseteq S_p \implies S_{pq}^{-1}\Z\subseteq S_p^{-1}\Z$, by our theorem (I missed it).
\end{itemize}
\begin{theorem}
    Let $R$ be a commutative ring, $S$ a nonempty multiplicity closed subset of $R$ such that $0\notin S$ and $S$ having no zero divisors. Let $Q$ be a ring with $\varphi \colon R \to Q$ an injective homomorphism such that $\varphi (S)\subseteq Q^{\times }$. Then there exists a $\Phi \colon S^{-1}R \to Q$ an injective ring homomorphism such that 
                \begin{figure}[H]
                \centering
\begin{tikzcd}
R \arrow[rr, "\varphi"] \arrow[rdd, "\iota"'] &                              & Q \\
                                              &                              &   \\
                                              & S^{-1}R \arrow[ruu, "\Phi"'] &  
\end{tikzcd}
            \end{figure} commutes (ie $S^{-1}R$ is the smallest ring containing $R$ where the elements of $S$ become units).
\end{theorem}
So we have our theorem that I missed. From there, we can show everything that happened above (not easy).
\subsection{The Chinese remainder theorem}
Recap: let $R$ be a commutative ring with identity $1\neq 0$, and $I,J$ be two ideals of $R$. Note that 
\begin{itemize}
    \item $I+J$ is an ideal of $R$, where addition is just addition of sets
    \item $IJ:=$ the ideal generated by $ij$ for all $i\in I,j\in J$. Note that $IJ\subseteq I\cap J$. 
\end{itemize}
\begin{definition}[Co-maximality]
    Two ideals $I,J\subseteq R$ are \textbf{co-maximal} if $I+J=R$. 
\end{definition}
\begin{theorem}[Chinese remainder theorem]
Let $I_1,\cdots ,I_k$ be ideals of $R$. The map \[
    R\to R/ I_1 \times \cdots \times R /I_k, \quad r\mapsto (r+I_1,\cdots ,r+I_k)
\] is a ring homomorphism with $I_1\cap \cdots \cap I_k$. If for each pair $(i,j)\in \{1,\cdots ,k\} $ the ideals $I_i $ and $I_j $ are maximal, then the map is surjective and \[
\ker \varphi =I_1\cdot I_2\cdot  ...\cdot I_k.
\] 
\end{theorem}
\begin{proof}
    Take $\varphi \colon R \to R /I_1 \times  \cdots \times  R /I_k$. We have $\varphi =(\varphi_1 ,\cdots , \varphi_k )$ where $\varphi _j  \colon R \to R /I_j $ a homomorphism with $I_j $ as its kernel implies that $\varphi $ is a homomorphism and $\ker \varphi =\bigcap_{j=1} ^k I_j $. We proceed by induction: note that our base case is when there are two ideals. For the base case, let $k=2$. We have $\varphi \colon R \to R /I_1\times R /I_2 $, and $I_1+I_2=R$. We want to show that
    \begin{enumerate}
    \item[(a)] $\operatorname{im}\varphi =R /I_1\times R /I_2$,
    \item[(b)] $\ker \varphi =I_1I_2$.
    \end{enumerate}
    For $(a)$, note that $I_1+I_2=R$ implies that $\varphi_1 (I_2)=R /I_1$, similarly $\varphi_2(I_1)=R /I_2 $. Also, $I_1+r=r+I_1$ for $r\in I_2$. We have $\varphi (I_2)= (R /I_1,0)$, $\varphi (I_1)=(0, R/I_2)$. So \[
        \varphi (R)=\varphi (I_1+I_2)=(0,R/I_2)+(R /I_1,0)=(R /I_1, R /I_2)
    \] which implies that $\varphi $ is surjective.

    For $(b)$, does $\ker \varphi =I_1\cap I_2\overset{?}{=}I_1I_2$? $I_1+I_2=R$ implies that $1=c_1+c_2$ with $c_1\in I_1 $, $c_2\in I_2$. So $c=c\cdot 1=\subseteq 1+cc_2$ for all $c\in R$. Let $c\in I_1\cap I_2$. $c\in I_1\implies c c_2\in I_1I_2$ and $c\in I_2\implies cc_1\in I_1I_2$, together these imply $c\in I_1I_2$, hence $I_1\cap I_2=I_1I_2$.
\end{proof}


