\section{October 30, 2020}
Missed this lecture due to Alg top fridays.
\section{November 2, 2020}
Gov test later today, writing notes on Arrow's impossibility theorem, showed up late.
\subsection{PID's and prime elements}
\begin{definition}[Prime and irreducible elements]
    Let $R$ be an integral domain, and $x\in R$. Then 
    \begin{itemize}
        \item $x$ is \textbf{prime} if $(x)$ is a prime ideal.
        \item $x$ is \textbf{irreducible} if $x=ab$ implies that $a$ or $b$ is a unit, for $a,b\in R$.
    \end{itemize}
\end{definition}
\begin{remark}
    Prime implies irreducible. If $x\in (x)$ is prime, then $x=a\cdot b$ implies that $a$ or $b$ are in $(x)$, so $b$ or $a$ is a unit, which is true iff $x$ is irreducible.
\end{remark}
\begin{cor}
    Let $R$ be a PID. Then $x$ is prime if and only if $x$ is irreducible.
\end{cor}
\begin{proof}
    ($\impliedby $) Let $x$ be irreducible, and $R$ a PID. Then $(x)$ is maximal implies that $(x)$ is prime, which is true iff $x$ is prime. To see that $(x)$ is maximal, $x\in (x)\subsetneq (y) \subsetneq R \implies x=y\cdot z$, $z$ is not a unit, $y$ is not a unit. (??)
\end{proof}

\subsection{$3$ is irreducible in $\Z[\sqrt{-5} ]$}
What are the units in $\Z[\sqrt{-5} ]?$ They are exactly the elements such that \[
    (a+b\sqrt{-5} )(a-b\sqrt{-5} )=a^2+5b^2=1.
\] This only happens if $a=\pm 1$ and $b=0$, so the units are just $\{\pm 1\} $. (Yay I answered the question correctly!)
\begin{claim}
    $3$ is irreducible in $\Z[\sqrt{-5} ]$.
\end{claim}
\begin{proof}
    Note that 
    \begin{align*}
        3 &=(a+b\sqrt{-5} )(c+d\sqrt{-5} )\\
          &= ac-5bd+\sqrt{-5} (bd+ad) \\
        \implies 3&=ac-5bd\quad \small{\text{since} \ \sqrt{-5} \ \text{is irrational}},\\
        0&=bd+ad.
    \end{align*}
    Then \[
        3dc=adc^2=5bcd^2=ad(c^2+5d^2).
    \] Now $d\left( 3c-a\left( c^2+5d^2 \right)  \right) =0$ implies that $d=0$ or $3c=a(c^2+5d^2)$. $d=0$ implies that $3=ac+bc\sqrt{-5} $ which implies that $3=ac$ and $bc=0$, since $\sqrt{-5} $ isn't rational. Now $bc=0$ implies that $b=0$ (since $c$ can't be zero or else the top part would fail), and $3=ac$. So either $a$ or $c$ is a unit, and $3$ is irreducible in $\Z[\sqrt{-5} ]$ (given that $d=0$).

    Let's go onto our second possibility, where  $
        3c=a(c^2+5d^2), \ a,c\in \Z^+.
$ Now $c^2\leq 3c$, and $c\leq 3$. So there are now four potential values for $c$: 
\begin{itemize}
    \item $c=1$. Then $3=a(1+5d^2)$, which implies that $3=a(1+5d^2)$, so $a=3,d=0$ (why does $d=1,a=2$ not work?) So $c+d\sqrt{-5} =1$ a unit.
    \item $c=2$. Then $6=a(4+5d^2)$, so $a=1$, and $6=4+5d^2$. This isn't possible.
    \item $c=3$. Then $9=a(9+5d^2)$, so $a=1$, $d=0$. Now $bd=-ad\implies 3b=0\implies b=0$, so $a+b\sqrt{-5} =1$, and we are done.
\end{itemize}
This finishes every case that we need to consider, and we conclude that $3$ is irreducible in $\Z[\sqrt{-5} ]$.
\end{proof}
This is a way of proving that things aren't PID's (irreducible elements that aren't prime). This takes us toward UFD's (which is not the direction we wanted to take), so take this an example, which shows the importance of PID's. Now rings where every ideal is generated by one element are super special, but a ring in which every ideal is finitely generated is also special.
\subsection{Noetherian rings}
\begin{definition}[Noetherian rings]
   A commutative ring is \textbf{Noetherian} if every ascending chain of ideals terminates. That is, for \[
   I_1 \subseteq I_2\subseteq \cdots \subseteq I_j \subseteq \cdots 
   \] then $I_j =I_{j+1}$ for all $j\gg 0$\footnote{This is standard notation.}.
\end{definition}
\begin{prop}
    A ring $R$ is Noetherian iff every ideal of $R$ is finitely generated.
\end{prop}
\begin{proof}
    Let $R$ be Noetherian, and $I$ be an ideal that is \textbf{not} finitely generated. Pick an element $r_1\in I$ and set $I_1=(r_1)$\footnote{I'm swapping over to this notation for an ideal generated by an element, since everyone else seems to do it.}. Since $I$ isn't finitely generated, then there exists an $r_2\in I \setminus I_1$. Set $I_2=(r_1,r_2)$. Note that $I_1 \subsetneq I_2$. Then by the fact that $I$ isn't finitely generated, there eixsts an $r_3\in I \setminus I_2$, set $I_3=(r_1,r_2,r_3)$. Repeat indefinitely, so \[
    I_1 \subsetneq I_2 \subsetneq I_3 \subsetneq \cdots \subsetneq I_j  \subsetneq \cdots \subsetneq I,
\] where $I_j =(r_1,\cdots,r_j )$. Since $I$ isn't finitely generated, there exists an $r_{j+1}\in I \setminus I_j $, set $I_{j+1}=(r_1,\cdots ,r_j ,r_{r+1})$. This produces an infinite strictly increasing chain of ideals, which is not possible since $R$ is Noetherian.
\end{proof}
