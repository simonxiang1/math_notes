\section{September 2, 2020}
Last time: Homomorphisms, Isormorphisms, Automorphisms, trivial maps.
\subsection{The Symmetric Group Rises from the Automorphism Group}
\begin{example}[Group of Automorphisms]
    Let $X$ be a finite set. Let \[
        S_x := \{ f \colon X \to X  \mid f \, \text{is bijective}\} 
    \]
    Bijections on $X$ preserve $X$: think of this set as the \emph{group of automorphisms} on $X$, defined as $\operatorname{Aut}(X)$. The group operation is simply function composition. Then the identity element is the identity map, and the inverse of any $f\in S_x$ is $f^{-1}\in S_x$.
\end{example}
Assume that $ g \colon X \to Y$ is a bijection. Then $g$ gives rise to a homomorphism $ \phi_g \colon S_y \to S_x$, $f \mapsto g^{-1}fg$. Verify that this map is well defined and a group homomorphism. Is $\phi_g$ an isomorphism? If $\phi_g^{-1} \colon S_x \to S_y$ were well-defined, then $\phi_g$ is a bijection. Consider $S_y (\phi_g) \to S_x (\phi_g^{-1}) \to S_y$, $f \mapsto g^{-1}fg \mapsto g(g^{-1}fg)g^{-1}=(gg^{-1})f(gg^{-1})=f$. So $\phi_{g^{-1}} \colon S_{x} \to S_{y}$, $h \mapsto (g^{-1})^{-1}fg^{-1}=gfg^{-1}$.

\begin{conclusion}
    Two finite sets $X,Y$ have the same cardinality if there exists a bijection $ g \colon X \to Y$. This bijection gives rise to the map $ \phi_g \colon S_y \to S_x$ an isomorphism, so the group of automorphisms $S_x$ depends only on the size of the group (when $X$ is a finite set). Let $|X|=n$, then $S_x \simeq S_n$.
\end{conclusion}

\subsection{On the Symmetric Group}
A cycle in $S_n$: $(\alpha_x , ... , \alpha_k)$ is a $k$-cycle. $\alpha_1 ,...,\alpha_k \in \{1, ...,n\} $, $a_i \neq a_j \ \forall i \neq j$. We have  
\begin{equation*}
    (\alpha_1,...,\alpha_k)(m) = 
\begin{cases}
    m \quad &\mbox{if}\, m \neq \alpha_i \ \forall i=1,...,k\\
    \alpha_{i+1} \quad &\mbox{if} \, m=\alpha_i, \, i\in \{1,...,k-1\} \\
    \alpha_1 \quad &\mbox{if} \, m=\alpha_k.

\end{cases}
\end{equation*}

\subsection{Transpositions and Cycles}
\begin{definition}[Transpositions]
   A \emph{transposition} is a $2$-cycle in $S_n$, denoted  \[
       (\alpha_1 \alpha_2),
   \]
   where $\alpha_1 \neq \alpha_2$.
\end{definition}

\begin{definition}
    Two cycles $(\alpha_1,...,\alpha_k)(\beta_1,...,\beta_m)$ are \emph{disjoint} if $\alpha_i \neq \beta_j$ for all $i \in \{1,...,k\}, \, j \in \{1,...,m\}$. 
    Disjoint cycles commute, that is, \[
        (\alpha_1,...,\alpha_k)(\beta_1,...,\beta_m)=(\beta_1,...,\beta_m)(\alpha_1,...,\alpha_k)
    \]
\end{definition}
\begin{lemma}
    Every element $s\in S_n$ can be written uniquely (up to reordering) as a product of disjoint cycles.
\end{lemma}
\begin{proof}
    \textbf{Step 1}: Let $s\in S_n$. If $s=\operatorname{id}_{\{1,...,n\} }$, then $s=1_{S_n}$. We have $s \neq 1_{S_n} \implies I_0 (\neq \O) := \{1 \leq k \leq n, \, s(k) \neq k\} $. Define $k_1 := \operatorname{min}I_0.$ Then \[
        \iota_1 := ( k_1 \,s(k_1) \, s^2(k_1)\, ... ) 
    \]
   is an $e_1$-cycle where \[
   \begin{cases}
       s^{e_1}(k_1) =k_1 \\
       e_1=\operatorname{min} \{d \in \N  \mid s^{d}(k_1)=k_1\}.
   \end{cases}
   \]
   \textbf{Step 2}: Now \[
       I_1=I_0\setminus \{k_1,...,s^{e_1}(k_1)\}.
   \]
   If $I_1=\O$, we are done: $s = c.$ If $I_1\neq\O$: $k_2=\operatorname{min}I_1$. Set $\iota_2=(k_2\,s(k_2)\,...)$ an $e_2$-cycle where $s^{e_2}(k_2)=k_2)$, $e_2=\operatorname{min}\{d\in \N \mid s^{d}(k_2)=k_2\} $. 
   \begin{note}
       $c_1, \, c_2$ are disjoint cycles.
   \end{note}
\noindent\textbf{Step 3}: $I_2=i_1\setminus \{k_2,s(k_2),...,s^{e_2-1}(k_2)\} $. If $I_2=\O$ then we are done, verify $s=c_1c_2$. If $I_2\neq\O$ then $k_3=\operatorname{min}I_2$. Repeat the steps until $I_j = \O \implies s=c_1 ... c_j$ disjoint cycles by construction. Verify the uniqueness in your free time.
\end{proof}
\begin{note}
    $s\in S_n \implies s=\prod_{i=1}^{n}c_i$, where the $c_{i}$ are \emph{disjoint} cycles.
\end{note}
\begin{claim}
   The order of $s$ defined as \[
       \operatorname{ord} s:= \operatorname{min} \{k\in \N  \mid s^{k}=1_{S_n}\}  
   \]
   is equal to \[
       \operatorname{lcm}\{\operatorname{ord} c_i  \mid i=1,\cdots,j\},
   \]
   where each $\operatorname{ord} c_i$ is the length of each cycle $c_i$.
\end{claim}
Verify that this claim holds in your free time.
\begin{note}
    We will show next time that every finite group is a subgroup of $S_n$ for some $n\in \N$ (Cayley's Theorem). This shows the importants of permutation groups: they contain all the information you need to know about groups.
\end{note}




