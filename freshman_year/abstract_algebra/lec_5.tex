\section{September 4, 2020}
\subsection{Group Actions}
\begin{definition}[Group Action]
    An \emph{action} of a group $G$ on a set $X$ is a map \[
        a \colon G \times X \to X, \quad (g,x) \mapsto g\cdot x
    \]
    such that 
    \begin{enumerate}
        \item $(1_G,x) \mapsto x$,
        \item $g_1(g_2\cdot x)=(g_1g_2)\cdot x$
    \end{enumerate}
    for all $x\in X$, $g_1,g_2\in G$. Notation: $G \hookrightarrow X$, $G$ acts on $X$.
\end{definition}
\begin{prop}
    Let $G$ be a group and $X$ a set. Actions of $G$ on $X \,(a \colon G\times X \to X)$ are in bijection with homomorphisms $\phi \colon G \to S_X$.
\end{prop}
\begin{proof}
    Given an action $a \colon G\times X \to X$, define $\phi_a \colon G \to S_X, \, g \mapsto (x \mapsto a(g,x))$, $a(g,x)\in X$. Verify that 
    \begin{enumerate}
        \item $x\mapsto a(g,x)$ is a bijection on $X$ ($\iff [x \mapsto a(g,x)]\in S_x$),
        \item $\phi_a$ is a homomorphism.
    \end{enumerate}
\end{proof}
Given $\phi \colon G \to S_X$ a homomorphism, define $a_{\phi} \colon G\times X \to X$, $(g,x) \mapsto \phi(g)(x) \in X$. We have to verify that 
\begin{enumerate}
    \item $a_{\phi}$ is a group action, i.e., $a_{\phi}$ is a well-defined map.
    \item $a_{\phi}(1_G,x)=x$. $\phi(1_G)(x)=1_{S_X}(x)=\operatorname{id}_X(x)=x$.

    \item $a_{\phi}(g_1,a_{\phi}(??)$
    %come back for this too
\end{enumerate}
Finally, we must verify that \[
a \mapsto \phi_a \mapsto a_{\phi_a}=a
\]
and \[
\phi \mapsto a_{\phi}\mapsto \phi_{a_{\phi}}=\phi.
\]

\subsection{Orbits and Stabilizers}
Given an action $a \colon G\times X \to X$ and an element $x\in X$, we can talk about the \emph{orbit} of this action under $x$.
\begin{definition}[Orbits]
    We define an \emph{orbit} of $x$ as \[
    G\cdot x = \{g\cdot x \mid g\in G\}.
    \]
\end{definition}
\begin{definition}[Stabilizer]
    We define the \emph{stabilizer} of $x$ as \[
    G_x= \{g\in G \mid g\cdot x=x\} .
    \]
\end{definition}
\begin{remark}
 We have $1_G\in G_x$ for all $x\in X$. 
\end{remark}
\begin{claim}
    $G_x$ is a subgroup of $G$. To show this, note that
    \begin{enumerate}
        \item $1_G\in G_x \impliedby (1_G,x)=x,$
        \item $g\in G_x \implies g^{-1}\in G_x$. To see this, note that $g^{-1}(gx)=g^{-1}x$ (since $g$ is in the stabilizer subgroup) and $(g^{-1}g)x=1_{G}x=x$, which implies $g^{-1}x=x$, so $g^{-1}\in G_x.$
        \item $g_1,g_2\in G_x \implies g_1g_2\in G_x$. $(g_1g_2)x=g_1(g_2x)=g_1(x)$ since $g_2$ stabilizes $x$, which implies $g_1x=x$ since $g_1$ also stabilizes $x$, and we are done.
    \end{enumerate}
\end{claim}
\begin{definition}[Transitive Action]
    An action is \emph{transitive} if \[
    Gx=X
    \]
    for some $x\in X$. Prove that if you have this property \emph{for some} $x\in X$, then this is the same as \emph{every} $x\in X$ having this property.
\end{definition}
\begin{lemma}
    If $x,y \in X$ lie in the same orbit (there exists a $g \in G$ such that $gx=y$), then $G_x=g^{-1}G_yg$.
\end{lemma}
\begin{proof} We have
    \begin{align*}
        h\in G_y &\iff hy=y\\
                 &\implies hgx=gx\\ 
                 &\implies g^{-1}hgx=g^{-1}(gx)=(g^{-1}g)x=x.
    \end{align*}
        So $g^{-1}hg\in G_x$, which implies $g^{-1}G_yg \subseteq G_x$. To prove the reverse inclusion, let $h\in G_x$. Then 
    \begin{align*}
        h\in G_x &\iff hx=x\\
                 &\implies hg^{-1}y=g^{-1}y\\ 
                 &\implies ghg^{-1}y=g(g^{-1}y)=(gg^{-1})y=y.
    \end{align*}
    So $ghg^{-1}\in G_y \implies gG_xg^{-1}\subseteq G_y \implies G_x \in g^{-1}G_yg$, and we are done. 
\end{proof}
\begin{lemma}
    Let $G \hookrightarrow X$. Then two orbits are either equal or disjoint.
\end{lemma}
\begin{proof}
    $G_x \cap G_y \neq \O \implies  G_x=G_y$. Let $z\in G_x \cap G_y \implies G_x=G_z=G_y$.
\end{proof}
General idea of group actions: for every element of the set, you have its stabilizer, and you can look at its orbits (are the same or are they disjoint?).

\subsection{Quotient Group of Orbits}
Let $G \hookrightarrow X$, $x\in X$. Consider the map \[
G/G_x \to G_x, \quad gG_x \mapsto g\cdot x.
\]
Notice this is well defined because $gh \mapsto gh\cdot x = g(hx)=gx$ since $h\in G_x$.

\begin{claim}
    The map $G/G_x \mapsto G_x$ is a bijection.
\end{claim}
Surjectivity follows from the definition of an orbit, and injectivity ... is up to you to prove. (Not hard, think about the definitions). But what does this mean?
\begin{prop}
    If $G$ is finite, then the size of each orbit divides the size of $G$. 
\end{prop}
\begin{proof}
    $x\in X$, $G_x \leftrightarrow G/G_x \implies |G_x|=|G/G_x| \, \big| \, |G|$.
\end{proof}
\begin{example}
    Every group acts on itself in three different ways, that is, $G \hookrightarrow X, \, X=G$.
    \begin{enumerate}
        \item Left multiplication: $g\cdot x=gx$,
        \item Conjugation: $g\cdot x=gxg^{-1}$,
        \item Right multiplication: $g\cdot x=xg^{-1}$ (if we define it as $xg$ some properties of group actions will not hold). Why? $(g_1g_2)^{-1}=g_2^{-1}g_1^{-1}$.
    \end{enumerate}
    Orbits and Stabilizers WRT the above actions: 
    \begin{enumerate}
        \item $Gx=X=G$ for all $x\in X$, $G_x={1_G}$,
        \item  $Gx=$ conjugacy class of $x$, $G_x=$centralizer of $x = \{g\in G \mid gx=xg\} $,
        \item $Gx=X=G$ for all $x\in X$, $G_x={1_G}.$
    \end{enumerate}
\end{example}
\begin{prop}
    Let $G$ be a group of order $n$, then $G \simeq $ subgroup of $S_n$.
\end{prop}
