\section{September 18, 2020}
Last time: proving a corollary. I wish I was in the Canvas... We deal with many many cases.
\begin{example}
$|G|=2\cdot 3\cdot 5$. 
    \begin{itemize}
        \item $n_5=1$ or $6 \implies G$ has $(5-1)6$ elements of order $5$.
        \item $n_3=1$ or $10 \implies G$ has $(3-1)6$ elements of order $3$.
    \end{itemize}
    So  $|G|>4\cdot 6+20=44$, which is false since $|G|=30$. Now let $|G|=2\cdot 3\cdot 7$. $n_7=1\implies G$ is not simple. $|G|=2^3\cdot \implies n_3=1$ or $4$, $n_2=1$ or $3$. Let $P_2$ be the $2$-Sylow: $[G:P_2]=3$. We have an older theorem: there exists an $N<P_2$ such that $N\trianglelefteq G$ and $3\leq[G:N]  \mid 3!=6$, which implies $N$ is nontrivial or the full group, so $G$ is not simple.
\end{example}

\subsection{Representation Theory}
We have group actions connected to some representations, and we use these representations to talk about our groups. Representation theory is the study of linear algebra to deduce things about groups.

\begin{claim}
    Group actions of $G$ gives rise to representations of $G$. (The other way holds, but is not as useful).
\end{claim}
We have seen for a group $G$ acting on a set $X$, we have a bijective map $\varphi \colon G \to S_X$ a group homomorphism. Each group action corresponds to a homomorphism, that is, \[
    g\cdot x \longrightarrow \varphi \colon g \mapsto (x\mapsto g\cdot x).
\] For the other way around, \[
g\cdot x:= \varphi (g)(x)\longleftarrow \varphi .
\] 
\begin{definition}[Representations]
    A representation of $G$ on an $\F$-vector space $V$ is a homomorphism  \[
        \varphi \colon G \to \operatorname{GL}(V),
    \] where $\operatorname{GL}(V)$ is just the set of automorphisms $\operatorname{Aut}(V\to V)$ from $V$ onto $V$ (automorphisms are just invertible linear maps). 
\end{definition}

\subsection{Linear Actions}
How are group actions related to representation theory?
\begin{definition}
   A group action $G$ on the vector space $V$ is \emph{linear} if the maps induced by the elements of your group $v\to g\cdot v$ are linear for all $g\in G$.
\end{definition}
\begin{prop}
    Linear actions of $G$ are in bijection with representations of $G$. To see this, $g\cdot v \longrightarrow \varphi \colon g\mapsto (v\mapsto g\cdot v)$, $v\in V,\,g\in G$. Verify that $\varphi $ is a homomorphism. For the other way, $\varphi \colon G \to \operatorname{GL}(V)$, $g\cdot v=\varphi (g)(v)$.
\end{prop}
\begin{proof}
   Things we have to do:
   \begin{enumerate}
       \item Verify that $g\cdot v$ is a linear group action.
       \item Verify that $\varphi (g)\in \operatorname{GL}(V)$. Also verify that $\varphi $ is a homomorphism (this is trivial).
   \end{enumerate}
\end{proof}
Let $G$ a group acting on a set $X$. We want to construct a linear action of $G$ using this given action. Let $V=\bigoplus\F e_x$, where $e_x$ is a basis element. Then the action of $G$ on $V$ is defined as follows: let \[
v\in V\implies v=\sum_{x\in X}^{} a_x e_x
\] where $a_x\in \F$, $a_x=0$ for all but finitely many $x\in X$ (denoted by the convention ``almost all''). Then \[
g\cdot := \sum_{}^{} a_x e_{g\cdot x}\in V.
\] 
\begin{claim}
    The action of $G$ on $V$ is linear. Verify this in your free time.
\end{claim}

\subsection{Regular Representations}
\begin{definition}[Regular Representations]
Consider the corresponding representation $\varphi \colon G \to \operatorname{GL}(V)$. This representation has a special name: observe $\varphi (g)$ is a \emph{permutation matrix}  whos entries are $0$ or $1$ if $x$ is finite. Permutation matrices simply permute (rearrange) the basis elements. This is called the \emph{regular} representation.
\end{definition}
\begin{example}
    Consider the action $G$ on $G$ by left multiplication. Then \[
        V_{\text{reg}}:=\bigoplus_{g\in G}\F e_g, \quad \varphi _{\text{reg}} \colon G \to \operatorname{GL}(V_{\text{reg}}).
    \] is the regular representation of $G$. If $G$ is finite, then $V_{\text{reg}}$ is finite dimensional. Multiplicity, irreducability (throwback to last semester!). We call a space irreducable if we can't find a subspace such that we can restrict this homomorphism to the subspace.

    If this is gibberish, just know that the regular representation will contain \textbf{all} the information you need to know about your group (wow!).
\end{example}
\begin{definition}
    Let $V$ be a finite dimensional vector space over $\F$. Then the \emph{character} of a representation $\varphi \colon G \to \operatorname{GL}(V)$ is defined as \[
        \operatorname{char}\varphi  \colon G \to F,\quad g\mapsto \operatorname{tr}\varphi (g).
    \] 
\end{definition}
The amazing thing is that your character will determine your representation uniquely. Let's continue this next time (this is making much more sense than Sylow whatever).

