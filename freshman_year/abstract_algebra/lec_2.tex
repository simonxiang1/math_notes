\section{August 26, 2020}
\subsection{Oops}
Unfortunately, I couldn't attend Lecture 1.
\section{August 28, 2020}

\subsection{Subgroups and Normal Subgroups}
\begin{lemma} Let $H \subseteq G$, $\langle G, \cdot \rangle$ a group and $H \neq \O.$ Then $H$ is a subgroup of $G$ if and only if $h_1h_2 \in H \implies h_1h_2^{-1}\in H$.
\end{lemma}
\begin{proof}
    For all $h_2 \in H$, $h_2^{-1} \in H$ since $H$ is a group. $H$ is closed under multiplication implies $h_1h_2^{-1}\in H$ for all $h_1,h_2 \in H$. Conversely, assume that $h_1h_2\in H\implies h_1h_2^{-1}\in H.$ Then for $h\in H$, $hh^{-1}\in H$ so $1\in H$. Now that we know $1\in H$, then for $h\in H$ we have $1\cdot h \in H \implies h^{-1}\cdot 1\in H$, so $H$ is closed under inverses. Finally, associativity follows from the fact that  $H \subseteq G \implies \forall h \in H, h \in G$ where $G$ is a group, and we are done.
\end{proof}

\begin{definition}[Normal Subgroup]
    A \emph{subgroup} $H$ of $G$ is normal if $gHg^{-1}=H$ for all $g \in G$.
\end{definition}
\begin{example}
    Let $G$ be abelian: then every subgroup is normal since $ghg^{-1}=gg^{-1}h=h$ for all $g \in G, h \in H$.
\end{example}
\begin{example}
    Take $G=S_3.$ Then the subgroup $\langle (123) \rangle$ is normal. However, the subgroup $\langle (1,2) \rangle$ is not normal, since $(13)(12)(13)^{-1}=(23) \notin \langle(12)\rangle$.
\end{example}
\begin{example}
    Take $\textrm{SL}_n\R \subseteq \textrm{GL}_n\R$, where $\textrm{SL}_n\R$ is the set of matrices with $\det(A)=1$ for $A \in \textrm{SL}_n\R$. We know $\textrm{SL}_n\R$ forms a subgroup. Question: is $\textrm{SL}_n\R$ normal? Answer: yes.
    \[
        \det(ABA^{-1})=\det(A)\det(B)\det(A^{-1})=\det(A)\det(A)^{-1}\det(B)=\det(B).
    \]
\end{example}

\begin{prop}
    Let $H, K$ be subgroups of $G$, then $H \cap K$ is a subgroup of $G.$ You can verify this in your free time.
\end{prop}

\textbf{Note:} is $H \cup K$ a subgroup? No!

\subsection{Product and Quotient Groups}
\begin{definition}[Product Groups]
    Let $G,H$ be groups. We define the \emph{direct product} $G \times H$ with the group operation $(g_1,h_1)\cdot(g_2,h_2)=(g_1g_2,h_1h_2)$. The identity is just $(1_G,1_H)$ where $1_G$ and $1_H$ denotes the respective identities for $G$ and $H$. Finally, the inverse is similarly defined as $(g_1^{-1},h_1^{-1})$ where $g_1^{-1}$ and $h_1^{-1}$ are the respective inverses for $g_1\in G$, $h_1\in H$.
\end{definition}

Some examples of product groups include $\Z \times \Z$ ($\Z$ denotes $\langle \Z,+ \rangle )$, and $\Z \times \langle \R\setminus\{0\} , \cdot \rangle $

\begin{example}[Quotient Groups]
    Let $n \in \Z$, for example $ \langle \Z /n\Z, + \rangle $, equivalence relations: modulo $n$. $a,b \in \Z, a \equiv b \mod n \iff n \mid (a-b)$. Equivalence classes: $a+n\Z = \{ a+nk \mid k \in \Z \}$. Notation: $\bar{a}=a+n\Z=[a]$. Our set $\Z /n\Z=\{a+n\Z \mid a \in \Z \} = \{ a + n\Z \mid a=0, ... , n-1 \}$. $(a+n\Z)+(b+n\Z)=(a+b)+n\Z$, so this is a group operation. In this case, the identity is just $0+n\Z=n\Z$. We have the inverse of $(a+n\Z)$ equal to $(a+n\Z)^{-1}=-a+n\Z$.
\end{example}

Remark: $ \langle \Z/n\Z,+  \rangle $ is a quotient of the group $\langle \Z,+ \rangle $ by the subgroup $\langle n\Z,+  \rangle $. $\langle 1 \rangle =\Z, \langle 1+n\Z \rangle = \langle \Z/n\Z \rangle.$

Quotient groups in general: $G$ a group, $H$ a \textbf{normal} subgroup. 

\subsection{Left and Right Cosets}
\begin{definition}[Cosets]
    Left cosets: $gH=\{ gh \mid h \in H \}.$
    Right cosets: $Hg=\{ hg \mid h \in H \}.$
    $G/H$ - set of left cosets.
    $H\setminus G$ - set of right cosets.
\end{definition}
Observe: Left and right cosets are in bijection with one another. $gH \mapsto Hg$, $gh \mapsto g^{-1}(gh)g=hg$. You can verify that this is a bijection. Let $g_1,g_2 \in G$, what map maps $g_1H \to g_2H$? $g_1h \mapsto (g_2g_1^{-1})g_1h=g_2h.$

\begin{note} We have
   \[
\bigcup_{g\in G} gH = G.
   \]
Also: $g_1H \cap g_2H $ is either $\O$ or they are equal. (Equivalence relation).
\end{note}

\subsection{Lagrange's Theorem}
\begin{prop}
If $G$ is finite and $H$ a subgroup of $G$, then $|H| \mid |G|$.
\end{prop}


\begin{proof}
    By the statement above,
    \[
    G = \bigcup_{i=1}^{n}g_iH
    \]
    since $G$ is finite for $n \in \N$. Note that this is a disjoint union. So \[
        |G|= \sum_{i=1}^{n} |g_iH| = n \cdot |H| \implies |H| \mid |G|.
    \]
\end{proof}
Quotient group: $G$ a group, $H$ a normal subgroup, $G/H = \{ gH \mid g \in G \}$. The multiplication is defined as $g_1H \cdot g_2H = g_1g_2H$. You can verify this operation is well defined (given that $H$ is normal).


