\section{September 25, 2020}
\subsection{Whoops}
Whoops, I was working on some Dehn presentation homework problem for Algebraic Topology (due at 1:00), and couldn't make it to class today.
\section{September 28, 2020}

\subsection{The alternating group}
We have $S_n$ generated by transpositions: this is because $S_n$ is generated by cycles, which are generated by transpositions. Within $S_n =\langle \text{transpositions} \rangle $, we have $A_n=\langle \text{pairs of transpositions} \rangle $. For $\tau \in S_n$, elements of $\tau A_n \tau^{-1}$ still has an even number of transpositions. For example, say $\tau=(12)(14)(46)$, then $\tau^{-1}=(46)(14)(12)$. We know $\sigma \in A_n$ is a product of an even number of transpositions, say $2n$. Then $\tau\sigma\tau^{-1}$ is a product of $2m+2\cdot 3$ transpositions, which is even, so $\tau\sigma\tau^{-1}\in A_n$ for all $\sigma\in A_n$. Hence $\tau A_n\tau^{-1}\subseteq A_n$ for all $\tau\in S_n$, which implies that $A_n\trianglelefteq S_n$.

Observe that $(12)(13)=(12)(31)=(132)$. If two cycles are adjacent and have a number in common, then we can transform it into a $3$-cycle as shown above. Also observe that $(12)(34)=(12)(23)(23)(34)=(123)(234)$\footnote{I like how the thing we struggle the most with is manually calculating what cycles are.}. So we can write any two adjacent cycles by a product of two or less $3$-cycles. Hence $A_n$ is generated by $3$-cycles, and any even transposition can be written as a product of $3$-cycles.

\subsection{$A_n$ is simple for $n\geq 5$}
Dr.\ Ciperiani just told some story that I lost, but it's in the book (Dummit and Foote).
\begin{lemma}
    Let $(abc),(ijk)\in A_n$ for $n\geq 5$. Then $\pi(abc)\pi^{-1}=(ijk)$ for some $\pi\in A_n.$ 
\end{lemma}
Note that this would be no suprise if $\pi\in S_n$.
\begin{proof}
    We know that there exists  $\pi'\in S_n$ such that $(ijk)=\pi'(abc)\pi'^{-1}$. If $\pi'\in A_n$, then set $\pi=\pi'$ and we are done. If $\pi'\notin A_n$, we know there exists some $(kd)$ that is a conjugate of $(abc)$, that is, $k,d\neq a,b,c$, which we can do since $n\geq 5$.So \[
        \pi'(kd)(abc)(kd)^{-1}\pi'^{-1}=\pi'(abc)\pi'^{-1}=(ijk).
    \] Then set $\pi=\pi^{-1}(kd)$, together with the fact that $\pi'\notin A_n$ we have $\pi\in A_n$.
\end{proof}
\begin{theorem}
    $A_n$ is simple in $S_n$ for every $n\geq 5$.
\end{theorem}
\begin{proof}
    Say we have some nontrivial subgroup $N \trianglelefteq A_n$. We want to show $N=A_n$. If $N=A_n$, then of course $N$ contains a $3$-cycle. By our lemma, if $N$ contains a $3$-cycle, then $N=A_n$ ($N$ is normal, conjugates). So $N=A_n\iff N$ contains a $3$-cycle.

    Let $\pi\in N\setminus \{\operatorname{id}\} $ such that $\pi$ fixes as many symbols as possible. We will choose an element of $N$ that fixes more symbols than $\pi$, which is a contradiction. Suppose $\pi$ is not a $3$-cycle (if not, then $N=A_n$). Then 
    \begin{enumerate}
        \item $\pi=(12)(34)\cdots$ (we get this by conjugation) – moves at least four symbols, starting with a product of two disjoint transpositions.
        \item $\pi=(123\cdots)\cdots$ – moves two more symbols, say $4,5$. $\pi$ cannot equal $(1234)$, since $\pi$ is even and $4$-cycles don't live in $A_n$.
    \end{enumerate}
    Consider $\sigma=(345),\, \pi'=\sigma^{-1}\pi^{-1}\sigma\pi\in N$, since $\pi\in N$, $\sigma\in A_n$ ($N\trianglelefteq A_n\trianglelefteq S_n$). Notice that $\pi(x)=x$ implies that $\pi'(x)=x$ for any $x>5$. For $\pi'$, 
    \[
    1\overset{\pi}{\mapsto }2\overset{\sigma}{\mapsto }2\overset{\pi^{-1}}{\mapsto }1\overset{\sigma^{-1}}{\mapsto }1,
    \] 
    which implies $\pi'(1)=1$. So $\pi'$ fixes more elements than $\pi$. Now we want to show that $\pi'\neq \operatorname{id}$. Now $\pi'(2)=2$, so this isn't very helpful. What's $\pi'(3)$? in case 1, $\pi' \colon 3 \mapsto 4 \mapsto 5$, and $5$ either maps to $5$ or some $k>5$ (because the cycles are disjoint). If $5\mapsto 5,$ then $5\overset{\sigma^{-1}}{\mapsto }4$, and if $5\mapsto k$, $k\overset{\sigma^{-1}}{\mapsto } k$ for some $k\geq 5$, which together imply $\pi'(3)>3$. For case 2, $\pi' \colon 2 \overset{\pi}{\mapsto }3 \overset{\sigma}{\mapsto }4 \overset{\pi^{-1}}{\mapsto }?$ Either $\pi=(12345)\cdots,\,(1234)(5\cdots),\,(123)(45)\cdots\,(123)(45\cdots)$. In the first two cases, $4\overset{\pi^{-1}}{\mapsto }3$, in the third case, $4\overset{\pi^{-1}}{\mapsto }5$, and in the fourth case, $4 \overset{\pi^{-1}}{\mapsto }k$ for $k>5$. None of these are $2$, so $\pi'(2)\neq 2$, which implies $\pi'$ is not the identity. This is a contradiction, since the $\pi'$ we constructed lives in $N$, is not the identity, and fixes more symbols than $\pi$. So $\pi$ is a $3-cycle$, which implies $N=A_n$ by the lemma. Therefore $A_n$ is simple in $S_n$ for all $n\geq 5$.
\end{proof}



