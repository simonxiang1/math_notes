\section{Solving the word problem with geometry (10/13/20)}
Today's speaker is Teddy Weisman, a 5\textsuperscript{th} year Ph.D student. We'll be talking about regular expression in computer science and their connection to geometry (connecting math and linux!) JK, I was baited :((( I really like regular expressions this is sad

Favorite thing to do in math: see a weird abstract problem, look at it hard enough, it turns into geometry.
\orbreak
Here's a simple puzzle, take an alphabet, write a word (finite seq of letters), basically group presentations. We have the free group on four letters with relations $\langle a,b,c,d \mid [a,b],[a,c],[b,d] \rangle $: wait, you can replace the commutators $[a,c]$ and $[b,d]$ with the empty word $\{\} $, but not $[a,b]$. Goal: is every word trivial? 
\begin{example}
    $abcd\to bacd\to bd\to \{\} $, so that's trivial. $bac bd\to bbd \to b$, non trivial.
\end{example}
\begin{definition}[]
    Two words are \textbf{equivalent} if you can get from one to the other with finitely many replacements.
\end{definition}
\subsection{The word problem}
\textbf{Question:} Can you write an algorithm that determines if any word is equivalent to the empty word? 

Answer: Yes, there is one. Rewrite our generating set as $a,b,a^{-1},b^{-1},\,ab=ba,\,aa^{-1}=a^{-1}a=\{\} ,\,bb^{-1}=b^{-1}b=\{\} $. This turns into algebra, take for example the word $aba^{-1}bbb^{-1}aa^{-1}ba$, we can switch around and simplify stuff. Furthermore, $ab^{-1}=b^{-1}a$ and $a^{-1}b=ba^{-1}$. How do we check if it's the empty word? Count the number of $a$'s, subtract the number of $a^{-1}$'s, for example the word from earlier is now $b^3a$ which is nontrivial.
\orbreak
\begin{definition}[]
    A \textbf{string rewriting system} $S$ is a pair $S=(A,R)$, where $A$ is a finite set of letters and $R$ is a finite set of pairs of words in $A$. If each letter $a\in A$ has an inverse $a^{-1}\in A$, with a replacement rule $a^{-1}a=aa^{-1}=\{\} $, then $S$ is a group presentation.
\end{definition}
\textbf{Question:} Can you find an algorithm with takes as input a string writing system $S$ and a word in an alphabet for $S$, which determines if the word is equivalent to $1$ (trivial)? (I think the answer is no...)

The answer is no! I got it right yay. Computers are not capable of doing this. No wonder the group presentations from van Kampens are so nasty. This is equivalent to the halting problem.

    You can na\"ively try a \emph{brute force algorithm} :
\begin{itemize}
    \item Set $n=1$.
    \item Apply all possible sequences of $n$ subsitutions to the original word.
    \item If any are the empty word, we're done.
    \item Otherwise, increment $n$ and try again.
\end{itemize}
However, if a word is nontrivial, this algorithm doesn't halt. It would work if: we know the max number (finite) of subsitutions we need to try, just try everything. The catch is we need to know this number, which we don't always know. 

\subsection{Applying geometry}
Let's go back to the original string rewriting system, with $A=\{a,a^{-1},b,b^{-1}\} $ and $R=\{[a,b],aa^{-1},a^{-1}a,bb^{-1},b^{-1}b\} $. Mapping works to $(x,y)$ pairs: consider \[
    \{\text{words in} \ A\} \longrightarrow \{(x,y) \ \text{pairs}\} ,
\] where $w \mapsto (x,y)$, $x=$ number of $a$'s minus number of $a^{-1}$'s, and $y=$ number of $b$'s minus the number of $b^{-1}$'s. There's a nice correspondence between equivalence classes of words to $(x,y)$ pairs. Equivalent words map to some $(x,y)$ pair, and inequivalent words map to different $(x,y)$ pairs. Unfortunately, now he's gonna draw a picture which I don't have the skills to do in real time yet. Think of the $(x,y)$ pairs as a grid, and draw some lines. Let $(0,0)=1,\,(0,1)=a,\,(1,0)=b,(1,1)=ab,(-1,0)=a^{-1},(-1,1)=a^{-1}b$ and so on. In this graph, the vertex set $V \mapsto \{\text{equivalence classes of words}\} $, and the edge set $E \mapsto \{\text{words}\} $. For example, for a word $aabba^{-1}a^{-1}b^{-1}$, draw a path starting at $a$ and you'll end up at $b$ (are these Cayley graphs??) Loops are special paths that correspond to trivial words. 

We can think about operations geometrically: cancelling an inverse means getting rid of backtracking. Another operation is given by the commutator relation: it turns out this is the same thing as adding or deleting loops at a point on the path. For example, for a word $aab^{-1}b^{-1}a^{-1}a^{-1}bb$, which is a loop. Apply the commutator relation, then this word maps to $aab^{-1}b^{-1}a^{-1}[aba^{-1}b^{-1}]a^{-1}bb$, this adds a unit loop at $(1,-2)$. However, we can cancel inverses, and then examining the interior it looks like we deleted the aformentioned loop. 

Idea: adding a substitution to a word changes the area enclosed by at most $1$. Good thing is, if our loop encloses an area $V$, then we need to apply at most $V$ substitutions to make it trivial. This is what we wanted for the brute force algorithm to work, the largest number of substitutions we can make.
\textbf{Question:} How much area can a loop of length $L$ in $\R^2$ enclose? This problem is called \emph{Dido's problem}, in which she was told to enclose the land she was to receive with a finite string (she somehow managed to get all of Carthage?) We have $A=\frac{L^2}{4\pi}$ for an arc of length $L$ (not entirely sure). Given a word of length $L$:
\begin{itemize}
    \item Apply a substitution of the form $aba^{-1}b^{-1}=1$,
    \item Get ride of backtracking,
    \item Check if the word is $1$,
    \item Keep going until we've applied $L^2$ subsitutions,
    \item If the word isn't $1$, start over with a different sequence.
\end{itemize}
The idea is, there's only a finite amount of words with length $L^2$, so this algorithm will work.

\subsection{Cayley graphs}
Here we are: Given a group presentation $(A,R)$, where the vertices are just the equivalence classes of words in $A$. $v_1\sim v_2$ if a word representing $v_2$ is $w_1a$, where $w_1$ represents $v_1$ and $a\in A$. Basically, edges on the Cayley graph represent words and points are equivalence classes. We can also look at the Cayley graph of $F_2=\Z*\Z$, the free group on two letters. Regular graph: every vertex has the same degree, in this case every vertex can serve as the basepoint of the graph. Basically, $F_2$ is the universal cover of $S^{1} \vee S^{1} $. 

\begin{example}
    Complicated example: $A=\{a,b,c,d,e\} ,$ $R=\{aa=bb=cc=dd=ee\} $ plus a lot of other relations. Then this is a tiling with five things of $90^{\circ}$, doesn't work in Euclidian space but it works in non-Euclidian space. How does this help with solving the word problem? Look at loops in hyperbolic space.

    Question: How much area can we enclose with a loop of length $L$ is the hyperbolic plane? This time, of length $L$, Area $\leq 3L+20$, need a long string to enclose less area. So this is a linear isometric inequality\footnote{I don't think I got the long words right...}. This solves the word problem! Go through $3L+20$ relations, if we aren't dont, it's not the identity. 
\end{example}

